\def\ChineseScale{1200}
\documentclass[12pt,a4paper,titlepage]{article}
\usepackage{times}

%\usepackage{CJK}
%CJK中文支持

%=========================xeCJK设置==============================
\usepackage{xeCJK}
%xeCJK中文支持

\def\CJKecglue{\hskip 0.15em}
%设置 CJK 文字与西文、文字与行内数学公式之间的间距, CJK默认值是一个空格。

\setCJKmainfont{Hiragino Sans GB}
\setCJKmonofont{Hiragino Sans GB}
%设置正文等宽族的 CJK 字体, 影响 \ttfamily 和 \texttt 的字体。 为了有利于等宽字体的 代码对齐等情形, xeCJK 在 ⟨font features⟩ 里增加了 Mono 这个选项。

%\setmainfont{Courier}
%================================================================

\usepackage{latexsym,bm}
\usepackage{indentfirst}
%自动首行缩进
\usepackage[top=1in, bottom=1in, left=0.5in, right=0.5in]{geometry}
\usepackage[CJKbookmarks,colorlinks=true]{hyperref}

%CJKbookmarks支持中文标签, colorlinks=true 就是把超链接的边框去掉,但字是红色的
%用来调页边距
\author{mayf3}
\title{Standard Source code Library}

\begin{document}

%CJK的时候使用
%\begin{CJK}{UTF8}{gbsn}

\maketitle
\tableofcontents

\newpage
\section{\LARGE 图论}
	\subsection{最短路算法}
    \subsubsection{Dijkstra最短路}
\begin{verbatim}
#include <cstdio>
#include <cstring>
#include <algorithm>
#include <map>

/*
 * name     :     dijkstra(STL)
 * usage     :    single-source shortest path(only non-negative weight)
 * develop    :    small label first optimization, negative circle
 * space complexity    :    O(M)
 * time complexity    :    O(NlogN)
 * checked    :    no
 */

const int N = 111111; //number of the vertices

int n, m;
int dist[N];
vector<PII> E[N];

int calc(int s, int e) {
    priority_queue<PII, vector<PII>, greater<PII> > Q;
    rep(i, n) dist[i] = -1;
    dist[s] = 0;
    Q.push(MP(0, s));
    int x, y, cost;
    while (!Q.empty()) {
        x = Q.top().Y, cost = Q.top().X;
        Q.pop();
        if (cost > dist[x]) continue;
        rep(i, E[x].size()){
            y = E[x][i].X, cost = E[x][i].Y;
            if (dist[y] == -1 || dist[y] != -1 && dist[x] + cost > dist[y]){
                dist[y] = dist[x] + cost;
                Q.push( make_pair(dist[y], y) );
            }
        }
    }
    return dist[e];
}

int main(){
    while(~scanf("%d%d", &n, &m)){
        rep(i, n) E[i].clear();
        int x, y, c;
        rep(i, m){
            scanf("%d%d%d", &x, &y, &c);
            x--, y--;
            E[x].PB(MP(y, c));
            E[y].PB(MP(x, c));
        }
        printf("%d\n", calc(0, n - 1));
    }
    return 0;
}
\end{verbatim}

    \subsubsection{spfa最短路}
\begin{verbatim}
#include <cstdio>
#include <cstring>
#include <algorithm>
#include <queue>

/*
 * name     :     spfa
 * usage     :    single-source shortest path, differential restraint system
 * develop    :    small label first optimization, negative circle
 * space complexity    :    O(M)
 * time complexity    :    O(k * M) (where k is usually less than 2)
 * checked    :    no
 */

const int N = 10000;

int n, m;
vector<PII> E[N];
int dist[N];

int spfa(int s, int e){
    static deque<int> Q;
    static bool inque[N];
    Cls(inque);
    memset(dist, -1, sizeof dist);
    dist[s] = 0;
    Q.PB(s);
    inque[s] = true;
    int x, y, c;
    while(!Q.empty()){
        x = Q.front();
        Q.pop_front();
        inque[x] = false;
        rep(i, E[x].size()){
            y = E[x][i].X;
            c = E[x][i].Y;
            if (dist[y] == -1 || dist[y] != -1 && dist[y] > dist[x] + c){
                dist[y] = dist[x] + c;
                if (!inque[y]){
                    Q.PB(y);
                    inque[y] = true;
                }
            }
        }
    }
    return dist[e];
}

int main(){
    while(~scanf("%d%d", &n, &m)){
        rep(i, n) E[i].clear();
        int x, y, c;
        rep(i, m){
            scanf("%d%d%d", &x, &y, &c);
            x--, y--;
            E[x].PB(MP(y, c));
            E[y].PB(MP(x, c));
        }
        printf("%d\n", spfa(0, n - 1));
    }
    return 0;
}
\end{verbatim}

	\subsubsection{k最短路(无环)}
\begin{verbatim}
#include <cstdio>
#include <cstring>
#include <algorithm>
#include <map>

using namespace std;

const int MAXN = 50 + 10; //number of vertices
const int MAXK = 200 + 10;
const int INF = 1000000000; //max dist

struct Tpath {
    int cnt, len, pos;
    int v[MAXN];
};

Tpath path[MAXK];
int g[MAXN][MAXN];
int len[MAXK], pos[MAXK], ans[MAXK];
bool used[MAXN];
int dist[MAXN], prev[MAXN], List[MAXN];
int N, M, K, S, T, cnt;

void Dijkstra() {
    int visited[MAXN];
    for (int i = 0; i <= N; ++i) dist[i] = INF, visited[i] = 0;
    dist[T] = 0;
    for (int k, i = T; i != N; i = k) {
        visited[i] = 1; k = N;
        for (int j = 0; j < N; ++j) {
            if (visited[j] || used[j]) continue;
            if (g[j][i] > -1 && dist[i] + g[j][i] < dist[j]) {
                dist[j] = dist[i] + g[j][i];
                prev[j] = i;
            }
            if (dist[j] < dist[k]) k = j;
        }
    }
}

void setPath(int v, Tpath &p) {
    p.len = 0;
    while (1) {
        p.v[p.cnt++] = v;
        if (v == T) return;
        p.len += g[v][prev[v]]; v = prev[v];
    }
}

void solve() {
    memset(used, 0, sizeof(used));
    Dijkstra();
    memset(ans, -1, sizeof(ans));
    if (dist[S] == INF)    return;
    multimap<int, int> Q; Q.clear();
    path[0].cnt = 0; path[0].pos = 0; setPath(S, path[0]); Q.insert( make_pair(path[0].len, 0) );
    int tot = 1;
    for (int i = 0; i < K; ++i) {
        if (Q.empty()) return;
        multimap<int, int> :: iterator p = Q.begin();
        int x = (*p).second;
        ans[i] = path[x].len;
        if (i == K - 1) break;
        memset(used, 0, sizeof(used));
        Tpath cur; cur.cnt = 0; cur.len = 0;
        for (int sum = 0, j = 0; j + 1 < path[x].cnt; ++j) {
            cur.v[cur.cnt++] = path[x].v[j]; used[path[x].v[j]] = 1;
            if (j) sum += g[path[x].v[j - 1]][path[x].v[j]];
            if (j >= path[x].pos) {
                Dijkstra();
                int u = path[x].v[j];
                for (int v = 0; v < N; ++v)
                    if (g[u][v] > -1 && !used[v] && dist[v] < INF && v != path[x].v[j + 1]) {
                        Tpath tp = cur; tp.pos = j + 1; setPath(v, tp); tp.len += sum + g[u][v];
                        if (tot < K) path[tot] = tp, Q.insert( make_pair(tp.len, tot++) );
                        else {
                            multimap<int, int> :: iterator p = Q.end(); --p;
                            if (tp.len >= (*p).first) continue;
                            path[(*p).second] = tp; Q.insert( make_pair(tp.len, (*p).second) );
                            Q.erase(p);
                        }
                    }
            }
        }
        Q.erase(p);
    }
}

void DFS(int step, int u, int len) {
    if (!cnt) return;
    if (u == T) {
        if (len == ans[K - 1]) {
            if (!(--cnt)) {
                for (int j = 0; j < step; ++j) {
                    if (j) printf("-");
                    printf("%d", List[j] + 1);
                }
                printf("\n");
            }
        }
        return;
    }
    Dijkstra();
    int tmp[MAXN];
    for (int i = 0; i < N; ++i) tmp[i] = dist[i];
    for (int i = 0; i < N; ++i)
        if (g[u][i] > -1 && !used[i] && tmp[i] < INF && len + g[u][i] + tmp[i] <= ans[K - 1]) {
            used[i] = 1; List[step] = i;
            DFS(step + 1, i, len + g[u][i]);
            if (!cnt) return;
            used[i] = 0;
        }
}

int main() {
    scanf("%d%d%d%d%d", &N, &M, &K, &S, &T);
    --S; --T;
    if (S == T) ++K;
    memset(g, -1, sizeof(g));
    for (int i = 0; i < M; ++i) {
        int u, v, w;
        scanf("%d%d%d", &u, &v, &w);
        --u; --v;
        g[u][v] = w;
    }
    solve();
    if (ans[K - 1] == -1) printf("None\n");
    else {
        cnt = 0;
        for (int i = 0; i < K; ++i)
            if (ans[i] == ans[K - 1]) ++cnt;
        memset(used, 0, sizeof(used));
        used[S] = 1; List[0] = S;
        DFS(1, S, 0);
    }
    return 0;
}

\end{verbatim}

	\input{code/graph/kth_shortest_path}

    \subsection{生成树}
	\input{code/graph/mst}
    %\input{code/graph/krusal}
    \subsubsection{prim最小生成树}
\begin{verbatim}
const int MAXN = 2000 + 1; //number of vertices + 1
const int MAXM = 10000; //number of edges
const int INF = 2000000000; //max weight

struct Tedge {
    int v, w, next;
};

Tedge edge[MAXM * 2];
int first[MAXN], dist[MAXN], heap[MAXN], pos[MAXN];
bool used[MAXN];
int N, M, cnt;

void init() {
    memset(first, -1, sizeof(first));
    M = 0;
}

inline void add_edge(int u, int v, int w) {
    edge[M].v = v; edge[M].w = w; edge[M].next = first[u]; first[u] = M++;
}

inline void moveup(int i) {
    int key = heap[i];
    while (i > 1 && dist[heap[i >> 1]] > dist[key]) heap[i] = heap[i >> 1], pos[heap[i]] = i, i >>= 1;
    heap[i] = key; pos[key] = i;
}

inline void movedown(int i) {
    int key = heap[i];
    while ((i << 1) <= cnt) {
        int j = i << 1;
        if (j < cnt && dist[heap[j + 1]] < dist[heap[j]]) ++j;
        if (dist[key] <= dist[heap[j]]) break;
        heap[i] = heap[j]; pos[heap[i]] = i; i = j;
    }
    heap[i] = key; pos[key] = i;
}

void Prim() {
    memset(used, 0, sizeof(used));
    for (int i = 0; i < N; ++i) pos[i] = -1, dist[i] = INF;
    cnt = 1; heap[1] = 0; dist[0] = 0;
    while (cnt) {
        int u = heap[1];
        used[u] = 1;
        heap[1] = heap[cnt--];
        movedown(1);
        for (int i = first[u]; i != -1; i = edge[i].next) {
            int v = edge[i].v, w = edge[i].w;
            if (!used[v] && w < dist[v]) {
                dist[v] = w;
                if (pos[v] == -1) pos[v] = ++cnt, heap[cnt] = v;
                moveup(pos[v]);
            }
        }
    }
}
\end{verbatim}

    \subsubsection{prim(stl)最小生成树}
\begin{verbatim}
const int MAXN = 2000; //number of vertices
const int INF = 2000000000; //max weight

vector < pair<int, int> > edge[MAXN];
int dist[MAXN];
bool used[MAXN];
int N;

void Prim() {
	priority_queue < pair<int, int>, vector< pair<int, int> >, greater< pair<int, int> > > Q;
	memset(used, 0, sizeof(used));
	for (int i = 0; i < N; ++i) dist[i] = INF;
	dist[0] = 0; Q.push( make_pair(0, 0) );
	while (!Q.empty()) {
		int u = Q.top().second, d = Q.top().first;
		Q.pop();
		used[u] = 1;
		if (d > dist[u]) continue;
		for (int i = 0; i < edge[u].size(); ++i) {
			int v = edge[u][i].first, w = edge[u][i].second;
			if (w < dist[v]) {
				dist[v] = w;
				Q.push( make_pair(dist[v], v) );
			}
		}
	}
}
\end{verbatim}

	\input{code/graph/aroad}
	\input{code/graph/mdst}

	\subsection{网络流}
    \input{code/graph/network_mode}
	\input{code/graph/maxflow}
    %\input{code/graph/sap}
	%\input{code/graph/dinic}
	\input{code/graph/min_cost_max_flow}
    \input{code/graph/networkflow_limit}
    \input{code/graph/euler_circuit}
    \input{code/graph/network_cutpoint}
	\subsubsection{stoer wagner最小割集}
\begin{verbatim}
const int MAXN = 50 + 10; //number of vertices
const int MAXM = 500; //number of MinCut edges
const int INF = 1000000000; //max capacity

int map[MAXN][MAXN], a[MAXN][MAXN], idx[MAXN][MAXN]; //map, tmp map, idx of edge
int root[MAXN], q[MAXN], w[MAXN], pre[MAXN];
int list[MAXM]; //MinCut Edges
bool used[MAXN];
int N, M;

int mincut(int n) {
    memset(used, 0, sizeof(used));
    memset(w, 0, sizeof(w));
    int last, cnt = 0;
    for (int k, i = 0; i != n; i = k) {
        last = i; used[i] = 1; k = n;
        for (int j = 0; j < n; ++j) {
            if (used[j]) continue;
            w[j] += a[q[i]][q[j]]; pre[j] = i;
            if (w[j] > w[k]) k = j;
        }
    }
    return last;
}

int find(int x) {
    if (root[x] == x) return x;
    else return root[x] = find(root[x]);
}

int stoer_wagner() {
    memcpy(a, map, sizeof(map));
    for (int i = 0; i < N; ++i) q[i] = root[i] = i;
    int ret = INF;
    for (int i = 0; i < N - 1; ++i) {
        int t = mincut(N - i);
        ret = min(ret, w[t]);
        int s = pre[t];
        for (int j = 0; j < N - i; ++j)
            if (j != s && j != t) a[q[t]][q[j]] = (a[q[j]][q[t]] += a[q[j]][q[s]]);
        root[find(q[s])] = find(q[t]); q[s] = q[N - i - 1]; 
    }
    return ret;
}

void cal(int ans) {
    memcpy(a, map, sizeof(map));
    for (int i = 0; i < N; ++i) q[i] = root[i] = i;
    int t;
    for (int i = 0; i < N - 2; ++i) {
        t = mincut(N - i - 1);
        if (w[t] == ans) break;
        int s = pre[t];
        for (int j = 0; j < N - i - 1; ++j)
            if (j != s && j != t) a[q[t]][q[j]] = (a[q[j]][q[t]] += a[q[j]][q[s]]);
        root[find(q[s])] = find(q[t]); q[s] = q[N - i - 2]; 
    }
    t = find(q[t]);
    M = 0; //number of MinCut edges
    for (int i = 0; i < N; ++i)
        if (find(root[i]) == t)
            for (int j = 0; j < N; ++j)
                if (find(root[j]) != t && idx[i][j]) list[M++] = idx[i][j];
}
\end{verbatim}

	\subsubsection{Dijkstra最短路}
\begin{verbatim}
#include <cstdio>
#include <cstring>
#include <algorithm>

using namespace std;

const int maxn = 200 + 10;
const int maxm = maxn * maxn;
const int INF = 100000000;

struct Tedge {
	int v, f, c, next;
};


Tedge edge[maxm];
int first[maxn], level[maxn], nedge[maxn], pedge[maxn], prev[maxn], queue[maxn], par[maxn], fl[maxn];
int a[maxn][maxn], cut[maxn][maxn];
int n, m, S, T;

inline void add_edge(int u, int v, int c1, int c2 = 0) {
	edge[m].v = v; edge[m].f = 0; edge[m].c = c1; edge[m].next = first[u]; first[u] = m++;
	edge[m].v = u; edge[m].f = 0; edge[m].c = c2; edge[m].next = first[v]; first[v] = m++;
}

bool newphase() {
	for (int i = 0; i < n; ++i) level[i] = n, nedge[i] = first[i];
	queue[0] = S; level[S] = 0;
	for (int h = 0, t = 1; h < t; ++h) {
		int u = queue[h];
		for (int i = first[u]; i != -1; i = edge[i].next)
			if (edge[i].f < edge[i].c && level[edge[i].v] == n) {
				level[edge[i].v] = level[u] + 1;
				if (edge[i].v == T) return 1;
				queue[t++] = edge[i].v;
			}
	}
	return 0;
}

bool find_path(int u) {
	for (int i = nedge[u]; i != -1; i = edge[i].next)
		if (edge[i].f < edge[i].c && level[edge[i].v] == level[u] + 1)
			if (edge[i].v == T || find_path(edge[i].v)) {
				pedge[edge[i].v] = nedge[u] = i;
				return 1;
			}
	nedge[u] = -1;
	return 0;
}

int Dinic() {
	for (int i = 0; i < m; ++i) edge[i].f = 0;
	int ret = 0;
	while (newphase())
		while (find_path(S)) {
			int delta = INF;
			for (int u = T, i = pedge[u]; u != S; u = edge[i ^ 1].v, i = pedge[u])
				delta = min(delta, edge[i].c - edge[i].f);
			for (int u = T, i = pedge[u]; u != S; u = edge[i ^ 1].v, i = pedge[u])
				edge[i].f += delta, edge[i ^ 1].f -= delta;
			ret += delta;
		}
	return ret;
}

int main() {
	int N;
	scanf("%d", &N);
	for (int tst = 1; tst <= N; ++tst) {
		m = 0;
		memset(first, -1, sizeof(first));
		scanf("%d", &n);
		for (int i = 0; i < n; ++i)
			for (int j = 0; j < n; ++j) {
				scanf("%d", &a[i][j]);
				if (i < j && a[i][j]) add_edge(i, j, a[i][j], a[i][j]);
			}

		memset(cut, 0, sizeof(cut));
		memset(par, 0, sizeof(par));
		for (S = 1; S < n; ++S) {
			T = par[S];
			fl[S] = cut[S][T] = cut[T][S] = Dinic();
			for (int i = 1; i < n; ++i)
				if (i != S && level[i] != n && par[i] == T) par[i] = S;
			if (level[par[T]] != n) {
				par[S] = par[T];
				par[T] = S;
				fl[S] = fl[T];
				fl[T] = cut[S][T];
			}
			for (int i = 0; i < S; ++i)
				if (i != T) cut[S][i] = cut[i][S] = min(cut[S][T], cut[T][i]);
		}
		// (i, par[i]) of value fl[i][par[i]] are the edges of GH cut tree

		printf("Case #%d:\n", tst);
		for (int i = 0; i < n; ++i) {
			for (int j = 0; j < n; ++j) {
				if (j) printf(" ");
				printf("%d", cut[i][j]);
			}
			printf("\n");
		}
	}

	return 0;
}
\end{verbatim}


	\subsection{匹配}
    \input{code/graph/hungarian}
	\input{code/graph/hopcroft_carp}
    \subsubsection{二分图最大权匹配}
\begin{verbatim}
const int N = 105, inf = 0x3F3F3F3F;

int n;
int graph[N][N];
int match[N], slack[N], lx[N], ly[N];
bool vx[N], vy[N];

bool find(int x) {
    vx[x] = true;
    rep(y, n){
        if (vy[y]) continue;
        if (lx[x] + ly[y] == graph[x][y]){
            vy[y] = true;
            if (match[y] == -1 || find(match[y])){
                match[y] = x;
                return true;
            }
        }
        else slack[y] = min(slack[y], lx[x] + ly[y] - graph[x][y]);
    }
    return false;
}

int max_match(int n) {
    rep(i, n) {
        lx[i] = *max_element(graph[i], graph[i] + n);
        ly[i] = 0;
    }
    memset(match, -1, sizeof(match));
    rep(x, n){
        memset(slack, -1, sizeof slack);
        while(true){
            Cls(vx);
            Cls(vy);
            if (find(x)) break;
            int sub = inf;
            rep(i, n) if (!vy[i]) sub = min(sub, slack[i]);
            rep(i, n) if (vx[i]) lx[i] -= sub;
            rep(i, n) {
                if (vy[i]) ly[i] += sub;
                else slack[i] -= sub;
            }
        }
    }
    int res = 0;
    rep(i, n) res += graph[match[i]][i];
    return res;
}

int min_match(int n) {
    rep(i, n) rep(j, n) graph[i][j] *= -1;
    return -max_match(n);
}

int main() {
    while(~scanf("%d", &n)){
        rep(i, n) rep(j, n) scanf("%d", &graph[i][j]);
        printf("%d\n", max_match(n));
    }
    return 0;
}
\end{verbatim}

	\input{code/graph/edmonds_blossom}

    \subsection{图}
	%\input{code/graph/tarjan}
	\subsubsection{最大团}
\begin{verbatim}
const int MAXN = 100; //number of vertices

int a[MAXN][MAXN];
int f[MAXN];
int N, ans;

bool DFS(int q[], int t, int cnt) {
    if (t == 0) {
        if (cnt > ans) {
            ans = cnt;
            return 1;
        }
        return 0;
    }

    int tq[MAXN];
    for (int i = 0; i < t; ++i) {
        if (f[q[i]] + cnt <= ans) return 0;
        int k = 0;
        for (int j = i + 1; j < t; ++j)
            if (a[q[i]][q[j]]) tq[k++] = q[j];
        if (DFS(tq, k, cnt + 1)) return 1;
    }
    return 0;
}

void MaxClique() {
    ans = 0;
    int q[MAXN];
    for (int i = N - 1; i >= 0; --i) {
        int t = 0;
        for (int j = i + 1; j < N; ++j) if (a[i][j]) q[t++] = j;
        DFS(q, t, 1);
        f[i] = ans;
    }
}
\end{verbatim}

	\subsubsection{树链剖分}
\begin{verbatim}
//By myf
//#pragma comment(linker, "/STACK:16777216")  //C++
#include <iomanip>
#include <iostream>
#include <algorithm>
#include <cmath>
#include <cstdio>
#include <cstdlib>
#include <cstring>
#include <bitset>
#include <complex>
#include <map>
#include <set>
#include <queue>
#include <deque>
#include <stack>
#include <vector>
#include <list>

#define rep(i,n) for(int i=0;i<(n);i++)
#define REP(i,l,r) for(int i=(l);i<(r);i++)
#define fab(i,a,b) for(int i=(a);i<=(b);i++)
#define fba(i,b,a) for(int i=(b);i>=(a);i--)
//#define foreach(i,n) for(__typeof(n.begin()) i=n.begin();i!=n.end();i++) //G++
#define MP make_pair
#define PB push_back
#define X first
#define Y second
#define Cls(x) memset(x,0,sizeof x)
#define Print(n,x) for(int i=0;i<(n);i++) cout<<x<<" ";cout<<endl;
#define Sqr(x) (x)*(x)

using namespace std;

typedef long long LL;
typedef pair<int,int> PII;
typedef pair<PII,int> PIII;
typedef pair<LL,int> PLI;
typedef vector<int> VI;
typedef LL T;

const int N=50005,M=1<<16;

int n,m,q,tot;
int v[N];
int t[M*2];
VI E[N];
int fa[N],dep[N],son[N],sz[N];
int id[N],top[N];

void dfs(int x){
    sz[x]=1,son[x]=0;
    rep(i,E[x].size()){
        int y=E[x][i];
        if (y==fa[x]) continue;
        dep[y]=dep[x]+1;
        fa[y]=x;
        dfs(y);
        sz[x]+=sz[y];
        if (sz[y]>sz[son[x]]) son[x]=y;
    }
}

void dfs(int x,int p){
    id[x]=++tot,top[x]=p;
    if (son[x]) dfs(son[x],p);
    rep(i,E[x].size()){
        int y=E[x][i];
        if (y==fa[x]||y==son[x]) continue;
        dfs(y,y);
    }
}

int ask(int x){
    x=id[x];
    int ret=0;
    for(x+=M;x;x>>=1) ret+=t[x];
    return ret;
}

void insert(int l,int r,int x){
    for(l+=M-1,r+=M+1;l^r^1;l>>=1,r>>=1){
        if (~l&1) t[l^1]+=x;
        if ( r&1) t[r^1]+=x;
    }
}

void add(int x,int y,int k){
    while(top[x]!=top[y]){
        if (dep[top[x]]<dep[top[y]]) swap(x,y);
        insert(id[top[x]],id[x],k);
        x=fa[top[x]];
    }
    if (dep[x]<dep[y]) swap(x,y);
    insert(id[y],id[x],k);
}

int main(){
    while(~scanf("%d%d%d",&n,&m,&q)){
        rep(i,n) scanf("%d",&v[i+1]);
        rep(i,n) E[i+1].clear();
        rep(i,m){
            int x,y;
            scanf("%d%d",&x,&y);
            E[x].PB(y);
            E[y].PB(x);
        }
        fa[1]=dep[1]=1;
        sz[0]=0,tot=0;
        dfs(1);
        dfs(1,1);
        Cls(t);
        fab(i,1,n) t[id[i]+M]=v[i];
        char ch;
        int x,y,k;
        rep(i,q){
            while((ch=getchar())&&ch!='D'&&ch!='Q'&&ch!='I');
            if (ch=='Q'){
                scanf("%d",&x);
                printf("%d\n",ask(x));
            }
            else{
                scanf("%d%d%d",&x,&y,&k);
                add(x,y,(ch=='I')?k:-k);
            }
        }
    }
    return 0;
}

    \input{code/graph/kosaraju}
    \subsubsection{割点和桥}
\begin{verbatim}
const int N = 1111, M = 1111111;

int n, m;
int root;
int low[N], dep[N];
bool cut[N], bri[N];
vector<int> E[N];
vector<int> id[N];
PII edge[N];

void dfs(int x, int f, int d){
    int e = 0, deg = 0;
    low[x] = dep[x] = d;
    rep(i, E[x].size()){
        int y = E[x][i];
        if (low[y] == -1){
            deg++;
            dfs(y, x, d + 1);
            low[x] = min(low[x], low[y]);
            if (low[y] > dep[x]) bri[id[x][i]] = true;
            cut[x] |= (x == root && deg > 1 || x != root && low[y] >= dep[x]);
        }
        else if (y != f || e){
            low[x] = min(low[x], dep[y]);
        }
        else e = 1;
    }
}


int main(){
    while(~scanf("%d%d",&n,&m)){
        rep(i, n) E[i].clear(), low[i] = dep[i] = -1, cut[i] = false;
        rep(i, n) id[i].clear();
        int x, y;
        rep(i, m){
            scanf("%d%d", &x, &y);
            x--, y--;
            bri[i] = false;
            edge[i] = MP(x, y);
            E[x].PB(y), id[x].PB(i);
            E[y].PB(x), id[y].PB(i);
        }
        dfs(root = 0, -1, 0);
    }
    return 0;
}
\end{verbatim}

    \input{code/graph/hamilton}
	\input{code/graph/tree_divide_edge}
	\subsubsection{点的分治,权值在点上}
\begin{verbatim}
const int N=50000+10,M=30;

int n,k;
VI E[N];
int tot,top,mi,root;
int size[N];
LL f[N];
map<LL,int> Q;
LL prime[N];
int sta[N][M];
LL q[N];
bool use[N];
LL base[M+1];

LL ans;

inline LL add(LL x,int sta[M]){
    LL y=0;
    rep(i,k){
        int tmp=(x%base[i+1])/base[i];
        tmp+=sta[i];
        tmp%=3;
        y+=(tmp*base[i]);
    }
    return y;
}

inline LL dec(LL a,LL b){
    LL y=0;
    rep(i,k){
        int tmp1=(a%base[i+1])/base[i];
        int tmp2=(b%base[i+1])/base[i];
        int tmp=(tmp1-tmp2+3)%3;
        y+=(tmp*base[i]);
    }
    return y;
}

inline LL add(LL a,LL b){
    LL y=0;
    rep(i,k){
        int tmp1=(a%base[i+1])/base[i];
        int tmp2=(b%base[i+1])/base[i];
        int tmp=(tmp1+tmp2)%3;
        y+=(tmp*base[i]);
    }
    return y;
}


void getVal(int x,LL val,int fa){
    q[top++]=val;
    rep(i,E[x].size()){
        int y=E[x][i];
        if (use[y]||y==fa)continue;
        getVal(y,add(val,sta[y]),x);
    }
}

void getRoot(int x,int fa){
    int big=-1;
    size[x]=1;
    rep(i,E[x].size()){
        int y=E[x][i];
        if (use[y]||y==fa) continue;
        getRoot(y,x);
        size[x]+=size[y];
        big=max(big,size[y]);
    }
    big=max(big,tot-size[x]);
    if (big<mi) mi=big,root=x;
}

void dfs(int x){
     tot=mi=size[x];
     getRoot(x,-1);
     x=root;
     use[x]=true;
     Q.clear();
     LL now=add(0,sta[x]);
     Q[now]=1;
     if (now==0) ans++;
     rep(i,E[x].size()){
         int y=E[x][i];
         if (use[y]) continue;
         top=0;
         getVal(y,add(0,sta[y]),x);
         rep(j,top){
             LL tmp=dec(0,q[j]);
             if (Q.count(tmp)) ans+=Q[tmp];
         }
         rep(j,top) Q[add(now,q[j])]++;
     }
     rep(i,E[x].size()) if (!use[E[x][i]]) dfs(E[x][i]);
}

int main(){
    base[0]=1;
    fab(i,1,M) base[i]=base[i-1]*3;
    while(~scanf("%d",&n)){
        Cls(use);
        rep(i,n) E[i].clear();
        scanf("%d",&k);
        rep(i,k) scanf("%I64d",prime+i);
        rep(i,n){
            LL x;
            scanf("%I64d",&x);
            Cls(sta[i]);
            rep(j,k) while (x%prime[j]==0) x/=prime[j],sta[i][j]++,sta[i][j]%=3;
        }
        rep(i,n-1){
            int x,y;
            scanf("%d%d",&x,&y);
            x--,y--;
            E[x].PB(y);
            E[y].PB(x);
        }
        size[0]=n;
        ans=0;
        dfs(0);
        printf("%I64d\n",ans);
    }
    return 0;
}
\end{verbatim}

	\subsubsection{图平面化}
\begin{verbatim}
// vertices numbered from 1 to N
// No self-loops and no duplicate edges

typedef pair<int, int> T;
const int maxn = 10000 + 10;

struct node {
    int dep, fa, infc, used, vst, dfi, ec, lowp, bflag, flag, lowpoint;
};

int n, m, indee, p1, p2, p, ps;
int lk[maxn * 3][2], child[maxn * 3][3], bedg[maxn * 3][2], sdlist[maxn * 6][3], 
    buk[maxn * 6][2], exf[maxn * 3][2], proots[maxn * 3][3], stk[maxn * 3][2], infap[maxn * 3];
int w1[maxn], w2[maxn], que[maxn];
node dot[maxn];

void init(T * ts) {
    ps = 0;
    for (int i = 1; i <= n; ++i) w1[i] = i;
    p1 = n;
    for (int i = 0; i < m; ++i) {
        int k1 = ts[i].first, k2 = ts[i].second;
        lk[++p1][0] = k2; lk[p1][1] = 0;
        lk[w1[k1]][1] = p1;
        w1[k1] = p1;
        lk[++p1][0] = k1; lk[p1][1] = 0;
        lk[w1[k2]][1] = p1;
        w1[k2] = p1;
    }
    for (int i = 1; i <= n; ++i) que[i] = i;
}

int deep(int a) {
    dot[a].used = 1; dot[a].dfi = ++indee;
    int t = lk[a][1];
    while (t != 0) {
        int tmp = lk[t][0];
        if (!dot[tmp].used) {
            dot[tmp].fa = a; dot[tmp].dep = dot[a].dep + 1; dot[tmp].ec = dot[a].dep; dot[tmp].lowp = dot[a].dep;
            child[++p1][0] = tmp; child[p1][1] = 0;
            child[w1[a]][1] = p1;
            w1[a] = p1;
            int s = deep(tmp);
            if (s < dot[a].ec) dot[a].ec = s;
        }
        else if (dot[a].fa != tmp) {
            if (dot[a].lowp > dot[tmp].dep) dot[a].lowp = dot[tmp].dep;
            if (dot[a].dfi > dot[tmp].dfi) {
                bedg[++p2][0] = a; bedg[p2][1] = 0;
                bedg[w2[tmp]][1] = p2;
                w2[tmp] = p2;
            }
        }
        t = lk[t][1];
    }
    if (dot[a].ec > dot[a].lowp) dot[a].ec = dot[a].lowp;
    return dot[a].ec;
}

void sortvtx() {
    for (int i = 1; i <= n; ++i) w1[i] = i;
    p1 = n; p2 = 0;
    for (int i = 1; i <= n; ++i) {
        buk[++p1][0] = i; buk[p1][1] = 0;
        buk[w1[dot[i].dfi]][1] = p1;
        w1[dot[i].dfi] = p1;
    }
    for (int i = n; i > 0; --i) {
        int tmp = buk[i][1];
        while (tmp != 0) {
            que[++p2] = buk[tmp][0];
            tmp = buk[tmp][1];
        }
    }
}

void getsdlist() {
    memset(buk, 0, sizeof(buk));
    for (int i = 1; i <= n; ++i) {
        w1[i] = w2[i] = i;
        buk[i][1] = 0;
    }
    p1 = p2 = n;
    for (int i = 1; i <= n; ++i) {
        buk[++p1][0] = i; buk[p1][1] = 0;
        buk[w1[dot[i].ec]][1] = w1[dot[i].ec] = p1;
    }
    for (int i = 1; i <= n; ++i) {
        int tmp = buk[i][1];
        while (tmp != 0) {
            int fa = dot[buk[tmp][0]].fa;
            sdlist[++p2][0] = i; sdlist[p2][1] = 0;
            sdlist[w2[fa]][1] = dot[buk[tmp][0]].infc = p2;
            sdlist[p2][2] = w2[fa]; w2[fa] = p2;
            tmp = buk[tmp][1];
        }
    }
}

void getnextvtx(int v, int v1, int &m, int &m1) {
    m = exf[v][v1 ^ 1];
    if (exf[m][0] == v) m1 = 0;
    else m1 = 1;
}

void addwei(int a) {
    int fa = dot[a - n].fa;
    ++p1;
    proots[p1][0] = a; proots[p1][1] = 0;
    proots[w1[fa]][1] = p1;
    proots[p1][2] = w1[fa]; w1[fa] = p1;
    infap[a] = p1;
}

void addsou(int a) {
    int fa = dot[a - n].fa;
    ++p1;
    proots[p1][0] = a; proots[p1][1] = proots[fa][1]; proots[p1][2] = fa;
    proots[fa][1] = p1;
    proots[proots[p1][1]][2] = p1;
    infap[a] = p1;
    if (w1[fa] == fa) w1[fa] = p1;
}

void walkup(int v, int w) {
    dot[w].bflag = v;
    int x = w, x1 = 1, y = w, y1 = 0;
    while (x != v) {
        if (dot[x].vst == v || dot[y].vst == v) break;
        dot[x].vst = v; dot[y].vst = v;
        int z1 = 0;
        if (x > n) z1 = x;
        if (y > n) z1 = y;
        if (z1 != 0) {
            int c = z1 - n, z = dot[c].fa;
            if (z != v) {
                if (dot[c].lowpoint < dot[v].dep) addwei(z1);
                else addsou(z1);
            }
            x = z; x1 = 1;
            y = z; y1 = 0;
        } else {
            getnextvtx(x, x1, x, x1);
            getnextvtx(y, y1, y, y1);
        }
    }
}

void getactivenext(int v, int v1, int &m, int &m1, int vt) {
    m = v; m1 = v1;
    getnextvtx(m, m1, m, m1);
    while (dot[m].bflag != vt && proots[m][1] == 0 && dot[m].ec >= dot[vt].dep && m != v) getnextvtx(m, m1, m, m1);
}

void addstack(int a, int b) {
    stk[++ps][0] = a; stk[ps][1] = b;
}

void mergestack() {
    int t = stk[ps][0], t1 = stk[ps][1], k = stk[ps - 1][0], k1 = stk[ps - 1][1];
    ps -= 2;
    int s1, s = exf[t][1 ^ t1];
    if (exf[s][1] == t) s1 = 1;
    else s1 = 0;
    exf[k][k1] = s;
    exf[s][s1] = k;
    int tmp = dot[t - n].infc;
    sdlist[sdlist[tmp][2]][1] = sdlist[tmp][1]; sdlist[sdlist[tmp][1]][2] = sdlist[tmp][2];
    tmp = dot[t - n].fa;
    if (sdlist[tmp][1] == 0) dot[tmp].ec = dot[tmp].lowp;
    else dot[tmp].ec = min(dot[tmp].lowp, sdlist[sdlist[tmp][1]][0]);
    tmp = infap[t];
    int fa = dot[t - n].fa;
    proots[proots[tmp][2]][1] = proots[tmp][1];
    if (proots[tmp][1] != 0) proots[proots[tmp][1]][2] = proots[tmp][2];
    else w1[fa] = proots[tmp][2];
}

void embededg(int v, int v1, int w, int w1) {
    exf[v][v1] = w; exf[w][w1] = v;
}

void walkdown(int v) {
    ps = 0;
    int vt = dot[v - n].fa;
    for (int v2 = 0; v2 <= 1; ++v2) {
        int w, w1;
        getnextvtx(v, 1 ^ v2, w, w1);
        while (w != v) {
            if (dot[w].bflag == vt) {
                while (ps != 0) mergestack();
                embededg(v, v2, w, w1);
                dot[w].bflag = 0;
            }
            if (proots[w][1] != 0) {
                addstack(w, w1);
                int x, x1, y, y1, w2, w0 = proots[proots[w][1]][0];
                getactivenext(w0, 1, x, x1, vt);
                getactivenext(w0, 0, y, y1, vt);
                if (dot[x].ec >= dot[vt].dep) w = x, w1 = x1;
                else if (dot[y].ec >= dot[vt].dep) w = y, w1 = y1;
                else if (dot[x].bflag == vt || proots[x][1] != 0) w = x, w1 = x1;
                else w = y, w1 = y1;
                if (w == x) w2 = 0;
                else w2 = 1;
                addstack(w0, w2);
            }
            else if (w > n || dot[w].ec >= dot[vt].dep) getnextvtx(w, w1, w, w1);
            else {
                if (w <= n && dot[w].ec < dot[vt].dep && ps == 0) embededg(v, v2, w, w1);
                break;
            }
        }
        if (ps != 0) break;
    }
}

bool chainvtx(int a) {
    for (int t = child[a][1]; t != 0; t = child[t][1]) {
        int tmp = child[t][0];
        exf[tmp][1] = tmp + n; exf[tmp][0] = tmp + n;
        exf[tmp + n][1] = tmp; exf[tmp + n][0] = tmp;
    }
    for (int t = bedg[a][1]; t != 0; t = bedg[t][1]) walkup(a, bedg[t][0]);
    for (int t = child[a][1]; t != 0; t = child[t][1]) walkdown(child[t][0] + n);
    for (int t = bedg[a][1]; t != 0; t = bedg[t][1]) if (dot[bedg[t][0]].bflag != 0) return false;
    return true;
}

bool judge(int N, int M, T * ts) {
    n = N;    m = M;
    if (n == 1) return true;
    if (m > 3 * n - 5) return false;
    init(ts);
    
    for (int i = 1; i <= n; ++i) {
        proots[i][1] = 0; proots[i + n][1] = 0;
        p = 0;
        child[i][1] = 0;
        buk[i][1] = 0; buk[i + n][1] = 0;
        sdlist[i][1] = 0; sdlist[i + n][1] = 0;
        dot[i].bflag = 0; dot[i + n].flag = 0;
    }
    for (int i = 1; i <= n; ++i) {
        w1[i] = i; w2[i] = i;
        child[i][1] = 0; bedg[i][1] = 0;
        dot[i].used = 0;
    }
    indee = 0; p1 = p2 = n;
    for (int i = 1; i <= n; ++i) {
        if (!dot[i].used) {
            dot[i].dep = 1;
            deep(i);
        }
    }
    sortvtx();
    getsdlist();
    for (int i = 1; i <= n; ++i) {
        dot[i].lowpoint = dot[i].ec;
        dot[i].vst = 0; dot[i + n].vst = 0;
        proots[i][1] = 0;
        w1[i] = i;
    }
    p1 = n;
    for (int i = 1; i <= n; ++i) if (!chainvtx(que[i])) return false;
    return true;
}

T ts[maxn];
bool a[3001][3001];

int main() {
    int N, M;
    scanf("%d%d", &N, &M);
    int m = 0;
    for(int i = 0; i < M; i ++) {
        scanf("%d%d", &ts[i].first, &ts[i].second);
        ++ts[i].first; ++ts[i].second;
        if (ts[i].first == ts[i].second || a[ts[i].first][ts[i].second]) continue;
        a[ts[i].first][ts[i].second] = a[ts[i].second][ts[i].first] = 1;
        ts[m++] = ts[i];
    }
    M = m;
    if(judge(N, M, ts)) puts("YES");
    else puts("NO");

    return 0;
}
\end{verbatim}

	\subsubsection{有向图割点}
\begin{verbatim}
const int MAXN = 5000 + 10; //number of vertices
const int MAXM = 200000 + 10; //number of edges

struct Tedge {
    int v, next;
};

Tedge edge[MAXM], back[MAXM]; //back is opposite to edge
bool ontree[MAXM];
int first1[MAXN], first2[MAXN], id[MAXN], low[MAXN], stack[MAXN];
bool critical[MAXN]; //1 - the node is CutVertex, 0 - not
int N, M, cnt;

void DFS(int u) {
    id[u] = cnt; stack[cnt++] = u; low[u] = u;
    for (int i = first1[u]; i != -1; i = edge[i].next)
        if (id[edge[i].v] == -1) {
            ontree[i] = 1;
            DFS(edge[i].v);
        }
}

void update(int u) {
    for (int i = first1[u]; i != -1; i = edge[i].next)
        if (ontree[i] && id[low[u]] < id[low[edge[i].v]]) {
            low[edge[i].v] = low[u];
            update(edge[i].v);
        }
}

void CV() {
    cnt = 0;
    memset(id, -1, sizeof(id));
    memset(ontree, 0, sizeof(ontree));
    DFS(0);
    memset(critical, 0, sizeof(critical));
    critical[0] = 1;
    for (int i = cnt - 1; i >= 0; --i) {
        int u = stack[i];
        for (int j = first1[u]; j != -1; j = edge[j].next)
            if (ontree[j] && low[edge[j].v] == u) {
                critical[u] = 1;
                break;
            }
        for (int j = first2[u]; j != -1; j = back[j].next)
            if (id[low[back[j].v]] < id[low[u]]) low[u] = low[back[j].v];
        update(u);
    }
}
\end{verbatim}


    %\subsection{最近公共祖先算法}
    %\input{code/graph/lca_rmq}
    %\input{code/graph/tarjan}
    %\input{code/graph/ancient_tree}

\section{\LARGE 数据结构}
	\subsection{平衡树}
    \input{code/data_structure/heap}
    \input{code/data_structure/treap}
	\input{code/data_structure/binary_tree}
    \input{code/data_structure/bst}
	\input{code/data_structure/splay}
	%\input{code/data_structure/zkw_tree}

	\subsection{图上的数据结构}
	\input{code/data_structure/leftist_tree}
    \subsubsection{动态树}
\begin{verbatim}
/*
    You are given a tree with N nodes.
    The tree’s nodes are numbered 1 through N and its edges are numbered 1 through N - 1.
    Each edge is associated with a weight. Then you are to execute a series of instructions on the tree.
    The instructions can be one of the following forms:
        CHANGE i v Change the weight of the ith edge to v 
        NEGATE a b Negate the weight of every edge on the path from a to b 
        QUERY a b Find the maximum weight of edges on the path from a to b 
*/
#include <cstdio>
#include <cstring>
#include <algorithm>

using namespace std;

const int MAXN = 10010;
const int INF = 2000000000;

struct Node {
    int child[2]; 
    int parent, typ, value, maxvalue, minvalue;
    bool neg;
}; 

struct Tedge {
    int v, w, next;
};

Tedge edge[MAXN * 2];
Node node[MAXN];
int first[MAXN], bottom[MAXN];
bool used[MAXN];
int N;

inline int cal_max(int x) {
    if (!x) return -INF;
    if (node[x].neg) return -node[x].minvalue;
    else return node[x].maxvalue;
}

inline int cal_min(int x) {
    if (!x) return INF;
    if (node[x].neg) return -node[x].maxvalue;
    else return node[x].minvalue;
}

inline void update(int x) {
    if (node[x].neg) {
        node[x].value = -node[x].value;
        swap(node[x].maxvalue, node[x].minvalue);
        node[x].maxvalue = -node[x].maxvalue; node[x].minvalue = -node[x].minvalue;
        node[node[x].child[0]].neg = !node[node[x].child[0]].neg;
        node[node[x].child[1]].neg = !node[node[x].child[1]].neg;
        node[x].neg = 0;
    }
    if (x > 1) node[x].maxvalue = node[x].value;
    else node[x].maxvalue = -INF;
    node[x].maxvalue = max(node[x].maxvalue, cal_max(node[x].child[0]));
    node[x].maxvalue = max(node[x].maxvalue, cal_max(node[x].child[1]));
    if (x > 1) node[x].minvalue = node[x].value;
    else node[x].minvalue = INF;
    node[x].minvalue = min(node[x].minvalue, cal_min(node[x].child[0]));
    node[x].minvalue = min(node[x].minvalue, cal_min(node[x].child[1]));
}

void cal(int x) {
    if (node[x].parent == 0) return;
    cal(node[x].parent);
    update(x);
}

void rotate(int x, int a) {
    int y = node[x].parent, b = node[y].typ;
    node[x].parent = node[y].parent; node[x].typ = b;
    if (b != 2) node[node[y].parent].child[b] = x;
    b = 1 - a;
    node[node[x].child[b]].parent = y; node[node[x].child[b]].typ = a;
    node[y].child[a] = node[x].child[b]; node[y].parent = x; node[y].typ = b;
    node[x].child[b] = y;
    update(y);
} 

void splay(int x) {
    cal(x);
    while (1) {
        int a = node[x].typ;
        if (a == 2) break;
        int y = node[x].parent, b = node[y].typ;
        if (a == b) rotate(y, a);
        else rotate(x, a);
        if (b == 2) break;
        rotate(x, b);
    }
    update(x);
} 

void expose(int v) {
    int u = v, w = 0;
    while (u) {
        splay(u);
        node[node[u].child[0]].typ = 2;
        node[u].child[0] = w;
        node[w].typ = 0;
        update(u);
        w = u; u = node[u].parent;
    }
    splay(v);
} 

inline void link(int v, int w) {
    expose(v);
    expose(w);
    node[v].parent = w;
}

void DFS(int u) {
    used[u] = 1;
    for (int i = first[u]; i != -1; i = edge[i].next) {
        int v = edge[i].v, w = edge[i].w;
        if (!used[v]) {
            node[v].value = node[v].maxvalue = node[v].minvalue = w;
            bottom[i >> 1] = v;
            link(v, u);
            DFS(v);
        }
    }
}

void readTrees() {
    memset(first, -1, sizeof(first));
    scanf("%d", &N);
    for (int i = 0; i < N - 1; ++i) {
        int u, v, w;
        scanf("%d%d%d", &u, &v, &w);
        edge[i * 2].v = v; edge[i * 2].w = w; edge[i * 2].next = first[u]; first[u] = i * 2;
        edge[i * 2 + 1].v = u; edge[i * 2 + 1].w = w; edge[i * 2 + 1].next = first[v]; first[v] = i * 2 + 1;
    }
    for (int i = 0; i <= N; ++i) {
        node[i].child[0] = node[i].child[1] = 0; node[i].typ = 2;
        node[i].parent = 0; node[i].neg = 0;
    }
    node[0].maxvalue = -INF; node[0].minvalue = INF;
    node[1].maxvalue = -INF; node[1].minvalue = INF;
    memset(used, 0, sizeof(used));
    DFS(1);
}

int query(int u, int v) {
    if (u == v) return 0;
    expose(u);
    int x = v, w = 0, ret = -INF;
    while (v) {
        splay(v);
        if (node[v].parent == 0) {
            ret = max(ret, cal_max(node[v].child[0]));
            ret = max(ret, cal_max(w));
        }
        node[node[v].child[0]].typ = 2;
        node[v].child[0] = w;
        node[w].typ = 0;
        update(v);
        w = v; v = node[v].parent;
    }
    splay(x);
    return ret;
}

void negate(int u, int v) {
    expose(u);
    int x = v, w = 0;
    while (v) {
        splay(v);
        if (node[v].parent == 0) {
            node[node[v].child[0]].neg = !node[node[v].child[0]].neg;
            node[w].neg = !node[w].neg;
        }
        node[node[v].child[0]].typ = 2;
        node[v].child[0] = w;
        node[w].typ = 0;
        update(v);
        w = v; v = node[v].parent;
    }
    splay(x);
}

void solve() {
    char cmd[10];
    while (scanf("%s", cmd), cmd[0] != 'D') {
        int x, y;
        scanf("%d%d", &x, &y);
        
        if (cmd[0] == 'Q') printf("%d\n", query(x, y));
        else if (cmd[0] == 'C') {
            x = bottom[x - 1];
            splay(x);
            node[x].value = y; node[x].neg = 0;
            update(x);
        }
        else negate(x, y);
    }
}

int main() {
    int t;
    for (scanf("%d", &t); t > 0; --t) {
        readTrees();
        solve();
    }
    return 0;
}
\end{verbatim}

    \subsubsection{支持子树操作的动态树}
\begin{verbatim}
//By myf
//#pragma comment(linker, "/STACK:16777216")  //C++
#include <iomanip>
#include <iostream>
#include <algorithm>
#include <cmath>
#include <cstdio>
#include <cstdlib>
#include <cstring>
#include <bitset>
#include <complex>
#include <map>
#include <set>
#include <queue>
#include <deque>
#include <stack>
#include <vector>
#include <list>

#define rep(i,n) for(int i=0;i<(n);i++)
#define REP(i,l,r) for(int i=(l);i<(r);i++)
#define fab(i,a,b) for(int i=(a);i<=(b);i++)
#define fba(i,b,a) for(int i=(b);i>=(a);i--)
//#define foreach(i,n) for(__typeof(n.begin()) i=n.begin();i!=n.end();i++) //G++
#define MP make_pair
#define PB push_back
#define X first
#define Y second
#define Cls(x) memset(x,0,sizeof x)
#define Print(n,x) for(int i=0;i<(n);i++) cout<<x<<" ";cout<<endl;
#define inf 0xc0c0c0c0
#define lch ch[x][0]
#define rch ch[x][1]

using namespace std;

typedef long long LL;
typedef pair<int,int> PII;
typedef pair<PII,int> PIII;
typedef pair<LL,int> PLI;
typedef vector<int> VI;
typedef int T;

const int N=333333;

int n;
int pre[N],fa[N],fat[N],val[N],ma[N],ch[N][2];
bool black[N];
multiset<int> Q[N];
VI E[N];

inline void up(int x){ma[x]=max(max(val[x],*Q[x].rbegin()),max(ma[lch],ma[rch]));}

inline void rot(int id,int tp){
    static int k;
    k=pre[id];
    ch[k][tp^1]=ch[id][tp];
    if(ch[id][tp]) pre[ch[id][tp]]=k;
    if(pre[k]) ch[pre[k]][k==ch[pre[k]][1]]=id;
    pre[id]=pre[k];
    ch[id][tp]=k;
    pre[k]=id;
    up(k);
}

inline void splay(int x){
    if (!pre[x]) return;
    int tmp;
    for(tmp=x;pre[tmp];tmp=pre[tmp]);
    for(swap(fa[x],fa[tmp]);pre[x];rot(x,x==ch[pre[x]][0]));
    up(x);
}

inline int access(int x){
    int nt;
    for(nt=0;x;x=fa[x]){
        splay(x);
        if (rch){
            fa[rch]=x;
            pre[rch]=0;
            Q[x].insert(ma[rch]);
        }
        rch=nt;
        if (nt){
            fa[nt]=0;
            pre[nt]=x;
            Q[x].erase(Q[x].find(ma[nt]));
        }
        up(nt=x);
    }
    return nt;
}

void make(int x,int f){
    fat[x]=f;
    rep(i,E[x].size()) if (E[x][i]!=f) make(E[x][i],x);
    int t;
    up(x+n);up(x+2*n);
    fa[t=x+(1+black[x])*n]=x;
    Q[x].insert(*Q[t].rbegin());
    up(x);
    fa[x]=t=f+(1+black[x])*n;
    Q[t].insert(ma[x]);
}

void cut(int x,int f){
    access(f);
    splay(f);
    splay(x);
    Q[f].erase(Q[f].find(ma[x]));
    fa[x]=0;
    up(f);
}

void link(int x,int f){
    access(f);
    splay(f);
    splay(x);
    fa[x]=f;
    Q[f].insert(ma[x]);
    up(f);
}

int main(){
    while(~scanf("%d",&n)){
        Cls(pre);
        Cls(ch);
        Cls(fa);
        rep(i,n+1) E[i].clear();
        rep(i,n-1){
            int x,y;
            scanf("%d%d",&x,&y);
            E[x].PB(y);
            E[y].PB(x);
        }
        n++;
        rep(i,3*n+1) Q[i].clear();
        rep(i,3*n+1) ma[i]=val[i]=inf,Q[i].insert(inf);
        REP(i,1,n) scanf("%d",black+i);
        REP(i,1,n) scanf("%d",val+i);
        make(1,n);
        int q,k,x;
        scanf("%d",&q);
        rep(i,q){
            scanf("%d%d",&k,&x);
            if (k==0){
                for(x=access(x);lch;x=lch);
                splay(x);
                printf("%d\n",ma[rch]);
            }
            if (k==1){
                cut(x,fat[x]+(1+black[x])*n);
                cut(x+(1+black[x])*n,x);
                black[x]^=1;
                link(x+(1+black[x])*n,x);
                link(x,fat[x]+(1+black[x])*n);
            }
            if (k==2){
                access(x);
                splay(x);
                scanf("%d",val+x);
                up(x);
            }
        }
    }
    return 0;
}
\end{verbatim}


	\subsection{可持久化数据结构}
	\input{code/data_structure/function_segment_tree}
	\subsubsection{函数式treap}
\begin{verbatim}
//By Lin
#include<cstdio>
#include<cstring>
#include<cstdlib>
using namespace std;

struct Node{
    int key,weight,size;
    Node *l,*r;
    Node(int _key , int _weight, Node *_l, Node* _r):
        key(_key),weight(_weight),l(_l),r(_r){
            size = 1;
            if ( l ) size += l->size;
            if ( r ) size += r->size;
        }
    Node *newnode(int key){
        return new Node(key,rand(),NULL,NULL);
    }
    inline int lsize(){ return l?l->size:0; }
    inline int rsize(){ return r?r->size:0; }
}*root[50005];

Node* Meger(Node *a , Node *b ){
    if ( !a || !b ) return a?a:b;
    return a->weight>b->weight?
         new Node(a->key,a->weight,a->l,Meger(a->r,b)):
         new Node(b->key,b->weight,Meger(a,b->l),b->r);
}

Node* Split_L(Node *a ,int size ){
    if ( !a || size == 0 ) return NULL;
    return a->lsize() < size?
        new Node(a->key,a->weight,a->l,Split_L(a->r,size-1-a->lsize())):
        Split_L(a->l,size);
}

Node* Split_R(Node *a ,int size ){
    if ( !a || size == 0 ) return NULL;
    return a->rsize() < size?
        new Node(a->key,a->weight,Split_R(a->l,size-1-a->rsize()),a->r):
        Split_R(a->r,size);
}

int Ask( Node *a ,int k ){
    if ( a->lsize() >= k ) return Ask(a->l,k);
    k -= a->lsize()+1;
    if ( k == 0 ) return a->key;
    return Ask(a->r,k);
}

int len = 0;

int main(){
    int d = 0 , cas;
    scanf("%d", &cas);
    root[0] = NULL;
    int cnt = 1,kind,v,p,c;
    char s[1005];
    while ( cas -- ) {
        scanf("%d", &kind );
        if ( kind == 1 ) {
            scanf("%d%s", &p , s );
            p-=d;
            Node *l = Split_L(root[cnt-1],p),
                 *r = Split_R(root[cnt-1],len-p);
            for (int i = 0; s[i]; i++ ){
                l = Meger(l,new Node(s[i],rand(),NULL,NULL));
                len++;
            }
            root[cnt++] = Meger(l,r);
        }
        else if ( kind == 2){
            scanf("%d%d", &p , &c );
            p-=d,c-=d;
            Node *l = Split_L(root[cnt-1],p-1),
                 *r = Split_R(root[cnt-1],len-p-c+1);
            len -= c;
            root[cnt++] = Meger(l,r);
        }
        else{
            scanf("%d%d%d", &v, &p , &c );
            v-=d,p-=d,c-=d;
            char ch;
            for (int i = p; i<p+c; i++) {
                printf("%c", ch = Ask(root[v],i) );
                if ( ch == 'c' ) d++;
            }
            puts("");
        }
    }
    return 0;
}
\end{verbatim}

    %\subsubsection{划分树}
\begin{verbatim}
const int D = 18;
const int N = 100000 + 1000;

struct Tree{
	int n;
	int v[N];
	int val[D][N], to_left[D][N];
	LL sum_l[D][N];

	void build(int l, int r, int deep){
		if (l == r) return;
		int mid = (l + r) / 2, left_same = mid - l + 1;
		for(int i = l; i <= r; i++){
			if (val[deep][i] < v[mid]) left_same--;
		}
		int le = l, ri = mid + 1, same = 0;
		sum_l[deep][0] = 0;
		for(int i = l; i <= r; i++){
			to_left[deep][i] = (i == l) ? 0 : to_left[deep][i - 1];
			sum_l[deep][i] = sum_l[deep][i - 1];
			if (val[deep][i] < v[mid]){
				to_left[deep][i]++;
				sum_l[deep][i] += val[deep][i];
				val[deep + 1][le++] = val[deep][i];
			}
			else if (val[deep][i] > v[mid]){
				val[deep + 1][ri++] = val[deep][i];
			}
			else if (same < left_same){
				to_left[deep][i]++;
				sum_l[deep][i] += val[deep][i];
				val[deep + 1][le++] = val[deep][i];
				same++;
			}
			else{
				val[deep + 1][ri++] = val[deep][i];
			}
		}
		build(l, mid, deep + 1);
		build(mid + 1, r, deep + 1);
	}

	pair<int, LL> ask(int ask_l, int ask_r, int l, int r, int deep, int kth){ 
		if (l == r) return MP(val[deep][l], val[deep][l]);
		int mid = (l + r) / 2, s1, s2;
		if (l == ask_l){
			s1 = 0;
			s2 = to_left[deep][ask_r];
		}
		else{
			s1 = to_left[deep][ask_l - 1];
			s2 = to_left[deep][ask_r];
		}
		if (s2 - s1 >= kth){
			ask_l = l + s1, ask_r = ask_l + s2 - s1 - 1;
			return ask(ask_l, ask_r, l, mid, deep + 1, kth);
		}
		else{
			LL ret = sum_l[deep][ask_r] - sum_l[deep][ask_l - 1];
			kth = kth - (s2 - s1);
			s2 = ask_r - ask_l + 1 - (s2 - s1);
			ask_l = mid + ask_l - l + 1 - s1;
			ask_r = ask_l + s2 - 1;
			pair<int, LL> tmp = ask(ask_l, ask_r, mid + 1, r, deep + 1, kth);
			return MP(tmp.F, tmp.S + ret);
		}
	}

	void init(int n, int other[]){
		for(int i = 1; i <= n; i++){
			v[i] = other[i];
			val[0][i] = v[i];
		}
		sort(v + 1, v + n + 1);
		build(1, n, 0);
	}
}tree;
\end{verbatim}


\section{\LARGE 字符串算法}
	\subsection{基础}
	\input{code/string/string}
	\input{code/string/hash}
    \subsubsection{kmp}
\begin{verbatim}
#include <cstdio>
#include <cstring>
#include <algorithm>
using namespace std;

const int N = 1000010;

int next[N];

int kmp(char *s, int n, char *t, int m) {
    int i, j;
    next[0] = -1;
    i = 0; j = -1;
    while (i < m) {
        if (j == -1 || t[i] == t[j]) {
            i++; j++;
            next[i] = (t[i] == t[j] ? next[j] : j);
        } else {
            j = next[j];
        }
    }
    i = j = 0;
    while (i < n && j < m) {
    if (j == -1 || s[i] == t[j]) {
        i++; j++;
    } else {
        j = next[j];
    }
}
return (j >= m ? i - m : -1);
}
\end{verbatim}

    %\input{code/string/trie_tree}
    \input{code/string/min_representation}
    \input{code/string/manacher}

	\subsection{进阶}
    %\input{code/string/suffix}
	\input{code/string/suffix_automata}
    %\input{code/string/trie_graph}

\section{\LARGE 计算几何}
	\subsection{平面几何基础}
    %\input{code/geometry/geometry}
    \subsubsection{Point}
\begin{verbatim}
const double EPS = 1E-8;
const double INF = 1E10;
const double PI = acos(-1.0);

typedef complex<double> Point;

double cross(Point a, Point b){
    return a.X * b.Y - a.Y * b.X;
}

double cross(Point a, Point b, Point c){
    return cross(b - a, c - a);
}

double dot(Point a, Point b){
    return a.X * b.X + a.Y * b.Y;
}

double dot(Point a, Point b, Point c){
    return dot(b - a, c - a);
}

double dist(Point a, Point b){
    return abs(a - b);
}

Point rotate(Point v, double alpha){
    double c = cos(alpha), s = sin(alpha);
    return Point(v.X * c - v.Y * s, v.X * s + v.Y * c);
}

double angle(Point a, Point b){
    return arg(b - a);
}
\end{verbatim}

    \subsubsection{Line}
\begin{verbatim}
typedef pair<Point, Point> Line;

bool  inter(Line a, Line b, Point &p){
    double s1 = cross(a.F, a.S, b.F);
    double s2 = cross(a.F, a.S, b.S);
    if (!sign(s1 - s2)) return false;
    p = (s1 * b.S - s2 * b.F) / (s1 - s2);
    return true;
}
\end{verbatim}

    \subsubsection{圆}
\begin{verbatim}
struct Circle{
    Point o;
    double r;
    Circle(Point o = Point(), double r = 1) : o(o), r(r){}
    Circle(double x, double y, double r = 1) : o(x, y), r(r){}
};

int intersected_circle_line(Circle c, Line l){
    return sign(dist_line_point(l, c.o) - c.r) < 0;
}

int ip_circle_line(Circle c, Line l, Point &p1, Point &p2){
    Point a = l.p, b = l.q;
    double dx = b.x - a.x;
    double dy = b.y - a.y;
    double sdr = Sqr(dx) + Sqr(dy);
    double dr = sqrt(sdr);
    double d, disc, x, y;
    a.x -= c.o.x; a.y -= c.o.y;
    b.x -= c.o.x; b.y -= c.o.y;
    d = a.x * b.y - b.x * a.y;
    disc = Sqr(c.r) * sdr - Sqr(d);
    if (disc < -EPS) return 0;
    if (disc < +EPS){
        disc = 0;
    }
    else{
        disc = sqrt(disc);
    }
    x = disc * dx * (dy > 0 ? 1 : -1);
    y = disc * fabs(dy);
    p1.x = (+d * dy + x) / sdr + c.o.x;
    p2.x = (+d * dy - x) / sdr + c.o.x;
    p1.y = (-d * dx + y) / sdr + c.o.y;
    p2.y = (-d * dx - y) / sdr + c.o.y;
    return disc > EPS ? 2 : 1;
}

int ip_circle_circle(const Circle &c1, const Circle &c2, Point &p1, Point &p2){
    double mx = c2.o.x - c1.o.x, sx = c2.o.x + c1.o.x, mx2 = Sqr(mx);
    double my = c2.o.y - c1.o.y, sy = c2.o.y + c1.o.y, my2 = Sqr(my);
    double sq = mx2 + my2, d = -(sq - Sqr(c1.r - c2.r)) * (sq - Sqr(c1.r + c2.r));
    if (!sign(sq)) return 0;
    if (d + EPS < 0) return 0;
    if (d < EPS){
        d = 0;
    }
    else{
        d = sqrt(d);
    }
    double x = mx * ((c1.r + c2.r) * (c1.r - c2.r) + mx * sx) + sx * my2;
    double y = my * ((c1.r + c2.r) * (c1.r - c2.r) + my * sy) + sy * mx2;
    double dx = mx * d, dy = my * d;
    sq *= 2;
    p1.x = (x + dy) / sq; p1.y = (y - dx) / sq;
    p2.x = (x - dy) / sq; p2.y = (y + dy) / sq;
    return d > EPS ? 2 : 1; 
}

double circle_circle_intersection_area(Circle A, Circle B){
    double d, dA, dB, tx, ty;
    d = hypot(B.o.x - A.o.x, B.o.y - A.o.y);
    if ((d < EPS) || (d + A.r <= B.r) || (d + B.r <= A.r)){
        return Sqr((B.r < A.r) ? B.r : A.r) * PI;
    }
    if (d >= A.r + B.r){
        return 0;
    }
    dA = tx = (Sqr(d) + Sqr(A.r) - Sqr(B.r)) / d / 2;
    ty = sqrt(Sqr(A.r) - Sqr(tx));
    dB = d - dA;
    return Sqr(A.r) * acos(dA / A.r) - dA * sqrt(Sqr(A.r) - Sqr(dA)) + Sqr(B.r) * acos(dB / B.r) - dB * sqrt(Sqr(B.r) - Sqr(dB));
}

/*
 * return 2 points of tangency of c and p
 */
void circle_tangents(Circle c, Point p, Point &a, Point &b){
    double d = Sqr(c.o.x - p.x) + Sqr(c.o.y - p.y);
    double para = Sqr(c.r) / d;
    double perp = c.r * sqrt(d - Sqr(c.r)) / d;
    a.x = c.o.x + (p.x - c.o.x) * para - (p.y - c.o.y) * perp;
    a.y = c.o.y + (p.y - c.o.y) * para + (p.x - c.o.x) * perp;
    b.x = c.o.x + (p.x - c.o.x) * para + (p.y - c.o.y) * perp;
    b.y = c.o.y + (p.y - c.o.y) * para - (p.x - c.o.x) * perp;
}

/*
 * +0 : on circle;
 * +1 : inside circle;
 * -1 : outside circle;
 */
int on_circle(Circle c, Point a){
    return sign(c.r - dist(a, c.o));
}

/*
 * minimum circle that covers 2 points
 */
Circle cc2(Point a, Point b){
    return Circle(mp(a, b), dist(a, b) / 2);
}

Circle cc3(Point p, Point q, Point r){
    Circle c;
    if (on_circle(c = cc2(p, q), r) >= 0) return c;
    if (on_circle(c = cc2(p, r), q) >= 0) return c;
    if (on_circle(c = cc2(q, r), p) >= 0) return c;
    c.o = ccc(p, q, r);
    c.r = dist(c.o, p);
    return c;
}

Circle min_circle_cover(Point p[], int n){
    if (n == 1) return Circle(p[0], 0);
    if (n == 2) return cc2(p[0], p[1]);
    random_shuffle(p, p + n);
    Point *ps[4] = {&p[0], &p[1], &p[2], &p[3]};
    Circle c = cc3(*ps[0], *ps[1], *ps[2]);
    while(true){
        Point *b = p;
        for(int i = 1; i < n; i++){
            if (dist(p[i], c.o) > dist(*b, c.o)) b = &p[i];
        }
        if (on_circle(c, *b) >= 0) return c;
        ps[3] = b;
        for(int i = 0; i < 3; i++){
            swap(ps[i], ps[3]);
            if (on_circle(c = cc3(*ps[0], *ps[1], *ps[2]), *ps[3]) >= 0) break;
        }
    }
}
\end{verbatim}

    \subsubsection{垂心,内心,外心}
\begin{verbatim}
point ip(line u, line v) {
  double n = (u.p.y - v.p.y) * (v.q.x - v.p.x) - (u.p.x - v.p.x) * (v.q.y - v.p.y);
  double d = (u.q.x - u.p.x) * (v.q.y - v.p.y) - (u.q.y - u.p.y) * (v.q.x - v.p.x);
  double r = n / d;
  return point(u.p.x + r * (u.q.x - u.p.x), u.p.y + r * (u.q.y - u.p.y));
}

Line perpendicular(Line l, Point a){
    return Line(a, Point(a.x + l.p.y - l.q.y, a.y + l.q.x - l.p.x));
}

Point pedal(Line l, Point a){
    return ip(l, perpendicular(l, a));
}

Point mirror(Line l, Point a){
    Point p = pedal(l, a);
    return Point(p.x * 2 - a.x, p.y * 2 - a.y);
}

//垂心
Point perpencenter(Point a, Point b, Point c){
    Line u = perpendicular(Line(b, c), a);
    Line v = perpendicular(Line(a, c), b);
    return ip(u, v);
}

//内心
Point icc(Point A, Point B, Point C){
    double a = dist(B, C);
    double b = dist(C, A);
    double c = dist(A, B);
    double p = (a + b + c) / 2;
    double s = sqrt(p * (p - a) * (p - b) * (p - c));
    Point cp;
    cp.x = (a * A.x + b * B.x + c * C.x) / (a + b + c);
    cp.y = (a * A.y + b * B.y + c * C.y) / (a + b + c);
    return cp;
}

//外心
Point ccc(Point A, Point B, Point C){
    double a1 = B.x - A.x, b1 = B.y - A.y, c1 = (Sqr(a1) + Sqr(b1)) / 2;;
    double a2 = C.x - A.x, b2 = C.y - A.y, c2 = (Sqr(a2) + Sqr(b2)) / 2;;
    double d = a1 * b2 - a2 * b1;
    Point cp;
    cp.x = A.x + (c1 * b2 - c2 * b1) / d;
    cp.y = A.y + (a1 * c2 - a2 * c1) / d;
    return cp;
}
\end{verbatim}

    \subsubsection{一般多边形}
\begin{verbatim}
/*
 * if point a inside polygon p[n]
 */
int inside_polygon(Point p[], int n, Point a){
	double sum = 0;
	for(int i = 0; i < n; i++){
		int j = (i + 1) % n;
		if (on_lineseg(Line(p[i], p[j]), a)) return 0;
		double angle = acos(dot(a, p[i], p[j]) / dist(a, p[i]) / dist(a, p[j]));
		sum += sign(cross(a, p[i], p[j])) * angle;
	}
	return sign(sum);
}

/*
 * if lineseg l strickly inside polygon p[n]
 */
int lineseg_inside_polygon(Point p[], int n, Line l){
	for(int i = 0; i < n; i++){
		int j = (i + 1) % n;
		Line l1(p[i], p[j]);
		if (on_lineseg_exclusive(l, p[i])) return 0;
		if (intersected_exclusive(l, l1)) return 0;
	}
	return inside_polygon(p, n, mp(l.p, l.q));
}

/*
 * if lineseg l intersect convex polygon p[n]
 */
int intersect_convex_lineseg(Point p[], int n, Line l){
	if (n < 3) return 0;
	Point q[4];
	int k = 0;
	q[k++] = l.p;
	q[k++] = l.q;
	for(int i = 0; i < n; i++){
		if (on_lineseg(l, p[i])){
			q[k++] = p[i];
		}
		else{
			int j = (i + 1) % n;
			Line a(p[i], p[j]);
			Point tmp = ip(a, l);
			if (on_lineseg(l, tmp) && on_lineseg(a, tmp)) q[k++] = tmp;
		}
	}
	sort(q, q + k);
	for(int i = 0; i + 1 < k; i++){
		if (inside_polygon(p, n, mp(q[i], q[i + 1]))) return 1;
	}
	return 0;
}

#define crossOp(p1,p2,p3) sign(cross(p1,p2,p3))

Point isSS(Point p1, Point p2, Point q1, Point q2) {
	double a1 = cross(q1,q2,p1), a2 = -cross(q1,q2,p2);
	return (p1 * a2 + p2 * a1) / (a1 + a2);
}

vector<Point> convexCut(const vector<Point>&ps, Point q1, Point q2) {
	vector<Point> qs;
	int n = ps.size();
	for (int i = 0; i < n; ++i) {
		Point p1 = ps[i], p2 = ps[(i + 1) % n];
		int d1 = crossOp(q1,q2,p1), d2 = crossOp(q1,q2,p2);
		if (d1 >= 0)
			qs.push_back(p1);
		if (d1 * d2 < 0)
			qs.push_back(isSS(p1, p2, q1, q2));
	}
	return qs;
}

typedef double Tdata;
typedef Point Tpoint;

struct Tline {
	Tdata a, b, c;
	double ang;
	Tline() {}
	Tline(Tdata a, Tdata b, Tdata c) : a(a), b(b), c(c) { ang = atan2(b, -a); }
	void get() { scanf("%lf%lf%lf", &a, &b, &c); }
	bool operator <(Tline l) const { return sign(ang - l.ang) < 0 || sign(ang - l.ang) == 0 && sign(c - l.c) < 0; }
};

inline int side(Tline l, Tpoint p) { return sign(l.a * p.x + l.b * p.y + l.c); }

// change line from two point form to general form
// O(1)
// return line(general form)
inline Tline change_line(Tpoint a, Tpoint b) {
	Tdata tmp, A = a.y - b.y, B = b.x - a.x, C = cross(a, b);
	if (sign(A)) tmp = fabs(A);
	else tmp = fabs(B);
	return Tline(A / tmp, B / tmp, C / tmp);
}

// calculate the area of polygon
// O(N)
// be careful the sign of the area
Tdata cal_area(Tpoint *P, int N) {
	if (N < 3) return 0;
	Tdata ret = 0;
	P[N] = P[0];
	for (int i = 0; i < N; ++i) ret += cross(P[i], P[i + 1]);
	return ret / 2.0;
}

// intersection of half-planes
// O(N log N)
// ax + by + c >= 0
// P - points form the intersection, M - number of points
void inter_hplane(Tline *H, int N, Tpoint *P, int &M) {
	int *queue = new int[N + 1];
	sort(H, H + N);
	M = 0;
	for (int i = 0; i < N; ++i)	if (!i || sign(H[i].ang - H[queue[M - 1]].ang)) queue[M++] = i;
	int h = 0, t = 2;
	for (int i = 2; i < M; ++i) {
		while (h + 1 < t && side(H[queue[i]], inter_point(H[queue[t - 1]], H[queue[t - 2]])) < 0) --t;
		while (h + 1 < t && side(H[queue[i]], inter_point(H[queue[h]], H[queue[h + 1]])) < 0) ++h;
		queue[t++] = queue[i];
	}
	while (h + 1 < t && side(H[queue[h]], inter_point(H[queue[t - 1]], H[queue[t - 2]])) < 0) --t;
	while (h + 1 < t && side(H[queue[t - 1]], inter_point(H[queue[h]], H[queue[h + 1]])) < 0) ++h;
	M = 0;
	for (int i = h; i < t; ++i) queue[M++] = queue[i];
	queue[M] = queue[0];
	for (int i = 0; i < M; ++i) P[i] = inter_point(H[queue[i]], H[queue[i + 1]]);
	delete [] queue;
}

// get the core of polygon
// O(N log N)
Tpoint core_of_poly(Tpoint *P, int N) {
	Tline *H = new Tline[N];
	Tpoint *A = new Tpoint[N];
	int M;
	P[N] = P[0];
	for (int i = 0; i < N; ++i) H[i] = change_line(P[i], P[i + 1]);
	inter_hplane(H, N, A, M);
	Tpoint ret = A[0];
	delete [] H; delete [] A;
	return ret;
}

// get the length of segment in convex polygon
// O(N)
Tdata seg_in_convex_poly(Tpoint a, Tpoint b, Tpoint *P, int N) {
	int d1 = point_in_convex_poly(a, P, N), d2 = point_in_convex_poly(b, P, N);
	if (d2 == 1) swap(d1, d2), swap(a, b);
	if (d2 == 1) return dist(a, b);
	Tpoint p;
	P[N] = P[0];
	if (d1 == 1)
		for (int i = 0; i < N; ++i) {
			int d = inter_seg(a, b, P[i], P[i + 1], p);
			if (d == 1 || d == 2) return dist(a, p); // not including the boundaries, add "d == 3" for including the boundaries
		}
	else {
		int cnt = 0;
		Tpoint u, v;
		for (int i = 0; i < N; ++i) {
			int d = inter_seg(a, b, P[i], P[i + 1], p);
			if (d == 3) return 0; // on the boundaries
			if (cnt == 2) continue;
			if (d)
				if (!cnt) u = p, ++cnt;
				else if (u != p) v = p, ++cnt;
		}
		return cnt == 2 ? dist(u, v) : 0;
	}
}

// get the centroid of polygon
// O(N)
Tpoint cal_centroid(Tpoint *P, int N) {
	P[N] = P[0];
	Tpoint c(0, 0);
	Tdata s = 0;
	for (int i = 0; i < N; ++i) {
		Tdata tmp = cross(P[i], P[i + 1]);
		c += (P[i] + P[i + 1]) * tmp; s += tmp;
	}
	return c / (3 * s);
}
\end{verbatim}


	\subsection{空间几何基础}
    %\input{code/geometry/geometry3d}

	\subsection{凸包}
	%\subsubsection{凸包}
\begin{verbatim}
// find the convex hull
Point __o;

bool cmp_p(Point a, Point b){
	int f = sign(a.X - b.X);
	if (f) return f < 0;
	return sign(a.Y - b.Y) < 0;
}

bool cmp(Point a, Point b){
	int f = sign(cross(o, a, b));
	if (f) return f > 0;
	return sign(abs(a - o) - abs(b - o)) < 0;
}

Point stack[1111]

int find_convex(Point p[], int n){
	__o = *min_element(p, p + n, cmp_p);
	sort(p, p + n, cmp);
	int top = 0;
	rep(i, n){
		while(top >= 2 && sign(cross(stack[top - 2], stack[top - 1], p[i])) <= 0) top--;
		stack[top++] = p[i];
	}
	rep(i, top) p[i] = stack[i];
	return top;
}

// -----intersection points convex hull--------
bool lcmp(Line u, Line v){
	int c = sign((u.p.x - u.q.x) * (v.p.y - v.q.y) - (v.p.x - v.q.x) * (u.p.y - u.q.y));
	return c < 0 || !c && sign(cross(u.p, u.q, v.p)) < 0;
}

/*
 * XXX sizeof(p) MUST be as large as n * 2
 * return # of points of resulting convex hull
 */
int ip_convex(Line l[], int n, Point p[]){
	for(int i = 0; i < n; i++){
		if (l[i].q < l[i].p) swap(l[i].p, l[i].q);
	}
	sort(l, l + n, lcmp);
	int n1 = 0;
	for(int i = 0, j = 0; i < n; i = j){
		while(j < n && parallel(l[i], l[j])) j++;
		if (j - i == 1){
			l[n1++] = l[i];
		}
		else{
			l[n1++] = l[i];
			l[n1++] = l[j - 1];
		}
	}
	n = n1;
	l[n + 0] = l[0];
	l[n + 1] = l[1];
	int m = 0;
	for(int i = 0, j = 0; i < n; i++){
		while(j < n + 2 && parallel(l[i], l[j])) j++;
		for(int k = j; k < n + 2 && parallel(l[j], l[k]);k++){
			p[m++] = ip(l[i], l[k]);
		}
	}
	return find_convex(p, m);
}

typedef double Tdata;
typedef Point Tpoint;
// get the diameter of convex polygon
// O(N)
// p1, p2 are the points forming diameter
Tdata diam_convex_poly(Tpoint *P, int N, Tpoint &p1, Tpoint &p2) {
	if (N == 1) {
		p1 = p2 = P[0];
		return 0;
	}
	double ret = -INF;
	for (int j = 1, i = 0; i < N; ++i) {
		while (sign(cross(P[i], P[i + 1], P[j + 1]) - cross(P[i], P[i + 1], P[j])) > 0) j = (j + 1) % N;
		ret = max(ret, max(dist2(P[i], P[j]), dist2(P[i + 1], P[j + 1])));
	}
	return ret;
}
\end{verbatim}

	%\input{code/geometry/convex3d}
	%\input{code/geometry/dynamic_convex}
	%\subsubsection{两凸包间最短距离}
\begin{verbatim}
const int maxn = 10000 + 10;
const double PI = acos(-1.0);
const double EPS = 1E-6;

struct Tpoint {
    double x, y;
};

Tpoint a[maxn], b[maxn];
int n, m;

inline double cross(double X1, double Y1, double X2, double Y2) {
    return X1 * Y2 - X2 * Y1;
}

double Area(Tpoint *a, int n) {
    double ret = 0;
    a[n] = a[0];
    for (int i = 0; i < n; ++i) ret += cross(a[i].x, a[i].y, a[i + 1].x, a[i + 1].y);
    return ret;
}

inline double Dist(Tpoint A, Tpoint B) {
    return sqrt((A.x - B.x) * (A.x - B.x) + (A.y - B.y) * (A.y - B.y));
}

inline double DistP2S(Tpoint P, Tpoint A, Tpoint B) {
    if ((B.x - A.x) * (P.x - A.x) + (B.y - A.y) * (P.y - A.y) < 0) return Dist(P, A);
    if ((A.x - B.x) * (P.x - B.x) + (A.y - B.y) * (P.y - B.y) < 0) return Dist(P, B);
    return fabs(cross(P.x - A.x, P.y - A.y, P.x - B.x, P.y - B.y)) / Dist(A, B);
}

double MinDist() {
    if (Area(a, n) < 0) reverse(a, a + n);
    if (Area(b, m) < 0) reverse(b, b + m);

    int p1 = 0, p2 = 0;
    for (int i = 0; i < n; ++i)
        if (a[i].x < a[p1].x) p1 = i;
    for (int i = 0; i < m; ++i)
        if (b[i].x > b[p2].x) p2 = i;

    int cnt = 0;
    double ret = dist(a[p1], b[p2]);
    while (cnt < n) {
        ret = min(ret, DistP2S(a[p1], b[p2], b[(p2 + 1) % m]));
        ret = min(ret, DistP2S(a[(p1 + 1) % n], b[p2], b[(p2 + 1) % m]));
        ret = min(ret, DistP2S(b[p2], a[p1], a[(p1 + 1) % n]));
        ret = min(ret, DistP2S(b[(p2 + 1) % m], a[p1], a[(p1 + 1) % n]));

        if (cross(a[(p1 + 1) % n].x - a[p1].x, a[(p1 + 1) % n].y - a[p1].y, b[p2].x - b[(p2 + 1) % m].x, b[p2].y - b[(p2 + 1) % m].y) > 0) p1 = (p1 + 1) % n, ++cnt;
        else p2 = (p2 + 1) % m;
    }

    return ret;
}
\end{verbatim}


	\subsection{平面}
	%\input{code/geometry/half_plane_intersection}
	%\subsubsection{动态半平面交}
\begin{verbatim}
#include <cstdio>
#include <set>
#include <map>

using namespace std;

const int N = 100000 + 10;
const int inf = 100001;
const double eps = 1e-10;

set<int> S;
int B[N];
double xval[N];
map<double, int> X;

void cal(int a1, int b1, int a2, int b2, double& x, double& y) {
	if (a2 == -inf) {
		x = 0;
		y = b1;
	}
	else {
		x = (b2 - b1) / double(a1 - a2);
		y = a1 * x + b1;
	}
}

int main() {
	int n, a, b, i;
	double x, y;
	char cmd[2];
	set<int>::iterator itr;
	S.insert(-inf);
	S.insert(0);
	B[inf] = 0;
	xval[inf] = inf;
	B[0] = 0;
	X[0] = 0;
	xval[0] = 0;
	scanf("%d", &n);
	for (i=1; i<=n; i++) {
		scanf("%s", cmd);
		if (cmd[0] == 'S') {
			scanf("%d %d", &a, &b);
			itr = S.lower_bound(a);
			if (*itr == a) {
				if (B[-*itr] >= b) continue; else B[-*itr] = b;
			}
			else {
				itr --;
				cal(a, b, *itr, B[-*itr], x, y);
				itr ++;
				if (y < *itr * x + B[-*itr] + eps) continue;
				itr = S.insert(a).first;
				B[-*itr] = b;
			}
			itr --;
			while (itr != S.begin() && *itr * xval[-*itr] + B[-*itr] < a * xval[-*itr] + b + eps) {
				x = xval[-*itr];
				X.erase(x);
				S.erase(itr --);
			}
			cal(a, b, *itr, B[-*itr], x, y);
			itr = S.find(a);
			itr ++;
			X.erase(xval[-*itr]);
			xval[-a] = x;
			X[x] = a;
			itr ++;
			while (itr != S.end() && *itr * xval[-*itr] + B[-*itr] < a * xval[-*itr] + b + eps) {
				itr --;
				S.erase(itr ++);
				x = xval[-*itr];
				X.erase(x);
				itr ++;
			}
			itr --;
			cal(a, b, *itr, B[-*itr], x, y);
			xval[-*itr] = x;
			X[x] = *itr;
		}
		else {
			scanf("%lf", &x);
			x = x * x;
			map<double, int>::iterator itr;
			if (x == 0) itr = X.lower_bound(0);
			else {
				itr = X.lower_bound(x);
				itr --;
			}
			printf("%.0lf\n", itr->second * x + B[-itr->second]);
		}
	}
	return 0;
}                                 
\end{verbatim}

    %\input{code/geometry/polygon_core}
	%\subsubsection{kd树,支持插入}
\begin{verbatim}

const int N = 500005,K = 2,D=6;
const LL inf = ((ULL)1<<63)-1;
//const int inf=~0U>>1;

struct kd{
    T x[K];
    kd(){rep(i,K)x[i]=0;}
} t[N];
int l[N],r[N];
int a[D],n,tot,root;

void insert(int &cur,kd p, int d) {
    if (!cur){
        cur=++tot;
        rep(i,K) t[cur].x[i] = p.x[i];
        l[cur]=r[cur]=0;
        return;
    }
    T dx = p.x[d] - t[cur].x[d];
    if (++d==K) d=0;
    insert(dx<0?l[cur]:r[cur],p,d);
}

T dis2(kd a,kd b) {
    T s=0;
    rep(i,K) s+=Sqr(a.x[i]-b.x[i]);
    return s;
}

void query(int cur, kd p, LL &ret, int d) {
    if (!cur) return;
    ret = min(ret, dis2(t[cur],p));
    T dx = p.x[d] - t[cur].x[d];
    if (++d == K) d = 0;
    if (dx < 0) {
        query(l[cur],p,ret,d);
        if (ret > Sqr(dx)) query(r[cur],p,ret,d);
    } else {
        query(r[cur],p,ret,d);
        if (ret > Sqr(dx)) query(l[cur],p,ret,d);
    }
}

void work() {
    root = tot = 0;
    T ans = inf, ret=0;
    kd p;
    rep(i,n){
        p.x[0] = (p.x[0] * a[0] + a[1]) % a[2];
        p.x[1] = (p.x[1] * a[3] + a[4]) % a[5];
        query(root, p, ans, 0);
        insert(root, p, 0);
        ret += ans * (i > 0);
    }
    printf("%I64d\n", ret);
}

int main() {
    int test;
    scanf("%d", &test);
    rep(cas,test){
        scanf("%d", &n);
        rep(i,D) scanf("%d", &a[i]);
        work();
    }
    return 0;
}
\end{verbatim}

	%\subsubsection{knn询问距离最近K个点}
\begin{verbatim}
double cross(Point a,Point b,Point c){return (b.X-a.X)*(c.Y-a.Y)-(c.X-a.X)*(b.Y-a.Y);}

double dot(Point a,Point b,Point c){return (b.X-a.X)*(c.X-a.X)+(b.Y-a.Y)*(c.Y-a.Y);}

bool inpoly(Point a, Point *p, int n){
    int wn = 0;
    rep(i,n){
        Point p1 = p[i], p2 = p[(i + 1) % n];
        double s = cross(a, p1, p2);
        if (!s && dot(a, p1, p2) <= 0) return true;
        double d1 = p1.Y - a.Y, d2 = p2.Y - a.Y;
        if (s > 0 && d1 <= 0 && d2 > 0) ++wn;
        if (s < 0 && d2 <= 0 && d1 > 0) --wn;
    }
    return wn != 0;
}

const int N = 20000, M = 20;

int n, m, r;
Point p[N], poly[M];

const int K = 2;
struct kd {
    LL x[K];
    int id;
}t[N];

double dis2(kd a, kd b){
    double s = 0;
    rep(i,K) s += Sqr(a.x[i] - b.x[i]);
    return s;
}

struct cmpk {
    int k;
    cmpk(int _k): k(_k) {}
    bool operator()(kd a, kd b){ return a.x[k] < b.x[k]; }
};

void build(int l, int r, int d){
    if (r - l <= 1) return;
    int mid = (l + r) >> 1;
    nth_element(t + l, t + mid, t + r, cmpk(d));
    if (++d == K) d = 0;
    build(l, mid, d); build(mid + 1, r, d);
}

typedef priority_queue<pair<double, int> > heap;
void knn(int l, int r, int d, kd p, size_t k, heap &h){
    if (r - l < 1) return;
    int mid = (l + r) >> 1;
    h.push(make_pair(dis2(p, t[mid]), t[mid].id));
    if (h.size() > k) h.pop();
    double dx = p.x[d] - t[mid].x[d];
    if (++d == K) d = 0;
    if (dx < 0) {
        knn(l, mid, d, p, k, h);
        if (h.top().first > Sqr(dx)) knn(mid + 1, r, d, p, k, h);
    } else {
        knn(mid + 1, r, d, p, k, h);
        if (h.top().first > Sqr(dx)) knn(l, mid, d, p, k, h);
    }
}

void solve(){
    scanf("%d", &m);
    rep(i,m) {
        int x,y;
        scanf("%d%d",&x,&y);
        poly[i]=MP(x,y);
    }
    int cnt = 0;
    rep(i,n){
        if (inpoly(p[i], poly, m)) {
            t[cnt].x[0] = p[i].X; t[cnt].x[1] = p[i].Y;
            t[cnt++].id = i + 1;
        }
    }
    build(0, cnt, 0);
    int q;
    scanf("%d", &q);
    while (q--) {
        kd p;
        scanf("%lld%lld", &p.x[0], &p.x[1]);
        heap h;
        knn(0, cnt, 0, p, 2, h);
        int a, b;
        b = h.top().second; h.pop();
        a = h.top().second;
        printf("%d %d\n", a, b);
    }
}

int main(){
    int dat;
    scanf("%d", &dat);
    rep(cas,dat){
        printf("Case #%d:\n", cas+1);
        scanf("%d",&n);
        rep(i,n){
            int x,y;
            scanf("%d%d",&x,&y);
            p[i]=MP(x,y);
        }
        scanf("%d", &r);
        rep(id,r){
            printf("Region %d\n", id+1);
            solve();
        }
    }
}
\end{verbatim}

	%\input{code/geometry/range_tree}

	\subsection{多边形和圆}
	%\subsubsection{k多边形面积交}
\begin{verbatim}
int n;
int v[MAXN]; // the number of vertexes
point p[MAXN][MAXV];
pair<double, int> c[MAXN * MAXV * 2];
double tot[MAXN + 1];

double pos(point p, line ln) {
    return dcmp(ln.second.X - ln.first.X) ?
        (p.X - ln.first.X) / (ln.second.X - ln.first.X) :
        (p.Y - ln.first.Y) / (ln.second.Y - ln.first.Y);
}

double area() {
    memset(tot, 0, sizeof(tot));
    for (int i = 0; i < n; ++i)
        for (int ii = 0; ii < v[i]; ++ii) {
            point A = p[i][ii], B = p[i][(ii + 1) % v[i]];
            line AB = line(A, B);
            int m = 0;
            for (int j = 0; j < n; ++j) if (i != j)
                for (int jj = 0; jj < v[j]; ++jj) {
                    point C = p[j][jj], D = p[j][(jj + 1) % v[j]];
                    line CD = line(C, D);
                    int f1 = dcmp(cross(A, B, C));
                    int f2 = dcmp(cross(A, B, D));
                    if (!f1 && !f2) {
                        if (i < j && dcmp(dot(dir(AB), dir(CD))) > 0) {
                            c[m++] = make_pair(pos(C, AB), 1);
                            c[m++] = make_pair(pos(D, AB), -1);
                        }
                    } else {
                        double s1 = cross(C, D, A);
                        double s2 = cross(C, D, B);
                        double t = s1 / (s1 - s2);
                        if (f1 >= 0 && f2 < 0) c[m++] = make_pair(t, 1);
                        if (f1 < 0 && f2 >= 0) c[m++] = make_pair(t, -1);
                    }
                }
            c[m++] = make_pair(0.0, 0);
            c[m++] = make_pair(1.0, 0);
            sort(c, c + m);
            double s = cross(A, B), z = min(max(c[0].first, 0.0), 1.0);
            for (int j = 1, k = c[0].second; j < m; ++j) {
                double w = min(max(c[j].first, 0.0), 1.0);
                tot[k] += s * (w - z);
                k += c[j].second;
                z = w;
            }
        }
    return tot[0];
}

/*
   tot[0] is the aera of union
   tot[n - 1] is the aera of intersection
   tot[k - 1] - tot[k] is the aera of region covered by k times
   */
\end{verbatim}

	%\input{code/geometry/circle_k_intersection}
	\subsubsection{圆与多边形面积交}
\begin{verbatim}
Point p[3];
double r;

double cross(Point a, Point b){
    return a.X * b.Y - a.Y * b.X;
}

double cross(Point a, Point b, Point c){
    return cross(b - a, c - a);
}

double dot(Point a, Point b){
    return a.X * b.X + a.Y * b.Y;
}

double dot(Point a, Point b, Point c){
    return dot(b - a, c - a);
}

double len(Line l){
    return abs(l.S - l.F);
}

double dis(Point p, Line l){
    return fabs(cross(p, l.F, l.S) / len(l));
}

bool inter(Line a, Line b, Point &p){
    double s1 = cross(a.F, a.S, b.F);
    double s2 = cross(a.F, a.S, b.S);
    if (!sign(s1 - s2)) return false;
    p  = (s1 * b.S - s2 * b.F) / (s1 - s2);
    return true;
}

Vec dir(Line l){
    return l.S - l.F;
}

Vec normal(Vec v){
    return Vec(-v.Y, v.X);
}

Vec unit(Vec v){
    return v / abs(v);
}

bool onseg(Point p, Line l){
    return sign(cross(p, l.F, l.S)) == 0 && sign(dot(p, l.F, l.S)) <= 0;
}

double arg(Vec a, Vec b){
    double d = arg(b) - arg(a);
    if (d > PI) d -= 2 * PI;
    if (d < -PI) d += 2 * PI;
    return d;
}

double area(Point a, Point b){
    double s1 = 0.5 * cross(a, b);
    double s2 = 0.5 * arg(a, b) * r * r;
    return fabs(s1) < fabs(s2) ? s1 : s2;
}

double area(){
    double s = 0;
    rep(i, n){
        Point O(0, 0), A = p[i], B = p[(i + 1) % 3];
        Line AB(A, B);
        double d = dis(O, AB);
        if (sign(d - r) >= 0){
            s += area(A, B);
        }
        else{
            Point P;
            inter(AB, Line(O, O + normal(dir(AB))), P);
            Vec v = sqrt(r * r - d * d) * unit(dir(AB));
            Point P1 = P - v, P2 = P + v;
            if (!onseg(P1, AB) && !onseg(P2, AB)){
                s += area(A, B);
            }
            else{
                s += area(A, P1);
                s += area(P1, P2);
                s += area(P2, B);
            }
        }
    }
    return fabs(s);
}

void init(){
    scanf("%d%d", &n, &r);
    rep(i, n){
        double x, y;
        scanf("%lf%lf", &x, &y);
        p[i] = Point(x, y);
    }
}

int main(){
    init();
    printf("%.12lf\n", area());
    return 0;
}
\end{verbatim}


	\subsection{其他}
	%\input{code/geometry/minium_circle_cover}
    %\input{code/geometry/rectangular}
	%\input{code/geometry/unionrectange}
	%\input{code/geometry/circle_tree}

\section{\LARGE 理论}
	\subsection{数学}
    %\subsubsection{数学结论}
\begin{verbatim}
五边形定理
五边形数 n * (3 * n +- 1) / 2
(1-x)*(1-x^2)*(1-x^3)....=sigma{(-1)^k * x^(n * (3 * n (+-) 1) / 2)}
即f[n] = f[n - 1] + f[n - 2] - f[n - 5] - f[n - 7] + f[n - 12] + f[n - 15] - .....

fibonacci数性质:
f[n] = f[n - 1] + f[n - 2]
f[n + m + 1] = f[n] * f[m] + f[n + 1] * f[m + 1]
gcd(f[n], f[n + 1]) = 1
gcd(f[n], f[n + 2]) = 1
gcd(f[n], f[m]) = f[gcd(n, m)]
f[n+1]*f[n+1]-f[n]*f[n+2] = (-1)^n
sigma{f[i]^2, 1<=i<=n} = f[n]*f[n+1]
sigma{f[i], 0<=i<=n} = f[n+2] - 1
sigma{f[2*i-1],1<=i<=n} = f[2*n]
sigma{f[2*i],1<=i<=n} = f[2*n+1]-1
sigma{(-1)^i*f[i],0<=i<=n} = (-1)^n*(f[n+1]-f[n])+1
f[2*n-1]=f[n]^2-f[n-2]^2
f[2*n+1]=f[n]^2+f[n+1]^2
3*f[n]=f[n+2]+f[n-2]
f[n]=c(n-1,0)+c(n-2,1)+..c(n-1-m,m) (m<=n-1-m)
sigma{f[i]*i,1<=i<=n}=n*f[n+2]-f[n+3]+2

catalan数性质:
凸多边形三角剖分数
简单有序根树的计数
(0,0)走到(n,n)经过的点(a,b)满足a<=b的路径数
乘法结合问题
c[n+1] = (4 * n - 2) / (n + 1) * c[n]
c[n] = (2*n)!/(n!)/((n+1)!)

第一类stirling数性质
有正有负,其绝对值是n个元素的项目分作k个环排列的数量,s[n,k](n个人分成k组,每组再按特定顺序围圈)
s[n][0] = 0, s[1][1] = 1;
s[n+1][k]= = s[n][k - 1] + n * s[n][k]
|s[n][1]| = (n-1)!
s[n][k] = (-1)^(n+k)*|s[n][k]|
s[n][n-1] = -C(n,2)
x*(x-1)*(x-2)..(x-n+1) = sigma{s[n][k] * x ^k}

第二类stirling数性质
n个元素的集定义k个等价类的方法数目(n个人分成k组的方法数)
s[n][n] = s[n][1] = 1
s[n][k] = s[n - 1][k - 1] + k * s[n - 1][k]
s[n][n - 1] = C(n, 2)
s[n][2] = 2^(n-1)-1
s[n][k] = 1/(k!)sigma{(-1)^k-j * C(k, j) * j ^n, 1<=j<=k}

bell数性质
B[n] = sigma{s[n][k], 1<=k<=n}
B[n+1] = simga{C(n,k)*B[k], 0<=k<=n}
B[p+n] = B[n] + B[n + 1] (mod p)
B[p^m+n] = B[n] + B[n+1] (mod p)

多项式性质
f(x)不存在重根<=>gcd(f(x), f‘(x))的次数小于1次
多项式gcd可以用来判断两多项式是否有公共根

多项式取模
f[x] = 0 (mod m) 
m = m1 * m2 * m3 ... mk
Ti 表示 f[x] = 0 (mod mi)的解数,则T = T1 * T2 * T3...Tk

数论
a^n % b = a^(n % phi(b) + phi(b)) % b (n >= phi(b))
lucas定理 c(n, m) = c(n % p, m % p) * c(n / p, m / p) % p
lucas函数 满足 f(n, m) = f(n % p, m % p) * f(n / p, m / p) % p, 可以猜测满足

原根
2,4,p^k,2*p^k存在原根,存在原根则原根数量为phi(phi(n))
验证原根x = phi(n), x = p1^a1*p2^a2..pk^ak
原根满足t ^ (x / pi) != 1 (mod n)

x*x+y*y==n的整数解:
x*x+y*y==n的整数解个数num = 4 * sigma{H(d), d | n}
H(d) =
(1) 奇数 : (-1)^((d-1)/2)
(2) 偶数 : 0

平方和定理:
(1)费马平方和定理:
    奇质数能表示为两个平方数之和的充分必要条件是该素数被4除余1
(2)费马平方和定理的拓展定理:
    正整数能表示为两平方数之和的充要条件是在它的标准分解式中,形如素因子的指数是偶数
(3)Brahmagupta–Fibonacci identity
    如果两个整数都能表示为两个平方数之和,则它们的积也能表示为两个平方数之和。公式及拓展公式为
\end{verbatim}
$$(a^{2}+b^{2})(c^{2}+d^{2})=(ac-bd)^{2}+(ad+bc)^{2}=(ac+bd)^{2}+(ad-bc)^{2}$$
$$(a^{2}+n*b^{2})(c^{2}+n*d^{2})=(ac-n*bd)^{2}+n*(ad+bc)^{2}=(ac+n*bd)^{2}+n(ad-bc)^{2}$$
\begin{verbatim}
    从这个定理可以看出:如果不能表示为三个数的平方和,那么也就不能表示为两个数的平方和。
(4)四平方和定理:
    每个正整数都可以表示成四个整数的平方数之和
(5)表为3个数的平方和条件: 
    正整数能表示为三个数的平方和的充要条件是不能表示成的形式,其中和为非负 整数。

连分数
连分数(a+(n^0.5)) / b
开始时,i满足,(a+i)/b=floor((a+(n^0.5))/b),之后过程一样
如果不成功,则可以变换为(ab+((nb^2)^0.5))/(b^2),之后再来

杨氏矩阵又叫杨氏图表,它是这样一个矩阵,满足条件:

 
杨氏矩阵
(1)如果格子(i,j)没有元素,则它右边和上边的相邻格子也一定没有元素。
(2)如果格子(i,j)有元素a[i][j],则它右边和上边的相邻格子要么没有元素,要么有元素且比a[i][j]大。
1 ~ n所组成杨氏矩阵的个数可以通过下面的递推式得到:
f[1] = 1; f[2] = 2;
f[n] = f[n - 1] + (n - 1) * f[n - 2];

钩子公式:
对于给定形状,不同的杨氏矩阵的个数为:n!除以每个格子的钩子长度加1的积。其中钩子长度定义为该格子
右边的格子数和它上边的格子数之和。
\end{verbatim}


	\subsection{数论}
    %\input{code/theory/number_conclusion}
	%\input{code/theory/mod}
    %\subsubsection{中国剩余定理(非互质)}
\begin{verbatim}
LL exgcd(LL a,LL b,LL &x,LL &y) {
    if (!a){
        x = 0;
        y = 1;
        return b;
    }
    LL g = exgcd(b % a, a, x, y);
    LL t = y;
    y = x;
    x = t - (b / a) * y;
    return g;
}

LL CRT(const vector<LL>& m,const vector<LL>& b) {
    bool flag = false;
    LL x, y, i,d,result,a1,m1,a2,m2,Size=m.size();
    m1 = m[0]; a1 = b[0];
    for(i = 1; i < Size; ++i){
        m2 = m[i]; a2 = b[i];
        d = exgcd(m1, m2, x, y );
        if((a2-a1) % d != 0) flag = true;
        result = (x * ((a2-a1) / d ) % m2 + m2 ) % m2;
        a1 = a1 + m1 * result; //对于求多个方程
        m1 = (m1 * m2) / d;    //lcm(m1,m2)最小公倍数
        a1 = (a1 % m1 + m1) % m1;
    }
    if (flag) return -1;
    else return a1;
}
\end{verbatim}

    %\input{code/theory/extended_gcd}
    %\input{code/theory/euler}
	%\input{code/theory/mu}
    %\input{code/theory/head}
	%\input{code/theory/detmod}

	\subsection{博弈论}
	%\input{code/theory/sg}

    \subsection{特殊数列}
    %\input{code/theory/fibonacci}
    %\input{code/theory/catalan}
    %\input{code/theory/stirling}
	%\input{code/theory/schroder}

\section{\LARGE 其他}
    %\input{code/other/biginteger}
    %\subsubsection{FFT}
\begin{verbatim}
typedef complex<long double> Comp;
class FFT {
    public:
        FFT(int n);
        void forward(Comp a[]) {
            compute(a, r);
        }
        void reverse(Comp a[]){
            compute(a, ir);
            for (int i = 0; i < n; i++) a[i] /= n;
        }
    private:
        int n, p;
        vector<int> rb;
        Comp r[20];
        Comp ir[20];
        void compute(Comp a[], Comp* r);
};

FFT::FFT(int n) : n(n) , rb(n) , p(0) {
    while ((1 << p) < n) ++p;
    for(int i = 0; i < n; i++){
        int x = i, y = 0;
        for (int j = 0; j < p; ++j) {
            y = (y << 1) | (x & 1);
            x >>= 1;
        }
        rb[i] = y;
    }
    for(int i = 0; i < p; i++){
        long double angle = 2 * PI / (1 << (i + 1));
        ir[i] = Comp(cos(angle), sin(angle));
        r[i] = std::conj(ir[i]);
    }
}

void FFT::compute(Comp a[], Comp* r) {
    for (int i = 0; i < n; ++i) if (rb[i] > i) swap(a[i], a[rb[i]]);
    for (int len = 2; len <= n; len <<= 1) {
        Comp root = *r++;
        for (int i = 0; i < n; i += len) {
            Comp w(1, 0);
            for (int j = 0; j < len / 2; ++j) {
                Comp u = a[i + j];
                Comp v = a[i + j + len / 2] * w;
                a[i + j] = u + v;
                a[i + j + len / 2] = u - v;
                w *= root;
            }
        }
    }
}
\end{verbatim}

    %\input{code/other/c++note}
    %\input{code/other/editplus}
    %\input{code/other/vim}

%CJK的时候使用
%\end{CJK}
\end{document}
