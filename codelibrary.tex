\def\ChineseScale{1200}
\documentclass[12pt,a4paper,titlepage]{article}
\usepackage{times}

%\usepackage{CJK}
%CJK中文支持

%=========================xeCJK设置==============================
\usepackage{xeCJK}
%xeCJK中文支持

\def\CJKecglue{\hskip 0.15em}
%设置 CJK 文字与西文、文字与行内数学公式之间的间距, CJK默认值是一个空格。

\setCJKmainfont{Hiragino Sans GB}
\setCJKmonofont{Hiragino Sans GB}
%设置正文等宽族的 CJK 字体, 影响 \ttfamily 和 \texttt 的字体。 为了有利于等宽字体的 代码对齐等情形, xeCJK 在 ⟨font features⟩ 里增加了 Mono 这个选项。

%\setmainfont{Courier}
%================================================================

\usepackage{latexsym,bm}
\usepackage{indentfirst}
%自动首行缩进
\usepackage[top=1in, bottom=1in, left=0.5in, right=0.5in]{geometry}
\usepackage[CJKbookmarks,colorlinks=true]{hyperref}

%CJKbookmarks支持中文标签, colorlinks=true 就是把超链接的边框去掉,但字是红色的
%用来调页边距
\author{mayf3}
\title{Standard Source code Library}

\begin{document}

%CJK的时候使用
%\begin{CJK}{UTF8}{gbsn}

\maketitle
\tableofcontents

\newpage
\section{\LARGE 图论}
	\subsection{最短路算法}
    \subsubsection{Dijkstra最短路}
\begin{verbatim}
#include <cstdio>
#include <cstring>
#include <algorithm>
#include <map>

/*
 * name     :     dijkstra(STL)
 * usage     :    single-source shortest path(only non-negative weight)
 * develop    :    small label first optimization, negative circle
 * space complexity    :    O(M)
 * time complexity    :    O(NlogN)
 * checked    :    no
 */

const int N = 111111; //number of the vertices

int n, m;
int dist[N];
vector<PII> E[N];

int calc(int s, int e) {
    priority_queue<PII, vector<PII>, greater<PII> > Q;
    rep(i, n) dist[i] = -1;
    dist[s] = 0;
    Q.push(MP(0, s));
    int x, y, cost;
    while (!Q.empty()) {
        x = Q.top().Y, cost = Q.top().X;
        Q.pop();
        if (cost > dist[x]) continue;
        rep(i, E[x].size()){
            y = E[x][i].X, cost = E[x][i].Y;
            if (dist[y] == -1 || dist[y] != -1 && dist[x] + cost > dist[y]){
                dist[y] = dist[x] + cost;
                Q.push( make_pair(dist[y], y) );
            }
        }
    }
    return dist[e];
}

int main(){
    while(~scanf("%d%d", &n, &m)){
        rep(i, n) E[i].clear();
        int x, y, c;
        rep(i, m){
            scanf("%d%d%d", &x, &y, &c);
            x--, y--;
            E[x].PB(MP(y, c));
            E[y].PB(MP(x, c));
        }
        printf("%d\n", calc(0, n - 1));
    }
    return 0;
}
\end{verbatim}

    \subsubsection{spfa最短路}
\begin{verbatim}
#include "template.cpp"

/*
 * name     :     spfa
 * usage     :    single-source shortest path, differential restraint system
 * develop    :    small label first optimization, negative circle
 * space complexity    :    O(M)
 * time complexity    :    O(k * M) (where k is usually less than 2)
 * checked    :    no
 */

const int N = 10000;

int n, m;
vector<PII> E[N];
int dist[N];

int spfa(int s, int e){
    static deque<int> Q;
    static bool inque[N];
    Cls(inque);
    memset(dist, -1, sizeof dist);
    dist[s] = 0;
    Q.PB(s);
    inque[s] = true;
    int x, y, c;
    while(!Q.empty()){
        x = Q.front();
        Q.pop_front();
        inque[x] = false;
        rep(i, E[x].size()){
            y = E[x][i].X;
            c = E[x][i].Y;
            if (dist[y] == -1 || dist[y] != -1 && dist[y] > dist[x] + c){
                dist[y] = dist[x] + c;
                if (!inque[y]){
                    Q.PB(y);
                    inque[y] = true;
                }
            }
        }
    }
    return dist[e];
}

int main(){
    while(~scanf("%d%d", &n, &m)){
        rep(i, n) E[i].clear();
        int x, y, c;
        rep(i, m){
            scanf("%d%d%d", &x, &y, &c);
            x--, y--;
            E[x].PB(MP(y, c));
            E[y].PB(MP(x, c));
        }
        printf("%d\n", spfa(0, n - 1));
    }
    return 0;
}
\end{verbatim}

	\subsubsection{k最短路(无环)}
\begin{verbatim}
#include <cstdio>
#include <cstring>
#include <algorithm>
#include <map>

using namespace std;

const int MAXN = 50 + 10; //number of vertices
const int MAXK = 200 + 10;
const int INF = 1000000000; //max dist

struct Tpath {
    int cnt, len, pos;
    int v[MAXN];
};

Tpath path[MAXK];
int g[MAXN][MAXN];
int len[MAXK], pos[MAXK], ans[MAXK];
bool used[MAXN];
int dist[MAXN], prev[MAXN], List[MAXN];
int N, M, K, S, T, cnt;

void Dijkstra() {
    int visited[MAXN];
    for (int i = 0; i <= N; ++i) dist[i] = INF, visited[i] = 0;
    dist[T] = 0;
    for (int k, i = T; i != N; i = k) {
        visited[i] = 1; k = N;
        for (int j = 0; j < N; ++j) {
            if (visited[j] || used[j]) continue;
            if (g[j][i] > -1 && dist[i] + g[j][i] < dist[j]) {
                dist[j] = dist[i] + g[j][i];
                prev[j] = i;
            }
            if (dist[j] < dist[k]) k = j;
        }
    }
}

void setPath(int v, Tpath &p) {
    p.len = 0;
    while (1) {
        p.v[p.cnt++] = v;
        if (v == T) return;
        p.len += g[v][prev[v]]; v = prev[v];
    }
}

void solve() {
    memset(used, 0, sizeof(used));
    Dijkstra();
    memset(ans, -1, sizeof(ans));
    if (dist[S] == INF)    return;
    multimap<int, int> Q; Q.clear();
    path[0].cnt = 0; path[0].pos = 0; setPath(S, path[0]); Q.insert( make_pair(path[0].len, 0) );
    int tot = 1;
    for (int i = 0; i < K; ++i) {
        if (Q.empty()) return;
        multimap<int, int> :: iterator p = Q.begin();
        int x = (*p).second;
        ans[i] = path[x].len;
        if (i == K - 1) break;
        memset(used, 0, sizeof(used));
        Tpath cur; cur.cnt = 0; cur.len = 0;
        for (int sum = 0, j = 0; j + 1 < path[x].cnt; ++j) {
            cur.v[cur.cnt++] = path[x].v[j]; used[path[x].v[j]] = 1;
            if (j) sum += g[path[x].v[j - 1]][path[x].v[j]];
            if (j >= path[x].pos) {
                Dijkstra();
                int u = path[x].v[j];
                for (int v = 0; v < N; ++v)
                    if (g[u][v] > -1 && !used[v] && dist[v] < INF && v != path[x].v[j + 1]) {
                        Tpath tp = cur; tp.pos = j + 1; setPath(v, tp); tp.len += sum + g[u][v];
                        if (tot < K) path[tot] = tp, Q.insert( make_pair(tp.len, tot++) );
                        else {
                            multimap<int, int> :: iterator p = Q.end(); --p;
                            if (tp.len >= (*p).first) continue;
                            path[(*p).second] = tp; Q.insert( make_pair(tp.len, (*p).second) );
                            Q.erase(p);
                        }
                    }
            }
        }
        Q.erase(p);
    }
}

void DFS(int step, int u, int len) {
    if (!cnt) return;
    if (u == T) {
        if (len == ans[K - 1]) {
            if (!(--cnt)) {
                for (int j = 0; j < step; ++j) {
                    if (j) printf("-");
                    printf("%d", List[j] + 1);
                }
                printf("\n");
            }
        }
        return;
    }
    Dijkstra();
    int tmp[MAXN];
    for (int i = 0; i < N; ++i) tmp[i] = dist[i];
    for (int i = 0; i < N; ++i)
        if (g[u][i] > -1 && !used[i] && tmp[i] < INF && len + g[u][i] + tmp[i] <= ans[K - 1]) {
            used[i] = 1; List[step] = i;
            DFS(step + 1, i, len + g[u][i]);
            if (!cnt) return;
            used[i] = 0;
        }
}

int main() {
    scanf("%d%d%d%d%d", &N, &M, &K, &S, &T);
    --S; --T;
    if (S == T) ++K;
    memset(g, -1, sizeof(g));
    for (int i = 0; i < M; ++i) {
        int u, v, w;
        scanf("%d%d%d", &u, &v, &w);
        --u; --v;
        g[u][v] = w;
    }
    solve();
    if (ans[K - 1] == -1) printf("None\n");
    else {
        cnt = 0;
        for (int i = 0; i < K; ++i)
            if (ans[i] == ans[K - 1]) ++cnt;
        memset(used, 0, sizeof(used));
        used[S] = 1; List[0] = S;
        DFS(1, S, 0);
    }
    return 0;
}

\end{verbatim}

	\subsubsection{k最短路}
\begin{verbatim}
#include <cstdio>
#include <cstring>
#include <algorithm>
#include <queue>

using namespace std;

const int MAXN = 1000 + 10; //number of vertices
const int MAXM = 100000 + 10; //number of edges
const int MAXK = 1000 + 10;
const int MAXH = 200000; //M + N log N
const int INF = 1000000000; //max dist

struct Theap {
    int idx, dep;
    int chd[3];
};

struct Tedge {
    int u, v, w, delta;
    bool inT;
};

Theap heap[MAXH];
Tedge edge[MAXM];
int first[MAXN], rfirst[MAXN], outdeg[MAXN], dist[MAXN], nextT[MAXN], list[MAXN], H1[MAXN], H2[MAXN];
int next[MAXM], rnext[MAXM];
int ans[MAXK];
int N, M, K, S, T, nlist, H, curedge;

void Dijkstra() {
    priority_queue < pair<int, int>, vector< pair<int, int> >, greater< pair<int, int> > > Q;
    for (int i = 0; i < N; ++i)    dist[i] = INF;
    dist[T] = 0; Q.push( make_pair(0, T) );
    while (!Q.empty()) {
        int u = Q.top().second, d = Q.top().first;
        Q.pop();
        if (d > dist[u]) continue;
        for (int i = rfirst[u]; i != -1; i = rnext[i]) {
            int v = edge[i].u, w = edge[i].w;
            if (dist[u] + w < dist[v]) {
                dist[v] = dist[u] + w;
                Q.push( make_pair(dist[v], v) );
            }
        }
    }
}

void DFS(int u) {
    list[nlist++] = u;
    for (int i = rfirst[u]; i != -1; i = rnext[i]) {
        int v = edge[i].u, w = edge[i].w;
        if (!edge[i].delta && nextT[v] == -1) {
            nextT[v] = u; edge[i].inT = 1;
            DFS(v);
        }
    }
}

int buildH1(int Size, int dep) {
    if (!Size) return 0;
    if (edge[curedge].inT) curedge = next[curedge];
    int cur = H++;
    heap[cur].idx = curedge; curedge = next[curedge];
    heap[cur].chd[2] = 0;
    if (!dep) heap[cur].chd[0] = buildH1(Size - 1, dep + 1), heap[cur].chd[1] = 0;
    else {
        int half = (Size - 1) / 2;
        heap[cur].chd[0] = buildH1(half, dep + 1); heap[cur].chd[1] = buildH1(Size - 1 - half, dep + 1);
    }
    int i = cur;
    while (1) {
        int k = i;
        for (int j = 0; j < 2; ++j)
            if (heap[i].chd[j] && edge[heap[heap[i].chd[j]].idx].delta < edge[heap[k].idx].delta) k = heap[i].chd[j];
        if (k == i) break;
        swap(heap[k].idx, heap[i].idx); i = k;
    }
    return cur;
}

int buildH2(int a, int b) {
    if (!a) {
        heap[b].chd[0] = heap[b].chd[1] = 0; heap[b].dep = 1;
        return b;
    }
    int Next = heap[heap[a].chd[0]].dep >= heap[heap[a].chd[1]].dep;
    int cur = H++;
    heap[cur] = heap[a];
    if (edge[heap[b].idx].delta < edge[heap[a].idx].delta) {
        heap[b].chd[0] = heap[a].chd[0]; heap[b].chd[1] = heap[a].chd[1];
        heap[b].chd[Next] = buildH2(heap[b].chd[Next], cur);
        heap[b].dep = min(heap[heap[b].chd[0]].dep, heap[heap[b].chd[1]].dep) + 1;
        return b;
    }
    else {
        heap[cur].chd[Next] = buildH2(heap[cur].chd[Next], b);
        heap[cur].dep = min(heap[heap[cur].chd[0]].dep, heap[heap[cur].chd[1]].dep) + 1;
        return cur;
    }
}

void solve() {
    Dijkstra();
    memset(ans, -1, sizeof(ans));
    if (dist[S] == INF) return;
    for (int i = 0; i < M; ++i) edge[i].delta = edge[i].w - dist[edge[i].u] + dist[edge[i].v];
    memset(nextT, -1, sizeof(nextT));
    nextT[T] = -2; nlist = 0;
    DFS(T);
    H = 1; heap[0].dep = 0;
    memset(H1, 0, sizeof(H1));
    for (int i = 0; i < N; ++i)
        if (dist[i] < INF) {
            int Size = outdeg[i];
            if (i != T) --Size;
            curedge = first[i];
            H1[i] = buildH1(Size, 0);
            if (H1[i]) {
                heap[H1[i]].chd[2] = heap[H1[i]].chd[0];
                heap[H1[i]].chd[0] = 0;
                heap[H1[i]].dep = 1;
            }
        }
    memset(H2, 0, sizeof(H2));
    H2[T] = H1[T];
    for (int i = 1; i < nlist; ++i) {
        int j = list[i];
        if (!H1[j]) H2[j] = H2[nextT[j]];
        else H2[j] = buildH2(H2[nextT[j]], H1[j]);
    }
    ans[0] = dist[S];
    priority_queue < pair<int, int>, vector< pair<int, int> >, greater< pair<int, int> > > Q;
    if (H2[S]) Q.push( make_pair(edge[heap[H2[S]].idx].delta, H2[S]) );
    for (int i = 1; i < K; ++i) {
        if (Q.empty()) break;
        int u = Q.top().second, d = Q.top().first;
        ans[i] = dist[S] + d;
        Q.pop();
        for (int j = 0; j < 3; ++j) {
            int v = heap[u].chd[j];
            if (v) Q.push( make_pair(d + edge[heap[v].idx].delta - edge[heap[u].idx].delta, v) );
        }
        int v = H2[edge[heap[u].idx].v];
        if (v) Q.push( make_pair(d + edge[heap[v].idx].delta, v) );
    }
}

int main() {
    memset(first, -1, sizeof(first));
    memset(rfirst, -1, sizeof(rfirst));
    memset(outdeg, 0, sizeof(outdeg));
    scanf("%d%d", &N, &M);
    for (int i = 0; i < M; ++i) {
        int u, v, w;
        scanf("%d%d%d", &u, &v, &w);
        --u; --v;
        edge[i].u = u; edge[i].v = v; edge[i].w = w; edge[i].inT = 0;
        next[i] = first[u]; first[u] = i; rnext[i] = rfirst[v]; rfirst[v] = i;
        ++outdeg[u];
    }
    scanf("%d%d%d", &S, &T, &K);
    --S; --T;
    if (S == T) ++K;
    solve();
    printf("%d\n", ans[K - 1]);
    return 0;
}
\end{verbatim}


    \subsection{生成树}
	\subsubsection{生成树计数}
	一个完全图$K_n$有$n^{n-2}$棵生成树,即$n$个节点的带标号无根树有$n^{n-2}$个。\par
	证明用到$\mathbf{prufer code}$:一棵$n$无根树的$\mathbf{prufer code}$是这样转化的:每次选一个编号最小的叶节点,删除它,并把它所连的父亲节点的编号写下,直到这棵树剩下2个节点为止。那么生成的这$n-2$个数组成的序列就是这棵树的$\mathbf{prufer code}$。$\mathbf{prufer code}$和树是一一对应的,而一个长度为$n-2$,每个数字的范围为$[1,n]$的序列一共有$n^{n-2}$种可能,所以$n$个节点的带标号无根树一共有$n^{n-2}$个。\par
	以上是Cayley公式,它的一个应用:$n$个节点,每个节点的度分别为$d_1,d_2,\ldots,d_n$,那么生成树的个数为$\frac{(n-2)!}{(d_1-1)!(d_2-1)!\cdots(d_n-1)!)}$。因为顶点$i$在序列中出现了$d_i-1$次。\par
	~\\
	~\\ \par
		对于完全二分图,两边的顶点分别为$n,m$,那么生成树的个数为$n^{m-1}*m^{n-1}$。\par
		~\\
		~\\ \par
\subsubsection{无向图的生成数计数-MatrixTree定理}
		给出一个无向图$G=(V,E)$,求生成树个数。做法是构造一个$n*n$的Kichhoff矩阵。矩阵的对角线$(i,i)$的位置填的是第$i$个顶点的度,对于$G$的边$(v_i,v_j)$在矩阵$(i,j)$和$(j,i)$的位置填-1(\textbf{若$(i,j)$有$k$条重边,那么矩阵$(i,j)$和$(j,i)$的位置填$-k$}),然后生成树的个数就是$n*n$的矩阵的$n-1$阶的行列式。具体做法就是删除任意的第$r$行$r$列,然后求矩阵的行列式。
	~\\
\subsubsection{无向带权图最小生成树计数}
		把所有的边按照边权从小到大排序,然后做Kruskal。\par
		假设已经处理了边权$w_i<w$的边,形成一个森林$T$,现在考虑所有边权为$w$的边。\par
		1、若一条边权为$w$的边$(u,v)$所连接的两个顶点在森林$T$中属于统一个块,那么$(u,v)$这条边是不可能存在于最小生成树的方案中,否则$(u,v)$可以存在于最小生成树的方案中。\par
		2、把所有边权为$w$,且根据森林$T$判断出可以存在于最小生成树方案中的边找出来,假设这些边集为$E$。我们可以把森林$T$中的每一个块缩成一个点,那么用$E$中的边去连接$T$,就形成了一些连通块。对于每一个连通块的方案数就是对这个连通块做一个\textbf{生成树计数}就可以了,然后把这些连通块各自的方案数相乘就是选择边权为$w$的边的方案数。\par
		3、算完边权为$w$的方案之后,就把这些边加入到$T$中,形成新的森林。\par
\subsubsection{带限制的最小生成树问题}
	\begin{itemize}
		\item Problem \par
			无向带权连通图,每条边是黑色或白色。让你求一棵最小权的恰好有K条白色边的生成树。
		\item Solution \par
			1、对于一个图,如果存在一棵生成树,它的白边数量为$x$,那么就称$x$是合法白边数。所有的合法白边数组成一个区间$[l,r]$。\par
			2、对于一个图,如果存在一棵最小生成树,它的白边数量为$x$,那么就称$x$是最小合法白边数。所有的最小合法白边数组成一个区间$[l,r]$。\par
			3、将所有白边追加权值$x$所得到的最小生成树,如果该树有$a$条白边。那么这可树就是$a$条白边最小生成树的一个最优解。

			~\\
			所以可以二分得到一个最大的$x$使得所求的最小生成树的白边的最小值和最大值所组成的一个区间$[l,r]$,若$K\in [l,r]$,则该最小生成树就是最优解。其实只要求出最大的一个$x$使得最小生成树中最大白边数量不小于$K$即可。记录答案的时候,必须把枚举的$x$加上,然后在最后减去$K*x$,如果直接在计算的时候加原来白边的长度的话,有可能超过$K$条边。\par
			对于黑、白不同的边,他们内部的顺序是一样的,所以一开始将黑白边分别排序,这样在二分判断的时候只需要$O(M)$的时间复杂度去合并排序的边了,其中$M$是边数。总的时间复杂度为$O(MlogW+NlogN)$其中$N$是顶点数,$W$是边权。

		\item Expansion \par
			1、限制某个节点$node$的度数恰好为$K$的最小生成树。解法就是把和$node$关联的边标记为白边,其余的边为黑边,然后就转化为上面的经典问题了。
	\end{itemize}


    %\input{code/graph/krusal}
    \subsubsection{prim最小生成树}
\begin{verbatim}
const int MAXN = 2000 + 1; //number of vertices + 1
const int MAXM = 10000; //number of edges
const int INF = 2000000000; //max weight

struct Tedge {
    int v, w, next;
};

Tedge edge[MAXM * 2];
int first[MAXN], dist[MAXN], heap[MAXN], pos[MAXN];
bool used[MAXN];
int N, M, cnt;

void init() {
    memset(first, -1, sizeof(first));
    M = 0;
}

inline void add_edge(int u, int v, int w) {
    edge[M].v = v; edge[M].w = w; edge[M].next = first[u]; first[u] = M++;
}

inline void moveup(int i) {
    int key = heap[i];
    while (i > 1 && dist[heap[i >> 1]] > dist[key]) heap[i] = heap[i >> 1], pos[heap[i]] = i, i >>= 1;
    heap[i] = key; pos[key] = i;
}

inline void movedown(int i) {
    int key = heap[i];
    while ((i << 1) <= cnt) {
        int j = i << 1;
        if (j < cnt && dist[heap[j + 1]] < dist[heap[j]]) ++j;
        if (dist[key] <= dist[heap[j]]) break;
        heap[i] = heap[j]; pos[heap[i]] = i; i = j;
    }
    heap[i] = key; pos[key] = i;
}

void Prim() {
    memset(used, 0, sizeof(used));
    for (int i = 0; i < N; ++i) pos[i] = -1, dist[i] = INF;
    cnt = 1; heap[1] = 0; dist[0] = 0;
    while (cnt) {
        int u = heap[1];
        used[u] = 1;
        heap[1] = heap[cnt--];
        movedown(1);
        for (int i = first[u]; i != -1; i = edge[i].next) {
            int v = edge[i].v, w = edge[i].w;
            if (!used[v] && w < dist[v]) {
                dist[v] = w;
                if (pos[v] == -1) pos[v] = ++cnt, heap[cnt] = v;
                moveup(pos[v]);
            }
        }
    }
}
const int MAXN = 2000 + 1; //number of vertices + 1
const int MAXM = 10000; //number of edges
const int INF = 2000000000; //max weight

struct Tedge {
    int v, w, next;
};

Tedge edge[MAXM * 2];
int first[MAXN], dist[MAXN], heap[MAXN], pos[MAXN];
bool used[MAXN];
int N, M, cnt;

void init() {
    memset(first, -1, sizeof(first));
    M = 0;
}

inline void add_edge(int u, int v, int w) {
    edge[M].v = v; edge[M].w = w; edge[M].next = first[u]; first[u] = M++;
}

inline void moveup(int i) {
    int key = heap[i];
    while (i > 1 && dist[heap[i >> 1]] > dist[key]) heap[i] = heap[i >> 1], pos[heap[i]] = i, i >>= 1;
    heap[i] = key; pos[key] = i;
}

inline void movedown(int i) {
    int key = heap[i];
    while ((i << 1) <= cnt) {
        int j = i << 1;
        if (j < cnt && dist[heap[j + 1]] < dist[heap[j]]) ++j;
        if (dist[key] <= dist[heap[j]]) break;
        heap[i] = heap[j]; pos[heap[i]] = i; i = j;
    }
    heap[i] = key; pos[key] = i;
}

void Prim() {
    memset(used, 0, sizeof(used));
    for (int i = 0; i < N; ++i) pos[i] = -1, dist[i] = INF;
    cnt = 1; heap[1] = 0; dist[0] = 0;
    while (cnt) {
        int u = heap[1];
        used[u] = 1;
        heap[1] = heap[cnt--];
        movedown(1);
        for (int i = first[u]; i != -1; i = edge[i].next) {
            int v = edge[i].v, w = edge[i].w;
            if (!used[v] && w < dist[v]) {
                dist[v] = w;
                if (pos[v] == -1) pos[v] = ++cnt, heap[cnt] = v;
                moveup(pos[v]);
            }
        }
    }
}
\end{verbatim}

    \subsubsection{prim(stl)最小生成树}
\begin{verbatim}
const int MAXN = 2000; //number of vertices
const int INF = 2000000000; //max weight

vector < pair<int, int> > edge[MAXN];
int dist[MAXN];
bool used[MAXN];
int N;

void Prim() {
	priority_queue < pair<int, int>, vector< pair<int, int> >, greater< pair<int, int> > > Q;
	memset(used, 0, sizeof(used));
	for (int i = 0; i < N; ++i) dist[i] = INF;
	dist[0] = 0; Q.push( make_pair(0, 0) );
	while (!Q.empty()) {
		int u = Q.top().second, d = Q.top().first;
		Q.pop();
		used[u] = 1;
		if (d > dist[u]) continue;
		for (int i = 0; i < edge[u].size(); ++i) {
			int v = edge[u][i].first, w = edge[u][i].second;
			if (w < dist[v]) {
				dist[v] = w;
				Q.push( make_pair(dist[v], v) );
			}
		}
	}
}
\end{verbatim}

	\subsubsection{K度限制最小生成树}
\begin{verbatim}

/*
    Find a minimum spanning tree whose vertex 1 has a degree limit D
*/
#include <cstdio>
#include <cstring>
#include <algorithm>

using namespace std;

const int MAXN = 1000 + 1; //number of vertices + 1
const int MAXM = 100000; //number of edges
const int INF = 2000000000;

struct Tedge {
    int v, w, next;
};

Tedge edge[MAXM * 2], mst_edge[MAXM * 2];
int first[MAXN], mst_first[MAXN], dist[MAXN], heap[MAXN], pos[MAXN], maxw[MAXN], path[MAXN], prev[MAXN];
bool used[MAXN];
int N, M, D, cnt, num, ans;

inline void add_edge(Tedge& e, int& first, int i, int v, int w) {
    e.v = v; e.w = w; e.next = first; first = i;
}

void init() {
    memset(first, -1, sizeof(first));
    scanf("%d%d%d", &N, &M, &D);
    for (int i = 0; i < M; ++i) {
        int u, v, w;
        scanf("%d%d%d", &u, &v, &w);
        --u; --v;
        add_edge(edge[i * 2], first[u], i * 2, v, w);
        add_edge(edge[i * 2 + 1], first[v], i * 2 + 1, u, w);
    }
}

inline void moveup(int i) {
    int key = heap[i];
    while (i > 1 && dist[heap[i >> 1]] > dist[key])    heap[i] = heap[i >> 1], pos[heap[i]] = i, i >>= 1;
    heap[i] = key; pos[key] = i;
}

inline void movedown(int i) {
    int key = heap[i];
    while ((i << 1) <= num) {
        int j = i << 1;
        if (j < num && dist[heap[j + 1]] < dist[heap[j]]) ++j;
        if (dist[key] <= dist[heap[j]]) break;
        heap[i] = heap[j]; pos[heap[i]] = i; i = j;
    }
    heap[i] = key; pos[key] = i;
}

void Prim(int u) {
    int minw = INF, s;
    num = 0;
    while (1) {
        used[u] = 1;
        for (int i = first[u]; i != -1; i = edge[i].next) {
            int v = edge[i].v, w = edge[i].w;
            if (!used[v] && (dist[v] == -1 || w < dist[v])) {
                dist[v] = w;
                prev[v] = u;
                if (pos[v] == -1) pos[v] = ++num, heap[num] = v;
                moveup(pos[v]);
            }
            else if (used[v] && v == 0 && w < minw) minw = w, s = i;
        }
        if (!num) break;
        u = heap[1]; heap[1] = heap[num--]; movedown(1);
        ans += dist[u];
        add_edge(mst_edge[cnt], mst_first[u], cnt, prev[u], dist[u]); ++cnt;
        add_edge(mst_edge[cnt], mst_first[prev[u]], cnt, u, dist[u]); ++cnt;
    }
    if (minw == INF) return;
    edge[s].w = -1; edge[s ^ 1].w = -1;
    s = edge[s ^ 1].v; ans += minw; --D;
    add_edge(mst_edge[cnt], mst_first[0], cnt, s, minw); ++cnt;
    add_edge(mst_edge[cnt], mst_first[s], cnt, 0, minw); ++cnt;
}

void DFS(int u) {
    used[u] = 1;
    for (int i = mst_first[u]; i != -1; i = mst_edge[i].next) {
        int v = mst_edge[i].v, w = mst_edge[i].w;
        if (w > -1 && !used[v]) {
            if (w > maxw[v]) maxw[v] = w, path[v] = i;
            if (maxw[u] > maxw[v]) maxw[v] = maxw[u], path[v] = path[u];
            DFS(v);
        }
    }
}

void work() {
    ans = cnt = 0;
    memset(mst_first, -1, sizeof(mst_first));
    memset(dist, -1, sizeof(dist));
    memset(pos, -1, sizeof(pos));
    memset(used, 0, sizeof(used));
    used[0] = 1;
    for (int i = first[0]; i != -1; i = edge[i].next)
        if (!used[edge[i].v]) Prim(edge[i].v);
    if (D < 0) {
        printf("NONE\n");
        return;
    }
    for (int i = 1; i < N; ++i)
        if (!used[i]) {
            printf("NONE\n");
            return;
        }
    memset(maxw, -1, sizeof(maxw));
    memset(used, 0, sizeof(used));
    used[0] = 1;
    for (int i = mst_first[0]; i != -1; i = mst_edge[i].next) DFS(mst_edge[i].v);
    for (int i = 0; i < D; ++i) {
        int minw = INF, s, x, y;
        for (int j = first[0]; j != -1; j = edge[j].next) {
            int v = edge[j].v, w = edge[j].w;
            if (w > -1 && maxw[v] > -1 && w - maxw[v] < minw) {
                minw = w - maxw[v]; s = v;
                x = path[v]; y = j;
            }
        }
        if (minw >= 0) break;
        ans += minw;
        mst_edge[x].w = mst_edge[x ^ 1].w = -1;
        add_edge(mst_edge[cnt], mst_first[0], cnt, s, edge[y].w); ++cnt;
        add_edge(mst_edge[cnt], mst_first[s], cnt, 0, edge[y].w); ++cnt;
        edge[y].w = edge[y ^ 1].w = -1;
        memset(used, 0, sizeof(used));
        used[0] = 1;
        for (int u = 0; u < N; ++u)
            if (path[u] == x) maxw[u] = -1;
        DFS(s);
    }

    printf("%d\n", ans);
}

int main() {
    int c;
    for (scanf("%d", &c); c > 0; --c) {
        init();
        work();
    }
    return 0;
}
\end{verbatim}

	\subsubsection{最小树形图}
\begin{verbatim}
#include <cstdio>
#include <cstring>
#include <algorithm>

using namespace std;

const int maxn = 200 + 1; //number of vertices
const int maxnum = 200000001; //max weight

int map[maxn][maxn], dist[maxn][maxn], list[maxn][maxn];
int Q[maxn], c[maxn], d1[maxn], d2[maxn];
bool used[maxn];
int n, m, ans, x, y, p;

void init() {
    scanf("%d%d", &n, &m);
    for (int i = 0; i < n; ++i) {
        for (int j = 0; j < n; ++j)
            map[i][j] = maxnum;
        map[i][i] = 0;
    }
    for (int i = 0; i < m; ++i) {
        int u, v, w;
        scanf("%d%d%d", &u, &v, &w);
        --u; --v;
        if (w > map[u][v]) continue;
        map[u][v] = map[v][u] = w;
    }
}

void APSP() {
    for (int s = 0; s < n; ++s) {
        memset(used, 0, sizeof(used));
        for (int i = 0; i <= n; ++i) dist[s][i] = maxnum, list[s][i] = n;
        dist[s][s] = 0;
        for (int k, u = s, cnt = 0; u < n; u = k, ++cnt) {
            list[s][cnt] = u; used[u] = 1; k = n;
            for (int v = 0; v < n; ++v) {
                if (used[v]) continue;
                dist[s][v] = min(dist[s][v], dist[s][u] + map[u][v]);
                if (dist[s][v] < dist[s][k]) k = v;
            }
        }
    }
}

void work() {
    ans = maxnum;
    for (int u = 0; u < n; ++u)
        if (dist[u][list[u][n - 1]] + dist[u][list[u][n - 2]] < ans) {
            ans = dist[u][list[u][n - 1]] + dist[u][list[u][n - 2]];
            x = y = u;
        }

    for (int u = 0; u < n; ++u)
        for (int v = u + 1; v < n; ++v)
            if (map[u][v] < maxnum) {
                for (int i = 0; i < n; ++i) d1[i] = dist[u][list[u][i]], d2[i] = dist[v][list[u][i]];
                for (int j = n - 1, i = n - 2; i >= 0; --i) {
                    if (d1[i] == d1[j]) {
                        if (d2[i] > d2[j]) j = i;
                        continue;
                    }
                    if (d2[i] > d2[j]) {
                        if (map[u][v] + d1[i] + d2[j] < ans) {
                            ans = map[u][v] + d1[i] + d2[j];
                            x = u; y = v;
                            p = map[u][v] + d2[j] - d1[i];
                        }
                        j = i;
                    }
                }
            }
}

void print() {
    printf("%d\n", ans);
    if (x == y) {
        for (int i = 0; i < n; ++i) c[i] = dist[x][i];
        p = 1;
    }
    else {
        printf("%d %d\n", x + 1, y + 1);
        for (int i = 0; i < n; ++i)
            if (dist[x][i] * 2 + p < (dist[y][i] + map[x][y]) * 2 - p) c[i] = dist[x][i] * 2 + p;
            else c[i] = (dist[y][i] + map[x][y]) * 2 - p;
        p = 2;
    }

    memset(used, 0, sizeof(used));
    int t = 1;
    Q[0] = x; used[x] = 1;
    if (x != y) Q[t++] = y, used[y] = 1;
    for (int h = 0; h < t; ++h) {
        int u = Q[h];
        for (int v = 0; v < n; ++v)
            if (!used[v] && c[v] == c[u] + map[u][v] * p) {
                used[v] = 1;
                if (u < v) printf("%d %d\n", u + 1, v + 1);
                else printf("%d %d\n", v + 1, u + 1);
                Q[t++] = v;
            }
    }
}

int main() {
    init();
    APSP();
    work();
    print();
    return 0;
}
\end{verbatim}


	\subsection{网络流}
    \subsubsection{网络流模型}
1、\textbf{最大权闭合图}:定义一个有向图$G=(V,E)$的闭合图是该有向图的一个点集,且该点集的所有出边都还指向该点集。即闭合图内的任一点的任意后继也一定在闭合图中。更形式化地说,闭合图是这样一个点集$V' \in V$,满足对于$\forall u \in V'$引出的$\forall \langle u,v \rangle \in E$,必有$v \in V'$成立。还有一种等价定义为:满足对于$\forall \langle u,v \rangle \in E$,若有$u \in V'$成立,必有$v \in V'$成立。闭合图允许超过一个连通块。\par
给每个点$v$分配一个点权$w_v$(任意实数,可正可负)。最大权闭合图,是一个点权之和最大的闭合图,即最大化$\sum_{v \in V'}w_v$。\par
在许多实际应用中,给出的有向图常常是一个有向无环图(DAG),闭合图的性质恰好反映了事件间的\textbf{必要条件}的关系:一个时间的发生,它所需要的所有前提也要发生。\par
\textbf{最大权闭合图转化成最小割模型}:在原图点集的基础上增加源$s$和汇$t$;将原图每条有向边$\langle u,v \rangle \in E$替换为容量为$c(u,v)=+\infty$的有向边$\langle u,v \rangle \in E_N$;增加连接源$s$到原图每个\textbf{正权点}$v(w_v>0)$的有向边$\langle s,v \rangle \in E_N$,容量为$c(s,v)=w_v$;增加连接原图的每个\textbf{负权点}$v(w_v<0)$到汇$t$的有向边$\langle v,t \rangle \in E_N$,容量为$c(v,t)=-w_v$。其中,正无限$\infty$定义为任意一个大于$\sum_{v\in V}|w_v|$的整数。

~\\
~\\ \par
2、\textbf{最大密度子图}:定义一个无向图$G=(V,E)$的密度$D$为该图的边数$|E|$与该图的点数$|V|$的比值$D=\frac{|E|}{|V|}$。给出一个无向图$G=(V,E)$,其具有最大密度的子图$G'=(V',E')$称为\textbf{最大密度子图},即最大化$D'=\frac{|E'|}{|V'|}$。\par
\textbf{性质}:无向图中,任意两个具有不同密度的子图$G_1,G_2$,它们的密度差不小于$\frac{1}{n^2}$。\par
\textbf{构图}:在原图点集$V$的基础上增加源$s$和汇$t$;将每条原无向边$(u,v)$替换为两条容量为1的有向边$\langle v,u \rangle$和$\langle u,v \rangle$;增加连接源$s$到原图每个点$v$的有向边$\langle s,v \rangle$,容量为$U$;增加连接原图每个点$v$到汇$t$的有向边$\langle v,t \rangle$,容量为$(U+2g-d_v)$。其中$U$是图总的边数,$g$是二分的答案,$d_v$表示顶点$v$的度。



	\subsubsection{最大流}
\begin{verbatim}
#include "template.cpp" 

/*
 * name     :     maxflow
 * usage     :    maxflow
 * develop    :    mincost
 * space complexity    :    O(N^2)
 * time complexity    :    O(N^2 * M)
 * checked    :    no
 */

const int N = 222;
const int E = N * N;

struct Cla{
    struct Edge{
        int t, u;
        Edge *nx, *op;
    }*e[N], mem[E], *C;
    int n, S, T;
    int vh[N], h[N];

    void init(int num){
        n = num;
        C = mem;
        rep(i, n) e[i] = NULL;
    }

    void add_edge(int u, int v, int c){
        C->t = v, C->u = c, C->nx = e[u], e[u] = C++;
        C->t = u, C->u = 0, C->nx = e[v], e[v] = C++;
        e[u]->op = e[v];
        e[v]->op = e[u];
    }

    int aug(int p, int m){
        if (p == T) return m;
        int d = m;
        for(Edge *i = e[p]; i; i = i->nx){
            if (i->u && h[p] == h[i->t] + 1){
                int f = aug(i->t, min(i->u, d));
                i->u -= f, i->op->u += f, d-= f;
                if (h[S] == n || !d) return m - d;
            }
        }
        int w = d < m ? min(n, h[p] + 2) : n;
        for(Edge *i = e[p]; i; i = i->nx){
            if (i->u) w = min(w, h[i->t] + 1);
        }
        ++vh[w];
        --vh[h[p]] ? h[p] = w : h[S] = n;
        return m - d;
    }

    int maxflow(int s, int t){
        S = s, T = t;
        rep(i, n) vh[i] = 0, h[i] = 0;
        vh[0] = n;
        int ret = 0;
        while(h[S] != n) ret += aug(S, INT_MAX);
        return ret;
    }
}graph;
\end{verbatim}

    %\input{code/graph/sap}
	%\input{code/graph/dinic}
	\subsubsection{最小费用最大流}
\begin{verbatim}
#include <cstdio>
#include <cstring>
#include <algorithm>
#include <iostream>
#include <climits>
#include <numeric>
#include <vector>
#include <queue>
using namespace std;

template<class Flow = int, class Cost = int>
struct MinCostFlow {
    struct Edge {
        int t;
        Flow f;
        Cost c;
        Edge*next, *rev;
        Edge(int _t, Flow _f, Cost _c, Edge*_next) :
                t(_t), f(_f), c(_c), next(_next) {
        }
    };

    vector<Edge*> E;

    int addV() {
        E.push_back((Edge*) 0);
        return E.size() - 1;
    }

    Edge* makeEdge(int s, int t, Flow f, Cost c) {
        return E[s] = new Edge(t, f, c, E[s]);
    }

    void addEdge(int s, int t, Flow f, Cost c) {
        Edge*e1 = makeEdge(s, t, f, c), *e2 = makeEdge(t, s, 0, -c);
        e1->rev = e2, e2->rev = e1;
    }

    pair<Flow, Cost> minCostFlow(int vs, int vt) { //flow,cost
        int n = E.size();
        Flow flow = 0;
        Cost cost = 0;
//        const Cost MAX_COST = numeric_limits<Cost>::max();
//        const Flow MAX_FLOW = numeric_limits<Flow>::max();
        const Cost MAX_COST = ~0U >> 1;
        const Flow MAX_FLOW = ~0U >> 1;
        for (;;) {
            vector<Cost> dist(n, MAX_COST);
            vector<Flow> am(n, 0);
            vector<Edge*> prev(n);
            vector<bool> inQ(n, false);
            queue<int> que;

            dist[vs] = 0;
            am[vs] = MAX_FLOW;
            que.push(vs);
            inQ[vs] = true;

            while (!que.empty()) {
                int u = que.front();
                Cost c = dist[u];
                que.pop();
                inQ[u] = false;
                for (Edge*e = E[u]; e; e = e->next)
                    if (e->f > 0) {
                        Cost nc = c + e->c;
                        if (nc < dist[e->t]) {
                            dist[e->t] = nc;
                            prev[e->t] = e;
                            am[e->t] = min(am[u], e->f);
                            if (!inQ[e->t]) {
                                que.push(e->t);
                                inQ[e->t] = true;
                            }
                        }
                    }
            }

            if (dist[vt] == MAX_COST)
                break;

            Flow by = am[vt];
            int u = vt;
            flow += by;
            cost += by * dist[vt];
            while (u != vs) {
                Edge*e = prev[u];
                e->f -= by;
                e->rev->f += by;
                u = e->rev->t;
            }
        }

        return make_pair(flow, cost);
    }
};

int main() {
    return 0;
}
\end{verbatim}

    \subsubsection{有上下界的最大流}

\begin{verbatim}
#include <algorithm>
#include <iostream>
#include <cstdio>
#include <cstring>
using namespace std;

const int INF=0x7FFFFFFF;

const int maxn=1001;
const int kMaxM=100001;

int n,m,s,t,ca,P,tot,S,T,NT,flow,maxtot;
int g[maxn],last[maxn],h[maxn],vh[maxn];
int a[kMaxM][4],f[kMaxM],adj[kMaxM],next[kMaxM];

void insert(int x, int y, int limit){
    f[tot]=limit;
    adj[tot]=y;
    next[tot]=g[x];
    g[x]=tot++;

    f[tot]=0;
    adj[tot]=x;
    next[tot]=g[y];
    g[y]=tot++;
}

int dfs(int now, int add){
    if (now==T) return add;
    int y, tmp, minh=NT+1, p=last[now];
    do{
        y=adj[p];
        if (f[p]>0 && p<maxtot){
            if (h[now]==h[y]+1){
                tmp=dfs(y,min(f[p],add));
                if (tmp!=0){
                    f[p]-=tmp;
                    f[p^1]+=tmp;
                    last[now]=p;
                    return tmp;
                }
            }
            minh=min(minh,h[y]+1);
            if (h[S]>NT) return 0;
        }
        p=next[p];
        if (p==-1) p=g[now];
    } while (p!=last[now]);

    if (--vh[h[now]]==0) h[S]=NT+1;
    h[now]=minh; ++vh[minh];
    return 0;
}

int getflow(){
    //a[i][0],a[i][1],a[i][2],a[i][3]表示(a[i][0],a[i][1])
    //这条边以及下界流量为a[i][2],上界流量为a[i][3]
    //tot是边的总数
    tot=0;
    memset(g,255,sizeof(g));
    for (int i=1; i<=m; i++)
    insert(a[i][0],a[i][1],a[i][3]-a[i][2]);
    int tmp=tot;
    //前面tot条边都是上界流量-下界流量
    //做第二次最大流的时候只能用这tot条边
    //所以在这里要用tmp纪录tot

    //S和T在第一次网络流中是超级源和超级汇。
    //对于一条下界边(u,v,lower),连(u,T,lower)和(S,v,lower)
    S=n+1; T=S+1;
    for (int i=1; i<=m; i++)
    {
        insert(a[i][0],T,a[i][2]);
        insert(S,a[i][1],a[i][2]);
    }
    //s和t是原图中的源和汇,第一次最大流要连一条(t,s,无穷大)的边
    insert(t,s,INF);

    memset(h,0,sizeof(h));
    memset(vh,0,sizeof(vh));
    for (int i=1; i<=T; i++) last[i]=g[i];
    //maxtot表示做网络流时所能经过的编号最大的边,NT表示有多少个顶点
    maxtot=tot; vh[0]=T; flow=0; NT=n+2;
    while (h[S]<=NT) flow+=dfs(S,INF);
    //第一次最大流,如果流量不等于所有边的下界总和,则方案不合法
    if (flow!=m*lower) return -1;

    //第二次最大流在残余图上做
    //且只能使用前tmp条边,源和汇是原图的源汇
    //f[tot-1]表示从原图的源流出的流量,加入到最后的流量中
    S=s; T=t; flow=f[tot-1];
    memset(h,0,sizeof(h));
    memset(vh,0,sizeof(vh));
    for (int i=1; i<=n; i++) last[i]=g[i];
    //若T比定点个数小,则设NT=n,所有的h[S]<=NT而不是<=T
    maxtot=tmp; vh[0]=n; NT=n;
    while (h[S]<=NT) flow+=dfs(S,INF);
    //flow就是这个有上下界流量的最大流
    return flow;
}
\end{verbatim}

    \subsubsection{混合图欧拉回路}
\begin{verbatim}
#include<iostream>
#include<cstdio>
#include<algorithm>
#include<cstring>
#include<ctime>
using namespace std;

const int INF=0x7FFFFFFF;
const int maxn=5000;
int n,m,num,s,t,all,correct,flow;
int a[maxn][3],adj[maxn],next[maxn],vt[maxn],h[maxn];
int vh[maxn],st[maxn],f[maxn],sum[maxn],sign[maxn][2],fa[maxn];

int maxflow(int x, int ff)
{
  if (x==t) return ff;

  int minh=t-s+2;

  for (int i=1; i<=sum[x]; i++){
    if (f[st[x]]>0){
    //因为欧拉回路中两点之间可能有多条连边,用邻接矩阵可能会出现问题
      if (h[x]==h[adj[st[x]]]+1){
        int k=maxflow(adj[st[x]],min(ff,f[st[x]]));
        if (k>0){
          f[st[x]]-=k;
          f[st[x]^1]+=k;
          return k;
        }
      }
      minh=min(minh,h[adj[st[x]]]+1);
      if (h[s]>t-s+1) return 0;
    }
    st[x]=next[st[x]];
    if (st[x]==0) st[x]=vt[x];
  }

  --vh[h[x]];
  if (vh[h[x]]==0) h[s]=t-s+2;
  h[x]=minh;
  ++vh[h[x]];

  return 0;
}
void connect(int x,int y)
{
  ++all;
  ++sum[x];
  adj[all]=y;
  next[all]=vt[x];
  vt[x]=all;
}
int ask(int x)
{
  if (fa[x]==x) return x;
  fa[x]=ask(fa[x]);
  return fa[x];
}
void solve()
{
  for (int i=2; i<=n; i++)
    if (ask(i)!=ask(i-1)){
      printf("impossible\n");
      return;
    }
    //并查集判断图的联通性

  s=0; t=n+1; correct=0; flow=0;

  for (int i=1; i<=n; i++)
    if ((abs(sign[i][0]-sign[i][1])%2)==1){
      printf("impossible\n");
      return;
    } else if (sign[i][0]>sign[i][1]){
      connect(s,i);
      f[all]=(sign[i][0]-sign[i][1])/2;
      connect(i,s);
      //将入度大于出度的点连到起点s
      correct+=(sign[i][0]-sign[i][1])/2;
    } else if (sign[i][1]>sign[i][0]){
      connect(i,t);
      f[all]=(sign[i][1]-sign[i][0])/2;
      connect(t,i);
      //将出度大于入度的点连接到汇点t
    }

  for (int i=s; i<=t; i++) st[i]=vt[i];

  memset(h,0,sizeof(h));
  memset(vh,0,sizeof(vh));
  vh[0]=t-s+1;

  while (h[s]<=t-s+1) flow+=maxflow(s,INF);

  if (flow==correct) printf("possible\n"); else printf("impossible\n");
  //满流时成立
}
void prepare()
{
  all=1;
  memset(sum,0,sizeof(sum));
  memset(vt,0,sizeof(vt));
  memset(sign,0,sizeof(sign));
  memset(f,0,sizeof(f));
  for (int i=1; i<=n; i++) fa[i]=i;
}
void init()
{
  scanf("%d",&num);
  while (num--){
    scanf("%d%d",&n,&m);

    prepare();
    for (int i=1; i<=m; i++){
      scanf("%d%d%d",&a[i][0],&a[i][1],&a[i][2]);
      if (a[i][2]==0){
        if ((rand()%2)==0) swap(a[i][0],a[i][1]);
        //随机双向边的方向
        connect(a[i][0],a[i][1]);
        f[all]=1;
        //构图时只连接双向边,流的方向为随机后的方向
        connect(a[i][1],a[i][0]);
      }
      ++sign[a[i][0]][0]; ++sign[a[i][1]][1];
      int x=ask(a[i][0]), y=ask(a[i][1]);
      fa[x]=fa[y];
    }
    solve();
  }
}
int main()
{
  srand(time(0));
  init();
  return 0;
}

//pku 1637

\end{verbatim} 

    \subsubsection{网络流割点}
网络流求割点:将一个点拆成两个点,两个点之间连一条边。求出最大流后,从源点开始DFS遍历,如果一条边没有满流就可以通过,记录源点S可以到达的点。如果一条边是割边,则这条边的两个点分别属于S集合和T集合。

	\subsubsection{Dijkstra最短路}
\begin{verbatim}
const int MAXN = 50 + 10; //number of vertices
const int MAXM = 500; //number of MinCut edges
const int INF = 1000000000; //max capacity

int map[MAXN][MAXN], a[MAXN][MAXN], idx[MAXN][MAXN]; //map, tmp map, idx of edge
int root[MAXN], q[MAXN], w[MAXN], pre[MAXN];
int list[MAXM]; //MinCut Edges
bool used[MAXN];
int N, M;

int mincut(int n) {
	memset(used, 0, sizeof(used));
	memset(w, 0, sizeof(w));
	int last, cnt = 0;
	for (int k, i = 0; i != n; i = k) {
		last = i; used[i] = 1; k = n;
		for (int j = 0; j < n; ++j) {
			if (used[j]) continue;
			w[j] += a[q[i]][q[j]]; pre[j] = i;
			if (w[j] > w[k]) k = j;
		}
	}
	return last;
}

int find(int x) {
	if (root[x] == x) return x;
	else return root[x] = find(root[x]);
}

int stoer_wagner() {
	memcpy(a, map, sizeof(map));
	for (int i = 0; i < N; ++i) q[i] = root[i] = i;
	int ret = INF;
	for (int i = 0; i < N - 1; ++i) {
		int t = mincut(N - i);
		ret = min(ret, w[t]);
		int s = pre[t];
		for (int j = 0; j < N - i; ++j)
			if (j != s && j != t) a[q[t]][q[j]] = (a[q[j]][q[t]] += a[q[j]][q[s]]);
		root[find(q[s])] = find(q[t]); q[s] = q[N - i - 1]; 
	}
	return ret;
}

void cal(int ans) {
	memcpy(a, map, sizeof(map));
	for (int i = 0; i < N; ++i) q[i] = root[i] = i;
	int t;
	for (int i = 0; i < N - 2; ++i) {
		t = mincut(N - i - 1);
		if (w[t] == ans) break;
		int s = pre[t];
		for (int j = 0; j < N - i - 1; ++j)
			if (j != s && j != t) a[q[t]][q[j]] = (a[q[j]][q[t]] += a[q[j]][q[s]]);
		root[find(q[s])] = find(q[t]); q[s] = q[N - i - 2]; 
	}
	t = find(q[t]);
	M = 0; //number of MinCut edges
	for (int i = 0; i < N; ++i)
		if (find(root[i]) == t)
			for (int j = 0; j < N; ++j)
				if (find(root[j]) != t && idx[i][j]) list[M++] = idx[i][j];
}
\end{verbatim}

	\subsubsection{Dijkstra最短路}
\begin{verbatim}
#include <cstdio>
#include <cstring>
#include <algorithm>

using namespace std;

const int maxn = 200 + 10;
const int maxm = maxn * maxn;
const int INF = 100000000;

struct Tedge {
	int v, f, c, next;
};


Tedge edge[maxm];
int first[maxn], level[maxn], nedge[maxn], pedge[maxn], prev[maxn], queue[maxn], par[maxn], fl[maxn];
int a[maxn][maxn], cut[maxn][maxn];
int n, m, S, T;

inline void add_edge(int u, int v, int c1, int c2 = 0) {
	edge[m].v = v; edge[m].f = 0; edge[m].c = c1; edge[m].next = first[u]; first[u] = m++;
	edge[m].v = u; edge[m].f = 0; edge[m].c = c2; edge[m].next = first[v]; first[v] = m++;
}

bool newphase() {
	for (int i = 0; i < n; ++i) level[i] = n, nedge[i] = first[i];
	queue[0] = S; level[S] = 0;
	for (int h = 0, t = 1; h < t; ++h) {
		int u = queue[h];
		for (int i = first[u]; i != -1; i = edge[i].next)
			if (edge[i].f < edge[i].c && level[edge[i].v] == n) {
				level[edge[i].v] = level[u] + 1;
				if (edge[i].v == T) return 1;
				queue[t++] = edge[i].v;
			}
	}
	return 0;
}

bool find_path(int u) {
	for (int i = nedge[u]; i != -1; i = edge[i].next)
		if (edge[i].f < edge[i].c && level[edge[i].v] == level[u] + 1)
			if (edge[i].v == T || find_path(edge[i].v)) {
				pedge[edge[i].v] = nedge[u] = i;
				return 1;
			}
	nedge[u] = -1;
	return 0;
}

int Dinic() {
	for (int i = 0; i < m; ++i) edge[i].f = 0;
	int ret = 0;
	while (newphase())
		while (find_path(S)) {
			int delta = INF;
			for (int u = T, i = pedge[u]; u != S; u = edge[i ^ 1].v, i = pedge[u])
				delta = min(delta, edge[i].c - edge[i].f);
			for (int u = T, i = pedge[u]; u != S; u = edge[i ^ 1].v, i = pedge[u])
				edge[i].f += delta, edge[i ^ 1].f -= delta;
			ret += delta;
		}
	return ret;
}

int main() {
	int N;
	scanf("%d", &N);
	for (int tst = 1; tst <= N; ++tst) {
		m = 0;
		memset(first, -1, sizeof(first));
		scanf("%d", &n);
		for (int i = 0; i < n; ++i)
			for (int j = 0; j < n; ++j) {
				scanf("%d", &a[i][j]);
				if (i < j && a[i][j]) add_edge(i, j, a[i][j], a[i][j]);
			}

		memset(cut, 0, sizeof(cut));
		memset(par, 0, sizeof(par));
		for (S = 1; S < n; ++S) {
			T = par[S];
			fl[S] = cut[S][T] = cut[T][S] = Dinic();
			for (int i = 1; i < n; ++i)
				if (i != S && level[i] != n && par[i] == T) par[i] = S;
			if (level[par[T]] != n) {
				par[S] = par[T];
				par[T] = S;
				fl[S] = fl[T];
				fl[T] = cut[S][T];
			}
			for (int i = 0; i < S; ++i)
				if (i != T) cut[S][i] = cut[i][S] = min(cut[S][T], cut[T][i]);
		}
		// (i, par[i]) of value fl[i][par[i]] are the edges of GH cut tree

		printf("Case #%d:\n", tst);
		for (int i = 0; i < n; ++i) {
			for (int j = 0; j < n; ++j) {
				if (j) printf(" ");
				printf("%d", cut[i][j]);
			}
			printf("\n");
		}
	}

	return 0;
}
\end{verbatim}


	\subsection{匹配}
    \subsubsection{匈牙利算法O(V*E)}
\begin{verbatim}
#include "template.cpp"

/*
 * name     :     hungarian    
 * usage     :    maximum cardinally bipartite matching
 * develop    :    Hopcroft-Karp(faster), Kuhn-Munkres(with weight)
 * space complexity    :    O(M)
 * time complexity    :    O(N * M)
 * checked    :    no
 */

const int N = 100 + 10;

int n, m;
vector<int> E[N];
bool visit[N];
int v[N];

int find(int x) {
    int y;
    rep(i, E[x].size()){
        y = E[x][i];
        if (visit[y]) continue;
        visit[y] = true;
        if (v[y] == -1 || find(v[y])){
            v[y] = x;
            return true;
        }
    }
    return false;
}

int max_match() {
    memset(v, -1, sizeof(v));
    int res = 0;
    rep(i, n){
        Cls(visit);
        if (find(i)) res++;
    }
    return res;
}

int main(){
    while(~scanf("%d%d", &n, &m)){
        rep(i, n) E[i].clear();
        int x, y;
        rep(i, m){
            scanf("%d%d", &x, &y);
            x--, y--;
            E[x].PB(y);
        }
        printf("%d\n", max_match());
    }
    return 0;
}
\end{verbatim}

	\subsubsection{二分图匹配O(sqrt(V)*E)}
\begin{verbatim}
const int MAXN = 1000; //number of vertices
const int MAXE = 10000; //number of edges

struct Tedge {
    int v, next;
};

Tedge edge[MAXE];
int first[MAXN], px[MAXN], py[MAXN], dx[MAXN], dy[MAXN], q[MAXN];
bool used[MAXN];
int N, E, len;

void init() {
    memset(first, -1, sizeof(first));
    E = 0;
}

inline void add_edge(int u, int v) {
    edge[E].v = v; edge[E].next = first[u]; first[u] = E++;
}

bool search(int u) {
    if (dx[u] > len) return 0;
    used[u] = 1;
    for (int i = first[u]; i != -1; i = edge[i].next) {
        int v = edge[i].v;
        if ((py[v] == -1 || !used[py[v]]) && dx[u] + 1 == dy[v]) {
            int tx = px[u], ty = py[v];
            px[u] = v; py[v] = u;
            if (ty == -1 || search(ty)) return 1;
            px[u] = tx; py[v] = ty;
        }
    }
    return 0;
}

void hopcroft() {
    memset(px, -1, sizeof(px));
    memset(py, -1, sizeof(py));
    while (1) {
        memset(dx, 0, sizeof(dx));
        memset(dy, 0, sizeof(dy));
        int t = len = 0;
        for (int i = 0; i < N; ++i)
            if (px[i] == -1) q[t++] = i, dx[i] = 1;
        for (int h = 0; h < t; ++h) {
            int u = q[h];
            for (int i = first[u]; i != -1; i = edge[i].next) {
                int v = edge[i].v;
                if (!dy[v]) {
                    dy[v] = dx[u] + 1;
                    if (py[v] != -1) q[t++] = py[v], dx[py[v]] = dy[v] + 1;
                    else len = max(len, dy[v]);
                }
            }
        }
        if (!len) break;
        memset(used, 0, sizeof(used));
        for (int i = 0; i < N; ++i)
            if (px[i] == -1) search(i);
    }
}
\end{verbatim}

    \subsubsection{二分图最大权匹配}
\begin{verbatim}
const int N = 105, inf = 0x3F3F3F3F;

int n;
int graph[N][N];
int match[N], slack[N], lx[N], ly[N];
bool vx[N], vy[N];

bool find(int x) {
    vx[x] = true;
    rep(y, n){
        if (vy[y]) continue;
        if (lx[x] + ly[y] == graph[x][y]){
            vy[y] = true;
            if (match[y] == -1 || find(match[y])){
                match[y] = x;
                return true;
            }
        }
        else slack[y] = min(slack[y], lx[x] + ly[y] - graph[x][y]);
    }
    return false;
}

int max_match(int n) {
    rep(i, n) {
        lx[i] = *max_element(graph[i], graph[i] + n);
        ly[i] = 0;
    }
    memset(match, -1, sizeof(match));
    rep(x, n){
        memset(slack, -1, sizeof slack);
        while(true){
            Cls(vx);
            Cls(vy);
            if (find(x)) break;
            int sub = inf;
            rep(i, n) if (!vy[i]) sub = min(sub, slack[i]);
            rep(i, n) if (vx[i]) lx[i] -= sub;
            rep(i, n) {
                if (vy[i]) ly[i] += sub;
                else slack[i] -= sub;
            }
        }
    }
    int res = 0;
    rep(i, n) res += graph[match[i]][i];
    return res;
}

int min_match(int n) {
    rep(i, n) rep(j, n) graph[i][j] *= -1;
    return -max_match(n);
}

int main() {
    while(~scanf("%d", &n)){
        rep(i, n) rep(j, n) scanf("%d", &graph[i][j]);
        printf("%d\n", max_match(n));
    }
    return 0;
}
\end{verbatim}

	\subsubsection{带花树}
\begin{verbatim}
const int N = 50, M = 150;

int n, m;
int x[M],y[M];int psz;
int next[N], match[N], v[N];
int f[N], rank[N];
int ans[M];

VI E[N];
deque<int> Q;

inline int find(int p) {return f[p]<0?p:f[p]=find(f[p]);}

void join(int x, int y){
    x = find(x); y = find(y);
    if (x != y) f[x] = y;
}

int lca(int x, int y){
    static int v[N];
    static int stamp = 0;
    ++stamp;
    for (;;) {
        if (x >= 0) {
            x = find(x);
            if (v[x] == stamp) return x;
            v[x] = stamp;
            if (match[x] >= 0) x = next[match[x]];
            else x = -1;
        }
        swap(x, y);
    }
}

void group(int a, int p){
    while (a != p) {
        int b = match[a], c = next[b];
        if (find(c) != p) next[c] = b;
        if (v[b] == 2) Q.PB(b),v[b] = 1;
        if (v[c] == 2) Q.PB(c),v[c] = 1;
        join(a, b); join(b, c);
        a = c;
    }
}

void aug(int s){
    Cls(v,0);
    Cls(next,-1);
    Cls(f,-1);
    Q.clear();
    Q.PB(s);
    v[s] = 1;
    while(!Q.empty()&&match[s]==-1){
        int x=Q.front();Q.pop_front();
        rep(i,E[x].size()){
            int y = E[x][i];
            if (match[x] == y || find(x) == find(y) || v[y] == 2) continue;
            if (v[y] == 1) {
                int p = lca(x, y);
                if (find(x) != p) next[x] = y;
                if (find(y) != p) next[y] = x;
                group(x, p);
                group(y, p);
            } else if (match[y] == -1) {
                next[y] = x;
                while (~y) {
                    int z = next[y];
                    int p = match[z];
                    match[y] = z; match[z] = y;
                    y = p;
                }
                break;
            } else {
                next[y] = x;
                Q.PB(match[y]);
                v[match[y]]=1;
                v[y] = 2;
            }
        }
    }
}

int work(int k){
    psz = 0;
    rep(i,n) E[i].clear();
    rep(i,m){
        if (x[i] == x[k]) continue;
        if (x[i] == y[k]) continue;
        if (y[i] == x[k]) continue;
        if (y[i] == y[k]) continue;
        E[x[i]].PB(y[i]);
        E[y[i]].PB(x[i]);
    }
    Cls(match,-1);
    rep(i,n) if (match[i]==-1) aug(i);
    int c = 0;
    rep(i,n) if (match[i]!=-1) c++;
    return c/2;
}

int main(){
    while(~scanf("%d%d", &n, &m)){
        rep(i,m) {
            scanf("%d%d",x+i,y+i);
            x[i]--,y[i]--;
        }
        x[m]=y[m]=n;
        int s=work(m);
        int tot=0;
        rep(i,m) if (work(i)!=s-1) ans[tot++]=i+1;
        printf("%d\n", tot);
        if (tot) rep(i,tot) printf(i==tot-1?"%d\n":"%d ",ans[i]);
        else puts("");
    }
}
\end{verbatim}


    \subsection{图}
	%\input{code/graph/tarjan}
	\subsubsection{最大团}
\begin{verbatim}
const int MAXN = 100; //number of vertices

int a[MAXN][MAXN];
int f[MAXN];
int N, ans;

bool DFS(int q[], int t, int cnt) {
    if (t == 0) {
        if (cnt > ans) {
            ans = cnt;
            return 1;
        }
        return 0;
    }

    int tq[MAXN];
    for (int i = 0; i < t; ++i) {
        if (f[q[i]] + cnt <= ans) return 0;
        int k = 0;
        for (int j = i + 1; j < t; ++j)
            if (a[q[i]][q[j]]) tq[k++] = q[j];
        if (DFS(tq, k, cnt + 1)) return 1;
    }
    return 0;
}

void MaxClique() {
    ans = 0;
    int q[MAXN];
    for (int i = N - 1; i >= 0; --i) {
        int t = 0;
        for (int j = i + 1; j < N; ++j) if (a[i][j]) q[t++] = j;
        DFS(q, t, 1);
        f[i] = ans;
    }
}
\end{verbatim}

	\subsubsection{树链剖分}
\begin{verbatim}
//By myf
//#pragma comment(linker, "/STACK:16777216")  //C++
#include <iomanip>
#include <iostream>
#include <algorithm>
#include <cmath>
#include <cstdio>
#include <cstdlib>
#include <cstring>
#include <bitset>
#include <complex>
#include <map>
#include <set>
#include <queue>
#include <deque>
#include <stack>
#include <vector>
#include <list>

#define rep(i,n) for(int i=0;i<(n);i++)
#define REP(i,l,r) for(int i=(l);i<(r);i++)
#define fab(i,a,b) for(int i=(a);i<=(b);i++)
#define fba(i,b,a) for(int i=(b);i>=(a);i--)
//#define foreach(i,n) for(__typeof(n.begin()) i=n.begin();i!=n.end();i++) //G++
#define MP make_pair
#define PB push_back
#define X first
#define Y second
#define Cls(x) memset(x,0,sizeof x)
#define Print(n,x) for(int i=0;i<(n);i++) cout<<x<<" ";cout<<endl;
#define Sqr(x) (x)*(x)

using namespace std;

typedef long long LL;
typedef pair<int,int> PII;
typedef pair<PII,int> PIII;
typedef pair<LL,int> PLI;
typedef vector<int> VI;
typedef LL T;

const int N=50005,M=1<<16;

int n,m,q,tot;
int v[N];
int t[M*2];
VI E[N];
int fa[N],dep[N],son[N],sz[N];
int id[N],top[N];

void dfs(int x){
    sz[x]=1,son[x]=0;
    rep(i,E[x].size()){
        int y=E[x][i];
        if (y==fa[x]) continue;
        dep[y]=dep[x]+1;
        fa[y]=x;
        dfs(y);
        sz[x]+=sz[y];
        if (sz[y]>sz[son[x]]) son[x]=y;
    }
}

void dfs(int x,int p){
    id[x]=++tot,top[x]=p;
    if (son[x]) dfs(son[x],p);
    rep(i,E[x].size()){
        int y=E[x][i];
        if (y==fa[x]||y==son[x]) continue;
        dfs(y,y);
    }
}

int ask(int x){
    x=id[x];
    int ret=0;
    for(x+=M;x;x>>=1) ret+=t[x];
    return ret;
}

void insert(int l,int r,int x){
    for(l+=M-1,r+=M+1;l^r^1;l>>=1,r>>=1){
        if (~l&1) t[l^1]+=x;
        if ( r&1) t[r^1]+=x;
    }
}

void add(int x,int y,int k){
    while(top[x]!=top[y]){
        if (dep[top[x]]<dep[top[y]]) swap(x,y);
        insert(id[top[x]],id[x],k);
        x=fa[top[x]];
    }
    if (dep[x]<dep[y]) swap(x,y);
    insert(id[y],id[x],k);
}

int main(){
    while(~scanf("%d%d%d",&n,&m,&q)){
        rep(i,n) scanf("%d",&v[i+1]);
        rep(i,n) E[i+1].clear();
        rep(i,m){
            int x,y;
            scanf("%d%d",&x,&y);
            E[x].PB(y);
            E[y].PB(x);
        }
        fa[1]=dep[1]=1;
        sz[0]=0,tot=0;
        dfs(1);
        dfs(1,1);
        Cls(t);
        fab(i,1,n) t[id[i]+M]=v[i];
        char ch;
        int x,y,k;
        rep(i,q){
            while((ch=getchar())&&ch!='D'&&ch!='Q'&&ch!='I');
            if (ch=='Q'){
                scanf("%d",&x);
                printf("%d\n",ask(x));
            }
            else{
                scanf("%d%d%d",&x,&y,&k);
                add(x,y,(ch=='I')?k:-k);
            }
        }
    }
    return 0;
}
\end{verbatim}

    \subsubsection{Kosaraju强联通分量}
\begin{verbatim}
#include <stdio.h>
#include <string.h>
#include <algorithm>
#include <vector>
using namespace std;

const int N = 10010;

int n, m, c;
int mark[N], stack[N], top;
vector<int> vout1[N], vout2[N];

void dfs1(int u) {
  mark[u] = -1;
  for (int z = 0; z < vout1[u].size(); z++) {
    int v = vout1[u][z];
    if (!mark[v]) dfs1(v);
  }
  stack[top++] = u;
}

void dfs2(int u, int k) {
  mark[u] = k;
  for (int z = 0; z < vout2[u].size(); z++) {
    int v = vout2[u][z];
    if (mark[v] == -1) dfs2(v, k);
  }
}

int main() {
  scanf("%d %d", &n, &m);
  while (m--) {
    int u, v;
    scanf("%d %d", &u, &v);
    vout1[u].push_back(v);
    vout2[v].push_back(u);
  }
    
  top = 0;
  memset(mark, 0, n * sizeof(mark[0]));
  for (int u = 0; u < n; u++)
    if (!mark[u]) dfs1(u);
  
  c = 0;
  while (top) {
    int u = stack[--top];
    if (mark[u] == -1) dfs2(u, c++);
  }
  
  return 0;
}
\end{verbatim}

    \subsubsection{割点和桥}
\begin{verbatim}
const int N = 1111, M = 1111111;

int n, m;
int root;
int low[N], dep[N];
bool cut[N], bri[N];
vector<int> E[N];
vector<int> id[N];
PII edge[N];

void dfs(int x, int f, int d){
    int e = 0, deg = 0;
    low[x] = dep[x] = d;
    rep(i, E[x].size()){
        int y = E[x][i];
        if (low[y] == -1){
            deg++;
            dfs(y, x, d + 1);
            low[x] = min(low[x], low[y]);
            if (low[y] > dep[x]) bri[id[x][i]] = true;
            cut[x] |= (x == root && deg > 1 || x != root && low[y] >= dep[x]);
        }
        else if (y != f || e){
            low[x] = min(low[x], dep[y]);
        }
        else e = 1;
    }
}


int main(){
    while(~scanf("%d%d",&n,&m)){
        rep(i, n) E[i].clear(), low[i] = dep[i] = -1, cut[i] = false;
        rep(i, n) id[i].clear();
        int x, y;
        rep(i, m){
            scanf("%d%d", &x, &y);
            x--, y--;
            bri[i] = false;
            edge[i] = MP(x, y);
            E[x].PB(y), id[x].PB(i);
            E[y].PB(x), id[y].PB(i);
        }
        dfs(root = 0, -1, 0);
    }
    return 0;
}
\end{verbatim}

    \subsubsection{哈密尔顿回路}
求一个图的哈密尔顿回路是一个NP问题,只能用搜索解决。当定点数$n$比较小的时候,可以用状态压缩Dp解决。当$n$比较大的时候,只能用
搜索解决。但是,关于哈密尔顿回路,有一个性质:\par
Ore性质:对所有不邻接的不同顶点对$x$和$y$,有$$deg(x)+deg(y)\geq n$$
那么这个图一定存在哈密尔顿回路,且可以用一下方法求回路,时间复杂度接近$O(n^2)$ \par
~\\
1)从任意一个顶点开始,在它的任意一端邻接一个顶点,构造一条越来越长的路径,直到不能再加长为止。设路径为$$\gamma:y_1-y_2-\cdots-y_m$$
~\\
2)检查$y_1$和$y_m$是否邻接。\par
.\quad \quad a)如果$y_1$和$y_m$不邻接,则转到3,否则,$y_1$和$y_m$是邻接的,转到b。\par
.\quad \quad b)如果$m=n$,则停止构造并输出哈密尔顿回路$y_1-y_2-\cdots-y_m-y_1$,否则,转到c。\par
.\quad \quad
c)找出一个不在$\gamma$上的顶点$z$和在$\gamma$上的顶点$y_k$,满足$z$和$y_k$是邻接的,将$\gamma$用下面的长度为$m+1$的路径来替代
$$z-y_k-\cdots-y_m-y_1-\cdots-y_{k-1}$$
.\quad \quad 转到2)
~\\
3)找出一个顶点$y_k(1<k<m)$,满足$y_1$和$y_k$是邻接的,且$y_{k-1}$和$y_{m}$也是邻接的,将$\gamma$用下面的路径来替代
$$y_1-\cdots-y_{k-1}-y_{m}-\cdots-y_k$$
.\quad \quad 转到2)

	\subsubsection{点的分治,权值在边上}
\begin{verbatim}
//By myf
#include <iomanip>
#include <iostream>
#include <algorithm>
#include <cmath>
#include <cstdio>
#include <cstdlib>
#include <cstring>
#include <bitset>
#include <complex>
#include <map>
#include <set>
#include <queue>
#include <deque>
#include <stack>
#include <vector>
#include <list>

#define rep(i,n) for(int i=0;i<(n);i++)
#define REP(i,l,r) for(int i=(l);i<(r);i++)
#define fab(i,a,b) for(int i=(a);i<=(b);i++)
#define fba(i,b,a) for(int i=(b);i>=(a);i--)
#define foreach(i,n) for(__typeof(n.begin()) i=n.begin();i!=n.end();i++)
#define MP make_pair
#define PB push_back
#define X first
#define Y second
#define Cls(x) memset(x,0,sizeof x)
#define Print(n,x) for(int i=0;i<(n);i++) cout<<x<<" ";cout<<endl;

using namespace std;

typedef long long LL;
typedef pair<int,int> PII;
typedef pair<PII,int> PIII;
typedef pair<LL,int> PLI;
typedef vector<int> VI;
typedef int T;

const int N = 10000 + 10;

int n, k;
vector<PII> E[N];
int tot, top, mi, root;
int size[N], f[N];
int q[N];
bool use[N];

void getDist(int x, int dist, int fa){
    q[top++] = dist;
    rep(i, E[x].size()){
        int y = E[x][i].X, c = E[x][i].Y;
        if (use[y] || y == fa) continue;
        getDist(y, dist + c, x);
    }
}

int count(int x, int dist){
    int s = 0;
    top = 0;
    getDist(x, dist, -1);
    sort(q, q + top);
    for(int i = 0,j = top - 1; i <= j; i++){
        while(q[i] + q[j] > k && i < j) j--;
        if (i < j) s += j - i;
    }
    return s;
}

void getRoot(int x, int fa){
    int big = -1;
    size[x] = 1;
    rep(i, E[x].size()){
        int y = E[x][i].X, c = E[x][i].Y;
        if (use[y] || y == fa) continue;
        getRoot(y, x);
        size[x] += size[y];
        big=max(big, size[y]);
    }
    big = max(big, tot - size[x]);
    if (big < mi) mi = big, root = x;
}

void dfs(int x){
     tot = mi = size[x];
     getRoot(x, -1);
     x = root;
     f[x] = count(x, 0);
     use[x] = true;
     rep(i, E[x].size()){
         int y = E[x][i].X, c = E[x][i].Y;
         if (use[y]) continue;
         f[x] -= count(y, c);
         dfs(y);
     }
}

int main(){
    while(scanf("%d%d", &n, &k)){
        if (!n && !k) break;
        Cls(use);
        int x,y,c;
        rep(i, n) E[i].clear();
        rep(i, n - 1){
            scanf("%d%d%d", &x, &y, &c);
            x--, y--;
            E[x].PB(MP(y, c));
            E[y].PB(MP(x, c));
        }
        size[0] = n;
        dfs(0);
        int ans = 0;
        rep(i,n) ans += f[i];
        printf("%d\n", ans);
    }
    return 0;
}
\end{verbatim}

	\subsubsection{点的分治,权值在点上}
\begin{verbatim}
//By myf
//#pragma comment(linker, "/STACK:16777216")  //C++
#include <iomanip>
#include <iostream>
#include <algorithm>
#include <cmath>
#include <cstdio>
#include <cstdlib>
#include <cstring>
#include <bitset>
#include <complex>
#include <map>
#include <set>
#include <queue>
#include <deque>
#include <stack>
#include <vector>
#include <list>

#define rep(i,n) for(int i=0;i<(n);i++)
#define REP(i,l,r) for(int i=(l);i<(r);i++)
#define fab(i,a,b) for(int i=(a);i<=(b);i++)
#define fba(i,b,a) for(int i=(b);i>=(a);i--)
//#define foreach(i,n) for(__typeof(n.begin()) i=n.begin();i!=n.end();i++) //G++
#define MP make_pair
#define PB push_back
#define X first
#define Y second
#define Cls(x) memset(x,0,sizeof x)
#define Print(n,x) for(int i=0;i<(n);i++) cout<<x<<" ";cout<<endl;

using namespace std;

typedef long long LL;
typedef pair<int,int> PII;
typedef pair<PII,int> PIII;
typedef pair<LL,int> PLI;
typedef vector<int> VI;
typedef int T;

const int N=50000+10,M=30;

int n,k;
VI E[N];
int tot,top,mi,root;
int size[N];
LL f[N];
map<LL,int> Q;
LL prime[N];
int sta[N][M];
LL q[N];
bool use[N];
LL base[M+1];

LL ans;

inline LL add(LL x,int sta[M]){
    LL y=0;
    rep(i,k){
        int tmp=(x%base[i+1])/base[i];
        tmp+=sta[i];
        tmp%=3;
        y+=(tmp*base[i]);
    }
    return y;
}

inline LL dec(LL a,LL b){
    LL y=0;
    rep(i,k){
        int tmp1=(a%base[i+1])/base[i];
        int tmp2=(b%base[i+1])/base[i];
        int tmp=(tmp1-tmp2+3)%3;
        y+=(tmp*base[i]);
    }
    return y;
}

inline LL add(LL a,LL b){
    LL y=0;
    rep(i,k){
        int tmp1=(a%base[i+1])/base[i];
        int tmp2=(b%base[i+1])/base[i];
        int tmp=(tmp1+tmp2)%3;
        y+=(tmp*base[i]);
    }
    return y;
}


void getVal(int x,LL val,int fa){
    q[top++]=val;
    rep(i,E[x].size()){
        int y=E[x][i];
        if (use[y]||y==fa)continue;
        getVal(y,add(val,sta[y]),x);
    }
}

void getRoot(int x,int fa){
    int big=-1;
    size[x]=1;
    rep(i,E[x].size()){
        int y=E[x][i];
        if (use[y]||y==fa) continue;
        getRoot(y,x);
        size[x]+=size[y];
        big=max(big,size[y]);
    }
    big=max(big,tot-size[x]);
    if (big<mi) mi=big,root=x;
}

void dfs(int x){
     tot=mi=size[x];
     getRoot(x,-1);
     x=root;
     use[x]=true;
     Q.clear();
     LL now=add(0,sta[x]);
     Q[now]=1;
     if (now==0) ans++;
     rep(i,E[x].size()){
         int y=E[x][i];
         if (use[y]) continue;
         top=0;
         getVal(y,add(0,sta[y]),x);
         rep(j,top){
             LL tmp=dec(0,q[j]);
             if (Q.count(tmp)) ans+=Q[tmp];
         }
         rep(j,top) Q[add(now,q[j])]++;
     }
     rep(i,E[x].size()) if (!use[E[x][i]]) dfs(E[x][i]);
}

int main(){
    base[0]=1;
    fab(i,1,M) base[i]=base[i-1]*3;
    while(~scanf("%d",&n)){
        Cls(use);
        rep(i,n) E[i].clear();
        scanf("%d",&k);
        rep(i,k) scanf("%I64d",prime+i);
        rep(i,n){
            LL x;
            scanf("%I64d",&x);
            Cls(sta[i]);
            rep(j,k) while (x%prime[j]==0) x/=prime[j],sta[i][j]++,sta[i][j]%=3;
        }
        rep(i,n-1){
            int x,y;
            scanf("%d%d",&x,&y);
            x--,y--;
            E[x].PB(y);
            E[y].PB(x);
        }
        size[0]=n;
        ans=0;
        dfs(0);
        printf("%I64d\n",ans);
    }
    return 0;
}
\end{verbatim}

	\subsubsection{图平面化}
\begin{verbatim}
// vertices numbered from 1 to N
// No self-loops and no duplicate edges

typedef pair<int, int> T;
const int maxn = 10000 + 10;

struct node {
    int dep, fa, infc, used, vst, dfi, ec, lowp, bflag, flag, lowpoint;
};

int n, m, indee, p1, p2, p, ps;
int lk[maxn * 3][2], child[maxn * 3][3], bedg[maxn * 3][2], sdlist[maxn * 6][3], 
    buk[maxn * 6][2], exf[maxn * 3][2], proots[maxn * 3][3], stk[maxn * 3][2], infap[maxn * 3];
int w1[maxn], w2[maxn], que[maxn];
node dot[maxn];

void init(T * ts) {
    ps = 0;
    for (int i = 1; i <= n; ++i) w1[i] = i;
    p1 = n;
    for (int i = 0; i < m; ++i) {
        int k1 = ts[i].first, k2 = ts[i].second;
        lk[++p1][0] = k2; lk[p1][1] = 0;
        lk[w1[k1]][1] = p1;
        w1[k1] = p1;
        lk[++p1][0] = k1; lk[p1][1] = 0;
        lk[w1[k2]][1] = p1;
        w1[k2] = p1;
    }
    for (int i = 1; i <= n; ++i) que[i] = i;
}

int deep(int a) {
    dot[a].used = 1; dot[a].dfi = ++indee;
    int t = lk[a][1];
    while (t != 0) {
        int tmp = lk[t][0];
        if (!dot[tmp].used) {
            dot[tmp].fa = a; dot[tmp].dep = dot[a].dep + 1; dot[tmp].ec = dot[a].dep; dot[tmp].lowp = dot[a].dep;
            child[++p1][0] = tmp; child[p1][1] = 0;
            child[w1[a]][1] = p1;
            w1[a] = p1;
            int s = deep(tmp);
            if (s < dot[a].ec) dot[a].ec = s;
        }
        else if (dot[a].fa != tmp) {
            if (dot[a].lowp > dot[tmp].dep) dot[a].lowp = dot[tmp].dep;
            if (dot[a].dfi > dot[tmp].dfi) {
                bedg[++p2][0] = a; bedg[p2][1] = 0;
                bedg[w2[tmp]][1] = p2;
                w2[tmp] = p2;
            }
        }
        t = lk[t][1];
    }
    if (dot[a].ec > dot[a].lowp) dot[a].ec = dot[a].lowp;
    return dot[a].ec;
}

void sortvtx() {
    for (int i = 1; i <= n; ++i) w1[i] = i;
    p1 = n; p2 = 0;
    for (int i = 1; i <= n; ++i) {
        buk[++p1][0] = i; buk[p1][1] = 0;
        buk[w1[dot[i].dfi]][1] = p1;
        w1[dot[i].dfi] = p1;
    }
    for (int i = n; i > 0; --i) {
        int tmp = buk[i][1];
        while (tmp != 0) {
            que[++p2] = buk[tmp][0];
            tmp = buk[tmp][1];
        }
    }
}

void getsdlist() {
    memset(buk, 0, sizeof(buk));
    for (int i = 1; i <= n; ++i) {
        w1[i] = w2[i] = i;
        buk[i][1] = 0;
    }
    p1 = p2 = n;
    for (int i = 1; i <= n; ++i) {
        buk[++p1][0] = i; buk[p1][1] = 0;
        buk[w1[dot[i].ec]][1] = w1[dot[i].ec] = p1;
    }
    for (int i = 1; i <= n; ++i) {
        int tmp = buk[i][1];
        while (tmp != 0) {
            int fa = dot[buk[tmp][0]].fa;
            sdlist[++p2][0] = i; sdlist[p2][1] = 0;
            sdlist[w2[fa]][1] = dot[buk[tmp][0]].infc = p2;
            sdlist[p2][2] = w2[fa]; w2[fa] = p2;
            tmp = buk[tmp][1];
        }
    }
}

void getnextvtx(int v, int v1, int &m, int &m1) {
    m = exf[v][v1 ^ 1];
    if (exf[m][0] == v) m1 = 0;
    else m1 = 1;
}

void addwei(int a) {
    int fa = dot[a - n].fa;
    ++p1;
    proots[p1][0] = a; proots[p1][1] = 0;
    proots[w1[fa]][1] = p1;
    proots[p1][2] = w1[fa]; w1[fa] = p1;
    infap[a] = p1;
}

void addsou(int a) {
    int fa = dot[a - n].fa;
    ++p1;
    proots[p1][0] = a; proots[p1][1] = proots[fa][1]; proots[p1][2] = fa;
    proots[fa][1] = p1;
    proots[proots[p1][1]][2] = p1;
    infap[a] = p1;
    if (w1[fa] == fa) w1[fa] = p1;
}

void walkup(int v, int w) {
    dot[w].bflag = v;
    int x = w, x1 = 1, y = w, y1 = 0;
    while (x != v) {
        if (dot[x].vst == v || dot[y].vst == v) break;
        dot[x].vst = v; dot[y].vst = v;
        int z1 = 0;
        if (x > n) z1 = x;
        if (y > n) z1 = y;
        if (z1 != 0) {
            int c = z1 - n, z = dot[c].fa;
            if (z != v) {
                if (dot[c].lowpoint < dot[v].dep) addwei(z1);
                else addsou(z1);
            }
            x = z; x1 = 1;
            y = z; y1 = 0;
        } else {
            getnextvtx(x, x1, x, x1);
            getnextvtx(y, y1, y, y1);
        }
    }
}

void getactivenext(int v, int v1, int &m, int &m1, int vt) {
    m = v; m1 = v1;
    getnextvtx(m, m1, m, m1);
    while (dot[m].bflag != vt && proots[m][1] == 0 && dot[m].ec >= dot[vt].dep && m != v) getnextvtx(m, m1, m, m1);
}

void addstack(int a, int b) {
    stk[++ps][0] = a; stk[ps][1] = b;
}

void mergestack() {
    int t = stk[ps][0], t1 = stk[ps][1], k = stk[ps - 1][0], k1 = stk[ps - 1][1];
    ps -= 2;
    int s1, s = exf[t][1 ^ t1];
    if (exf[s][1] == t) s1 = 1;
    else s1 = 0;
    exf[k][k1] = s;
    exf[s][s1] = k;
    int tmp = dot[t - n].infc;
    sdlist[sdlist[tmp][2]][1] = sdlist[tmp][1]; sdlist[sdlist[tmp][1]][2] = sdlist[tmp][2];
    tmp = dot[t - n].fa;
    if (sdlist[tmp][1] == 0) dot[tmp].ec = dot[tmp].lowp;
    else dot[tmp].ec = min(dot[tmp].lowp, sdlist[sdlist[tmp][1]][0]);
    tmp = infap[t];
    int fa = dot[t - n].fa;
    proots[proots[tmp][2]][1] = proots[tmp][1];
    if (proots[tmp][1] != 0) proots[proots[tmp][1]][2] = proots[tmp][2];
    else w1[fa] = proots[tmp][2];
}

void embededg(int v, int v1, int w, int w1) {
    exf[v][v1] = w; exf[w][w1] = v;
}

void walkdown(int v) {
    ps = 0;
    int vt = dot[v - n].fa;
    for (int v2 = 0; v2 <= 1; ++v2) {
        int w, w1;
        getnextvtx(v, 1 ^ v2, w, w1);
        while (w != v) {
            if (dot[w].bflag == vt) {
                while (ps != 0) mergestack();
                embededg(v, v2, w, w1);
                dot[w].bflag = 0;
            }
            if (proots[w][1] != 0) {
                addstack(w, w1);
                int x, x1, y, y1, w2, w0 = proots[proots[w][1]][0];
                getactivenext(w0, 1, x, x1, vt);
                getactivenext(w0, 0, y, y1, vt);
                if (dot[x].ec >= dot[vt].dep) w = x, w1 = x1;
                else if (dot[y].ec >= dot[vt].dep) w = y, w1 = y1;
                else if (dot[x].bflag == vt || proots[x][1] != 0) w = x, w1 = x1;
                else w = y, w1 = y1;
                if (w == x) w2 = 0;
                else w2 = 1;
                addstack(w0, w2);
            }
            else if (w > n || dot[w].ec >= dot[vt].dep) getnextvtx(w, w1, w, w1);
            else {
                if (w <= n && dot[w].ec < dot[vt].dep && ps == 0) embededg(v, v2, w, w1);
                break;
            }
        }
        if (ps != 0) break;
    }
}

bool chainvtx(int a) {
    for (int t = child[a][1]; t != 0; t = child[t][1]) {
        int tmp = child[t][0];
        exf[tmp][1] = tmp + n; exf[tmp][0] = tmp + n;
        exf[tmp + n][1] = tmp; exf[tmp + n][0] = tmp;
    }
    for (int t = bedg[a][1]; t != 0; t = bedg[t][1]) walkup(a, bedg[t][0]);
    for (int t = child[a][1]; t != 0; t = child[t][1]) walkdown(child[t][0] + n);
    for (int t = bedg[a][1]; t != 0; t = bedg[t][1]) if (dot[bedg[t][0]].bflag != 0) return false;
    return true;
}

bool judge(int N, int M, T * ts) {
    n = N;    m = M;
    if (n == 1) return true;
    if (m > 3 * n - 5) return false;
    init(ts);
    
    for (int i = 1; i <= n; ++i) {
        proots[i][1] = 0; proots[i + n][1] = 0;
        p = 0;
        child[i][1] = 0;
        buk[i][1] = 0; buk[i + n][1] = 0;
        sdlist[i][1] = 0; sdlist[i + n][1] = 0;
        dot[i].bflag = 0; dot[i + n].flag = 0;
    }
    for (int i = 1; i <= n; ++i) {
        w1[i] = i; w2[i] = i;
        child[i][1] = 0; bedg[i][1] = 0;
        dot[i].used = 0;
    }
    indee = 0; p1 = p2 = n;
    for (int i = 1; i <= n; ++i) {
        if (!dot[i].used) {
            dot[i].dep = 1;
            deep(i);
        }
    }
    sortvtx();
    getsdlist();
    for (int i = 1; i <= n; ++i) {
        dot[i].lowpoint = dot[i].ec;
        dot[i].vst = 0; dot[i + n].vst = 0;
        proots[i][1] = 0;
        w1[i] = i;
    }
    p1 = n;
    for (int i = 1; i <= n; ++i) if (!chainvtx(que[i])) return false;
    return true;
}

T ts[maxn];
bool a[3001][3001];

int main() {
    int N, M;
    scanf("%d%d", &N, &M);
    int m = 0;
    for(int i = 0; i < M; i ++) {
        scanf("%d%d", &ts[i].first, &ts[i].second);
        ++ts[i].first; ++ts[i].second;
        if (ts[i].first == ts[i].second || a[ts[i].first][ts[i].second]) continue;
        a[ts[i].first][ts[i].second] = a[ts[i].second][ts[i].first] = 1;
        ts[m++] = ts[i];
    }
    M = m;
    if(judge(N, M, ts)) puts("YES");
    else puts("NO");

    return 0;
}
\end{verbatim}

	\subsubsection{有向图割点}
\begin{verbatim}
const int MAXN = 5000 + 10; //number of vertices
const int MAXM = 200000 + 10; //number of edges

struct Tedge {
    int v, next;
};

Tedge edge[MAXM], back[MAXM]; //back is opposite to edge
bool ontree[MAXM];
int first1[MAXN], first2[MAXN], id[MAXN], low[MAXN], stack[MAXN];
bool critical[MAXN]; //1 - the node is CutVertex, 0 - not
int N, M, cnt;

void DFS(int u) {
    id[u] = cnt; stack[cnt++] = u; low[u] = u;
    for (int i = first1[u]; i != -1; i = edge[i].next)
        if (id[edge[i].v] == -1) {
            ontree[i] = 1;
            DFS(edge[i].v);
        }
}

void update(int u) {
    for (int i = first1[u]; i != -1; i = edge[i].next)
        if (ontree[i] && id[low[u]] < id[low[edge[i].v]]) {
            low[edge[i].v] = low[u];
            update(edge[i].v);
        }
}

void CV() {
    cnt = 0;
    memset(id, -1, sizeof(id));
    memset(ontree, 0, sizeof(ontree));
    DFS(0);
    memset(critical, 0, sizeof(critical));
    critical[0] = 1;
    for (int i = cnt - 1; i >= 0; --i) {
        int u = stack[i];
        for (int j = first1[u]; j != -1; j = edge[j].next)
            if (ontree[j] && low[edge[j].v] == u) {
                critical[u] = 1;
                break;
            }
        for (int j = first2[u]; j != -1; j = back[j].next)
            if (id[low[back[j].v]] < id[low[u]]) low[u] = low[back[j].v];
        update(u);
    }
}
\end{verbatim}


    %\subsection{最近公共祖先算法}
    %\input{code/graph/lca_rmq}
    %\input{code/graph/tarjan}
    %\subsubsection{祖先树}
\begin{verbatim}
#include <vector>
#include <cstdio>
#include <cstring>
#include <iostream>
using namespace std;

const int kMaxN=100001;
const int maxDeep=17;

/*
 * The ancestor-tree can use to calc LCA with O(logN) Complexity
 * also maintain some information from node i to i's ancestor
 * deep[node] is the distance from root to node
 * father[node][i] is the 2^i ancestor of node
*/

int n,m;
int deep[kMaxN],father[kMaxN][maxDeep];
vector<int> g[kMaxN];

void init(){
    scanf("%d",&n);
    memset(g,0,sizeof(g));
    int x,y;
    for (int i=0; i<n-1; i++){
        scanf("%d%d",&x,&y);
        g[x].push_back(y);
        g[y].push_back(x);
    }
}

void dfs(int node, int fa){
    //update father[node][] use the information calc before
    for (int i=1; i<maxDeep; i++) 
        father[node][i]=father[father[node][i-1]][i-1];

    for (int i=0; i<g[node].size(); i++)
        if (g[node][i]!=fa){
            deep[g[node][i]]=deep[node]+1;
            father[g[node][i]][0]=node;
            dfs(g[node][i],node);
        }
}

int LCA(int x, int y){
    //choose the farther one from root and jump up until x,y has 
    //the same distance from root.
    if (deep[x]<deep[y]) swap(x,y);
    int delta=deep[x]-deep[y];
    for (int i=0; i<maxDeep; i++)
        if (delta & (1<<i)) x=father[x][i];

    //when the x's,y's (2^i)-th father is different,both of them
    //should jump up for 2^i step
    for (int i=maxDeep-1; i>=0; i--)
        if (father[x][i]!=father[y][i]){
            x=father[x][i],y=father[y][i];
        }

    //there are two situtation at last
    return x!=y?father[x][0]:x;
}

void solve(){
    memset(father,0,sizeof(father));
    deep[1]=0;
    dfs(1,0);

    scanf("%d",&m);
    int x,y;
    for (int i=0; i<m; i++){
        scanf("%d%d",&x,&y);
        printf("%d\n",LCA(x,y));
    }
}

int main(){
    init();
    solve();
    return 0;
}
\end{verbatim}


\section{\LARGE 数据结构}
	\subsection{平衡树}
    \subsubsection{heap}
\begin{verbatim}
#include <stdio.h>
#include <string.h>
#include <algorithm>
using namespace std;

typedef int T;

const int N = 10000 + 10;

T value[N]; 
int n;
int heap[N], pos[N], hn;

void heap_init() {
  hn = 0;
}

int heap_size() {
  return hn;
}

void heap_up(int x) {
  int p;
  while (x && value[heap[x]] < value[heap[p = (x - 1) >> 1]]) {
    swap(heap[x], heap[p]);
    swap(pos[heap[x]], pos[heap[p]]);
    x = p;
  }
}

void heap_down(int x) {
  int c;
  while ((c = (x << 1) + 1) < hn) {
    if (c + 1 < hn && value[heap[c + 1]] < value[heap[c]]) c++;
    if (value[heap[c]] < value[heap[x]]) {
      swap(heap[x], heap[c]);
      swap(pos[heap[x]], pos[heap[c]]);
      x = c;
    } else break;
  }
}

void heap_push(int i) {
  pos[heap[hn] = i] = hn;
  heap_up(hn++);
}

void heap_remove(int i) {
  int x = pos[i];
  pos[heap[x] = heap[--hn]] = x;
  heap_down(x);
}

int heap_top() {
  return heap[0];
}

int heap_pop() {
  int t;
  heap_remove(t = heap[0]);
  return t;
}
\end{verbatim}

    \subsubsection{treap}
\begin{verbatim}
#include <stdio.h>
#include <stdlib.h>
#include <string.h>
#include <algorithm>
#include <ctype.h>
using namespace std;

typedef int T;

const int N = 1000010;

struct node {
  T d; 
  int t, s;
  node *p, *l, *r;
} nodes[N], *next_alloc;

void init() {
  next_alloc = nodes;
}

node *node_new(T d) {
  node *x = next_alloc++;
  x->d = d;
  x->l = x->r = NULL;
  x->t = rand();
  x->s = 1;
  return x;
}

inline int size(node *x) {
  return x ? x->s : 0;
}

node *root(node *x) {
  while (x->p) x = x->p;
  return x;
}

void add_size(node *x, int d) {
  for ( ; x; x = x->p)
    x->s += d;
}

void rotate(node *x) {
  node *y = x->p;
  node *g = y->p;

  x->p = g;
  if (g) (g->l == y ? g->l : g->r) = x;
  y->p = x;

  if (x == y->l) {
    y->l = x->r;
    if (x->r) x->r->p = y;
    x->r = y;
  } else {
    y->r = x->l;
    if (x->l) x->l->p = y;
    x->l = y;
  }

  y->s = size(y->l) + size(y->r) + 1;
  x->s = size(x->l) + size(x->r) + 1;
}

void adjust(node *x) {
  while (x->p && x->t < x->p->t)
    rotate(x);
}

/* return new root
 * duplicated elements are not allowed */
node *_insert(node *p, node *x) {
  if (!p)
    return x;

  while (1) {
    if (x->d < p->d) {
      if (!p->l) {
        add_size(x->p = p, 1);
        adjust(p->l = x);
        break;
      }
      p = p->l;
    } else if (x->d > p->d) {
      if (!p->r) {
        add_size(x->p = p, 1);
        adjust(p->r = x);
        break;
      }
      p = p->r;
    } else break;
  }
  return root(p);
}

/* return new root */
node *insert(node *p, T d) {
  return _insert(p, node_new(d));
}

node *kth(node *p, int k) {
  while (p) {
    if (k == size(p->l)) 
      return p;
    if (k < size(p->l)) 
      p = p->l;
    else 
      k -= size(p->l) + 1, p = p->r;
  }
  return NULL;
}

/* # nodes with value < d */
int count(node *x, T d) {
  int res = 0;
  while (x) {
    if (d <= x->d)
      x = x->l;
    else
      res += size(x->l) + 1, x = x->r;
  }
  return res;
}

node *find(node *p, T d) {
  while (p) {
    if (d == p->d) return p;
    p = (d < p->d ? p->l : p->r);
  }
  return NULL;
}

node *lower_bound(node *x, T d) {
  node *y;
  if (!x)
    return NULL;
  if (d <= x->d && (y = lower_bound(x->l, d))) 
    return y;
  if (d <= x->d)
    return x;
  return lower_bound(x->r, d);
}

node *upper_bound(node *x, T d) {
  node *y;
  if (!x)
    return NULL;
  if (d < x->d && (y = upper_bound(x->l, d))) 
    return y;
  if (d < x->d)
    return x;
  return upper_bound(x->r, d);
}

node *next(node *x) {
  if (x->r) {
    x = x->r;
    while (x->l) x = x->l;
    return x;
  } else {
    while (x->p && x->p->r == x)
      x = x->p;
    return x->p;
  }
}

node *prev(node *x) {
  if (x->l) {
    x = x->l;
    while (x->r) x = x->r;
    return x;
  } else {
    while (x->p && x->p->l == x)
      x = x->p;
    return x->p;
  }
}

/* return new root */
node *erase(node *x) {
  while (x->l || x->r) {
    if (x->l && (!x->r || x->l->t < x->r->t))
      rotate(x->l);
    else
      rotate(x->r);
  }

  node *y = x->p;
  if (y) (y->l == x ? y->l : y->r) = NULL, add_size(y, -1);
  x->p = NULL;
  return y ? root(y) : NULL;
}

node *__merge(node *x, node *y) {
  node *l = y->l, *r = y->r;
  y->l = y->r = NULL, y->s = 1;

  if (r) x = __merge(x, r);
  x = _insert(x, y);
  if (l) x = __merge(x, l);
  return x;
}

/* return new root */
node *merge(node *x, node *y) {
  if (size(x) < size(y)) swap(x, y);
  if (!y) return x;
  return __merge(x, y);
}

void dfs(node *x) {
  putchar('(');
  if (x) {
    if (x->l) { dfs(x->l); putchar(' '); }
    printf("%d", x->d);
    if (x->r) { putchar(' '); dfs(x->r); }
  }
  putchar(')');
}
\end{verbatim}

	\subsubsection{treap}
\begin{verbatim}
struct Tree{
    #define maxn 10000
    #define B 17
    int n;
    int c[maxn];
    Tree(){Cls(c);}
    void add(int x,int d){ for(x++; x <= n; x += x & -x) c[x] += d;}
    int sum(int x) {
        int ret = 0;
        for (x++; x; x -= x & -x) ret += c[x];
        return ret;
    }
    int kth(int k) {
        int ret = 0;
        for (int i = B; i >= 0; i--) {
            ret += 1 << i;
            if (ret > n || c[ret] > k) ret -= 1 << i;
            else k -= c[ret];
        }
        return ret;
    }
};
\end{verbatim}

    \subsubsection{BST}
\begin{verbatim}
\*
BST和Treap基本一样,Treap是用Max-heap作为平衡条件,BST用子树节点个数作为平衡条件
一般情况下BST比Treap稍微平衡一点
*\
#include <cstdio>
#include <algorithm>
const int INF=0x7FFFFFFF;
const int kMaxN=1000001;

int n,total,head;
int key[kMaxN],size[kMaxN],le[kMaxN],ri[kMaxN];

void init(){
    scanf("%d\n",&n);
}

void RightRotate(int &head){
    int tmp=le[head];
    le[head]=ri[tmp];
    ri[tmp]=head;
    size[tmp]=size[head];
    size[head]=size[le[head]]+size[ri[head]]+1;
    head=tmp;
}

void LeftRotate(int &head){
    int tmp=ri[head];
    ri[head]=le[tmp];
    le[tmp]=head;
    size[tmp]=size[head];
    size[head]=size[le[head]]+size[ri[head]]+1;
    head=tmp;
}

void insert(int &head, int number){
    if (head==0){
        ++total;
        key[total]=number;
        size[total]=1;
        le[total]=ri[total]=0;
        head=total;
    } else{
        ++size[head];
        if (key[head]>number){
            insert(le[head],number);
            if (size[le[le[head]]]>size[ri[head]] || size[ri[le[head]]]>size[ri[head]])
                RightRotate(head);
        } else{
            insert(ri[head],number);
            if (size[le[ri[head]]]>size[le[head]] || size[ri[ri[head]]]>size[le[head]])
                LeftRotate(head);
        }
    }
}

void del(int &head, int number){
    --size[head];
    if (key[head]>number) del(le[head],number); else
    if (key[head]<number) del(ri[head],number); else{
        if (size[head]==0){
            head=0;
            return;
        }
        if (size[le[head]]>size[ri[head]]){
            RightRotate(head);
            del(ri[head],number);
        } else{
            LeftRotate(head);
            del(le[head],number);
        }
    }
}

void solve(){
    char ch;
    int number;
    head=total=0;
    for (int i=0; i<n; i++){
        scanf("%c %d\n",&ch,&number);
        if (ch=='i'){
            insert(head,number);
        } else
        if (ch=='d'){
            del(head,number);
        }
    }
    printf("%d\n",size[head]);
}

int main(){
    init();
    solve();
    return 0;
}
\end{verbatim}

	\subsubsection{splay}
\begin{verbatim}
#include<cstdio>
#include<iostream>
#include<algorithm>
using namespace std;
const int MAX_N = 50000 + 10;
const int INF = ~0U >> 1;
struct Node {
    Node*ch[2], *p;
    int size, val, mx;
    int add;
    bool rev;
    Node() {
        size = 0;
        val = mx = -INF;
        add = 0;
    }
    bool d() {
        return this == p->ch[1];
    }
    void setc(Node*c, int d) {
        ch[d] = c;
        c->p = this;
    }
    void addIt(int ad) {
        add += ad;
        mx += ad;
        val += ad;
    }
    void revIt() {
        rev ^= 1;
    }
    void relax();
    void upd() {
        size = ch[0]->size + ch[1]->size + 1;
        mx = max(val, max(ch[0]->mx, ch[1]->mx));
    }
} Tnull, *null = &Tnull;
Node mem[MAX_N], *C = mem;

void Node::relax() {
    if (add != 0) {
        for (int i = 0; i < 2; ++i) {
            if (ch[i] != null)
                ch[i]->addIt(add);
        }
        add = 0;
    }
    if (rev) {
        swap(ch[0], ch[1]);
        for (int i = 0; i < 2; ++i) {
            if (ch[i] != null)
                ch[i]->revIt();
        }
        rev = 0;
    }
}

Node*make(int v) {
    C->ch[0] = C->ch[1] = null;
    C->size = 1;
    C->val = v;
    C->mx = v;
    C->add = 0;
    C->rev = 0;
    return C++;
}

Node*build(int l, int r) {
    if (l >= r)
        return null;
    int m = (l + r) >> 1;
    Node*t = make(0);
    t->setc(build(l, m), 0);
    t->setc(build(m + 1, r), 1);
    t->upd();
    return t;
}

Node*root;

Node*rot(Node*t) {
    Node*p = t->p;
    p->relax();
    t->relax();
    int d = t->d();
    p->p->setc(t, p->d());
    p->setc(t->ch[!d], d);
    t->setc(p, !d);
    p->upd();
    if (p == root)
        root = t;
}

void splay(Node*t, Node*f = null) {
    while (t->p != f) {
        if (t->p->p == f)
            rot(t);
        else
            t->d() == t->p->d() ? (rot(t->p), rot(t)) : (rot(t), rot(t));
    }
    t->upd();
}

Node* select(int k) {
    for (Node*t = root;;) {
        t->relax();
        int c = t->ch[0]->size;
        if (k == c)
            return t;
        if (k > c)
            k -= c + 1, t = t->ch[1];
        else
            t = t->ch[0];
    }
}

Node*&get(int l, int r) { //[l,r)
    Node*L = select(l - 1);
    Node*R = select(r);
    splay(L);
    splay(R, L);
    return R->ch[0];
}

int n, m;

int main() {
    cin >> n >> m;
    root = build(0, n + 2);
    root->p = null;
    for (int i = 0; i < m; ++i) {
        int k, l, r, v;
        scanf("%d%d%d", &k, &l, &r);
        Node*&t = get(l, r + 1);
        if (k == 1) {
            scanf("%d", &v);
            t->addIt(v);
            splay(t);
        } else if (k == 2) {
            t->revIt();
            splay(t);
        } else {
            printf("%d\n", t->mx);
        }
    }
}
\end{verbatim}

	%\input{code/data_structure/zkw_tree}

	\subsection{图上的数据结构}
	\subsubsection{左偏树}
\begin{verbatim}
const int MAXN = 2000; //number of nodes

struct node {
	int key, npl, parent, left, right;
};

node heap[MAXN];

void init() {
	heap[0].npl = -1;
}

int merge(int a, int b) {
	if (!a) return b;
	if (!b) return a;
	if (heap[a].key > heap[b].key) swap(a, b);
	heap[a].right = merge(heap[a].right, b);
	heap[b].parent = a;
	if (heap[heap[a].right].npl > heap[heap[a].left].npl) swap(heap[a].left, heap[a].right);
	heap[a].npl = heap[heap[a].right].npl + 1;
	return a;
}

void remove(int x) {
	int q = heap[x].parent, p = merge(heap[x].left, heap[x].right);
	heap[p].parent = q;
	if (q && heap[q].left == x) heap[q].left = p;
	if (q && heap[q].right == x) heap[q].right = p;
	while (q) {
		if (heap[heap[q].left].npl < heap[heap[q].right].npl) swap(heap[q].left, heap[q].right);
		if (heap[heap[q].right].npl + 1 == heap[q].npl) break;
		heap[q].npl = heap[heap[q].right].npl + 1;
		p = q; q = heap[q].parent;
	}
}
\end{verbatim}

    \subsubsection{动态树}
\begin{verbatim}
/*
    You are given a tree with N nodes.
    The tree’s nodes are numbered 1 through N and its edges are numbered 1 through N - 1.
    Each edge is associated with a weight. Then you are to execute a series of instructions on the tree.
    The instructions can be one of the following forms:
        CHANGE i v Change the weight of the ith edge to v 
        NEGATE a b Negate the weight of every edge on the path from a to b 
        QUERY a b Find the maximum weight of edges on the path from a to b 
*/
#include <cstdio>
#include <cstring>
#include <algorithm>

using namespace std;

const int MAXN = 10010;
const int INF = 2000000000;

struct Node {
    int child[2]; 
    int parent, typ, value, maxvalue, minvalue;
    bool neg;
}; 

struct Tedge {
    int v, w, next;
};

Tedge edge[MAXN * 2];
Node node[MAXN];
int first[MAXN], bottom[MAXN];
bool used[MAXN];
int N;

inline int cal_max(int x) {
    if (!x) return -INF;
    if (node[x].neg) return -node[x].minvalue;
    else return node[x].maxvalue;
}

inline int cal_min(int x) {
    if (!x) return INF;
    if (node[x].neg) return -node[x].maxvalue;
    else return node[x].minvalue;
}

inline void update(int x) {
    if (node[x].neg) {
        node[x].value = -node[x].value;
        swap(node[x].maxvalue, node[x].minvalue);
        node[x].maxvalue = -node[x].maxvalue; node[x].minvalue = -node[x].minvalue;
        node[node[x].child[0]].neg = !node[node[x].child[0]].neg;
        node[node[x].child[1]].neg = !node[node[x].child[1]].neg;
        node[x].neg = 0;
    }
    if (x > 1) node[x].maxvalue = node[x].value;
    else node[x].maxvalue = -INF;
    node[x].maxvalue = max(node[x].maxvalue, cal_max(node[x].child[0]));
    node[x].maxvalue = max(node[x].maxvalue, cal_max(node[x].child[1]));
    if (x > 1) node[x].minvalue = node[x].value;
    else node[x].minvalue = INF;
    node[x].minvalue = min(node[x].minvalue, cal_min(node[x].child[0]));
    node[x].minvalue = min(node[x].minvalue, cal_min(node[x].child[1]));
}

void cal(int x) {
    if (node[x].parent == 0) return;
    cal(node[x].parent);
    update(x);
}

void rotate(int x, int a) {
    int y = node[x].parent, b = node[y].typ;
    node[x].parent = node[y].parent; node[x].typ = b;
    if (b != 2) node[node[y].parent].child[b] = x;
    b = 1 - a;
    node[node[x].child[b]].parent = y; node[node[x].child[b]].typ = a;
    node[y].child[a] = node[x].child[b]; node[y].parent = x; node[y].typ = b;
    node[x].child[b] = y;
    update(y);
} 

void splay(int x) {
    cal(x);
    while (1) {
        int a = node[x].typ;
        if (a == 2) break;
        int y = node[x].parent, b = node[y].typ;
        if (a == b) rotate(y, a);
        else rotate(x, a);
        if (b == 2) break;
        rotate(x, b);
    }
    update(x);
} 

void expose(int v) {
    int u = v, w = 0;
    while (u) {
        splay(u);
        node[node[u].child[0]].typ = 2;
        node[u].child[0] = w;
        node[w].typ = 0;
        update(u);
        w = u; u = node[u].parent;
    }
    splay(v);
} 

inline void link(int v, int w) {
    expose(v);
    expose(w);
    node[v].parent = w;
}

void DFS(int u) {
    used[u] = 1;
    for (int i = first[u]; i != -1; i = edge[i].next) {
        int v = edge[i].v, w = edge[i].w;
        if (!used[v]) {
            node[v].value = node[v].maxvalue = node[v].minvalue = w;
            bottom[i >> 1] = v;
            link(v, u);
            DFS(v);
        }
    }
}

void readTrees() {
    memset(first, -1, sizeof(first));
    scanf("%d", &N);
    for (int i = 0; i < N - 1; ++i) {
        int u, v, w;
        scanf("%d%d%d", &u, &v, &w);
        edge[i * 2].v = v; edge[i * 2].w = w; edge[i * 2].next = first[u]; first[u] = i * 2;
        edge[i * 2 + 1].v = u; edge[i * 2 + 1].w = w; edge[i * 2 + 1].next = first[v]; first[v] = i * 2 + 1;
    }
    for (int i = 0; i <= N; ++i) {
        node[i].child[0] = node[i].child[1] = 0; node[i].typ = 2;
        node[i].parent = 0; node[i].neg = 0;
    }
    node[0].maxvalue = -INF; node[0].minvalue = INF;
    node[1].maxvalue = -INF; node[1].minvalue = INF;
    memset(used, 0, sizeof(used));
    DFS(1);
}

int query(int u, int v) {
    if (u == v) return 0;
    expose(u);
    int x = v, w = 0, ret = -INF;
    while (v) {
        splay(v);
        if (node[v].parent == 0) {
            ret = max(ret, cal_max(node[v].child[0]));
            ret = max(ret, cal_max(w));
        }
        node[node[v].child[0]].typ = 2;
        node[v].child[0] = w;
        node[w].typ = 0;
        update(v);
        w = v; v = node[v].parent;
    }
    splay(x);
    return ret;
}

void negate(int u, int v) {
    expose(u);
    int x = v, w = 0;
    while (v) {
        splay(v);
        if (node[v].parent == 0) {
            node[node[v].child[0]].neg = !node[node[v].child[0]].neg;
            node[w].neg = !node[w].neg;
        }
        node[node[v].child[0]].typ = 2;
        node[v].child[0] = w;
        node[w].typ = 0;
        update(v);
        w = v; v = node[v].parent;
    }
    splay(x);
}

void solve() {
    char cmd[10];
    while (scanf("%s", cmd), cmd[0] != 'D') {
        int x, y;
        scanf("%d%d", &x, &y);
        
        if (cmd[0] == 'Q') printf("%d\n", query(x, y));
        else if (cmd[0] == 'C') {
            x = bottom[x - 1];
            splay(x);
            node[x].value = y; node[x].neg = 0;
            update(x);
        }
        else negate(x, y);
    }
}

int main() {
    int t;
    for (scanf("%d", &t); t > 0; --t) {
        readTrees();
        solve();
    }
    return 0;
}
\end{verbatim}

    \subsubsection{支持子树操作的动态树}
\begin{verbatim}
const int N=333333;

int n;
int pre[N],fa[N],fat[N],val[N],ma[N],ch[N][2];
bool black[N];
multiset<int> Q[N];
VI E[N];

inline void up(int x){ma[x]=max(max(val[x],*Q[x].rbegin()),max(ma[lch],ma[rch]));}

inline void rot(int id,int tp){
    static int k;
    k=pre[id];
    ch[k][tp^1]=ch[id][tp];
    if(ch[id][tp]) pre[ch[id][tp]]=k;
    if(pre[k]) ch[pre[k]][k==ch[pre[k]][1]]=id;
    pre[id]=pre[k];
    ch[id][tp]=k;
    pre[k]=id;
    up(k);
}

inline void splay(int x){
    if (!pre[x]) return;
    int tmp;
    for(tmp=x;pre[tmp];tmp=pre[tmp]);
    for(swap(fa[x],fa[tmp]);pre[x];rot(x,x==ch[pre[x]][0]));
    up(x);
}

inline int access(int x){
    int nt;
    for(nt=0;x;x=fa[x]){
        splay(x);
        if (rch){
            fa[rch]=x;
            pre[rch]=0;
            Q[x].insert(ma[rch]);
        }
        rch=nt;
        if (nt){
            fa[nt]=0;
            pre[nt]=x;
            Q[x].erase(Q[x].find(ma[nt]));
        }
        up(nt=x);
    }
    return nt;
}

void make(int x,int f){
    fat[x]=f;
    rep(i,E[x].size()) if (E[x][i]!=f) make(E[x][i],x);
    int t;
    up(x+n);up(x+2*n);
    fa[t=x+(1+black[x])*n]=x;
    Q[x].insert(*Q[t].rbegin());
    up(x);
    fa[x]=t=f+(1+black[x])*n;
    Q[t].insert(ma[x]);
}

void cut(int x,int f){
    access(f);
    splay(f);
    splay(x);
    Q[f].erase(Q[f].find(ma[x]));
    fa[x]=0;
    up(f);
}

void link(int x,int f){
    access(f);
    splay(f);
    splay(x);
    fa[x]=f;
    Q[f].insert(ma[x]);
    up(f);
}

int main(){
    while(~scanf("%d",&n)){
        Cls(pre);
        Cls(ch);
        Cls(fa);
        rep(i,n+1) E[i].clear();
        rep(i,n-1){
            int x,y;
            scanf("%d%d",&x,&y);
            E[x].PB(y);
            E[y].PB(x);
        }
        n++;
        rep(i,3*n+1) Q[i].clear();
        rep(i,3*n+1) ma[i]=val[i]=inf,Q[i].insert(inf);
        REP(i,1,n) scanf("%d",black+i);
        REP(i,1,n) scanf("%d",val+i);
        make(1,n);
        int q,k,x;
        scanf("%d",&q);
        rep(i,q){
            scanf("%d%d",&k,&x);
            if (k==0){
                for(x=access(x);lch;x=lch);
                splay(x);
                printf("%d\n",ma[rch]);
            }
            if (k==1){
                cut(x,fat[x]+(1+black[x])*n);
                cut(x+(1+black[x])*n,x);
                black[x]^=1;
                link(x+(1+black[x])*n,x);
                link(x,fat[x]+(1+black[x])*n);
            }
            if (k==2){
                access(x);
                splay(x);
                scanf("%d",val+x);
                up(x);
            }
        }
    }
    return 0;
}
\end{verbatim}


	\subsection{可持久化数据结构}
	\subsubsection{可持久化线段树}
持久化数据结构最重要的思想是无论询问还是插入操作,都不修改原来的数据,而是新建立节点。
~\\
~\\
\Large{题目描述} \par
\small
1. C l r d: Adding a constant d for every $A_i(l\leq i\leq r)$, and increase the time $t$ add by 1, this is the only operation that will cause the time increase. \par
2. Q l r: Querying the current sum of $A_i(l \leq i \leq r)$. \par
3. H l r t: Querying a history sum of $A_i(l \leq i\leq r)$ in time $t$. \par
4. B t: Back to time $t$. And once you decide return to a past, you can never be access to a forward edition anymore.\par
~\\
\Large{程序} \par
\small
\begin{verbatim}
#include <cmath>
#include <cstdio>
#include <cstring>
#include <vector>
#include <iostream>
#include <algorithm>
using namespace std;

const int kMaxN=100001;
const int MAXLIMIT=500000;

struct data{
    long long add;
    long long sum;
    data *lt,*rt;
    data(){}
    data(data *lt, data *rt, long long sum, int add=0):lt(lt),rt(rt),sum(sum),add(add){}
    inline long long getSum(int len){
        if (!this) return 0;
        return sum+add*len;
    }
};

data buffer[MAXLIMIT];
data* tree[kMaxN];
int n,m,total,number;
int record[kMaxN];

data *build(int l, int r){
    if (l==r){
        scanf("%d",&number);
        data *node=&buffer[total++];
        node->lt=node->rt=NULL;
        node->sum=number; node->add=0; 
        return node;
    }

    int mid=(l+r)/2;
    data *node=&buffer[total++];
    node->lt=build(l,mid);
    node->rt=build(mid+1,r);
    node->sum=node->lt->getSum(mid-l+1)+
        node->rt->getSum(r-mid);

    node->add=0;
    return node;
}

data *insert(int l, int r, data *node, int L, int R, long long Add, int add){
    if (L>r || l>R){
        if (Add!=0){
            buffer[total]=data(*node);
            buffer[total].add+=Add;
            return &buffer[total++];
        }
        return node;
    }
    if (L<=l && r<=R){
        buffer[total]=data(*node);
        buffer[total].add+=Add+add;
        return &buffer[total++];
    }

    int mid=(l+r)/2;
    data *lt=insert(l,mid,node->lt,L,R,
        Add+node->add,add);

    data *rt=insert(mid+1,r,node->rt,L,R,
        Add+node->add,add);

    long long sum=lt->getSum(mid-l+1)+
        rt->getSum(r-mid);

    buffer[total++]=data(lt,rt,sum,0);
    return &buffer[total-1];
}

long long ask(int l, int r, data *node, int L, int R, long long Add){
    if (L>r || l>R) return 0;
    if (L<=l && r<=R){
        return node->getSum(r-l+1)+Add*(r-l+1);
    }

    int mid=(l+r)/2;
    long long lt=ask(l,mid,node->lt,L,R,
        Add+node->add);

    long long rt=ask(mid+1,r,node->rt,L,R,
        Add+node->add);

    return lt+rt;
}


void init(){
    total=0;
    tree[0]=build(1,n);
}

void solve(){
    char opt[2];
    int L,R,add;
    int cur=0,Time;
    record[cur]=total;
    for (int i=0; i<m; i++){
        scanf("%s",opt);
        if (opt[0]=='C'){
            scanf("%d%d%d",&L,&R,&add);
            tree[cur+1]=insert(1,n,tree[cur],L,R,0,add);
            record[++cur]=total;
        } else
        if (opt[0]=='H'){
            scanf("%d%d%d",&L,&R,&Time);
            long long temp=ask(1,n,tree[Time],L,R,0);
            printf("%I64d\n",temp);
        } else
        if (opt[0]=='Q'){
            scanf("%d%d",&L,&R);
            long long temp=ask(1,n,tree[cur],L,R,0);
            printf("%I64d\n",temp);
        } else{
            scanf("%d",&cur);
            total=record[cur];
        }
    }
}

int main(){
    while (scanf("%d%d",&n,&m)!=EOF){
        init();
        solve();
    }
    return 0;
}
\end{verbatim}

	\subsubsection{函数式treap}
\begin{verbatim}
//By Lin
#include<cstdio>
#include<cstring>
#include<cstdlib>
using namespace std;

struct Node{
    int key,weight,size;
    Node *l,*r;
    Node(int _key , int _weight, Node *_l, Node* _r):
        key(_key),weight(_weight),l(_l),r(_r){
            size = 1;
            if ( l ) size += l->size;
            if ( r ) size += r->size;
        }
    Node *newnode(int key){
        return new Node(key,rand(),NULL,NULL);
    }
    inline int lsize(){ return l?l->size:0; }
    inline int rsize(){ return r?r->size:0; }
}*root[50005];

Node* Meger(Node *a , Node *b ){
    if ( !a || !b ) return a?a:b;
    return a->weight>b->weight?
         new Node(a->key,a->weight,a->l,Meger(a->r,b)):
         new Node(b->key,b->weight,Meger(a,b->l),b->r);
}

Node* Split_L(Node *a ,int size ){
    if ( !a || size == 0 ) return NULL;
    return a->lsize() < size?
        new Node(a->key,a->weight,a->l,Split_L(a->r,size-1-a->lsize())):
        Split_L(a->l,size);
}

Node* Split_R(Node *a ,int size ){
    if ( !a || size == 0 ) return NULL;
    return a->rsize() < size?
        new Node(a->key,a->weight,Split_R(a->l,size-1-a->rsize()),a->r):
        Split_R(a->r,size);
}

int Ask( Node *a ,int k ){
    if ( a->lsize() >= k ) return Ask(a->l,k);
    k -= a->lsize()+1;
    if ( k == 0 ) return a->key;
    return Ask(a->r,k);
}

int len = 0;

int main(){
    int d = 0 , cas;
    scanf("%d", &cas);
    root[0] = NULL;
    int cnt = 1,kind,v,p,c;
    char s[1005];
    while ( cas -- ) {
        scanf("%d", &kind );
        if ( kind == 1 ) {
            scanf("%d%s", &p , s );
            p-=d;
            Node *l = Split_L(root[cnt-1],p),
                 *r = Split_R(root[cnt-1],len-p);
            for (int i = 0; s[i]; i++ ){
                l = Meger(l,new Node(s[i],rand(),NULL,NULL));
                len++;
            }
            root[cnt++] = Meger(l,r);
        }
        else if ( kind == 2){
            scanf("%d%d", &p , &c );
            p-=d,c-=d;
            Node *l = Split_L(root[cnt-1],p-1),
                 *r = Split_R(root[cnt-1],len-p-c+1);
            len -= c;
            root[cnt++] = Meger(l,r);
        }
        else{
            scanf("%d%d%d", &v, &p , &c );
            v-=d,p-=d,c-=d;
            char ch;
            for (int i = p; i<p+c; i++) {
                printf("%c", ch = Ask(root[v],i) );
                if ( ch == 'c' ) d++;
            }
            puts("");
        }
    }
    return 0;
}
\end{verbatim}

    %\subsubsection{划分树}
\begin{verbatim}
const int D = 18;
const int N = 100000 + 1000;

struct Tree{
	int n;
	int v[N];
	int val[D][N], to_left[D][N];
	LL sum_l[D][N];

	void build(int l, int r, int deep){
		if (l == r) return;
		int mid = (l + r) / 2, left_same = mid - l + 1;
		for(int i = l; i <= r; i++){
			if (val[deep][i] < v[mid]) left_same--;
		}
		int le = l, ri = mid + 1, same = 0;
		sum_l[deep][0] = 0;
		for(int i = l; i <= r; i++){
			to_left[deep][i] = (i == l) ? 0 : to_left[deep][i - 1];
			sum_l[deep][i] = sum_l[deep][i - 1];
			if (val[deep][i] < v[mid]){
				to_left[deep][i]++;
				sum_l[deep][i] += val[deep][i];
				val[deep + 1][le++] = val[deep][i];
			}
			else if (val[deep][i] > v[mid]){
				val[deep + 1][ri++] = val[deep][i];
			}
			else if (same < left_same){
				to_left[deep][i]++;
				sum_l[deep][i] += val[deep][i];
				val[deep + 1][le++] = val[deep][i];
				same++;
			}
			else{
				val[deep + 1][ri++] = val[deep][i];
			}
		}
		build(l, mid, deep + 1);
		build(mid + 1, r, deep + 1);
	}

	pair<int, LL> ask(int ask_l, int ask_r, int l, int r, int deep, int kth){ 
		if (l == r) return MP(val[deep][l], val[deep][l]);
		int mid = (l + r) / 2, s1, s2;
		if (l == ask_l){
			s1 = 0;
			s2 = to_left[deep][ask_r];
		}
		else{
			s1 = to_left[deep][ask_l - 1];
			s2 = to_left[deep][ask_r];
		}
		if (s2 - s1 >= kth){
			ask_l = l + s1, ask_r = ask_l + s2 - s1 - 1;
			return ask(ask_l, ask_r, l, mid, deep + 1, kth);
		}
		else{
			LL ret = sum_l[deep][ask_r] - sum_l[deep][ask_l - 1];
			kth = kth - (s2 - s1);
			s2 = ask_r - ask_l + 1 - (s2 - s1);
			ask_l = mid + ask_l - l + 1 - s1;
			ask_r = ask_l + s2 - 1;
			pair<int, LL> tmp = ask(ask_l, ask_r, mid + 1, r, deep + 1, kth);
			return MP(tmp.F, tmp.S + ret);
		}
	}

	void init(int n, int other[]){
		for(int i = 1; i <= n; i++){
			v[i] = other[i];
			val[0][i] = v[i];
		}
		sort(v + 1, v + n + 1);
		build(1, n, 0);
	}
}tree;
\end{verbatim}


\section{\LARGE 字符串算法}
	%\subsubsection{字符串循环节}
对于一个长度为$n$的字符串,枚举$n$的因子$d$,然后判断$d$是否为循环节。只要字符串st[1..n-d]和st[d+1..n]相等,则$d$为循环节,这一个判断可以用hash来完成,时间复杂度为O(1)。根据循环节判断的性质,求出KMP的next数组之后(假设下标从1开始),那么这个字符串的最小循环长度为$n-next[n]$。\par
对于字符串的循环节还有这样一个规律:如果$\frac{n}{i}$和$\frac{n}{j}$是循环节,且$i,j$互质,那么$\frac{n}{i*j}$也是循环节。\par

    %\subsubsection{hash}
\begin{verbatim}
const int maxn = 1000000 + 10;

char s[maxn * 2];
int n;

int MinRepresentation() {
	n = strlen(s);
	for (int i = 0; i < n; ++i) s[n + i] = s[i];

	int i = 0, j = 1;
	while (i < n && j < n) {
		int k = 0;
		while (k <= n && s[i + k] == s[j + k]) ++k;
		if (k > n) break;
		if (s[i + k] < s[j + k]) j += k + 1;
		else i += k + 1;
		if (i == j) ++j;
	}
	return (i < j) ? i : j;
}
\end{verbatim}

    %\subsubsection{kmp}
\begin{verbatim}
#include <cstdio>
#include <cstring>
#include <algorithm>
using namespace std;

const int N = 1000010;

int next[N];

int kmp(char *s, int n, char *t, int m) {
  int i, j;
  next[0] = -1;
  i = 0; j = -1;
  while (i < m) {
    if (j == -1 || t[i] == t[j]) {
      i++; j++;
      next[i] = (t[i] == t[j] ? next[j] : j);
    } else {
      j = next[j];
    }
  }
  i = j = 0;
  while (i < n && j < m) {
    if (j == -1 || s[i] == t[j]) {
      i++; j++;
    } else {
      j = next[j];
    }
  }
  return (j >= m ? i - m : -1);
}
\end{verbatim}

    %\subsubsection{manacher求最长回文}
\begin{verbatim}
#include <stdio.h>
#include <string.h>
#include <algorithm>
using namespace std;

void manacher(char a[], int n, int rad[]) {
    int i, j, k;
    rad[0] = rad[n] = 0;
    for (i = 1, j = 0; i < n; i += k, j = max(0, j - k)) {
        while (i > j && i + j < n && a[i - j - 1] == a[i + j]) j++;
        for (rad[i] = j, k = 1; k <= rad[i] && rad[i - k] != rad[i] - k; k++)
        rad[i + k] = min(rad[i - k], rad[i] - k);
    }
}

const int N = 500010;
int n;
char a[N];
int rad[N];

int main() {
    scanf("%d %s", &n, a);
    manacher(a, n, rad);
    return 0;
}
\end{verbatim}

    %\input{code/string/suffix}
	%\input{code/string/suffixautomata}
    %\input{code/string/trie_tree}
    %\subsubsection{Trie图}
\begin{verbatim}
/*
trie图用于多模式串匹配,一般的题目都是要求生成的字符串中不包含一些字符串
建立trie图,然后从根节点开始走,只要不走到危险节点就可以了
注意:trie图的2个性质:1、一个节点的危险性和它后缀节点相同 2、根节点的直接儿子的后缀是根节点
*/
#include <string>
#include <queue>
#include <cstdio>
#include <cstring>
#include <iostream>
using namespace std;

class Trie{
    public:
        int tot;
        const static int kMaxN=100;

    public:
        char ch[kMaxN];
        bool danger[kMaxN];
        int g[kMaxN],next[kMaxN],suffix[kMaxN];

    public:
        Trie(){
            tot=1;
            memset(g,255,sizeof(g));
            memset(danger,0,sizeof(danger));
        }
        int getson(int node, char c, bool flag);
        void insert(string st);
        void makegraph();
};

int Trie::getson(int node, char c, bool flag){
    while (true){
        int now=g[node];
        while (now!=-1 && ch[now]!=c) now=next[now];
        if (now!=-1 || !flag) return now;
        if (node==1) return node;
        node=suffix[node];
    }
}

void Trie::insert(string st){
    int len=st.size();
    int now=1,son;
    for (int i=0; i<len; i++){
        son=getson(now,st[i],false);
        if (son==-1){
            ++tot;
            ch[tot]=st[i];
            next[tot]=g[now];
            g[now]=tot;
            now=tot;
        } else now=son;
    }
    danger[now]=true;
}

void Trie::makegraph(){
    queue<int> q;
    q.push(1);
    while (!q.empty()){
        int node=q.front(); q.pop();
        for (int p=g[node]; p!=-1; p=next[p]){
            q.push(p);
            if (node==1){
                suffix[p]=1;
            } else{
                suffix[p]=getson(suffix[node],ch[p],true);
            }
            danger[p]|=danger[suffix[p]];
        }
    }
}

int main(){
    string st;
    Trie graph;
    cin>>st;
    graph.insert(st);
    cin>>st;
    graph.insert(st);
    return 0;
}

\end{verbatim} 


\section{\LARGE 计算几何}
	\subsection{基础}
    %\subsubsection{几何}
\begin{verbatim}
#include <cstdio>
#include <cstdlib>
#include <cstring>
#include <algorithm>
#include <cmath>

#define Sqr(x) (x) * (x)
#define sign(x) ((x < -EPS) ? -1 : x > EPS)

using namespace std;

const double EPS = 1E-8;
const double INF = 1E+9;
const double PI = acos(-1.0);

typedef complex<double> Point;

bool operator<(const Point &a, const Point &b){
    int f = sign(a.X - b.X);
    if (f) return f < 0;
    return sign(a.Y - b.Y) < 0;
}

double cross(Point a, Point b){
    return a.X * b.Y - a.Y * b.X;
}

double cross(Point a, Point b, Point c){
    return cross(b - a, c - a);
}

double dot(Point a, Point b){
    return a.X * b.X + a.Y * b.Y;
}

double dot(Point a, Point b, Point c){
    return dot(b - a, c - a);
}

double dist(Point a, Point b){
    return abs(a - b);
}

Point rotate(Point v, double alpha){
    double c = cos(alpha), s = sin(alpha);
    return Point(v.X * c - v.Y * s, v.X * s + v.Y * c);
}

double fix(double a, double b = 0) {
  a -= b;
  if (sign(a) < 0) a += 2 * pi;
  if (sign(a - 2 * pi) >= 0) a -= 2 * pi;
  return a;
}

double angle(Point a, Point b){
    return fix(arg(b - a));
}

Point centroid(Point a[], int n){
    double area = 0;
    Point c = Point(0, 0);
    a[n] = a[0];
    for(int i = 0; i < n; i++){
        area += cross(a[i], a[i + 1]);
        c.x += (a[i].X + a[i + 1].X) * cross(a[i], a[i + 1]);
        c.y += (a[i].Y + a[i + 1].Y) * cross(a[i], a[i + 1]);
    }
    area = fabs(area) / 2;
    c.x /= 6 * area;
    c.y /= 6 * area;
    return c;
}

/*
 * convex hull
 */

Point __o;

bool cmp_p(Point a, Point b){
    int f = sign(a.X - b.X);
    if (f) return f < 0;
    return sign(a.Y - b.Y) < 0;
}

bool cmp(Point a, Point b){
    int f = sign(cross(o, a, b));
    if (f) return f > 0;
    return sign(abs(a - o) - abs(b - o)) < 0;
}

/*
 * find convex hull of p[n] in place
 * return # of points of resulting convex hull
 */
Point stack[1111]
int find_convex(Point p[], int n){
    __o = *min_element(p, p + n, cmp_p);
    sort(p, p + n, cmp);
    int top = 0;
    rep(i, n){
        while(top >= 2 && sign(cross(stack[top - 2], stack[top - 1], p[i])) <= 0) top--;
        stack[top++] = p[i];
    }
    rep(i, top) p[i] = stack[i];
    return top;
}

/*
 * rotate calipers
 */
double shadow_length(double alpha, Point a, Point b){
    double dx = a.x - b.x;
    double dy = a.y - b.y;
    double c = cos(alpha);
    double s = sin(alpha);
    return fabs(dx * c + dy * s);
}

/*
 * min area & min peri rectangle covering, using rotate calipers
 */
void rotate_calipers(Point ps[], int n, double &area, double &peri){
    area = peri = INF;
    n = find_convex(ps, n);
    ps[n] = ps[0];
    Point *q[4] = {NULL, NULL, NULL, NULL};
    for(int i = 0; i < n; i++){
        Point *p = &ps[i];
        if (!q[0] || q[0]->Y > p->Y || q[0]->Y == p->Y && q[0]->X > p->X) q[0] = p;
        if (!q[1] || q[1]->X < p->X || q[1]->X == p->X && q[1]->Y > p->Y) q[1] = p;
        if (!q[2] || q[2]->Y < p->Y || q[2]->Y == p->Y && q[2]->X < p->X) q[2] = p;
        if (!q[3] || q[3]->X > p->X || q[3]->X == p->X && q[3]->Y < p->Y) q[3] = p;
    }
    double alpha = 0;
    for(int k = 0; k < n + 5; k++){
        int bi = -1;
        double gap_min = INF;
        for(int i = 0; i < 4; i++){
            double gap = fix(angle(q[i][0], q[i][1]), alpha + i * PI / 2);
            if (gap < gap_min){
                gap_min = gap;
                bi = i;
            }
        }
        if (++q[bi] == ps + n) q[bi] = ps + 0;
        alpha = fix(alpha + gap_min);
        double a = shadow_length(alpha + PI / 2, *q[0], *q[2]);
        double b = shadow_length(alpha, *q[1], *q[3]);
        area = min(area, a * b);
        peri = min(peri, a + a + b + b);
    }
}

/*
 * lines
 */
typedef pair<Point, Point> Line;

int parallel(Line a, Line b){
    return !sign(cross(a.Y - a.X, b.Y - b.X));
}

/*
 * same side : 1
 * at least one of a, b touches l : 0;;
 * other wise : -1
 */
int side(Line m, Point p, Point q){
    return sign(cross(m.X, m.Y, p)) * sign(cross(m.X, m.Y, q));
}

int on_line(Line l, Point p){
    return !sign(cross(l.p, l.q, p));
}

/*
 * u, v : line
 */
int coinside(Line u, Line v){
    return on_line(u, v.X) && on_line(u, v.Y);
}

/*
 * u, v : line segment, inclusive
 */
int intersected(Line u, Line v){
    return !parallel(u, v) && side(u, v.X, v.Y) <= 0 && side(v, u.X, u.Y) <= 0;
}

/*
 * u, v : line segment , exclusive
 */
int intersected_exclusive(Line u, Line v){
    return !parallel(u, v) && side(u, v.X, v.Y) < 0 && side(v, u.X, u.Y) < 0;
}

/*
 * intersection point
 * must check whether exist or not
 */
Point ip(Line u, Line v){
    double n = (u.p.y - v.p.y) * (v.q.x - v.p.x) - (u.p.x - v.p.x) * (v.q.y - v.p.y);
    double d = (u.q.x - u.p.x) * (v.q.y - v.p.y) - (u.q.y - u.p.y) * (v.q.x - v.p.x);
    double r = n / d;
    return Point(u.p.x + r * (u.q.x - u.p.x), u.p.y + r * (u.q.y - u.p.y));
}

bool  inter(Line a, Line b, Point &p){
    double s1 = cross(a.F, a.S, b.F);
    double s2 = cross(a.F, a.S, b.S);
    if (!sign(s1 - s2)) return false;
    p = (s1 * b.S - s2 * b.F) / (s1 - s2);
    return true;
}

/*
 * if P on the Line segment l, inclusive
 */
int on_lineseg(Line l, Point p){
    return on_line(l, p) && sign(dot(p, l.p, l.q)) <= 0;
}

/*
 * if P on the segment l, exclusive 
 */
int on_lineseg_exclusive(Line l, Point p){
    return on_line(l, p) && sign(dot(p, l.p, l.q)) < 0;
}

double dist_line_point(Line l, Point a){
    return fabs(cross(l.p, l.q, a)) / dist(l.p, l.q);
}

double dist_lineseg_point(Line l, Point a){
    if (on_lineseg(l, a)) return 0;
    if (on_line(l, a) || !sharp(l.p, a, l.q)) return min(dist(l.p, a), dist(l.q, a));
    return dist_line_point(l, a);
}

/*
 * u : line segment
 * ab : ray, if p is the resulting intersection point
 */
int intersected_lineseg_ray(Line u, Line v, Point &p){
    if (parallel(u, v)) return 0;
    p = ip(u, v);
    return on_lineseg(u, p) && (on_lineseg(v, p) || on_lineseg(Line(v.p, p), v.q));
}

/*
 * if point a inside polygon p[n]
 */
int inside_polygon(Point p[], int n, Point a){
    double sum = 0;
    for(int i = 0; i < n; i++){
        int j = (i + 1) % n;
        if (on_lineseg(Line(p[i], p[j]), a)) return 0;
        double angle = acos(dot(a, p[i], p[j]) / dist(a, p[i]) / dist(a, p[j]));
        sum += sign(cross(a, p[i], p[j])) * angle;
    }
    return sign(sum);
}

/*
 * if lineseg l strickly inside polygon p[n]
 */
int lineseg_inside_polygon(Point p[], int n, Line l){
    for(int i = 0; i < n; i++){
        int j = (i + 1) % n;
        Line l1(p[i], p[j]);
        if (on_lineseg_exclusive(l, p[i])) return 0;
        if (intersected_exclusive(l, l1)) return 0;
    }
    return inside_polygon(p, n, mp(l.p, l.q));
}

/*
 * if lineseg l intersect convex polygon p[n]
 */
int intersect_convex_lineseg(Point p[], int n, Line l){
    if (n < 3) return 0;
    Point q[4];
    int k = 0;
    q[k++] = l.p;
    q[k++] = l.q;
    for(int i = 0; i < n; i++){
        if (on_lineseg(l, p[i])){
            q[k++] = p[i];
        }
        else{
            int j = (i + 1) % n;
            Line a(p[i], p[j]);
            Point tmp = ip(a, l);
            if (on_lineseg(l, tmp) && on_lineseg(a, tmp)) q[k++] = tmp;
        }
    }
    sort(q, q + k);
    for(int i = 0; i + 1 < k; i++){
        if (inside_polygon(p, n, mp(q[i], q[i + 1]))) return 1;
    }
    return 0;
}

Line perpendicular(Line l, Point a){
    return Line(a, Point(a.x + l.p.y - l.q.y, a.y + l.q.x - l.p.x));
}

Point pedal(Line l, Point a){
    return ip(l, perpendicular(l, a));
}

Point mirror(Line l, Point a){
    Point p = pedal(l, a);
    return Point(p.x * 2 - a.x, p.y * 2 - a.y);
}

Point perpencenter(Point a, Point b, Point c){
    Line u = perpendicular(Line(b, c), a);
    Line v = perpendicular(Line(a, c), b);
    return ip(u, v);
}

/*
 * Inscribed circle center
 */
Point icc(Point A, Point B, Point C){
    double a = dist(B, C);
    double b = dist(C, A);
    double c = dist(A, B);
    double p = (a + b + c) / 2;
    double s = sqrt(p * (p - a) * (p - b) * (p - c));
    Point cp;
    cp.x = (a * A.x + b * B.x + c * C.x) / (a + b + c);
    cp.y = (a * A.y + b * B.y + c * C.y) / (a + b + c);
    return cp;
}

/*
 * Perpendicular bisector
 */
Line pb(Point a, Point b){
    return perpendicular(Line(a, b), mp(a, b));
}

/*
 * circumcicle center
 */
Point ccc(Point A, Point B, Point C){
    double a1 = B.x - A.x, b1 = B.y - A.y, c1 = (Sqr(a1) + Sqr(b1)) / 2;;
    double a2 = C.x - A.x, b2 = C.y - A.y, c2 = (Sqr(a2) + Sqr(b2)) / 2;;
    double d = a1 * b2 - a2 * b1;
    Point cp;
    cp.x = A.x + (c1 * b2 - c2 * b1) / d;
    cp.y = A.y + (a1 * c2 - a2 * c1) / d;
    return cp;
}

/*
 * translate l with distance e and direction s
 */
Line translate(Line l, double e, int s){
    double d = dist(l.p, l.q);
    double x = l.p.y - l.q.y;
    double y = l.q.x - l.p.x;
    x *= s * e / d;
    y *= s * e / d;
    l.p.x += x; l.p.y += y;
    l.q.x += x; l.q.y += y;
    return l;
}

/*
 * area of the part of convex polygon p[n] on the positive side of l
 */
double cut_area(Point *p, int n, Line l){
    int ai, bi;
    Point ap, bp;
    ai = bi = -1;
    for(int i = 0; i < n; i++){
        Line v(p[i], p[i + 1]);
        if (parallel(v, l)) continue;
        Point cp = ip(v, l);
        if (cp == p[i] || on_lineseg_exclusive(v, cp)){
            if (ai == -1){
                ai = i;
                ap = cp;
            }
            else{
                bi = i;
                bp = cp;
            }
        }
    }
    Point *q = new Point[n + 2]; //XXX new here
    int m = 0;
    for(int i = 0; i < n; i++){
        if (sign(cross(l.p, l.q, p[i])) >= 0){
            q[m++] = p[i];
        }
        if (i == ai){
            q[m++] = ap;
        }
        if (i == bi){
            q[m++] = bp;
        }
        double res = area_polygon(q, m);
        free(q);
        return res;
    }
}

/*
 * -----intersection points convex hull--------
 */
bool lcmp(Line u, Line v){
    int c = sign((u.p.x - u.q.x) * (v.p.y - v.q.y) - (v.p.x - v.q.x) * (u.p.y - u.q.y));
    return c < 0 || !c && sign(cross(u.p, u.q, v.p)) < 0;
}

/*
 * XXX sizeof(p) MUST be as large as n * 2
 * return # of points of resulting convex hull
 */
int ip_convex(Line l[], int n, Point p[]){
    for(int i = 0; i < n; i++){
        if (l[i].q < l[i].p) swap(l[i].p, l[i].q);
    }
    sort(l, l + n, lcmp);
    int n1 = 0;
    for(int i = 0, j = 0; i < n; i = j){
        while(j < n && parallel(l[i], l[j])) j++;
        if (j - i == 1){
            l[n1++] = l[i];
        }
        else{
            l[n1++] = l[i];
            l[n1++] = l[j - 1];
        }
    }
    n = n1;
    l[n + 0] = l[0];
    l[n + 1] = l[1];
    int m = 0;
    for(int i = 0, j = 0; i < n; i++){
        while(j < n + 2 && parallel(l[i], l[j])) j++;
        for(int k = j; k < n + 2 && parallel(l[j], l[k]);k++){
            p[m++] = ip(l[i], l[k]);
        }
    }
    return find_convex(p, m);
}

/*
 * ---------circles----------------
 */
struct Circle{
    Point o;
    double r;
    Circle(Point o = Point(), double r = 1) : o(o), r(r){}
    Circle(double x, double y, double r = 1) : o(x, y), r(r){}
};

int intersected_circle_line(Circle c, Line l){
    return sign(dist_line_point(l, c.o) - c.r) < 0;
}

int ip_circle_line(Circle c, Line l, Point &p1, Point &p2){
    Point a = l.p, b = l.q;
    double dx = b.x - a.x;
    double dy = b.y - a.y;
    double sdr = Sqr(dx) + Sqr(dy);
    double dr = sqrt(sdr);
    double d, disc, x, y;
    a.x -= c.o.x; a.y -= c.o.y;
    b.x -= c.o.x; b.y -= c.o.y;
    d = a.x * b.y - b.x * a.y;
    disc = Sqr(c.r) * sdr - Sqr(d);
    if (disc < -EPS) return 0;
    if (disc < +EPS){
        disc = 0;
    }
    else{
        disc = sqrt(disc);
    }
    x = disc * dx * (dy > 0 ? 1 : -1);
    y = disc * fabs(dy);
    p1.x = (+d * dy + x) / sdr + c.o.x;
    p2.x = (+d * dy - x) / sdr + c.o.x;
    p1.y = (-d * dx + y) / sdr + c.o.y;
    p2.y = (-d * dx - y) / sdr + c.o.y;
    return disc > EPS ? 2 : 1;
}

int ip_circle_circle(const Circle &c1, const Circle &c2, Point &p1, Point &p2){
    double mx = c2.o.x - c1.o.x, sx = c2.o.x + c1.o.x, mx2 = Sqr(mx);
    double my = c2.o.y - c1.o.y, sy = c2.o.y + c1.o.y, my2 = Sqr(my);
    double sq = mx2 + my2, d = -(sq - Sqr(c1.r - c2.r)) * (sq - Sqr(c1.r + c2.r));
    if (!sign(sq)) return 0;
    if (d + EPS < 0) return 0;
    if (d < EPS){
        d = 0;
    }
    else{
        d = sqrt(d);
    }
    double x = mx * ((c1.r + c2.r) * (c1.r - c2.r) + mx * sx) + sx * my2;
    double y = my * ((c1.r + c2.r) * (c1.r - c2.r) + my * sy) + sy * mx2;
    double dx = mx * d, dy = my * d;
    sq *= 2;
    p1.x = (x + dy) / sq; p1.y = (y - dx) / sq;
    p2.x = (x - dy) / sq; p2.y = (y + dy) / sq;
    return d > EPS ? 2 : 1; 
}

double circle_circle_intersection_area(Circle A, Circle B){
    double d, dA, dB, tx, ty;
    d = hypot(B.o.x - A.o.x, B.o.y - A.o.y);
    if ((d < EPS) || (d + A.r <= B.r) || (d + B.r <= A.r)){
        return Sqr((B.r < A.r) ? B.r : A.r) * PI;
    }
    if (d >= A.r + B.r){
        return 0;
    }
    dA = tx = (Sqr(d) + Sqr(A.r) - Sqr(B.r)) / d / 2;
    ty = sqrt(Sqr(A.r) - Sqr(tx));
    dB = d - dA;
    return Sqr(A.r) * acos(dA / A.r) - dA * sqrt(Sqr(A.r) - Sqr(dA)) + Sqr(B.r) * acos(dB / B.r) - dB * sqrt(Sqr(B.r) - Sqr(dB));
}

/*
 * return 2 points of tangecy of c and p
 */
void circle_tangents(Circle c, Point p, Point &a, Point &b){
    double d = Sqr(c.o.x - p.x) + Sqr(c.o.y - p.y);
    double para = Sqr(c.r) / d;
    double perp = c.r * sqrt(d - Sqr(c.r)) / d;
    a.x = c.o.x + (p.x - c.o.x) * para - (p.y - c.o.y) * perp;
    a.y = c.o.y + (p.y - c.o.y) * para + (p.x - c.o.x) * perp;
    b.x = c.o.x + (p.x - c.o.x) * para + (p.y - c.o.y) * perp;
    b.y = c.o.y + (p.y - c.o.y) * para - (p.x - c.o.x) * perp;
}

/*
 * +0 : on circle;
 * +1 : inside circle;
 * -1 : outside circle;
 */
int on_circle(Circle c, Point a){
    return sign(c.r - dist(a, c.o));
}

/*
 * minimum circle that covers 2 points
 */
Circle cc2(Point a, Point b){
    return Circle(mp(a, b), dist(a, b) / 2);
}

Circle cc3(Point p, Point q, Point r){
    Circle c;
    if (on_circle(c = cc2(p, q), r) >= 0) return c;
    if (on_circle(c = cc2(p, r), q) >= 0) return c;
    if (on_circle(c = cc2(q, r), p) >= 0) return c;
    c.o = ccc(p, q, r);
    c.r = dist(c.o, p);
    return c;
}

Circle min_circle_cover(Point p[], int n){
    if (n == 1) return Circle(p[0], 0);
    if (n == 2) return cc2(p[0], p[1]);
    random_shuffle(p, p + n);
    Point *ps[4] = {&p[0], &p[1], &p[2], &p[3]};
    Circle c = cc3(*ps[0], *ps[1], *ps[2]);
    while(true){
        Point *b = p;
        for(int i = 1; i < n; i++){
            if (dist(p[i], c.o) > dist(*b, c.o)) b = &p[i];
        }
        if (on_circle(c, *b) >= 0) return c;
        ps[3] = b;
        for(int i = 0; i < 3; i++){
            swap(ps[i], ps[3]);
            if (on_circle(c = cc3(*ps[0], *ps[1], *ps[2]), *ps[3]) >= 0) break;
        }
    }
}

\end{verbatim}

    %\subsubsection{三维几何}
\begin{verbatim}

#include <cstdio>
#include <cstring>
#include <cstdlib>
#include <algorithm>
#include <cmath>

#define Sqr(x) (x) * (x)

using namespace std;

const double EPS = 1E-8;

inline int sign(double x){
    return x < -EPS ? -1 : x > EPS;
}

inline double frand(){
    return rand() / (RAND_MAX + 1.0);
}

/*
 * ------points & vectors-----
 */
struct Point3{
    double x, y, z;
    Point3(){}
    Point3(double x, double y, double z) : x(x), y(y), z(z) {}
    void in(){
        scanf("%lf%lf%lf", &x, &y, &z);
    }
    void print(){
        printf("%lf %lf %lf", x, y, z);
    }
};
typedef Point3 Vector3;

inline Vector3 operator+(Point3 a, Point3 b){
    return Vector3(a.x + b.x, a.y + b.y, a.z + b.z);
}

inline Vector3 operator-(Point3 a, Point3 b){
    return Vector3(a.x - b.x, a.y - b.y, a.z - b.z);
}

inline Vector3 operator*(double t, Vector3 a){
    return Vector3(a.x * t, a.y * t, a.z * t);
}

inline Vector3 operator*(Vector3 a, double t){
    return Vector3(a.x * t, a.y * t, a.z * t);
}

inline Vector3 operator/(Vector3 a, double t){
    return Vector3(a.x / t, a.y / t, a.z / t);
}

inline Vector3 operator*(Vector3 a, Vector3 b){
    return Vector3(a.y * b.z - a.z * b.y, a.z * b.x - a.x * b.z, a.x * b.y - a.y * b.x);
}

inline double operator^(Vector3 a, Vector3 b){
    return a.x * b.x + a.y * b.y + a.z * b.z;
}

inline double len(Vector3 a){
    return sqrt(a ^ a);
}

inline double len2(Vector3 a){
    return a ^ a;
}

inline int zero(Vector3 a){
    return !sign(a.x) && !sign(a.y) && !sign(a.z);
}

/*
 * ----------lines, line segment & planes----------
 */
struct Line{
    Point3 p, q;
    Line(){}
    Line(Point3 p, Point3 q) : p(p), q(q) {}
    double len2(){
        return (q - p) ^ (q - p);
    }
    double len(){
        return sqrt((q - p) ^ (q - p));
    }
};

/*
 * returns a vector that perps to u
 */
Vector3 perp_vector(Vector3 u){
    Vector3 v, n;
    while(true){
        v.x = frand();
        v.y = frand();
        v.z = frand();
        if (!zero(n = u * v)) return v;
    }
}

/*
 * check if point a inside line l
 */
int on_seg(Line l, Point3 a){
    return zero((a - l.p) * (a - l.q)) && sign((l.p - a) ^ (l.q - a)) <= 0;
}

/*
 * relation of a & b base on l
 * same side : +1;
 * opposite side : -1;
 * otherwise : 0;
 */
inline int side(Line l, Point3 a, Point3 b){
    return sign(((l.p - l.q) * (a - l.p)) ^ ((l.p - l.q) * (b - l.p)));
}

/*
 * intersetion point of plane(norm, A) and lineseg l 
 * ret is the result
 */
int ip_plane_seg(Vector3 norm, Point3 a, Line l, Point3 &ret){
    double lhs = norm ^ (l.q - l.p);
    double rhs = norm ^ (a - l.p);
    double t = rhs / lhs;
    if (sign(t) >= 0 && sign(t- 1) <= 0){
        ret = l.p + t * (l.q - l.p);
        return 1;
    }
    return 0;
}

/*
 * check if 2 linesegs l1 & l2 touched with each other
 */
int touched_segs(Line l1, Line l2){
    if (zero((l1.q - l1.p) * (l2.q - l2.p))){
        return on_seg(l1, l2.p) || on_seg(l1, l2.q) || on_seg(l2, l1.p) || on_seg(l2, l1.q);
    }
    else{
        return side(l1, l2.p, l2.q) <= 0 && side(l2, l1.p, l1.q) <= 0;
    }
}

/*
 * return the projection of point a to line l
 */
Point3 project(Line l, Point3 a){
    double t = ((l.q - l.p) ^ (a - l.p)) / ((l.q - l.p) ^ (l.q - l.p));
    return l.p + t * (l.q - l.p);
}

/*
 * return the closest point in line l
 */
Point3 closest_point_seg(Line l, Point3 a){
    double t = ((l.q - l.p) ^ (a - l.p)) / ((l.q - l.p) ^ (l.q - l.p));
    return l.p + max(0.0, min(t, 1.0)) * (l.q - l.p);
}


/*
 * ----------plane----------
 */
struct Plane{
    Point3 a;
    Vector3 n;
};

/*
 * check if the point in the plane
 */
int on_plane(Plane pl, Point3 p){
    return !sign(pl.n ^ (p - pl.a));
}

/*
 * return the distance between point and the plane
 */
double dist_plane_point(Plane pl, Point3 a){
    return fabs(pl.n ^ (a - pl.a)) / len(pl.n);
}

/*
 * closest point in the plane
 */
Point3 closest_point_plane(Plane pl, Point3 a){
    return a + ((pl.n ^ (pl.a - a)) / (pl.n ^ pl.n)) * pl.n;
}

/*
 * mappint from 3D point to 2D point
 */
Point3 to_plane(Point3 a, Point3 b, Point3 c, Point3 p){
    Vector3 norm, ydir, xdir;
    Point3 res;
    norm = (b - a) * (c - a);
    xdir = b - a;
    xdir = xdir / len(xdir);
    ydir = norm * xdir;
    ydir = ydir / len(ydir);
    res.x = (p - a) ^ xdir;
    res.y = (p - a) ^ ydir;
    res.z = 0;
    return res;
}

/*
 * given two lines in 3D space , find distance of closest approach
 */
double dist_line_line(Line l1, Line l2){
    Vector3 v = (l1.q - l1.p) * (l2.q - l2.p);
    if (zero(v)){
        if (zero((l1.q - l1.p) * (l2.p - l1.p))) return 0;
        return len((l2.p - l1.p) * (l2.q - l1.p)) / len(l2.p - l2.q);
    }
    return fabs((l1.p - l2.p) ^ v) / len(v);
}

/*
 * this is the same as dist_line_line, but it also return s the points of closest approach
 */
double closest_approach(Line l1, Line l2, Point3 &p, Point3 &q){
    double s = (l2.q - l2.p) ^ (l1.q - l1.p);
    double t = (l1.p - l2.p) ^ (l2.q - l2.p);
    double num, den, tmp;
    den = l1.len2() * l2.len2() - s * s;
    num = t * s - l2.len2() * ((l1.p - l2.p) ^ (l1.q - l1.p));
    if (!sign(den)){
        p = l1.p;
        q = l2.p + (l2.q - l2.p) * t / l2.len();
        if (!sign(s)) q = l1.p;
    }
    else{
        tmp = num / den;
        p = l1.p + (l1.q - l1.p) * tmp;
        q = l2.p + (l2.q - l2.p) * (t + s * tmp) / l2.len2();
    }
    return len(p - q);
}

/*
 * --------balls(spheres)----------
 */
struct Ball{
    Point3 o;
    double r;
    Ball(Point3 o = Point3(0, 0, 0), double r = 1) : o(o), r(r) {}
};

/*
 * ip between ball o and line l
 */
int ip_ball_line(Ball o, Line l, Point3 &p, Point3 &q){
    Vector3 v;
    Point3 d = project(l, o.o);
    if (len2(o.o - d) > o.r * o.r) return 0;
    v = sqrt((o.r * o.r - len2(o.o - d)) / l.len2()) * (l.p - l.q);
    p = d + v;
    q = d - v;
    return 1;
}

/*
 * Given the latitude and longitude of two points in degrees
 * calculates the distance over the sphere between them.
 * Latitude is given in the range [-PI/2,PI/2] degrees,
 * Longitude is given in the range [-PI,PI] degrees.
 */
double greatcircle(double lat1, double long1, double lat2, double long2){
    return acos(sin(lat1) * sin(lat2) + cos(lat1) * cos(lat2) * cos(long2 - long1));
}

/*
 * Solves the determinant of a n*n matrix recursively
 */
double det(double m[4][4], int n){
    double s[4][4], res = 0, x;
    int i, j, skip, ssize;
    if (n == 2){
        return m[0][0] * m[1][1] - m[0][1] * m[1][0];
    }
    for(skip = 0; skip < n; skip++){
        for(i = 1; i < n; i++){
            for(j = 0, ssize = 0; j < n; j++){
                if (j == skip) continue;
                s[i - 1][ssize++] = m[i][j];
            }
        }
        x = det(s, n - 1);
        if (skip % 2){
            res -= m[0][skip] * x;
        }
        else{
            res += m[0][skip] * x;
        }
    }
    return res;
}

/*
 * Given 4 points:
 * Returns 0 if the points are coplanar
 * Returns 1 if the points are not coplanar with:
 *     o = center of sphere
 *     r = radius of sphere
 */
int make_sphere(Point3 p[4], Ball o){
    double m[4][5], s[4][4], sol[5];
    int ssize, skip, i, j;
    for(i = 0; i < 4; i++){
        s[i][0] = p[i].x;
        s[i][1] = p[i].y;
        s[i][2] = p[i].z;
        s[i][3] = 1;
    }
    if (!sign(det(s, 4))) return 0;
    for(i = 0; i < 4; i++){
        m[i][0] = 0;
        m[i][0] += Sqr(m[i][1] = p[i].x);
        m[i][0] += Sqr(m[i][2] = p[i].y);
        m[i][0] += Sqr(m[i][3] = p[i].z);
        m[i][4] = 1;
    }
    for(skip = 0; skip < 5; skip++){
        for(i = 0; i < 4; i++){
            for(j = 0, ssize = 0; j < 5; j++){
                if (j == skip) continue;
                s[i][ssize++] = m[i][j];
            }
        }
        sol[skip] = det(s, 4);
    }
    for(i = 1; i < 5; i++){
        sol[i] /= (sol[0] * ((i % 2) ? 1 : -1));
    }
    for(i = 1; i < 4; i++){
        sol[4] += Sqr(sol[i] /= 2);
    }
    o.o.x = sol[1];
    o.o.y = sol[2];
    o.o.z = sol[3];
    o.r = sqrt(sol[4]);
    return 1;
}

/*
 * ----------polygons---------- 
 */

/*
 * check if point A inside polygon p[n]
 */
int inside_polygon(Point3 *p, int n, Vector3 norm, Point3 A){
    if (sign(norm ^ (A - p[0]))) return 0;
    p[n] = p[0];
    for(int i = 0; i < n; i++){
        if (on_seg(Line(p[i], p[i + 1]), A)) return 1;
    }
    double sum = 0;
    for(int i = 0; i < n; i++){
        Vector3 a = p[i] - A;
        Vector3 b = p[i + 1] - A;
        sum += sign(norm ^ (a * b)) * acos((a ^ b) / (len(a) * len(b)));
    }
    return sign(sum);
}

/*
 * check if lineseg l touches polygon p[n] with normal vector norm
 */
int intersected_polygon_seg(Point3 *p, int n, Vector3 norm, Line l){
    p[n] = p[0];
    if (!sign((l.p - l.q) ^ norm)){
        if (sign(norm ^ (l.p - p[0]))) return 0;
        if (inside_polygon(p, n, norm, l.p) || inside_polygon(p, n, norm, l.q)) return 1;
        for(int i = 0; i < n; i++){
            if (touched_segs(l, Line(p[i], p[i + 1]))) return 1;
        }
        return 0;
    }
    Point3 ret;
    if (ip_plane_seg(norm, p[0], l, ret)){
        return inside_polygon(p, n, norm, ret);
    }
    return 0;
}

/*
 * normal vector of polygon p[n]
 */
Vector3 normal(Point3 *p, int n){
    Vector3 b, norm;
    p[n] = p[0];
    p[n + 1] = p[1];
    for(int i = 0; i < n; i++){
        norm = (p[i + 1] - p[i + 2]) * (p[i] - p[i + 1]);
        if (!zero(norm)) return norm;
    }
    return perp_vector(p[0] - p[1]);
}

/*
 * check if 2 polygons p[n] & q[m] touched with each other
 */
int touched_polygons(Point3 *p, int n, Point3 *q, int m){
    Vector3 norm;
    norm = normal(q, m);
    p[n] = p[0];
    for(int i = 0; i < n; i++){
        if (intersected_polygon_seg(q, m, norm, Line(p[i], p[i + 1]))) return 1;
    }
    norm = normal(p, n);
    q[m] = q[0];
    for(int i = 0; i < m; i++){
        if (intersected_polygon_seg(p, n, norm, Line(q[i], q[i + 1]))) return 1;
    }
    return 0;
}

/*
 * new add by myf
 */

struct Plane3{
    Point3 a, b, c;
    Plane3(){}
    Plane3(Point3 a, Point3 b, Point3 c) : a(a), b(b), c(c) {}
};

double triple(Point3 a, Point3 b, Point3 c){
    return a ^ (b * c);;
}

double polygon_volume(Plane3 *p, int n){
    double volume = 0.0;
    for(int i = 0; i < n; i++){
        volume += triple(p[i].a, p[i].b, p[i].c);
    }
    return fabs(volume) / 6.0;
}
\end{verbatim}


	\subsection{凸包}
	%\subsubsection{凸包}
\begin{verbatim}
// find the convex hull
Point __o;

bool cmp_p(Point a, Point b){
    int f = sign(a.X - b.X);
    if (f) return f < 0;
    return sign(a.Y - b.Y) < 0;
}

bool cmp(Point a, Point b){
    int f = sign(cross(o, a, b));
    if (f) return f > 0;
    return sign(abs(a - o) - abs(b - o)) < 0;
}

Point stack[1111]

int find_convex(Point p[], int n){
    __o = *min_element(p, p + n, cmp_p);
    sort(p, p + n, cmp);
    int top = 0;
    rep(i, n){
        while(top >= 2 && sign(cross(stack[top - 2], stack[top - 1], p[i])) <= 0) top--;
        stack[top++] = p[i];
    }
    rep(i, top) p[i] = stack[i];
    return top;
}

// -----intersection points convex hull--------
bool lcmp(Line u, Line v){
    int c = sign((u.p.x - u.q.x) * (v.p.y - v.q.y) - (v.p.x - v.q.x) * (u.p.y - u.q.y));
    return c < 0 || !c && sign(cross(u.p, u.q, v.p)) < 0;
}

/*
 * XXX sizeof(p) MUST be as large as n * 2
 * return # of points of resulting convex hull
 */
int ip_convex(Line l[], int n, Point p[]){
    for(int i = 0; i < n; i++){
        if (l[i].q < l[i].p) swap(l[i].p, l[i].q);
    }
    sort(l, l + n, lcmp);
    int n1 = 0;
    for(int i = 0, j = 0; i < n; i = j){
        while(j < n && parallel(l[i], l[j])) j++;
        if (j - i == 1){
            l[n1++] = l[i];
        }
        else{
            l[n1++] = l[i];
            l[n1++] = l[j - 1];
        }
    }
    n = n1;
    l[n + 0] = l[0];
    l[n + 1] = l[1];
    int m = 0;
    for(int i = 0, j = 0; i < n; i++){
        while(j < n + 2 && parallel(l[i], l[j])) j++;
        for(int k = j; k < n + 2 && parallel(l[j], l[k]);k++){
            p[m++] = ip(l[i], l[k]);
        }
    }
    return find_convex(p, m);
}

typedef double Tdata;
typedef Point Tpoint;
// get the diameter of convex polygon
// O(N)
// p1, p2 are the points forming diameter
Tdata diam_convex_poly(Tpoint *P, int N, Tpoint &p1, Tpoint &p2) {
    if (N == 1) {
        p1 = p2 = P[0];
        return 0;
    }
    double ret = -INF;
    for (int j = 1, i = 0; i < N; ++i) {
        while (sign(cross(P[i], P[i + 1], P[j + 1]) - cross(P[i], P[i + 1], P[j])) > 0) j = (j + 1) % N;
        ret = max(ret, max(dist2(P[i], P[j]), dist2(P[i + 1], P[j + 1])));
    }
    return ret;
}
\end{verbatim}

	%\subsubsection{三维凸包n*n}
\begin{verbatim}
typedef double Tdata;

const int MAXN = 1000 + 10;
const int MAXF = MAXN * 6;
const double EPS = 1E-6;

inline int sign(Tdata x) { return x < -EPS ? -1 : x > EPS ? 1 : 0; }

struct Tpoint {
    Tdata x, y, z;
    
    Tpoint() {}
    Tpoint(Tdata x, Tdata y, Tdata z) : x(x), y(y), z(z) {}
    void get() { scanf("%lf%lf%lf", &x, &y, &z); }
    bool operator <(Tpoint p) const {
        int s = sign(x - p.x); if (s) return s < 0;
        s = sign(y - p.y); if (s) return s < 0;
        return sign(z - p.z) < 0;
    }
    bool operator ==(Tpoint p) const { return !sign(x - p.x) && !sign(y - p.y) && !sign(z - p.z); }
    void operator -=(Tpoint p) { x -= p.x; y -= p.y; z -= p.z; }
    void operator +=(Tpoint p) { x += p.x; y += p.y; z += p.z; }
    void operator *=(Tdata c) { x *= c; y *= c; z *= c; }
    void operator /=(Tdata c) { x /= c; y /= c; z /= c; }
    Tpoint operator +(Tpoint p) const { return Tpoint(x + p.x, y + p.y, z + p.z); }
    Tpoint operator -(Tpoint p) const { return Tpoint(x - p.x, y - p.y, z - p.z); }
    Tpoint operator *(Tdata c) const { return Tpoint(x * c, y * c, z * c); }
    Tpoint operator /(Tdata c) const { return Tpoint(x / c, y / c, z / c); }
};

inline Tdata sqr(Tdata x) { return x * x; }

inline Tdata norm2(Tpoint p) { return sqr(p.x) + sqr(p.y) + sqr(p.z); }

inline Tdata norm(Tpoint p) { return sqrt(norm2(p)); }

inline Tpoint cross(Tpoint a, Tpoint b) { return Tpoint(a.y * b.z - b.y * a.z, a.z * b.x - b.z * a.x, a.x * b.y - b.x * a.y); }

inline Tpoint cross(Tpoint o, Tpoint a, Tpoint b) { return cross(a - o, b - o); }

inline Tdata det(Tpoint a, Tpoint b, Tpoint c) {
    #define D2(a, b, x, y) (a.x * b.y - a.y * b.x)
    return a.x * D2(b, c, y, z) - a.y * D2(b, c, x, z) + a.z * D2(b, c, x, y);
    #undef D2
}

inline Tdata dot(Tpoint a, Tpoint b) { return a.x * b.x + a.y * b.y + a.z * b.z; }

inline double volume(Tpoint p, Tpoint a, Tpoint b, Tpoint c) { return det(a - p, b - p, c - p); }

struct Chull3D {
    Tpoint P[MAXN];
    int face[MAXF][3];
    int del[MAXF];
    int lnk[MAXN][MAXN];
    bool used[MAXN];
    int N, F, face_num;
    Tdata vol, area;
    Tpoint cen;

    inline int vol_sgn(int o, int a, int b, int c) {
        Tdata v = volume(P[o], P[a], P[b], P[c]);
        return sign(v);
    }

    inline void add_face(int a, int b, int c) {
        face[F][0] = a; face[F][1] = b; face[F][2] = c; del[F] = 0;
        lnk[a][b] = lnk[b][c] = lnk[c][a] = F++;
    }

    inline bool can_see(int p, int f) { return vol_sgn(p, face[f][0], face[f][1], face[f][2]) < 0; }

    //return 0 if all in one plane or line
    bool find_tet() {
        for (int i = 1; i < N; ++i) if (P[i].x < P[0].x) swap(P[i], P[0]);
        for (int i = 2; i < N; ++i) if (P[i].x > P[1].x) swap(P[i], P[1]);
        for (int i = 3; i < N; ++i)
            if (fabs(norm2(cross(P[0], P[1], P[i]))) > fabs(norm2(cross(P[0], P[1], P[2])))) swap(P[2], P[i]);
        if (cross(P[0], P[1], P[2]) == Tpoint(0, 0, 0)) return 0;
        for (int i = 4; i < N; ++i)
            if (fabs(volume(P[0], P[1], P[2], P[i])) > fabs(volume(P[0], P[1], P[2], P[3]))) swap(P[3], P[i]);
        if (!vol_sgn(0, 1, 2, 3)) return 0;
        for (int i = 0; i < 4; ++i) {
            int a = (i + 1) % 4, b = (i + 2) % 4, c = (i + 3) % 4;
            if (vol_sgn(i, a, b, c) < 0) swap(b, c);
            add_face(a, b, c);
        }
        return 1;
    }

    void add(int p, int f) {
        if (del[f]) return;
        del[f] = 1;
        for (int i = 0; i < 3; ++i) {
            int opp = lnk[face[f][(i + 1) % 3]][face[f][i]];
            if (!del[opp]) {
                if (can_see(p, opp)) add(p, opp);
                else add_face(face[f][i], face[f][(i + 1) % 3], p);
            }
        }
    }

    bool coplanar(int f1, int f2, int p1, int p2) {
        int vs[4], m = 0;
        for (int i = 0; i < 3; ++i) {
            int v = face[f1][i];
            if (v != p1 && v != p2) vs[m++] = v;
        }
        for (int i = 0; i < 3; ++i) vs[m++] = face[f2][i];
        return vol_sgn(vs[0], vs[1], vs[2], vs[3]) == 0;
    }

    int cal_face() {
        int E = 0, V = 0;
        memset(used, 0, sizeof(used));
        for (int i = 0; i < F; ++i)
            if (!del[i])
                for (int j = 0; j < 3; ++j) {
                    int k = lnk[face[i][(j + 1) % 3]][face[i][j]];
                    if (!del[k] && !coplanar(i, k, face[i][j], face[i][(j + 1) % 3])) ++E, used[face[i][j]] = used[face[i][(j + 1) % 3]] = 1;
                }
        for (int i = 0; i < N; ++i) if (used[i]) ++V;
        return 2 + E / 2 - V;
    }

    double cal_volume() {
        double ret = 0;
        for (int i = 0; i < F; ++i)
            if (!del[i]) {
                Tpoint a = P[face[i][0]], b = P[face[i][1]], c = P[face[i][2]];
                ret += volume(Tpoint(0, 0, 0), a, b, c);
            }
        return fabs(ret) / 6.0;
    }

    double cal_area() {
        double ret = 0;
        for (int i = 0; i < F; ++i)
            if (!del[i]) {
                Tpoint a = P[face[i][0]], b = P[face[i][1]], c = P[face[i][2]];
                ret += fabs(norm(cross(a, b, c)) / 2.0);
            }
        return ret;
    }

    Tpoint cal_centroid() {
        Tpoint ret = Tpoint(0, 0, 0);
        for (int i = 0; i < F; ++i)
            if (!del[i]) {
                Tpoint a = P[face[i][0]], b = P[face[i][1]], c = P[face[i][2]];
                ret += (a + b + c) * volume(Tpoint(0, 0, 0), a, b, c);
            }
        return ret / cal_volume() / 24.0;
    }

    void get() {
        scanf("%d", &N);
        for (int i = 0; i < N; ++i) P[i].get();
        sort(P, P + N);
        N = unique(P, P + N) - P; F = 0;    
        vol = area = 0;
        memset(del, 0, sizeof(del));
        if (!find_tet()) return;
        random_shuffle(P + 4, P + N);
        for (int i = 4; i < N; ++i)
            for (int j = 0; j < F; ++j)
                if (!del[j] && can_see(i, j)) {
                    add(i, j);
                    break;
                }
        face_num = cal_face(); vol = cal_volume(); area = cal_area(); cen = cal_centroid();
    }
};
\end{verbatim}

	%\subsubsection{动态凸包}
\begin{verbatim}
//By Lin
#include<cstdio>
#include<cstring>
#include<map>
#define mp(x,y) make_pair(x,y)
#define foreach(i,n) for(__typeof(n.begin()) i = n.begin(); i!=n.end(); i++)
#define X first
#define Y second
using namespace std;
typedef long long LL;
typedef pair<int,int> pii;

map<int,int> up[2];
map<int,int>::iterator iter,p,q;

int strcmp( pii a, pii b, pii c){
    LL ret = ((LL)b.X-a.X)*((LL)c.Y-a.Y)-((LL)b.Y-a.Y)*((LL)c.X-a.X);
    return ret>0?1:(ret==0?0:-1);
}

bool pan( map<int,int> &g ,int x,int y){
    if ( g.size() == 0 ) return false;
    if ( g.find(x) != g.end() ) return y>=g[x];
    if ( g.begin()->X > x || (--g.end())->X < x ) return false;
    iter = g.lower_bound(x);
    p = q = iter;
    p--;
    return strcmp(*p,*q,mp(x,y))>=0;
}

void insert( map<int,int> &g, int x,int y){
    if ( pan(g,x,y) ) return;
    g[x] = y;
    iter = g.find(x);
    while ( iter != g.begin() ){
        p = iter;
        p--;
        if ( p == g.begin() ) break;
        q = p;
        q--;
        if ( strcmp(*q,*iter,*p)>=0 ) g.erase(p);
        else break;
    }
    iter = g.find(x);
    while ( true ){
        p = iter;
        p++;
        if ( p == g.end() ) break;
        q = p;
        q++;
        if ( q == g.end() ) break;
        if ( strcmp(*iter,*q,*p)>=0 ) g.erase(p);
        else break;
    }
}

int main(){
    int cas;
    scanf("%d", &cas );
    while ( cas -- ){
        int k,x,y
        scanf("%d%d%d", &k, &x, &y );
        if ( k == 1 )
            insert( up[0], x,y ),
            insert( up[1], x,-y);
        else
            printf( pan(up[0],x,y)&&pan(up[1],x,-y)?"YES\n":"NO\n" );
    }
    return 0;
}
\end{verbatim}

	%\input{code/geometry/min_distance_between_two_convex_polygon}

	\subsection{平面}
	%\subsubsection{半平面交}
\begin{verbatim}
#include <cstdio>
#include <iostream>
#include <algorithm>
#include <climits>
#include <cstring>
#include <cmath>
#define foreach(e,x) for(__typeof(x.begin()) e=x.begin();e!=x.end();++e)
using namespace std;

struct Point {
    long double x, y;
    Point() {
    }
    Point(long double _x, long double _y) :
            x(_x), y(_y) {
    }
    Point operator+(const Point&p) const {
        return Point(x + p.x, y + p.y);
    }
    Point operator-(const Point&p) const {
        return Point(x - p.x, y - p.y);
    }
    Point operator*(long double d) const {
        return Point(x * d, y * d);
    }
    Point operator/(long double d) const {
        return Point(x / d, y / d);
    }
    long double det(const Point&p) const {
        return x * p.y - y * p.x;
    }
    long double dot(const Point&p) const {
        return x * p.x + y * p.y;
    }
    Point rot90() const {
        return Point(-y, x);
    }
    void read() {
        cin >> x >> y;
    }
    void write() const {
        printf("%lf %lf", x, y);
    }
};

#define cross(p1,p2,p3) ((p2.x-p1.x)*(p3.y-p1.y)-(p3.x-p1.x)*(p2.y-p1.y))

const long double EPS = 1e-12;
inline int sign(long double a) {
    return a < -EPS ? -1 : a > EPS;
}

#define crossOp(p1,p2,p3) (sign(cross(p1,p2,p3)))

Point isSS(Point p1, Point p2, Point q1, Point q2) {
    long double a1 = cross(q1,q2,p1), a2 = -cross(q1,q2,p2);
    return (p1 * a2 + p2 * a1) / (a1 + a2);
}

struct Border {
    Point p1, p2;
    long double alpha;
    void setAlpha() {
        alpha = atan2(p2.y - p1.y, p2.x - p1.x);
    }
    void read() {
        p1.read();
        p2.read();
        setAlpha();
    }
};

int n;
const int MAX_N_BORDER = 20000 + 10;
Border border[MAX_N_BORDER];

bool operator<(const Border&a, const Border&b) {
    int c = sign(a.alpha - b.alpha);
    if (c != 0)
        return c == 1;
    return crossOp(b.p1,b.p2,a.p1) >= 0;
}

bool operator==(const Border&a, const Border&b) {
    return sign(a.alpha - b.alpha) == 0;
}

const long double LARGE = 10000;

void add(long double x, long double y, long double nx, long double ny) {
    border[n].p1 = Point(x, y);
    border[n].p2 = Point(nx, ny);
    border[n].setAlpha();
    n++;
}

Point isBorder(const Border&a, const Border&b) {
    return isSS(a.p1, a.p2, b.p1, b.p2);
}

Border que[MAX_N_BORDER];
int qh, qt;

bool check(const Border&a, const Border&b, const Border&me) {
    Point is = isBorder(a, b);
    return crossOp(me.p1,me.p2,is) > 0;
}

void convexIntersection() {
    qh = qt = 0;
    sort(border, border + n);
    n = unique(border, border + n) - border;
    for (int i = 0; i < n; ++i) {
        Border cur = border[i];
        while (qh + 1 < qt && !check(que[qt - 2], que[qt - 1], cur))
            --qt;
        while (qh + 1 < qt && !check(que[qh], que[qh + 1], cur))
            ++qh;
        que[qt++] = cur;
    }
    while (qh + 1 < qt && !check(que[qt - 2], que[qt - 1], que[qh]))
        --qt;
    while (qh + 1 < qt && !check(que[qh], que[qh + 1], que[qt - 1]))
        ++qh;
}

void calcArea() {
    static Point ps[MAX_N_BORDER];
    int cnt = 0;
    if (qt - qh <= 2) {
        puts("0.0");
        return;
    }
    for (int i = qh; i < qt; ++i) {
        int next = i + 1 == qt ? qh : i + 1;
        ps[cnt++] = isBorder(que[i], que[next]);
    }
    long double area = 0;
    for (int i = 0; i < cnt; ++i) {
        area += ps[i].det(ps[(i + 1) % cnt]);
    }
    area /= 2;
    area = fabsl(area);
    cout.setf(ios::fixed);
    cout.precision(1);
    cout << area << endl;
}

int main() {
    cin >> n;
    for (int i = 0; i < n; ++i) {
        border[i].read();
    }
    add(0, 0, LARGE, 0);
    add(LARGE, 0, LARGE, LARGE);
    add(LARGE, LARGE, 0, LARGE);
    add(0, LARGE, 0, 0);
    convexIntersection();
    calcArea();
}
\end{verbatim}

	%\subsubsection{动态半平面交}
\begin{verbatim}
#include <cstdio>
#include <set>
#include <map>

using namespace std;

const int N = 100000 + 10;
const int inf = 100001;
const double eps = 1e-10;

set<int> S;
int B[N];
double xval[N];
map<double, int> X;

void cal(int a1, int b1, int a2, int b2, double& x, double& y) {
    if (a2 == -inf) {
        x = 0;
        y = b1;
    }
    else {
        x = (b2 - b1) / double(a1 - a2);
        y = a1 * x + b1;
    }
}

int main() {
    int n, a, b, i;
    double x, y;
    char cmd[2];
    set<int>::iterator itr;
    S.insert(-inf);
    S.insert(0);
    B[inf] = 0;
    xval[inf] = inf;
    B[0] = 0;
    X[0] = 0;
    xval[0] = 0;
    scanf("%d", &n);
    for (i=1; i<=n; i++) {
        scanf("%s", cmd);
        if (cmd[0] == 'S') {
            scanf("%d %d", &a, &b);
            itr = S.lower_bound(a);
            if (*itr == a) {
                if (B[-*itr] >= b) continue; else B[-*itr] = b;
            }
            else {
                itr --;
                cal(a, b, *itr, B[-*itr], x, y);
                itr ++;
                if (y < *itr * x + B[-*itr] + eps) continue;
                itr = S.insert(a).first;
                B[-*itr] = b;
            }
            itr --;
            while (itr != S.begin() && *itr * xval[-*itr] + B[-*itr] < a * xval[-*itr] + b + eps) {
                x = xval[-*itr];
                X.erase(x);
                S.erase(itr --);
            }
            cal(a, b, *itr, B[-*itr], x, y);
            itr = S.find(a);
            itr ++;
            X.erase(xval[-*itr]);
            xval[-a] = x;
            X[x] = a;
            itr ++;
            while (itr != S.end() && *itr * xval[-*itr] + B[-*itr] < a * xval[-*itr] + b + eps) {
                itr --;
                S.erase(itr ++);
                x = xval[-*itr];
                X.erase(x);
                itr ++;
            }
            itr --;
            cal(a, b, *itr, B[-*itr], x, y);
            xval[-*itr] = x;
            X[x] = *itr;
        }
        else {
            scanf("%lf", &x);
            x = x * x;
            map<double, int>::iterator itr;
            if (x == 0) itr = X.lower_bound(0);
            else {
                itr = X.lower_bound(x);
                itr --;
            }
            printf("%.0lf\n", itr->second * x + B[-itr->second]);
        }
    }
    return 0;
}                                 
\end{verbatim}

    %\input{code/geometry/polygon_core}
	%\subsubsection{Point}
\begin{verbatim}
//By myf
//#pragma comment(linker, "/STACK:16777216")  //C++
#include <iomanip>
#include <iostream>
#include <algorithm>
#include <cmath>
#include <cstdio>
#include <cstdlib>
#include <cstring>
#include <bitset>
#include <complex>
#include <map>
#include <set>
#include <queue>
#include <deque>
#include <stack>
#include <vector>
#include <list>

#define rep(i,n) for(int i=0;i<(n);i++)
#define REP(i,l,r) for(int i=(l);i<(r);i++)
#define fab(i,a,b) for(int i=(a);i<=(b);i++)
#define fba(i,b,a) for(int i=(b);i>=(a);i--)
//#define foreach(i,n) for(__typeof(n.begin()) i=n.begin();i!=n.end();i++) //G++
#define MP make_pair
#define PB push_back
#define X first
#define Y second
#define Cls(x) memset(x,0,sizeof x)
#define Print(n,x) for(int i=0;i<(n);i++) cout<<x<<" ";cout<<endl;
#define Sqr(x) (x)*(x)

using namespace std;

typedef long long LL;
typedef unsigned long long ULL;
typedef pair<int,int> PII;
typedef pair<PII,int> PIII;
typedef pair<LL,int> PLI;
typedef vector<int> VI;
typedef LL T;

const int N = 500005,K = 2,D=6;
const LL inf = ((ULL)1<<63)-1;
//const int inf=~0U>>1;

struct kd{
    T x[K];
    kd(){rep(i,K)x[i]=0;}
} t[N];
int l[N],r[N];
int a[D],n,tot,root;

void insert(int &cur,kd p, int d) {
    if (!cur){
        cur=++tot;
        rep(i,K) t[cur].x[i] = p.x[i];
        l[cur]=r[cur]=0;
        return;
    }
    T dx = p.x[d] - t[cur].x[d];
    if (++d==K) d=0;
    insert(dx<0?l[cur]:r[cur],p,d);
}

T dis2(kd a,kd b) {
    T s=0;
    rep(i,K) s+=Sqr(a.x[i]-b.x[i]);
    return s;
}

void query(int cur, kd p, LL &ret, int d) {
    if (!cur) return;
    ret = min(ret, dis2(t[cur],p));
    T dx = p.x[d] - t[cur].x[d];
    if (++d == K) d = 0;
    if (dx < 0) {
        query(l[cur],p,ret,d);
        if (ret > Sqr(dx)) query(r[cur],p,ret,d);
    } else {
        query(r[cur],p,ret,d);
        if (ret > Sqr(dx)) query(l[cur],p,ret,d);
    }
}

void work() {
    root = tot = 0;
    T ans = inf, ret=0;
    kd p;
    rep(i,n){
        p.x[0] = (p.x[0] * a[0] + a[1]) % a[2];
        p.x[1] = (p.x[1] * a[3] + a[4]) % a[5];
        query(root, p, ans, 0);
        insert(root, p, 0);
        ret += ans * (i > 0);
    }
    printf("%I64d\n", ret);
}

int main() {
    int test;
    scanf("%d", &test);
    rep(cas,test){
        scanf("%d", &n);
        rep(i,D) scanf("%d", &a[i]);
        work();
    }
    return 0;
}
\end{verbatim}

	%\subsubsection{knn询问距离最近K个点}
\begin{verbatim}
//By myf
//#pragma comment(linker, "/STACK:16777216")  //C++
#include <iomanip>
#include <iostream>
#include <algorithm>
#include <cmath>
#include <cstdio>
#include <cstdlib>
#include <cstring>
#include <bitset>
#include <complex>
#include <map>
#include <set>
#include <queue>
#include <deque>
#include <stack>
#include <vector>
#include <list>

#define rep(i,n) for(int i=0;i<(n);i++)
#define REP(i,l,r) for(int i=(l);i<(r);i++)
#define fab(i,a,b) for(int i=(a);i<=(b);i++)
#define fba(i,b,a) for(int i=(b);i>=(a);i--)
//#define foreach(i,n) for(__typeof(n.begin()) i=n.begin();i!=n.end();i++) //G++
#define MP make_pair
#define PB push_back
#define X first
#define Y second
#define Cls(x) memset(x,0,sizeof x)
#define Print(n,x) for(int i=0;i<(n);i++) cout<<x<<" ";cout<<endl;
#define Sqr(x) (x)*(x)

using namespace std;

typedef long long LL;
typedef pair<int,int> PII;
typedef pair<PII,int> PIII;
typedef pair<LL,int> PLI;
typedef vector<int> VI;
typedef LL T;
typedef pair<T,T> Point;

T cross(Point a,Point b,Point c){return (b.X-a.X)*(c.Y-a.Y)-(c.X-a.X)*(b.Y-a.Y);}

T dot(Point a,Point b,Point c){return (b.X-a.X)*(c.X-a.X)+(b.Y-a.Y)*(c.Y-a.Y);}

bool inpoly(Point a, Point *p, int n){
    int wn = 0;
    rep(i,n){
        Point p1 = p[i], p2 = p[(i + 1) % n];
        T s = cross(a, p1, p2);
        if (!s && dot(a, p1, p2) <= 0) return true;
        T d1 = p1.Y - a.Y, d2 = p2.Y - a.Y;
        if (s > 0 && d1 <= 0 && d2 > 0) ++wn;
        if (s < 0 && d2 <= 0 && d1 > 0) --wn;
    }
    return wn != 0;
}

const int N = 20000, M = 20;

int n, m, r;
Point p[N], poly[M];

const int K = 2;
struct kd {
    LL x[K];
    int id;
}t[N];

T dis2(kd a, kd b){
    T s = 0;
    rep(i,K) s += Sqr(a.x[i] - b.x[i]);
    return s;
}

struct cmpk {
    int k;
    cmpk(int _k): k(_k) {}
    bool operator()(kd a, kd b){ return a.x[k] < b.x[k]; }
};

void build(int l, int r, int d){
    if (r - l <= 1) return;
    int mid = (l + r) >> 1;
    nth_element(t + l, t + mid, t + r, cmpk(d));
    if (++d == K) d = 0;
    build(l, mid, d); build(mid + 1, r, d);
}

typedef priority_queue<pair<T, int> > heap;
void knn(int l, int r, int d, kd p, size_t k, heap &h){
    if (r - l < 1) return;
    int mid = (l + r) >> 1;
    h.push(make_pair(dis2(p, t[mid]), t[mid].id));
    if (h.size() > k) h.pop();
    T dx = p.x[d] - t[mid].x[d];
    if (++d == K) d = 0;
    if (dx < 0) {
        knn(l, mid, d, p, k, h);
        if (h.top().first > Sqr(dx)) knn(mid + 1, r, d, p, k, h);
    } else {
        knn(mid + 1, r, d, p, k, h);
        if (h.top().first > Sqr(dx)) knn(l, mid, d, p, k, h);
    }
}

void solve(){
    scanf("%d", &m);
    rep(i,m) {
        int x,y;
        scanf("%d%d",&x,&y);
        poly[i]=MP(x,y);
    }
    int cnt = 0;
    rep(i,n){
        if (inpoly(p[i], poly, m)) {
            t[cnt].x[0] = p[i].X; t[cnt].x[1] = p[i].Y;
            t[cnt++].id = i + 1;
        }
    }
    build(0, cnt, 0);
    int q;
    scanf("%d", &q);
    while (q--) {
        kd p;
        scanf("%lld%lld", &p.x[0], &p.x[1]);
        heap h;
        knn(0, cnt, 0, p, 2, h);
        int a, b;
        b = h.top().second; h.pop();
        a = h.top().second;
        printf("%d %d\n", a, b);
    }
}

int main(){
    int dat;
    scanf("%d", &dat);
    rep(cas,dat){
        printf("Case #%d:\n", cas+1);
        scanf("%d",&n);
        rep(i,n){
            int x,y;
            scanf("%d%d",&x,&y);
            p[i]=MP(x,y);
        }
        scanf("%d", &r);
        rep(id,r){
            printf("Region %d\n", id+1);
            solve();
        }
    }
}
\end{verbatim}

	%\subsubsection{区域树(查询区域内点数量)}
\begin{verbatim}
/* MIPT Range Query 
 * surport 3 types of operations:
 *   add x, y
 *   delete x, y
 *   count [x1, x2] * [y1, y2] */
#include <stdio.h>
#include <string.h>
#include <algorithm>
using namespace std;

/*********** point ***********/

struct point {
  int x, y;
};

bool operator<(point p, point q) {
  return p.x < q.x || p.x == q.x && p.y < q.y;
}

bool operator==(point p, point q) {
  return p.x == q.x && p.y == q.y;
}

bool xcmp(point *a, point *b) {
  return a->x < b->x || a->x == b->x && a->y < b->y; 
}

bool ycmp(point *a, point *b) {
  return a->y < b->y || a->y == b->y && a->x < b->x; 
}

/*********** binary indexed tree ***********/

void ta_init(int *ta, int n) {
  memset(ta, 0, sizeof(*ta) * (n + 1));
}

void ta_add(int *ta, int n, int k, int d) {
  while (k <= n) {
    ta[k] += d;
    k += k & -k;
  }
}

int ta_sum(int *ta, int n, int k) {
  int res = 0;
  while (k) {
    res += ta[k];
    k -= k & -k;
  }
  return res;
}

/*********** range tree ***********/

struct node {
  int size;
  int x1, x2;
  node *l, *r;
  int *la, *lb;
  int *ta;
};

struct range_tree {
  node *root;
  int size;
  point **yl;
};

const int N = 100010;
const int M = N * 20; // N log N
const int INF = 2010000000;

range_tree __rt;
node nodes[N << 1], *next;
point *xs[N], *ys[N], *yt[N];
int links[M << 1], *ln;
int ts[M], *tn;

node *__build(point **xl, point **yl, point **yt, int n) {
  int i, d, na, nb;
  node *p;
  point **ya, **yb;

  p = next++;
  p->x1 = xl[0]->x;
  p->x2 = xl[n - 1]->x;
  p->size = n;
  p->ta = tn; tn += n + 1;
  ta_init(p->ta, n + 1);

  if (n > 1) {
    d = n / 2;
    ya = yt;
    yb = yt + d;
    na = d;
    nb = n - d;
    p->la = ln; ln += n + 1;
    p->lb = ln; ln += n + 1;
    p->la[n] = na;
    p->lb[n] = nb;
    for (i = n - 1; i >= 0; i--) {
      if (xcmp(yl[i], xl[d]))
        ya[--na] = yl[i];
      else
        yb[--nb] = yl[i];
      p->la[i] = na;
      p->lb[i] = nb;
    }
    p->l = __build(xl, yt, yl, d);
    p->r = __build(xl + d, yt + d, yl + d, n - d);
  } else {
    p->l = p->r = NULL;
  }

  return p;
}

/* NOTE: no duplicated points are allowed 
 *       only one range_tree can be maintained at a time */
range_tree *range_tree_build(point *p, int n) {
  range_tree *rt = &__rt;

  for (int i = 0; i < n; i++) {
    xs[i] = &p[i];
    ys[i] = &p[i];
  }
  sort(xs, xs + n, xcmp);
  sort(ys, ys + n, ycmp);
  ln = links;
  tn = ts;
  next = nodes;
  rt->root = n ? __build(xs, ys, yt, n) : NULL;

  for (int i = 0; i < n; i++)
    ys[i] = &p[i];
  sort(ys, ys + n, ycmp);
  rt->yl = ys;
  rt->size = n;

  return rt;
}

int __query(node *p, int x1, int x2, int lb, int ub) {
  if (!p || x2 < p->x1 || p->x2 < x1 || lb >= ub)
    return 0;

  if (x1 <= p->x1 && p->x2 <= x2)
    return ta_sum(p->ta, p->size, ub) - ta_sum(p->ta, p->size, lb);

  return __query(p->l, x1, x2, p->la[lb], p->la[ub]) + 
         __query(p->r, x1, x2, p->lb[lb], p->lb[ub]);
}

int range_tree_query(range_tree *rt, int x1, int x2, int y1, int y2) {
  int lb, ub;
  point a, b;
  node *root = rt->root;

  if (!root) return 0;

  a.x = -INF; a.y = y1; 
  b.x = +INF; b.y = y2;
  lb = lower_bound(rt->yl, rt->yl + rt->size, &a, ycmp) - rt->yl;
  ub = upper_bound(rt->yl, rt->yl + rt->size, &b, ycmp) - rt->yl;
  return __query(root, x1, x2, lb, ub);
}

int range_tree_add(range_tree *rt, int x, int y, int d) {
  int i;
  point a;
  node *p = rt->root;

  if (!p) return 0;

  a.x = x; a.y = y;
  i = lower_bound(rt->yl, rt->yl + rt->size, &a, ycmp) - rt->yl;
  if (i == rt->size || rt->yl[i]->x != x || rt->yl[i]->y != y)
    return 0;
  if ((ta_sum(p->ta, p->size, i) < ta_sum(p->ta, p->size, i + 1)) ^ (d < 0))
    return 0;

  while (p) {
    ta_add(p->ta, p->size, i + 1, d);
    if (p->size <= 1) break;
    if (p->la[i] != p->la[i + 1])
      i = p->la[i], p = p->l;
    else
      i = p->lb[i], p = p->r;
  }
  return 1;
}

/*********** main ***********/

struct query {
  char t;
  point p;
  int x1, y1, x2, y2;
};

int main() {
  char cmd[10];
  int n, m;
  static query qs[N];
  static point p[N];

  n = m = 0;
  while (scanf("%s", cmd) != EOF) {
    switch (qs[m].t = cmd[0]) {
      case 'A':
        scanf("%d %d", &qs[m].p.x, &qs[m].p.y);
        p[n++] = qs[m].p;
        break;
      case 'D':
        scanf("%d %d", &qs[m].p.x, &qs[m].p.y);
        break;
      case 'C':
        scanf("%d %d %d %d", &qs[m].x1, &qs[m].y1, &qs[m].x2, &qs[m].y2);
        break;
    }
    m++;
  }

  sort(p, p + n);
  n = unique(p, p + n) - p;
  range_tree *rt = range_tree_build(p, n);

  for (int i = 0; i < m; i++) {
    query *q = &qs[i];
    switch (q->t) {
      case 'A':
        if (range_tree_add(rt, q->p.x, q->p.y, +1)) {
          puts("ADDED");
        } else {
          puts("ALREADY EXISTS");
        }
        break;
      case 'D':
        if (range_tree_add(rt, q->p.x, q->p.y, -1)) {
          puts("DELETED");
        } else {
          puts("NOT FOUND");
        }
        break;
      case 'C':
        printf("%d\n", range_tree_query(rt, q->x1, q->x2, q->y1, q->y2));
        break;
    }
  }

  return 0;
}
\end{verbatim}


	\subsection{多边形和圆}
	%\subsubsection{k多边形面积交}
\begin{verbatim}
int n;
int v[MAXN]; // the number of vertexes
point p[MAXN][MAXV];
pair<double, int> c[MAXN * MAXV * 2];
double tot[MAXN + 1];

double pos(point p, line ln) {
    return dcmp(ln.second.X - ln.first.X) ?
        (p.X - ln.first.X) / (ln.second.X - ln.first.X) :
        (p.Y - ln.first.Y) / (ln.second.Y - ln.first.Y);
}

double area() {
    memset(tot, 0, sizeof(tot));
    for (int i = 0; i < n; ++i)
        for (int ii = 0; ii < v[i]; ++ii) {
            point A = p[i][ii], B = p[i][(ii + 1) % v[i]];
            line AB = line(A, B);
            int m = 0;
            for (int j = 0; j < n; ++j) if (i != j)
                for (int jj = 0; jj < v[j]; ++jj) {
                    point C = p[j][jj], D = p[j][(jj + 1) % v[j]];
                    line CD = line(C, D);
                    int f1 = dcmp(cross(A, B, C));
                    int f2 = dcmp(cross(A, B, D));
                    if (!f1 && !f2) {
                        if (i < j && dcmp(dot(dir(AB), dir(CD))) > 0) {
                            c[m++] = make_pair(pos(C, AB), 1);
                            c[m++] = make_pair(pos(D, AB), -1);
                        }
                    } else {
                        double s1 = cross(C, D, A);
                        double s2 = cross(C, D, B);
                        double t = s1 / (s1 - s2);
                        if (f1 >= 0 && f2 < 0) c[m++] = make_pair(t, 1);
                        if (f1 < 0 && f2 >= 0) c[m++] = make_pair(t, -1);
                    }
                }
            c[m++] = make_pair(0.0, 0);
            c[m++] = make_pair(1.0, 0);
            sort(c, c + m);
            double s = cross(A, B), z = min(max(c[0].first, 0.0), 1.0);
            for (int j = 1, k = c[0].second; j < m; ++j) {
                double w = min(max(c[j].first, 0.0), 1.0);
                tot[k] += s * (w - z);
                k += c[j].second;
                z = w;
            }
        }
    return tot[0];
}

/*
   tot[0] is the aera of union
   tot[n - 1] is the aera of intersection
   tot[k - 1] - tot[k] is the aera of region covered by k times
   */
\end{verbatim}

	%\subsubsection{Point}
\begin{verbatim}
/* Spoj CIRUT
 * Given n circles, find the area of all k-union regions
 * NOTE: No duplicated circles are allowed!
 * O(n ^ 2 log(n)) */
#include <stdio.h>
#include <string.h>
#include <algorithm>
#include <math.h>

#define Sqr(x) (x)*(x)

using namespace std;

const double eps = 1e-8, inf = 1e+9, pi = acos(-1.0);

inline int sign(double x) { return x < -eps ? -1 : x > eps;}

struct point {
  double x, y;
};

struct circle {
  point o;
  double r;
};

struct event {
  double a;
  int t;
  point p;
  event() {}
  event(double a, int t, point p) : a(a), t(t), p(p) {}
  bool operator<(const event e) const { return a < e.a; }
};

inline int ip_circle_circle(const circle &c1, const circle &c2, point &p1, point &p2) {
  double mx = c2.o.x - c1.o.x, sx = c2.o.x + c1.o.x, mx2 = sqr(mx);
  double my = c2.o.y - c1.o.y, sy = c2.o.y + c1.o.y, my2 = sqr(my);
  double sq = mx2 + my2, d = -(sq - sqr(c1.r - c2.r)) * (sq - sqr(c1.r + c2.r));
  if (!sign(sq)) return 0;
  if (d + eps < 0) return 0;
  if (d < eps) d = 0; else d = sqrt(d);
  double x = mx * ((c1.r + c2.r) * (c1.r - c2.r) + mx * sx) + sx * my2;
  double y = my * ((c1.r + c2.r) * (c1.r - c2.r) + my * sy) + sy * mx2;
  double dx = mx * d, dy = my * d; sq *= 2;
  p1.x = (x + dy) / sq; p1.y = (y - dx) / sq;
  p2.x = (x - dy) / sq; p2.y = (y + dx) / sq;
  return d > eps ? 2 : 1;
}

inline double fix(double a, double b = 0) {
  a -= b;
  if (sign(a) < 0) a += 2 * pi;
  return a;
}

inline double angle(point a, point b) {
  return fix(atan2(b.y - a.y, b.x - a.x));
}

inline int contains(const circle &c1, const circle &c2) {
  return c1.r > c2.r && sign(sqr(c1.o.x - c2.o.x) + sqr(c1.o.y - c2.o.y) - sqr(c1.r - c2.r)) <= 0;
}

inline double cross(point a, point b, point c) {
  return (b.x - a.x) * (c.y - b.y) - (b.y - a.y) * (c.x - b.x);
}

const int N = 1000 + 10;
int n, en;
circle cs[N];
event events[N + N];
point o;
double area[N];

int main() {
  scanf("%d", &n);
  for (int i = 0; i < n; i++)
    scanf("%lf %lf %lf", &cs[i].o.x, &cs[i].o.y, &cs[i].r);

  memset(area, 0, sizeof area);
  for (circle *a = cs; a < cs + n; a++) {
    int cover = 1;
    en = 0;
    for (circle *b = cs; b < cs + n; b++) if (a != b) {
      if (contains(*b, *a)) cover++;
      point p1, p2;
      if (ip_circle_circle(*a, *b, p1, p2) >= 2) {
        events[en++] = event(angle(a->o, p1), -sign(cross(a->o, b->o, p1)), p1);
        events[en++] = event(angle(a->o, p2), -sign(cross(a->o, b->o, p2)), p2);
        if ((events[en - 2].a < events[en - 1].a) ^ (events[en - 2].t > events[en - 1].t)) cover++;
      }
    }
    sort(events, events + en);
    events[en] = events[0];
    for (int i = 0; i < en; i++) {
      event *e1 = &events[i];
      event *e2 = &events[i + 1];
      cover += e1->t;
      double da = fix(e2->a, e1->a);
      area[cover] += cross(o, e1->p, e2->p) + sqr(a->r) * (da - sin(da));
    }
    if (!en) area[cover] += sqr(a->r) * pi * 2;
  }
  for (int i = 1; i < n; i++)
    area[i] -= area[i + 1];

  for (int i = 1; i <= n; i++)
    printf("[%d] = %.3lf\n", i, area[i] / 2);

  return 0;
}
\end{verbatim}

	%\subsubsection{圆与多边形面积交}
\begin{verbatim}
Point p[3];
double r;

double cross(Point a, Point b){
	return a.X * b.Y - a.Y * b.X;
}

double cross(Point a, Point b, Point c){
	return cross(b - a, c - a);
}

double dot(Point a, Point b){
	return a.X * b.X + a.Y * b.Y;
}

double dot(Point a, Point b, Point c){
	return dot(b - a, c - a);
}

double len(Line l){
	return abs(l.S - l.F);
}

double dis(Point p, Line l){
	return fabs(cross(p, l.F, l.S) / len(l));
}

bool inter(Line a, Line b, Point &p){
	double s1 = cross(a.F, a.S, b.F);
	double s2 = cross(a.F, a.S, b.S);
	if (!sign(s1 - s2)) return false;
	p  = (s1 * b.S - s2 * b.F) / (s1 - s2);
	return true;
}

Vec dir(Line l){
	return l.S - l.F;
}

Vec normal(Vec v){
	return Vec(-v.Y, v.X);
}

Vec unit(Vec v){
	return v / abs(v);
}

bool onseg(Point p, Line l){
	return sign(cross(p, l.F, l.S)) == 0 && sign(dot(p, l.F, l.S)) <= 0;
}

double arg(Vec a, Vec b){
	double d = arg(b) - arg(a);
	if (d > PI) d -= 2 * PI;
	if (d < -PI) d += 2 * PI;
	return d;
}

double area(Point a, Point b){
	double s1 = 0.5 * cross(a, b);
	double s2 = 0.5 * arg(a, b) * r * r;
	return fabs(s1) < fabs(s2) ? s1 : s2;
}

double area(){
	double s = 0;
	rep(i, n){
		Point O(0, 0), A = p[i], B = p[(i + 1) % 3];
		Line AB(A, B);
		double d = dis(O, AB);
		if (sign(d - r) >= 0){
			s += area(A, B);
		}
		else{
			Point P;
			inter(AB, Line(O, O + normal(dir(AB))), P);
			Vec v = sqrt(r * r - d * d) * unit(dir(AB));
			Point P1 = P - v, P2 = P + v;
			if (!onseg(P1, AB) && !onseg(P2, AB)){
				s += area(A, B);
			}
			else{
				s += area(A, P1);
				s += area(P1, P2);
				s += area(P2, B);
			}
		}
	}
	return fabs(s);
}

void init(){
	scanf("%d%d", &n, &r);
	rep(i, n){
		double x, y;
		scanf("%lf%lf", &x, &y);
		p[i] = Point(x, y);
	}
}

int main(){
	init();
	printf("%.12lf\n", area());
	return 0;
}
\end{verbatim}


	\subsection{其他}
	%\input{code/geometry/minium_circle_cover}
    %\input{code/geometry/rectangular}
	%\input{code/geometry/unionrectange}
	%\input{code/geometry/circle_tree}

\section{\LARGE 理论}
	\subsection{数学}
    %\subsubsection{数学结论}
\begin{verbatim}
五边形定理
五边形数 n * (3 * n +- 1) / 2
(1-x)*(1-x^2)*(1-x^3)....=sigma{(-1)^k * x^(n * (3 * n (+-) 1) / 2)}
即f[n] = f[n - 1] + f[n - 2] - f[n - 5] - f[n - 7] + f[n - 12] + f[n - 15] - .....

fibonacci数性质:
f[n] = f[n - 1] + f[n - 2]
f[n + m + 1] = f[n] * f[m] + f[n + 1] * f[m + 1]
gcd(f[n], f[n + 1]) = 1
gcd(f[n], f[n + 2]) = 1
gcd(f[n], f[m]) = f[gcd(n, m)]
f[n+1]*f[n+1]-f[n]*f[n+2] = (-1)^n
sigma{f[i]^2, 1<=i<=n} = f[n]*f[n+1]
sigma{f[i], 0<=i<=n} = f[n+2] - 1
sigma{f[2*i-1],1<=i<=n} = f[2*n]
sigma{f[2*i],1<=i<=n} = f[2*n+1]-1
sigma{(-1)^i*f[i],0<=i<=n} = (-1)^n*(f[n+1]-f[n])+1
f[2*n-1]=f[n]^2-f[n-2]^2
f[2*n+1]=f[n]^2+f[n+1]^2
3*f[n]=f[n+2]+f[n-2]
f[n]=c(n-1,0)+c(n-2,1)+..c(n-1-m,m) (m<=n-1-m)
sigma{f[i]*i,1<=i<=n}=n*f[n+2]-f[n+3]+2

catalan数性质:
凸多边形三角剖分数
简单有序根树的计数
(0,0)走到(n,n)经过的点(a,b)满足a<=b的路径数
乘法结合问题
c[n+1] = (4 * n - 2) / (n + 1) * c[n]
c[n] = (2*n)!/(n!)/((n+1)!)

第一类stirling数性质
有正有负,其绝对值是n个元素的项目分作k个环排列的数量,s[n,k](n个人分成k组,每组再按特定顺序围圈)
s[n][0] = 0, s[1][1] = 1;
s[n+1][k]= = s[n][k - 1] + n * s[n][k]
|s[n][1]| = (n-1)!
s[n][k] = (-1)^(n+k)*|s[n][k]|
s[n][n-1] = -C(n,2)
x*(x-1)*(x-2)..(x-n+1) = sigma{s[n][k] * x ^k}

第二类stirling数性质
n个元素的集定义k个等价类的方法数目(n个人分成k组的方法数)
s[n][n] = s[n][1] = 1
s[n][k] = s[n - 1][k - 1] + k * s[n - 1][k]
s[n][n - 1] = C(n, 2)
s[n][2] = 2^(n-1)-1
s[n][k] = 1/(k!)sigma{(-1)^k-j * C(k, j) * j ^n, 1<=j<=k}

bell数性质
B[n] = sigma{s[n][k], 1<=k<=n}
B[n+1] = simga{C(n,k)*B[k], 0<=k<=n}
B[p+n] = B[n] + B[n + 1] (mod p)
B[p^m+n] = B[n] + B[n+1] (mod p)

多项式性质
f(x)不存在重根<=>gcd(f(x), f‘(x))的次数小于1次
多项式gcd可以用来判断两多项式是否有公共根

多项式取模
f[x] = 0 (mod m) 
m = m1 * m2 * m3 ... mk
Ti 表示 f[x] = 0 (mod mi)的解数,则T = T1 * T2 * T3...Tk

数论
a^n % b = a^(n % phi(b) + phi(b)) % b (n >= phi(b))
lucas定理 c(n, m) = c(n % p, m % p) * c(n / p, m / p) % p
lucas函数 满足 f(n, m) = f(n % p, m % p) * f(n / p, m / p) % p, 可以猜测满足

原根
2,4,p^k,2*p^k存在原根,存在原根则原根数量为phi(phi(n))
验证原根x = phi(n), x = p1^a1*p2^a2..pk^ak
原根满足t ^ (x / pi) != 1 (mod n)

x*x+y*y==n的整数解:
x*x+y*y==n的整数解个数num = 4 * sigma{H(d), d | n}
H(d) =
(1) 奇数 : (-1)^((d-1)/2)
(2) 偶数 : 0

平方和定理:
(1)费马平方和定理:
    奇质数能表示为两个平方数之和的充分必要条件是该素数被4除余1
(2)费马平方和定理的拓展定理:
    正整数能表示为两平方数之和的充要条件是在它的标准分解式中,形如素因子的指数是偶数
(3)Brahmagupta–Fibonacci identity
    如果两个整数都能表示为两个平方数之和,则它们的积也能表示为两个平方数之和。公式及拓展公式为
\end{verbatim}
$$(a^{2}+b^{2})(c^{2}+d^{2})=(ac-bd)^{2}+(ad+bc)^{2}=(ac+bd)^{2}+(ad-bc)^{2}$$
$$(a^{2}+n*b^{2})(c^{2}+n*d^{2})=(ac-n*bd)^{2}+n*(ad+bc)^{2}=(ac+n*bd)^{2}+n(ad-bc)^{2}$$
\begin{verbatim}
    从这个定理可以看出:如果不能表示为三个数的平方和,那么也就不能表示为两个数的平方和。
(4)四平方和定理:
    每个正整数都可以表示成四个整数的平方数之和
(5)表为3个数的平方和条件: 
    正整数能表示为三个数的平方和的充要条件是不能表示成的形式,其中和为非负 整数。

连分数
连分数(a+(n^0.5)) / b
开始时,i满足,(a+i)/b=floor((a+(n^0.5))/b),之后过程一样
如果不成功,则可以变换为(ab+((nb^2)^0.5))/(b^2),之后再来

杨氏矩阵又叫杨氏图表,它是这样一个矩阵,满足条件:

 
杨氏矩阵
(1)如果格子(i,j)没有元素,则它右边和上边的相邻格子也一定没有元素。
(2)如果格子(i,j)有元素a[i][j],则它右边和上边的相邻格子要么没有元素,要么有元素且比a[i][j]大。
1 ~ n所组成杨氏矩阵的个数可以通过下面的递推式得到:
f[1] = 1; f[2] = 2;
f[n] = f[n - 1] + (n - 1) * f[n - 2];

钩子公式:
对于给定形状,不同的杨氏矩阵的个数为:n!除以每个格子的钩子长度加1的积。其中钩子长度定义为该格子
右边的格子数和它上边的格子数之和。
\end{verbatim}


	\subsection{数论}
    %\input{code/theory/number_conclusion}
	%\input{code/theory/mod}
    %\subsubsection{中国剩余定理(非互质)}
\begin{verbatim}
long long exgcd(long long a,long long b,long long &x,long long &y) {
	if (!a){
		x = 0;
		y = 1;
		return b;
	}
	LL g = exgcd(b % a, a, x, y);
	LL t = y;
	y = x;
	x = t - (b / a) * y;
    return g;
}

long long CRT(const vector<long long>& m,const vector<long long>& b,long long& lcm) {
    bool flag = false;
    long long x, y, i,d,result,a1,m1,a2,m2,Size=m.size();
    m1 = m[0]; a1 = b[0];
    for(i = 1; i < Size; ++i){
        m2 = m[i]; a2 = b[i];
        d = exgcd(m1, m2, x, y );
        if((a2-a1) % d != 0) flag = true;
        result = (x * ((a2-a1) / d ) % m2 + m2 ) % m2;
        a1 = a1 + m1 * result; //对于求多个方程
        m1 = (m1 * m2) / d;    //lcm(m1,m2)最小公倍数
        a1 = (a1 % m1 + m1) % m1;
    }
    lcm = m1;
    if (flag) return -1;
    else return a1;
}
\end{verbatim}

    %\input{code/theory/extended_gcd}
    %\input{code/theory/euler}
	%\input{code/theory/mu}
    %\input{code/theory/head}
	%\input{code/theory/detmod}

	\subsection{博弈论}
	%\input{code/theory/sg}

    \subsection{特殊数列}
    %\input{code/theory/fibonacci}
    %\input{code/theory/catalan}
    %\input{code/theory/stirling}
	%\input{code/theory/schroder}

\section{\LARGE 其他}
    %\input{code/other/biginteger}
    %\subsubsection{FFT}
\begin{verbatim}
typedef complex<long double> Comp;
class FFT {
    public:
        FFT(int n);
        void forward(Comp a[]) {
            compute(a, r);
        }
        void reverse(Comp a[]){
            compute(a, ir);
            for (int i = 0; i < n; i++) a[i] /= n;
        }
    private:
        int n, p;
        vector<int> rb;
        Comp r[20];
        Comp ir[20];
        void compute(Comp a[], Comp* r);
};

FFT::FFT(int n) : n(n) , rb(n) , p(0) {
    while ((1 << p) < n) ++p;
    for(int i = 0; i < n; i++){
        int x = i, y = 0;
        for (int j = 0; j < p; ++j) {
            y = (y << 1) | (x & 1);
            x >>= 1;
        }
        rb[i] = y;
    }
    for(int i = 0; i < p; i++){
        long double angle = 2 * PI / (1 << (i + 1));
        ir[i] = Comp(cos(angle), sin(angle));
        r[i] = std::conj(ir[i]);
    }
}

void FFT::compute(Comp a[], Comp* r) {
    for (int i = 0; i < n; ++i) if (rb[i] > i) swap(a[i], a[rb[i]]);
    for (int len = 2; len <= n; len <<= 1) {
        Comp root = *r++;
        for (int i = 0; i < n; i += len) {
            Comp w(1, 0);
            for (int j = 0; j < len / 2; ++j) {
                Comp u = a[i + j];
                Comp v = a[i + j + len / 2] * w;
                a[i + j] = u + v;
                a[i + j + len / 2] = u - v;
                w *= root;
            }
        }
    }
}
\end{verbatim}

    %\input{code/other/c++note}
    %\input{code/other/editplus}
    %\input{code/other/vim}

%CJK的时候使用
%\end{CJK}
\end{document}
