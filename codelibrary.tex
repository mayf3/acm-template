\def\ChineseScale{1200}
\documentclass[12pt,a4paper,titlepage]{article}
\usepackage{times}

%\usepackage{CJK}
%CJK中文支持

%=========================xeCJK设置==============================
\usepackage{xeCJK}
%xeCJK中文支持

\def\CJKecglue{\hskip 0.15em}
%设置 CJK 文字与西文、文字与行内数学公式之间的间距, CJK默认值是一个空格。

\setCJKmainfont{Hiragino Sans GB}
\setCJKmonofont{Hiragino Sans GB}
%设置正文等宽族的 CJK 字体, 影响 \ttfamily 和 \texttt 的字体。 为了有利于等宽字体的 代码对齐等情形, xeCJK 在 ⟨font features⟩ 里增加了 Mono 这个选项。

%\setmainfont{Courier}
%================================================================

\usepackage{latexsym,bm}
\usepackage{indentfirst}
%自动首行缩进
\usepackage[top=1in, bottom=1in, left=0.5in, right=0.5in]{geometry}
\usepackage[CJKbookmarks,colorlinks=true]{hyperref}

%CJKbookmarks支持中文标签, colorlinks=true 就是把超链接的边框去掉,但字是红色的
%用来调页边距
\author{mayf3}
\title{Standard Source code Library}

\begin{document}

%CJK的时候使用
%\begin{CJK}{UTF8}{gbsn}

\maketitle
\tableofcontents

\newpage
\section{\LARGE 图论}
	\subsection{最短路算法}
    %\subsubsection{Dijkstra最短路}
\begin{verbatim}
#include <cstdio>
#include <cstring>
#include <algorithm>
#include <map>

/*
 * name     :     dijkstra(STL)
 * usage     :    single-source shortest path(only non-negative weight)
 * develop    :    small label first optimization, negative circle
 * space complexity    :    O(M)
 * time complexity    :    O(NlogN)
 * checked    :    no
 */

const int N = 111111; //number of the vertices

int n, m;
int dist[N];
vector<PII> E[N];

int calc(int s, int e) {
    priority_queue<PII, vector<PII>, greater<PII> > Q;
    rep(i, n) dist[i] = -1;
    dist[s] = 0;
    Q.push(MP(0, s));
    int x, y, cost;
    while (!Q.empty()) {
        x = Q.top().Y, cost = Q.top().X;
        Q.pop();
        if (cost > dist[x]) continue;
        rep(i, E[x].size()){
            y = E[x][i].X, cost = E[x][i].Y;
            if (dist[y] == -1 || dist[y] != -1 && dist[x] + cost > dist[y]){
                dist[y] = dist[x] + cost;
                Q.push( make_pair(dist[y], y) );
            }
        }
    }
    return dist[e];
}

int main(){
    while(~scanf("%d%d", &n, &m)){
        rep(i, n) E[i].clear();
        int x, y, c;
        rep(i, m){
            scanf("%d%d%d", &x, &y, &c);
            x--, y--;
            E[x].PB(MP(y, c));
            E[y].PB(MP(x, c));
        }
        printf("%d\n", calc(0, n - 1));
    }
    return 0;
}
\end{verbatim}

    %\subsubsection{spfa最短路}
\begin{verbatim}
#include <cstdio>
#include <cstring>
#include <algorithm>
#include <queue>

/*
 * name     :     spfa
 * usage     :    single-source shortest path, differential restraint system
 * develop    :    small label first optimization, negative circle
 * space complexity    :    O(M)
 * time complexity    :    O(k * M) (where k is usually less than 2)
 * checked    :    no
 */

const int N = 10000;

int n, m;
vector<PII> E[N];
int dist[N];

int spfa(int s, int e){
    static deque<int> Q;
    static bool inque[N];
    Cls(inque);
    memset(dist, -1, sizeof dist);
    dist[s] = 0;
    Q.PB(s);
    inque[s] = true;
    int x, y, c;
    while(!Q.empty()){
        x = Q.front();
        Q.pop_front();
        inque[x] = false;
        rep(i, E[x].size()){
            y = E[x][i].X;
            c = E[x][i].Y;
            if (dist[y] == -1 || dist[y] != -1 && dist[y] > dist[x] + c){
                dist[y] = dist[x] + c;
                if (!inque[y]){
                    Q.PB(y);
                    inque[y] = true;
                }
            }
        }
    }
    return dist[e];
}

int main(){
    while(~scanf("%d%d", &n, &m)){
        rep(i, n) E[i].clear();
        int x, y, c;
        rep(i, m){
            scanf("%d%d%d", &x, &y, &c);
            x--, y--;
            E[x].PB(MP(y, c));
            E[y].PB(MP(x, c));
        }
        printf("%d\n", spfa(0, n - 1));
    }
    return 0;
}
\end{verbatim}

	\subsubsection{k最短路(无环)}
\begin{verbatim}
#include <cstdio>
#include <cstring>
#include <algorithm>
#include <map>

using namespace std;

const int MAXN = 50 + 10; //number of vertices
const int MAXK = 200 + 10;
const int INF = 1000000000; //max dist

struct Tpath {
    int cnt, len, pos;
    int v[MAXN];
};

Tpath path[MAXK];
int g[MAXN][MAXN];
int len[MAXK], pos[MAXK], ans[MAXK];
bool used[MAXN];
int dist[MAXN], prev[MAXN], List[MAXN];
int N, M, K, S, T, cnt;

void Dijkstra() {
    int visited[MAXN];
    for (int i = 0; i <= N; ++i) dist[i] = INF, visited[i] = 0;
    dist[T] = 0;
    for (int k, i = T; i != N; i = k) {
        visited[i] = 1; k = N;
        for (int j = 0; j < N; ++j) {
            if (visited[j] || used[j]) continue;
            if (g[j][i] > -1 && dist[i] + g[j][i] < dist[j]) {
                dist[j] = dist[i] + g[j][i];
                prev[j] = i;
            }
            if (dist[j] < dist[k]) k = j;
        }
    }
}

void setPath(int v, Tpath &p) {
    p.len = 0;
    while (1) {
        p.v[p.cnt++] = v;
        if (v == T) return;
        p.len += g[v][prev[v]]; v = prev[v];
    }
}

void solve() {
    memset(used, 0, sizeof(used));
    Dijkstra();
    memset(ans, -1, sizeof(ans));
    if (dist[S] == INF)    return;
    multimap<int, int> Q; Q.clear();
    path[0].cnt = 0; path[0].pos = 0; setPath(S, path[0]); Q.insert( make_pair(path[0].len, 0) );
    int tot = 1;
    for (int i = 0; i < K; ++i) {
        if (Q.empty()) return;
        multimap<int, int> :: iterator p = Q.begin();
        int x = (*p).second;
        ans[i] = path[x].len;
        if (i == K - 1) break;
        memset(used, 0, sizeof(used));
        Tpath cur; cur.cnt = 0; cur.len = 0;
        for (int sum = 0, j = 0; j + 1 < path[x].cnt; ++j) {
            cur.v[cur.cnt++] = path[x].v[j]; used[path[x].v[j]] = 1;
            if (j) sum += g[path[x].v[j - 1]][path[x].v[j]];
            if (j >= path[x].pos) {
                Dijkstra();
                int u = path[x].v[j];
                for (int v = 0; v < N; ++v)
                    if (g[u][v] > -1 && !used[v] && dist[v] < INF && v != path[x].v[j + 1]) {
                        Tpath tp = cur; tp.pos = j + 1; setPath(v, tp); tp.len += sum + g[u][v];
                        if (tot < K) path[tot] = tp, Q.insert( make_pair(tp.len, tot++) );
                        else {
                            multimap<int, int> :: iterator p = Q.end(); --p;
                            if (tp.len >= (*p).first) continue;
                            path[(*p).second] = tp; Q.insert( make_pair(tp.len, (*p).second) );
                            Q.erase(p);
                        }
                    }
            }
        }
        Q.erase(p);
    }
}

void DFS(int step, int u, int len) {
    if (!cnt) return;
    if (u == T) {
        if (len == ans[K - 1]) {
            if (!(--cnt)) {
                for (int j = 0; j < step; ++j) {
                    if (j) printf("-");
                    printf("%d", List[j] + 1);
                }
                printf("\n");
            }
        }
        return;
    }
    Dijkstra();
    int tmp[MAXN];
    for (int i = 0; i < N; ++i) tmp[i] = dist[i];
    for (int i = 0; i < N; ++i)
        if (g[u][i] > -1 && !used[i] && tmp[i] < INF && len + g[u][i] + tmp[i] <= ans[K - 1]) {
            used[i] = 1; List[step] = i;
            DFS(step + 1, i, len + g[u][i]);
            if (!cnt) return;
            used[i] = 0;
        }
}

int main() {
    scanf("%d%d%d%d%d", &N, &M, &K, &S, &T);
    --S; --T;
    if (S == T) ++K;
    memset(g, -1, sizeof(g));
    for (int i = 0; i < M; ++i) {
        int u, v, w;
        scanf("%d%d%d", &u, &v, &w);
        --u; --v;
        g[u][v] = w;
    }
    solve();
    if (ans[K - 1] == -1) printf("None\n");
    else {
        cnt = 0;
        for (int i = 0; i < K; ++i)
            if (ans[i] == ans[K - 1]) ++cnt;
        memset(used, 0, sizeof(used));
        used[S] = 1; List[0] = S;
        DFS(1, S, 0);
    }
    return 0;
}

\end{verbatim}

	\input{code/graph/kth_shortest_path}

    \subsection{生成树}
	\input{code/graph/mst}
    %\input{code/graph/krusal}
    %\subsubsection{prim最小生成树}
\begin{verbatim}
const int MAXN = 2000 + 1; //number of vertices + 1
const int MAXM = 10000; //number of edges
const int INF = 2000000000; //max weight

struct Tedge {
    int v, w, next;
};

Tedge edge[MAXM * 2];
int first[MAXN], dist[MAXN], heap[MAXN], pos[MAXN];
bool used[MAXN];
int N, M, cnt;

void init() {
    memset(first, -1, sizeof(first));
    M = 0;
}

inline void add_edge(int u, int v, int w) {
    edge[M].v = v; edge[M].w = w; edge[M].next = first[u]; first[u] = M++;
}

inline void moveup(int i) {
    int key = heap[i];
    while (i > 1 && dist[heap[i >> 1]] > dist[key]) heap[i] = heap[i >> 1], pos[heap[i]] = i, i >>= 1;
    heap[i] = key; pos[key] = i;
}

inline void movedown(int i) {
    int key = heap[i];
    while ((i << 1) <= cnt) {
        int j = i << 1;
        if (j < cnt && dist[heap[j + 1]] < dist[heap[j]]) ++j;
        if (dist[key] <= dist[heap[j]]) break;
        heap[i] = heap[j]; pos[heap[i]] = i; i = j;
    }
    heap[i] = key; pos[key] = i;
}

void Prim() {
    memset(used, 0, sizeof(used));
    for (int i = 0; i < N; ++i) pos[i] = -1, dist[i] = INF;
    cnt = 1; heap[1] = 0; dist[0] = 0;
    while (cnt) {
        int u = heap[1];
        used[u] = 1;
        heap[1] = heap[cnt--];
        movedown(1);
        for (int i = first[u]; i != -1; i = edge[i].next) {
            int v = edge[i].v, w = edge[i].w;
            if (!used[v] && w < dist[v]) {
                dist[v] = w;
                if (pos[v] == -1) pos[v] = ++cnt, heap[cnt] = v;
                moveup(pos[v]);
            }
        }
    }
}
\end{verbatim}

    %\subsubsection{prim(stl)最小生成树}
\begin{verbatim}
const int MAXN = 2000; //number of vertices
const int INF = 2000000000; //max weight

vector < pair<int, int> > edge[MAXN];
int dist[MAXN];
bool used[MAXN];
int N;

void Prim() {
	priority_queue < pair<int, int>, vector< pair<int, int> >, greater< pair<int, int> > > Q;
	memset(used, 0, sizeof(used));
	for (int i = 0; i < N; ++i) dist[i] = INF;
	dist[0] = 0; Q.push( make_pair(0, 0) );
	while (!Q.empty()) {
		int u = Q.top().second, d = Q.top().first;
		Q.pop();
		used[u] = 1;
		if (d > dist[u]) continue;
		for (int i = 0; i < edge[u].size(); ++i) {
			int v = edge[u][i].first, w = edge[u][i].second;
			if (w < dist[v]) {
				dist[v] = w;
				Q.push( make_pair(dist[v], v) );
			}
		}
	}
}
\end{verbatim}

	\input{code/graph/aroad}
	\input{code/graph/mdst}
	\subsubsection{最小树形图,N*N}
\begin{verbatim}
const int MAXN = 1000 + 10; //number of vertices
const int INF = 1000000000;

struct Edge {
	int u, c;
	friend bool operator<(const Edge &a, const Edge &b) { return a.c < b.c; }
};

Edge edge[MAXN][MAXN];
int top[MAXN], add[MAXN];

int set_find(int *s, int x) {
	return s[x] == x ? x : s[x] = set_find(s, s[x]);
}

inline void set_union(int *s, int x, int y) {
	int fx = set_find(s, x), fy = set_find(s, y);
	if (fx != fy) s[fy] = fx;
}

void merge(int a, int b) {
	static Edge ret[MAXN];
	static bool used[MAXN];
	int size = 0;
	memset(used, 0, sizeof(used));
	for (int i = 0; i < top[a]; ++i) edge[a][i].c += add[a];
	for (int i = 0; i < top[b]; ++i) edge[b][i].c += add[b];
	while (top[a] > 0 || top[b] > 0) {
		int id = (top[a] > 0 && (top[b] == 0 || edge[a][top[a] - 1].c > edge[b][top[b] - 1].c)) ? a : b;
		if (!used[edge[id][top[id] - 1].u]) {
			ret[size] = edge[id][top[id] - 1];
			used[ret[size++].u] = true;
		}
		--top[id];
	}
	top[a] = size;
	for (int i = 0; i < size; ++i) edge[a][i] = ret[size - i - 1];
	add[a] = 0;
}

// find maximal branchings, O(N^2)
// make sure that there exists a solution
// delete all edge(i, root) from c to find a rooted solution
int optimum_branchings(int c[MAXN][MAXN], int N) {
	static int S[MAXN], W[MAXN];
	static Edge enter[MAXN];
	memset(add, 0, sizeof(add));
	memset(top, 0, sizeof(top));
	for (int j = 0; j < N; ++j) {
		for (int i = 0; i < N; ++i)
			if (c[i][j] < INF) {
				edge[j][top[j]].u = i;
				edge[j][top[j]++].c = c[i][j];
			}
		sort(edge[j], edge[j] + top[j]);
		S[j] = W[j] = j;
		enter[j].u = -1;
	}
	int ret = 0;
	for (int u = 0; u < N; ) {
		if (top[u] == 0) {
			++u;
			continue;
		}
		Edge e = edge[u][--top[u]], t;
		e.c += add[u];
		if (set_find(S, e.u) != u) {
			if (set_find(W, e.u) != set_find(W, u)) {
				set_union(W, e.u, u);
				enter[u++] = e;
			}
			else {
				int val = e.c, v;
				for (t = enter[set_find(S, e.u)]; t.u != -1; t = enter[set_find(S, t.u)]) val = min(val, t.c);
				add[u] += val - e.c;
				e.c -= val;
				for (v = set_find(S, e.u), t = enter[v]; t.u != -1; v = set_find(S, t.u), t = enter[v]) {
					add[v] += val - t.c;
					merge(u, v);
					set_union(S, u, v);
				}
			}
			ret += e.c;
		}
	}
	return ret;
}
\end{verbatim}

	%\input{code/graph/mdst_NM}

	\subsection{网络流}
	Graph modification :
Given graph G, we have to find the flow from node s to t so that:
F[u][v]: amount of flow passes through edge(u, v).
F[u][v] must satisfy: D[u][v] <= F[u][v] <= C[u][v].

We add two new nodes s' and t' into the graph and rebuild the graph G to make new graph G':
Each edge(u, v) in the graph G will correspond to 3 edges in graph G': edge(s', v) with capacity = D[u][v], edge(u,  t') with capacity = D[u][v] and edge(u, v) with capacity = C[u][v] - D[u][v].
In addition , we add new edge(t, s) with capacity = infinity ( i.e very very big ).

Theorem: There exists the flow in graph G that satisfy D[u][v] <= F[u][v] <= C[u][v] if and only if there exists one flow from s' to t' in graph G' = Sum( D[u][v] ) with edge(u, v) in G .

It's clear that if there exists such flow in graph G' , it must be the maxflow from s' to t'. So we only have to check whether maxflow from s' to t' is equal to Sum ( D[u][v] ) or not.
If there exists such one, now to define the flow graph in G, we just add D[u][v] into F'[u][v] ( F'[u][v]: amount of flow passes through edge(u,v) in graph G' ) and F[u][v] = F'[u][v] + D[u][v].

Maximum feasible flow: Modify C[v][u] into D[u][v], then use any maxflow algorithm of ZERO lower bounds from S to T to get the Maximum feasible flow.

Minimum feasible flow: Modify C[v][u] into D[u][v], then use any maxflow algorithm of ZERO lower bounds from T to S to get the Minimum feasible flow.
	\input{code/graph/max_flow}
	\input{code/graph/max_flow_fast}
	\subsubsection{最大流(快)}
\begin{verbatim}
const int inf = 0x3f3f3f3f;
const LL infLL = 0x3f3f3f3f3f3f3f3fLL;

const int maxn = 100000 + 5;
const int maxe = 200000 + 5;

class SAP {
	int n, psz, s, t;
	struct Edge {
		int v, r;
		Edge *next, *cp;
	} epool[maxe], *e[maxn], *cur[maxn];
	int dis[maxn], gap[maxn], pre[maxn];

	void bfs() {
		queue<int> que;
		memset(dis, 0x3f, sizeof(dis));
		memset(gap, 0, sizeof(gap));
		dis[t] = 0; que.push(t);
		while (!que.empty()) {
			int u = que.front(); que.pop(); ++gap[dis[u]];
			for (Edge *i = e[u]; i; i = i->next) {
				int v = i->v;
				if (i->cp->r && dis[v] == inf) {
					dis[v] = dis[u] + 1;
					que.push(v);
				}
			}
		}
	}

	int aug(int &u) {
		if (u == t) {
			int d = inf;
			for (int i = s; i != t; i = cur[i]->v)
				if (cur[i]->r < d) d = cur[i]->r, u = i;
			for (int i = s; i != t; i = cur[i]->v)
				cur[i]->r -= d, cur[i]->cp->r += d;
			return d;
		}
		for (Edge *i = e[u]; i; i = i->next) {
			int v = i->v;
			if (i->r && dis[u] == dis[v] + 1) {
				cur[u] = i; pre[v] = u; u = v;
				return 0;
			}
		}
		if (--gap[dis[u]]) {
			int d = n;
			for (Edge *i = e[u]; i; i = i->next) {
				int v = i->v;
				if (i->r && dis[v] < d) d = dis[v], cur[u] = i;
			}
			++gap[dis[u]=d+1];
			if (u != s) u = pre[u];
		} else {
			dis[s] = n;
		}
		return 0;
	}
	public:
	void init(int n) {
		this->n = n; psz = 0;
		memset(e, 0, sizeof(e));
	}
	void add_edge(int u, int v, int w) {
		Edge *i = epool + psz++;
		i->v = v; i->r = w; i->next = e[u]; e[u] = i;
		i->cp = epool + ((psz - 1) ^ 1);
		if (psz & 1) add_edge(v, u, w);
	}
	int max_flow(int s, int t) {
		int flow = 0;
		this->s = s, this->t = t;
		bfs();
		// memset(dis, 0, sizeof(dis));
		// memset(gap, 0, sizeof(gap)); gap[0] = n;
		for (int i = 0; i < n; ++i) cur[i] = e[i];
		for (int i = s; dis[s] < n; flow += aug(i));
		return flow;
	}
} graph;
\end{verbatim}

	\input{code/graph/min_cost_max_flow}
	\input{code/graph/min_cost_max_flow_fast}
    %\input{code/graph/networkflow_limit}
    \input{code/graph/euler_circuit}
	\subsubsection{stoer wagner最小割集}
\begin{verbatim}
const int MAXN = 50 + 10; //number of vertices
const int MAXM = 500; //number of MinCut edges
const int INF = 1000000000; //max capacity

int map[MAXN][MAXN], a[MAXN][MAXN], idx[MAXN][MAXN]; //map, tmp map, idx of edge
int root[MAXN], q[MAXN], w[MAXN], pre[MAXN];
int list[MAXM]; //MinCut Edges
bool used[MAXN];
int N, M;

int mincut(int n) {
    memset(used, 0, sizeof(used));
    memset(w, 0, sizeof(w));
    int last, cnt = 0;
    for (int k, i = 0; i != n; i = k) {
        last = i; used[i] = 1; k = n;
        for (int j = 0; j < n; ++j) {
            if (used[j]) continue;
            w[j] += a[q[i]][q[j]]; pre[j] = i;
            if (w[j] > w[k]) k = j;
        }
    }
    return last;
}

int find(int x) {
    if (root[x] == x) return x;
    else return root[x] = find(root[x]);
}

int stoer_wagner() {
    memcpy(a, map, sizeof(map));
    for (int i = 0; i < N; ++i) q[i] = root[i] = i;
    int ret = INF;
    for (int i = 0; i < N - 1; ++i) {
        int t = mincut(N - i);
        ret = min(ret, w[t]);
        int s = pre[t];
        for (int j = 0; j < N - i; ++j)
            if (j != s && j != t) a[q[t]][q[j]] = (a[q[j]][q[t]] += a[q[j]][q[s]]);
        root[find(q[s])] = find(q[t]); q[s] = q[N - i - 1]; 
    }
    return ret;
}

void cal(int ans) {
    memcpy(a, map, sizeof(map));
    for (int i = 0; i < N; ++i) q[i] = root[i] = i;
    int t;
    for (int i = 0; i < N - 2; ++i) {
        t = mincut(N - i - 1);
        if (w[t] == ans) break;
        int s = pre[t];
        for (int j = 0; j < N - i - 1; ++j)
            if (j != s && j != t) a[q[t]][q[j]] = (a[q[j]][q[t]] += a[q[j]][q[s]]);
        root[find(q[s])] = find(q[t]); q[s] = q[N - i - 2]; 
    }
    t = find(q[t]);
    M = 0; //number of MinCut edges
    for (int i = 0; i < N; ++i)
        if (find(root[i]) == t)
            for (int j = 0; j < N; ++j)
                if (find(root[j]) != t && idx[i][j]) list[M++] = idx[i][j];
}
\end{verbatim}

	\subsubsection{Dijkstra最短路}
\begin{verbatim}
#include <cstdio>
#include <cstring>
#include <algorithm>

using namespace std;

const int maxn = 200 + 10;
const int maxm = maxn * maxn;
const int INF = 100000000;

struct Tedge {
	int v, f, c, next;
};


Tedge edge[maxm];
int first[maxn], level[maxn], nedge[maxn], pedge[maxn], prev[maxn], queue[maxn], par[maxn], fl[maxn];
int a[maxn][maxn], cut[maxn][maxn];
int n, m, S, T;

inline void add_edge(int u, int v, int c1, int c2 = 0) {
	edge[m].v = v; edge[m].f = 0; edge[m].c = c1; edge[m].next = first[u]; first[u] = m++;
	edge[m].v = u; edge[m].f = 0; edge[m].c = c2; edge[m].next = first[v]; first[v] = m++;
}

bool newphase() {
	for (int i = 0; i < n; ++i) level[i] = n, nedge[i] = first[i];
	queue[0] = S; level[S] = 0;
	for (int h = 0, t = 1; h < t; ++h) {
		int u = queue[h];
		for (int i = first[u]; i != -1; i = edge[i].next)
			if (edge[i].f < edge[i].c && level[edge[i].v] == n) {
				level[edge[i].v] = level[u] + 1;
				if (edge[i].v == T) return 1;
				queue[t++] = edge[i].v;
			}
	}
	return 0;
}

bool find_path(int u) {
	for (int i = nedge[u]; i != -1; i = edge[i].next)
		if (edge[i].f < edge[i].c && level[edge[i].v] == level[u] + 1)
			if (edge[i].v == T || find_path(edge[i].v)) {
				pedge[edge[i].v] = nedge[u] = i;
				return 1;
			}
	nedge[u] = -1;
	return 0;
}

int Dinic() {
	for (int i = 0; i < m; ++i) edge[i].f = 0;
	int ret = 0;
	while (newphase())
		while (find_path(S)) {
			int delta = INF;
			for (int u = T, i = pedge[u]; u != S; u = edge[i ^ 1].v, i = pedge[u])
				delta = min(delta, edge[i].c - edge[i].f);
			for (int u = T, i = pedge[u]; u != S; u = edge[i ^ 1].v, i = pedge[u])
				edge[i].f += delta, edge[i ^ 1].f -= delta;
			ret += delta;
		}
	return ret;
}

int main() {
	int N;
	scanf("%d", &N);
	for (int tst = 1; tst <= N; ++tst) {
		m = 0;
		memset(first, -1, sizeof(first));
		scanf("%d", &n);
		for (int i = 0; i < n; ++i)
			for (int j = 0; j < n; ++j) {
				scanf("%d", &a[i][j]);
				if (i < j && a[i][j]) add_edge(i, j, a[i][j], a[i][j]);
			}

		memset(cut, 0, sizeof(cut));
		memset(par, 0, sizeof(par));
		for (S = 1; S < n; ++S) {
			T = par[S];
			fl[S] = cut[S][T] = cut[T][S] = Dinic();
			for (int i = 1; i < n; ++i)
				if (i != S && level[i] != n && par[i] == T) par[i] = S;
			if (level[par[T]] != n) {
				par[S] = par[T];
				par[T] = S;
				fl[S] = fl[T];
				fl[T] = cut[S][T];
			}
			for (int i = 0; i < S; ++i)
				if (i != T) cut[S][i] = cut[i][S] = min(cut[S][T], cut[T][i]);
		}
		// (i, par[i]) of value fl[i][par[i]] are the edges of GH cut tree

		printf("Case #%d:\n", tst);
		for (int i = 0; i < n; ++i) {
			for (int j = 0; j < n; ++j) {
				if (j) printf(" ");
				printf("%d", cut[i][j]);
			}
			printf("\n");
		}
	}

	return 0;
}
\end{verbatim}


	\subsection{匹配}
    \input{code/graph/hungarian}
	\input{code/graph/hopcroft_carp}
    \subsubsection{二分图最大权匹配}
\begin{verbatim}
const int N = 105, inf = 0x3F3F3F3F;

int n;
int graph[N][N];
int match[N], slack[N], lx[N], ly[N];
bool vx[N], vy[N];

bool find(int x) {
    vx[x] = true;
    rep(y, n){
        if (vy[y]) continue;
        if (lx[x] + ly[y] == graph[x][y]){
            vy[y] = true;
            if (match[y] == -1 || find(match[y])){
                match[y] = x;
                return true;
            }
        }
        else slack[y] = min(slack[y], lx[x] + ly[y] - graph[x][y]);
    }
    return false;
}

int max_match(int n) {
    rep(i, n) {
        lx[i] = *max_element(graph[i], graph[i] + n);
        ly[i] = 0;
    }
    memset(match, -1, sizeof(match));
    rep(x, n){
        memset(slack, -1, sizeof slack);
        while(true){
            Cls(vx);
            Cls(vy);
            if (find(x)) break;
            int sub = inf;
            rep(i, n) if (!vy[i]) sub = min(sub, slack[i]);
            rep(i, n) if (vx[i]) lx[i] -= sub;
            rep(i, n) {
                if (vy[i]) ly[i] += sub;
                else slack[i] -= sub;
            }
        }
    }
    int res = 0;
    rep(i, n) res += graph[match[i]][i];
    return res;
}

int min_match(int n) {
    rep(i, n) rep(j, n) graph[i][j] *= -1;
    return -max_match(n);
}

int main() {
    while(~scanf("%d", &n)){
        rep(i, n) rep(j, n) scanf("%d", &graph[i][j]);
        printf("%d\n", max_match(n));
    }
    return 0;
}
\end{verbatim}

	\input{code/graph/edmonds_blossom}

    \subsection{图}
    \input{code/graph/hamilton}
    %\input{code/graph/kosaraju}
    \subsubsection{割点和桥}
\begin{verbatim}
const int N = 1111, M = 1111111;

int n, m;
int root;
int low[N], dep[N];
bool cut[N], bri[N];
vector<int> E[N];
vector<int> id[N];
PII edge[N];

void dfs(int x, int f, int d){
    int e = 0, deg = 0;
    low[x] = dep[x] = d;
    rep(i, E[x].size()){
        int y = E[x][i];
        if (low[y] == -1){
            deg++;
            dfs(y, x, d + 1);
            low[x] = min(low[x], low[y]);
            if (low[y] > dep[x]) bri[id[x][i]] = true;
            cut[x] |= (x == root && deg > 1 || x != root && low[y] >= dep[x]);
        }
        else if (y != f || e){
            low[x] = min(low[x], dep[y]);
        }
        else e = 1;
    }
}


int main(){
    while(~scanf("%d%d",&n,&m)){
        rep(i, n) E[i].clear(), low[i] = dep[i] = -1, cut[i] = false;
        rep(i, n) id[i].clear();
        int x, y;
        rep(i, m){
            scanf("%d%d", &x, &y);
            x--, y--;
            bri[i] = false;
            edge[i] = MP(x, y);
            E[x].PB(y), id[x].PB(i);
            E[y].PB(x), id[y].PB(i);
        }
        dfs(root = 0, -1, 0);
    }
    return 0;
}
\end{verbatim}

	\subsubsection{有向图割点}
\begin{verbatim}
const int MAXN = 5000 + 10; //number of vertices
const int MAXM = 200000 + 10; //number of edges

struct Tedge {
    int v, next;
};

Tedge edge[MAXM], back[MAXM]; //back is opposite to edge
bool ontree[MAXM];
int first1[MAXN], first2[MAXN], id[MAXN], low[MAXN], stack[MAXN];
bool critical[MAXN]; //1 - the node is CutVertex, 0 - not
int N, M, cnt;

void DFS(int u) {
    id[u] = cnt; stack[cnt++] = u; low[u] = u;
    for (int i = first1[u]; i != -1; i = edge[i].next)
        if (id[edge[i].v] == -1) {
            ontree[i] = 1;
            DFS(edge[i].v);
        }
}

void update(int u) {
    for (int i = first1[u]; i != -1; i = edge[i].next)
        if (ontree[i] && id[low[u]] < id[low[edge[i].v]]) {
            low[edge[i].v] = low[u];
            update(edge[i].v);
        }
}

void CV() {
    cnt = 0;
    memset(id, -1, sizeof(id));
    memset(ontree, 0, sizeof(ontree));
    DFS(0);
    memset(critical, 0, sizeof(critical));
    critical[0] = 1;
    for (int i = cnt - 1; i >= 0; --i) {
        int u = stack[i];
        for (int j = first1[u]; j != -1; j = edge[j].next)
            if (ontree[j] && low[edge[j].v] == u) {
                critical[u] = 1;
                break;
            }
        for (int j = first2[u]; j != -1; j = back[j].next)
            if (id[low[back[j].v]] < id[low[u]]) low[u] = low[back[j].v];
        update(u);
    }
}
\end{verbatim}

	\subsubsection{树链剖分}
\begin{verbatim}
//By myf
//#pragma comment(linker, "/STACK:16777216")  //C++
#include <iomanip>
#include <iostream>
#include <algorithm>
#include <cmath>
#include <cstdio>
#include <cstdlib>
#include <cstring>
#include <bitset>
#include <complex>
#include <map>
#include <set>
#include <queue>
#include <deque>
#include <stack>
#include <vector>
#include <list>

#define rep(i,n) for(int i=0;i<(n);i++)
#define REP(i,l,r) for(int i=(l);i<(r);i++)
#define fab(i,a,b) for(int i=(a);i<=(b);i++)
#define fba(i,b,a) for(int i=(b);i>=(a);i--)
//#define foreach(i,n) for(__typeof(n.begin()) i=n.begin();i!=n.end();i++) //G++
#define MP make_pair
#define PB push_back
#define X first
#define Y second
#define Cls(x) memset(x,0,sizeof x)
#define Print(n,x) for(int i=0;i<(n);i++) cout<<x<<" ";cout<<endl;
#define Sqr(x) (x)*(x)

using namespace std;

typedef long long LL;
typedef pair<int,int> PII;
typedef pair<PII,int> PIII;
typedef pair<LL,int> PLI;
typedef vector<int> VI;
typedef LL T;

const int N=50005,M=1<<16;

int n,m,q,tot;
int v[N];
int t[M*2];
VI E[N];
int fa[N],dep[N],son[N],sz[N];
int id[N],top[N];

void dfs(int x){
    sz[x]=1,son[x]=0;
    rep(i,E[x].size()){
        int y=E[x][i];
        if (y==fa[x]) continue;
        dep[y]=dep[x]+1;
        fa[y]=x;
        dfs(y);
        sz[x]+=sz[y];
        if (sz[y]>sz[son[x]]) son[x]=y;
    }
}

void dfs(int x,int p){
    id[x]=++tot,top[x]=p;
    if (son[x]) dfs(son[x],p);
    rep(i,E[x].size()){
        int y=E[x][i];
        if (y==fa[x]||y==son[x]) continue;
        dfs(y,y);
    }
}

int ask(int x){
    x=id[x];
    int ret=0;
    for(x+=M;x;x>>=1) ret+=t[x];
    return ret;
}

void insert(int l,int r,int x){
    for(l+=M-1,r+=M+1;l^r^1;l>>=1,r>>=1){
        if (~l&1) t[l^1]+=x;
        if ( r&1) t[r^1]+=x;
    }
}

void add(int x,int y,int k){
    while(top[x]!=top[y]){
        if (dep[top[x]]<dep[top[y]]) swap(x,y);
        insert(id[top[x]],id[x],k);
        x=fa[top[x]];
    }
    if (dep[x]<dep[y]) swap(x,y);
    insert(id[y],id[x],k);
}

int main(){
    while(~scanf("%d%d%d",&n,&m,&q)){
        rep(i,n) scanf("%d",&v[i+1]);
        rep(i,n) E[i+1].clear();
        rep(i,m){
            int x,y;
            scanf("%d%d",&x,&y);
            E[x].PB(y);
            E[y].PB(x);
        }
        fa[1]=dep[1]=1;
        sz[0]=0,tot=0;
        dfs(1);
        dfs(1,1);
        Cls(t);
        fab(i,1,n) t[id[i]+M]=v[i];
        char ch;
        int x,y,k;
        rep(i,q){
            while((ch=getchar())&&ch!='D'&&ch!='Q'&&ch!='I');
            if (ch=='Q'){
                scanf("%d",&x);
                printf("%d\n",ask(x));
            }
            else{
                scanf("%d%d%d",&x,&y,&k);
                add(x,y,(ch=='I')?k:-k);
            }
        }
    }
    return 0;
}

	\input{code/graph/tree_divide_edge}
	\subsubsection{点的分治,权值在点上}
\begin{verbatim}
const int N=50000+10,M=30;

int n,k;
VI E[N];
int tot,top,mi,root;
int size[N];
LL f[N];
map<LL,int> Q;
LL prime[N];
int sta[N][M];
LL q[N];
bool use[N];
LL base[M+1];

LL ans;

inline LL add(LL x,int sta[M]){
    LL y=0;
    rep(i,k){
        int tmp=(x%base[i+1])/base[i];
        tmp+=sta[i];
        tmp%=3;
        y+=(tmp*base[i]);
    }
    return y;
}

inline LL dec(LL a,LL b){
    LL y=0;
    rep(i,k){
        int tmp1=(a%base[i+1])/base[i];
        int tmp2=(b%base[i+1])/base[i];
        int tmp=(tmp1-tmp2+3)%3;
        y+=(tmp*base[i]);
    }
    return y;
}

inline LL add(LL a,LL b){
    LL y=0;
    rep(i,k){
        int tmp1=(a%base[i+1])/base[i];
        int tmp2=(b%base[i+1])/base[i];
        int tmp=(tmp1+tmp2)%3;
        y+=(tmp*base[i]);
    }
    return y;
}


void getVal(int x,LL val,int fa){
    q[top++]=val;
    rep(i,E[x].size()){
        int y=E[x][i];
        if (use[y]||y==fa)continue;
        getVal(y,add(val,sta[y]),x);
    }
}

void getRoot(int x,int fa){
    int big=-1;
    size[x]=1;
    rep(i,E[x].size()){
        int y=E[x][i];
        if (use[y]||y==fa) continue;
        getRoot(y,x);
        size[x]+=size[y];
        big=max(big,size[y]);
    }
    big=max(big,tot-size[x]);
    if (big<mi) mi=big,root=x;
}

void dfs(int x){
     tot=mi=size[x];
     getRoot(x,-1);
     x=root;
     use[x]=true;
     Q.clear();
     LL now=add(0,sta[x]);
     Q[now]=1;
     if (now==0) ans++;
     rep(i,E[x].size()){
         int y=E[x][i];
         if (use[y]) continue;
         top=0;
         getVal(y,add(0,sta[y]),x);
         rep(j,top){
             LL tmp=dec(0,q[j]);
             if (Q.count(tmp)) ans+=Q[tmp];
         }
         rep(j,top) Q[add(now,q[j])]++;
     }
     rep(i,E[x].size()) if (!use[E[x][i]]) dfs(E[x][i]);
}

int main(){
    base[0]=1;
    fab(i,1,M) base[i]=base[i-1]*3;
    while(~scanf("%d",&n)){
        Cls(use);
        rep(i,n) E[i].clear();
        scanf("%d",&k);
        rep(i,k) scanf("%I64d",prime+i);
        rep(i,n){
            LL x;
            scanf("%I64d",&x);
            Cls(sta[i]);
            rep(j,k) while (x%prime[j]==0) x/=prime[j],sta[i][j]++,sta[i][j]%=3;
        }
        rep(i,n-1){
            int x,y;
            scanf("%d%d",&x,&y);
            x--,y--;
            E[x].PB(y);
            E[y].PB(x);
        }
        size[0]=n;
        ans=0;
        dfs(0);
        printf("%I64d\n",ans);
    }
    return 0;
}
\end{verbatim}

	\subsubsection{最大团}
\begin{verbatim}
const int MAXN = 100; //number of vertices

int a[MAXN][MAXN];
int f[MAXN];
int N, ans;

bool DFS(int q[], int t, int cnt) {
    if (t == 0) {
        if (cnt > ans) {
            ans = cnt;
            return 1;
        }
        return 0;
    }

    int tq[MAXN];
    for (int i = 0; i < t; ++i) {
        if (f[q[i]] + cnt <= ans) return 0;
        int k = 0;
        for (int j = i + 1; j < t; ++j)
            if (a[q[i]][q[j]]) tq[k++] = q[j];
        if (DFS(tq, k, cnt + 1)) return 1;
    }
    return 0;
}

void MaxClique() {
    ans = 0;
    int q[MAXN];
    for (int i = N - 1; i >= 0; --i) {
        int t = 0;
        for (int j = i + 1; j < N; ++j) if (a[i][j]) q[t++] = j;
        DFS(q, t, 1);
        f[i] = ans;
    }
}
\end{verbatim}

	\subsubsection{图平面化}
\begin{verbatim}
// vertices numbered from 1 to N
// No self-loops and no duplicate edges

typedef pair<int, int> T;
const int maxn = 10000 + 10;

struct node {
    int dep, fa, infc, used, vst, dfi, ec, lowp, bflag, flag, lowpoint;
};

int n, m, indee, p1, p2, p, ps;
int lk[maxn * 3][2], child[maxn * 3][3], bedg[maxn * 3][2], sdlist[maxn * 6][3], 
    buk[maxn * 6][2], exf[maxn * 3][2], proots[maxn * 3][3], stk[maxn * 3][2], infap[maxn * 3];
int w1[maxn], w2[maxn], que[maxn];
node dot[maxn];

void init(T * ts) {
    ps = 0;
    for (int i = 1; i <= n; ++i) w1[i] = i;
    p1 = n;
    for (int i = 0; i < m; ++i) {
        int k1 = ts[i].first, k2 = ts[i].second;
        lk[++p1][0] = k2; lk[p1][1] = 0;
        lk[w1[k1]][1] = p1;
        w1[k1] = p1;
        lk[++p1][0] = k1; lk[p1][1] = 0;
        lk[w1[k2]][1] = p1;
        w1[k2] = p1;
    }
    for (int i = 1; i <= n; ++i) que[i] = i;
}

int deep(int a) {
    dot[a].used = 1; dot[a].dfi = ++indee;
    int t = lk[a][1];
    while (t != 0) {
        int tmp = lk[t][0];
        if (!dot[tmp].used) {
            dot[tmp].fa = a; dot[tmp].dep = dot[a].dep + 1; dot[tmp].ec = dot[a].dep; dot[tmp].lowp = dot[a].dep;
            child[++p1][0] = tmp; child[p1][1] = 0;
            child[w1[a]][1] = p1;
            w1[a] = p1;
            int s = deep(tmp);
            if (s < dot[a].ec) dot[a].ec = s;
        }
        else if (dot[a].fa != tmp) {
            if (dot[a].lowp > dot[tmp].dep) dot[a].lowp = dot[tmp].dep;
            if (dot[a].dfi > dot[tmp].dfi) {
                bedg[++p2][0] = a; bedg[p2][1] = 0;
                bedg[w2[tmp]][1] = p2;
                w2[tmp] = p2;
            }
        }
        t = lk[t][1];
    }
    if (dot[a].ec > dot[a].lowp) dot[a].ec = dot[a].lowp;
    return dot[a].ec;
}

void sortvtx() {
    for (int i = 1; i <= n; ++i) w1[i] = i;
    p1 = n; p2 = 0;
    for (int i = 1; i <= n; ++i) {
        buk[++p1][0] = i; buk[p1][1] = 0;
        buk[w1[dot[i].dfi]][1] = p1;
        w1[dot[i].dfi] = p1;
    }
    for (int i = n; i > 0; --i) {
        int tmp = buk[i][1];
        while (tmp != 0) {
            que[++p2] = buk[tmp][0];
            tmp = buk[tmp][1];
        }
    }
}

void getsdlist() {
    memset(buk, 0, sizeof(buk));
    for (int i = 1; i <= n; ++i) {
        w1[i] = w2[i] = i;
        buk[i][1] = 0;
    }
    p1 = p2 = n;
    for (int i = 1; i <= n; ++i) {
        buk[++p1][0] = i; buk[p1][1] = 0;
        buk[w1[dot[i].ec]][1] = w1[dot[i].ec] = p1;
    }
    for (int i = 1; i <= n; ++i) {
        int tmp = buk[i][1];
        while (tmp != 0) {
            int fa = dot[buk[tmp][0]].fa;
            sdlist[++p2][0] = i; sdlist[p2][1] = 0;
            sdlist[w2[fa]][1] = dot[buk[tmp][0]].infc = p2;
            sdlist[p2][2] = w2[fa]; w2[fa] = p2;
            tmp = buk[tmp][1];
        }
    }
}

void getnextvtx(int v, int v1, int &m, int &m1) {
    m = exf[v][v1 ^ 1];
    if (exf[m][0] == v) m1 = 0;
    else m1 = 1;
}

void addwei(int a) {
    int fa = dot[a - n].fa;
    ++p1;
    proots[p1][0] = a; proots[p1][1] = 0;
    proots[w1[fa]][1] = p1;
    proots[p1][2] = w1[fa]; w1[fa] = p1;
    infap[a] = p1;
}

void addsou(int a) {
    int fa = dot[a - n].fa;
    ++p1;
    proots[p1][0] = a; proots[p1][1] = proots[fa][1]; proots[p1][2] = fa;
    proots[fa][1] = p1;
    proots[proots[p1][1]][2] = p1;
    infap[a] = p1;
    if (w1[fa] == fa) w1[fa] = p1;
}

void walkup(int v, int w) {
    dot[w].bflag = v;
    int x = w, x1 = 1, y = w, y1 = 0;
    while (x != v) {
        if (dot[x].vst == v || dot[y].vst == v) break;
        dot[x].vst = v; dot[y].vst = v;
        int z1 = 0;
        if (x > n) z1 = x;
        if (y > n) z1 = y;
        if (z1 != 0) {
            int c = z1 - n, z = dot[c].fa;
            if (z != v) {
                if (dot[c].lowpoint < dot[v].dep) addwei(z1);
                else addsou(z1);
            }
            x = z; x1 = 1;
            y = z; y1 = 0;
        } else {
            getnextvtx(x, x1, x, x1);
            getnextvtx(y, y1, y, y1);
        }
    }
}

void getactivenext(int v, int v1, int &m, int &m1, int vt) {
    m = v; m1 = v1;
    getnextvtx(m, m1, m, m1);
    while (dot[m].bflag != vt && proots[m][1] == 0 && dot[m].ec >= dot[vt].dep && m != v) getnextvtx(m, m1, m, m1);
}

void addstack(int a, int b) {
    stk[++ps][0] = a; stk[ps][1] = b;
}

void mergestack() {
    int t = stk[ps][0], t1 = stk[ps][1], k = stk[ps - 1][0], k1 = stk[ps - 1][1];
    ps -= 2;
    int s1, s = exf[t][1 ^ t1];
    if (exf[s][1] == t) s1 = 1;
    else s1 = 0;
    exf[k][k1] = s;
    exf[s][s1] = k;
    int tmp = dot[t - n].infc;
    sdlist[sdlist[tmp][2]][1] = sdlist[tmp][1]; sdlist[sdlist[tmp][1]][2] = sdlist[tmp][2];
    tmp = dot[t - n].fa;
    if (sdlist[tmp][1] == 0) dot[tmp].ec = dot[tmp].lowp;
    else dot[tmp].ec = min(dot[tmp].lowp, sdlist[sdlist[tmp][1]][0]);
    tmp = infap[t];
    int fa = dot[t - n].fa;
    proots[proots[tmp][2]][1] = proots[tmp][1];
    if (proots[tmp][1] != 0) proots[proots[tmp][1]][2] = proots[tmp][2];
    else w1[fa] = proots[tmp][2];
}

void embededg(int v, int v1, int w, int w1) {
    exf[v][v1] = w; exf[w][w1] = v;
}

void walkdown(int v) {
    ps = 0;
    int vt = dot[v - n].fa;
    for (int v2 = 0; v2 <= 1; ++v2) {
        int w, w1;
        getnextvtx(v, 1 ^ v2, w, w1);
        while (w != v) {
            if (dot[w].bflag == vt) {
                while (ps != 0) mergestack();
                embededg(v, v2, w, w1);
                dot[w].bflag = 0;
            }
            if (proots[w][1] != 0) {
                addstack(w, w1);
                int x, x1, y, y1, w2, w0 = proots[proots[w][1]][0];
                getactivenext(w0, 1, x, x1, vt);
                getactivenext(w0, 0, y, y1, vt);
                if (dot[x].ec >= dot[vt].dep) w = x, w1 = x1;
                else if (dot[y].ec >= dot[vt].dep) w = y, w1 = y1;
                else if (dot[x].bflag == vt || proots[x][1] != 0) w = x, w1 = x1;
                else w = y, w1 = y1;
                if (w == x) w2 = 0;
                else w2 = 1;
                addstack(w0, w2);
            }
            else if (w > n || dot[w].ec >= dot[vt].dep) getnextvtx(w, w1, w, w1);
            else {
                if (w <= n && dot[w].ec < dot[vt].dep && ps == 0) embededg(v, v2, w, w1);
                break;
            }
        }
        if (ps != 0) break;
    }
}

bool chainvtx(int a) {
    for (int t = child[a][1]; t != 0; t = child[t][1]) {
        int tmp = child[t][0];
        exf[tmp][1] = tmp + n; exf[tmp][0] = tmp + n;
        exf[tmp + n][1] = tmp; exf[tmp + n][0] = tmp;
    }
    for (int t = bedg[a][1]; t != 0; t = bedg[t][1]) walkup(a, bedg[t][0]);
    for (int t = child[a][1]; t != 0; t = child[t][1]) walkdown(child[t][0] + n);
    for (int t = bedg[a][1]; t != 0; t = bedg[t][1]) if (dot[bedg[t][0]].bflag != 0) return false;
    return true;
}

bool judge(int N, int M, T * ts) {
    n = N;    m = M;
    if (n == 1) return true;
    if (m > 3 * n - 5) return false;
    init(ts);
    
    for (int i = 1; i <= n; ++i) {
        proots[i][1] = 0; proots[i + n][1] = 0;
        p = 0;
        child[i][1] = 0;
        buk[i][1] = 0; buk[i + n][1] = 0;
        sdlist[i][1] = 0; sdlist[i + n][1] = 0;
        dot[i].bflag = 0; dot[i + n].flag = 0;
    }
    for (int i = 1; i <= n; ++i) {
        w1[i] = i; w2[i] = i;
        child[i][1] = 0; bedg[i][1] = 0;
        dot[i].used = 0;
    }
    indee = 0; p1 = p2 = n;
    for (int i = 1; i <= n; ++i) {
        if (!dot[i].used) {
            dot[i].dep = 1;
            deep(i);
        }
    }
    sortvtx();
    getsdlist();
    for (int i = 1; i <= n; ++i) {
        dot[i].lowpoint = dot[i].ec;
        dot[i].vst = 0; dot[i + n].vst = 0;
        proots[i][1] = 0;
        w1[i] = i;
    }
    p1 = n;
    for (int i = 1; i <= n; ++i) if (!chainvtx(que[i])) return false;
    return true;
}

T ts[maxn];
bool a[3001][3001];

int main() {
    int N, M;
    scanf("%d%d", &N, &M);
    int m = 0;
    for(int i = 0; i < M; i ++) {
        scanf("%d%d", &ts[i].first, &ts[i].second);
        ++ts[i].first; ++ts[i].second;
        if (ts[i].first == ts[i].second || a[ts[i].first][ts[i].second]) continue;
        a[ts[i].first][ts[i].second] = a[ts[i].second][ts[i].first] = 1;
        ts[m++] = ts[i];
    }
    M = m;
    if(judge(N, M, ts)) puts("YES");
    else puts("NO");

    return 0;
}
\end{verbatim}


    %\subsection{最近公共祖先算法}
    %\input{code/graph/ancient_tree}
    \input{code/graph/LCA_offline}
    \input{code/graph/LCA_online}

\section{\LARGE 数据结构}
	\subsection{平衡树}
    \input{code/data_structure/heap}
    \input{code/data_structure/bst}
    \input{code/data_structure/treap}
	\input{code/data_structure/splay}

	\subsection{图上的数据结构}
	\input{code/data_structure/leftist_tree}
    \subsubsection{支持子树操作的动态树}
\begin{verbatim}
//By myf
//#pragma comment(linker, "/STACK:16777216")  //C++
#include <iomanip>
#include <iostream>
#include <algorithm>
#include <cmath>
#include <cstdio>
#include <cstdlib>
#include <cstring>
#include <bitset>
#include <complex>
#include <map>
#include <set>
#include <queue>
#include <deque>
#include <stack>
#include <vector>
#include <list>

#define rep(i,n) for(int i=0;i<(n);i++)
#define REP(i,l,r) for(int i=(l);i<(r);i++)
#define fab(i,a,b) for(int i=(a);i<=(b);i++)
#define fba(i,b,a) for(int i=(b);i>=(a);i--)
//#define foreach(i,n) for(__typeof(n.begin()) i=n.begin();i!=n.end();i++) //G++
#define MP make_pair
#define PB push_back
#define X first
#define Y second
#define Cls(x) memset(x,0,sizeof x)
#define Print(n,x) for(int i=0;i<(n);i++) cout<<x<<" ";cout<<endl;
#define inf 0xc0c0c0c0
#define lch ch[x][0]
#define rch ch[x][1]

using namespace std;

typedef long long LL;
typedef pair<int,int> PII;
typedef pair<PII,int> PIII;
typedef pair<LL,int> PLI;
typedef vector<int> VI;
typedef int T;

const int N=333333;

int n;
int pre[N],fa[N],fat[N],val[N],ma[N],ch[N][2];
bool black[N];
multiset<int> Q[N];
VI E[N];

inline void up(int x){ma[x]=max(max(val[x],*Q[x].rbegin()),max(ma[lch],ma[rch]));}

inline void rot(int id,int tp){
    static int k;
    k=pre[id];
    ch[k][tp^1]=ch[id][tp];
    if(ch[id][tp]) pre[ch[id][tp]]=k;
    if(pre[k]) ch[pre[k]][k==ch[pre[k]][1]]=id;
    pre[id]=pre[k];
    ch[id][tp]=k;
    pre[k]=id;
    up(k);
}

inline void splay(int x){
    if (!pre[x]) return;
    int tmp;
    for(tmp=x;pre[tmp];tmp=pre[tmp]);
    for(swap(fa[x],fa[tmp]);pre[x];rot(x,x==ch[pre[x]][0]));
    up(x);
}

inline int access(int x){
    int nt;
    for(nt=0;x;x=fa[x]){
        splay(x);
        if (rch){
            fa[rch]=x;
            pre[rch]=0;
            Q[x].insert(ma[rch]);
        }
        rch=nt;
        if (nt){
            fa[nt]=0;
            pre[nt]=x;
            Q[x].erase(Q[x].find(ma[nt]));
        }
        up(nt=x);
    }
    return nt;
}

void make(int x,int f){
    fat[x]=f;
    rep(i,E[x].size()) if (E[x][i]!=f) make(E[x][i],x);
    int t;
    up(x+n);up(x+2*n);
    fa[t=x+(1+black[x])*n]=x;
    Q[x].insert(*Q[t].rbegin());
    up(x);
    fa[x]=t=f+(1+black[x])*n;
    Q[t].insert(ma[x]);
}

void cut(int x,int f){
    access(f);
    splay(f);
    splay(x);
    Q[f].erase(Q[f].find(ma[x]));
    fa[x]=0;
    up(f);
}

void link(int x,int f){
    access(f);
    splay(f);
    splay(x);
    fa[x]=f;
    Q[f].insert(ma[x]);
    up(f);
}

int main(){
    while(~scanf("%d",&n)){
        Cls(pre);
        Cls(ch);
        Cls(fa);
        rep(i,n+1) E[i].clear();
        rep(i,n-1){
            int x,y;
            scanf("%d%d",&x,&y);
            E[x].PB(y);
            E[y].PB(x);
        }
        n++;
        rep(i,3*n+1) Q[i].clear();
        rep(i,3*n+1) ma[i]=val[i]=inf,Q[i].insert(inf);
        REP(i,1,n) scanf("%d",black+i);
        REP(i,1,n) scanf("%d",val+i);
        make(1,n);
        int q,k,x;
        scanf("%d",&q);
        rep(i,q){
            scanf("%d%d",&k,&x);
            if (k==0){
                for(x=access(x);lch;x=lch);
                splay(x);
                printf("%d\n",ma[rch]);
            }
            if (k==1){
                cut(x,fat[x]+(1+black[x])*n);
                cut(x+(1+black[x])*n,x);
                black[x]^=1;
                link(x+(1+black[x])*n,x);
                link(x,fat[x]+(1+black[x])*n);
            }
            if (k==2){
                access(x);
                splay(x);
                scanf("%d",val+x);
                up(x);
            }
        }
    }
    return 0;
}
\end{verbatim}


	\subsection{可持久化数据结构}
	\subsubsection{函数式treap}
\begin{verbatim}
//By Lin
#include<cstdio>
#include<cstring>
#include<cstdlib>
using namespace std;

struct Node{
    int key,weight,size;
    Node *l,*r;
    Node(int _key , int _weight, Node *_l, Node* _r):
        key(_key),weight(_weight),l(_l),r(_r){
            size = 1;
            if ( l ) size += l->size;
            if ( r ) size += r->size;
        }
    Node *newnode(int key){
        return new Node(key,rand(),NULL,NULL);
    }
    inline int lsize(){ return l?l->size:0; }
    inline int rsize(){ return r?r->size:0; }
}*root[50005];

Node* Meger(Node *a , Node *b ){
    if ( !a || !b ) return a?a:b;
    return a->weight>b->weight?
         new Node(a->key,a->weight,a->l,Meger(a->r,b)):
         new Node(b->key,b->weight,Meger(a,b->l),b->r);
}

Node* Split_L(Node *a ,int size ){
    if ( !a || size == 0 ) return NULL;
    return a->lsize() < size?
        new Node(a->key,a->weight,a->l,Split_L(a->r,size-1-a->lsize())):
        Split_L(a->l,size);
}

Node* Split_R(Node *a ,int size ){
    if ( !a || size == 0 ) return NULL;
    return a->rsize() < size?
        new Node(a->key,a->weight,Split_R(a->l,size-1-a->rsize()),a->r):
        Split_R(a->r,size);
}

int Ask( Node *a ,int k ){
    if ( a->lsize() >= k ) return Ask(a->l,k);
    k -= a->lsize()+1;
    if ( k == 0 ) return a->key;
    return Ask(a->r,k);
}

int len = 0;

int main(){
    int d = 0 , cas;
    scanf("%d", &cas);
    root[0] = NULL;
    int cnt = 1,kind,v,p,c;
    char s[1005];
    while ( cas -- ) {
        scanf("%d", &kind );
        if ( kind == 1 ) {
            scanf("%d%s", &p , s );
            p-=d;
            Node *l = Split_L(root[cnt-1],p),
                 *r = Split_R(root[cnt-1],len-p);
            for (int i = 0; s[i]; i++ ){
                l = Meger(l,new Node(s[i],rand(),NULL,NULL));
                len++;
            }
            root[cnt++] = Meger(l,r);
        }
        else if ( kind == 2){
            scanf("%d%d", &p , &c );
            p-=d,c-=d;
            Node *l = Split_L(root[cnt-1],p-1),
                 *r = Split_R(root[cnt-1],len-p-c+1);
            len -= c;
            root[cnt++] = Meger(l,r);
        }
        else{
            scanf("%d%d%d", &v, &p , &c );
            v-=d,p-=d,c-=d;
            char ch;
            for (int i = p; i<p+c; i++) {
                printf("%c", ch = Ask(root[v],i) );
                if ( ch == 'c' ) d++;
            }
            puts("");
        }
    }
    return 0;
}
\end{verbatim}

    \subsubsection{划分树}
\begin{verbatim}
const int D = 18;
const int N = 100000 + 1000;

struct Tree{
	int n;
	int v[N];
	int val[D][N], to_left[D][N];
	LL sum_l[D][N];

	void build(int l, int r, int deep){
		if (l == r) return;
		int mid = (l + r) / 2, left_same = mid - l + 1;
		for(int i = l; i <= r; i++){
			if (val[deep][i] < v[mid]) left_same--;
		}
		int le = l, ri = mid + 1, same = 0;
		sum_l[deep][0] = 0;
		for(int i = l; i <= r; i++){
			to_left[deep][i] = (i == l) ? 0 : to_left[deep][i - 1];
			sum_l[deep][i] = sum_l[deep][i - 1];
			if (val[deep][i] < v[mid]){
				to_left[deep][i]++;
				sum_l[deep][i] += val[deep][i];
				val[deep + 1][le++] = val[deep][i];
			}
			else if (val[deep][i] > v[mid]){
				val[deep + 1][ri++] = val[deep][i];
			}
			else if (same < left_same){
				to_left[deep][i]++;
				sum_l[deep][i] += val[deep][i];
				val[deep + 1][le++] = val[deep][i];
				same++;
			}
			else{
				val[deep + 1][ri++] = val[deep][i];
			}
		}
		build(l, mid, deep + 1);
		build(mid + 1, r, deep + 1);
	}

	pair<int, LL> ask(int ask_l, int ask_r, int l, int r, int deep, int kth){ 
		if (l == r) return MP(val[deep][l], val[deep][l]);
		int mid = (l + r) / 2, s1, s2;
		if (l == ask_l){
			s1 = 0;
			s2 = to_left[deep][ask_r];
		}
		else{
			s1 = to_left[deep][ask_l - 1];
			s2 = to_left[deep][ask_r];
		}
		if (s2 - s1 >= kth){
			ask_l = l + s1, ask_r = ask_l + s2 - s1 - 1;
			return ask(ask_l, ask_r, l, mid, deep + 1, kth);
		}
		else{
			LL ret = sum_l[deep][ask_r] - sum_l[deep][ask_l - 1];
			kth = kth - (s2 - s1);
			s2 = ask_r - ask_l + 1 - (s2 - s1);
			ask_l = mid + ask_l - l + 1 - s1;
			ask_r = ask_l + s2 - 1;
			pair<int, LL> tmp = ask(ask_l, ask_r, mid + 1, r, deep + 1, kth);
			return MP(tmp.F, tmp.S + ret);
		}
	}

	void init(int n, int other[]){
		for(int i = 1; i <= n; i++){
			v[i] = other[i];
			val[0][i] = v[i];
		}
		sort(v + 1, v + n + 1);
		build(1, n, 0);
	}
}tree;
\end{verbatim}


	\subsection{其他}
	\subsubsection{矩形切割}
\begin{verbatim}

int n, m;
LL ans;
vi v[N][2];

void work(int lev, vi &a, vi &b);

bool is_in(vi &a, vi &b, vi &a2, vi &b2){
	rep(i, n){
		if (!(a2[i] <= a[i] && b[i] <= b2[i])) return false;
	}
	return true;
}

void dfs(vi &a, vi &b, vi &a2, vi &b2, int now_d, int lev){
	rep(i, n){
		if (a[i] == b[i]) return;
	}
	if (is_in(a, b, a2, b2)) return;
	if (now_d == n) return;
	int l = max(a[now_d], a2[now_d]);
	int r = min(b[now_d], b2[now_d]);
	int tmp_l = a[now_d], tmp_r = b[now_d];

	a[now_d] = l, b[now_d] = r;
	dfs(a, b, a2, b2, now_d + 1, lev);
	a[now_d] = tmp_l, b[now_d] = tmp_r;

	b[now_d] = l;
	work(lev + 1, a, b);
	b[now_d] = tmp_r;

	a[now_d] = r;
	work(lev + 1, a, b);
	a[now_d] = tmp_l;
}

void work(int lev, vi &a, vi &b){
	rep(i, n){
		if (a[i] == b[i]) return;
	}
	if (lev == m){
		LL ret = 1;
		rep(i, n) ret = ret * (b[i] - a[i]) % MD;
		ans = (ans + ret) % MD;
		return;
	}
	vi &a2 = v[lev][0], &b2 = v[lev][1];
	bool no_cover = false;
	rep(i, n){
		int l = max(a[i], a2[i]);
		int r = min(b[i], b2[i]);
		if (l >= r) no_cover = true;
	}
	if (no_cover) work(lev + 1, a, b);
	else dfs(a, b, a2, b2, 0, lev);
}

int main(){
	while(~scanf("%d%d", &m, &n)){
		rep(i, m){
			v[i][0].clear();
			rep(j, n){
				int x;
				scanf("%d", &x);
				v[i][0].PB(x);
			}
			v[i][1].clear();
			rep(j, n){
				int x;
				scanf("%d", &x);
				v[i][1].PB(x);
			}
			rep(j, n) if (v[i][0][j] > v[i][1][j]) swap(v[i][0][j], v[i][1][j]);
		}
		ans = 0;
		rep(i, m){
			work(i + 1, v[i][0], v[i][1]);
		}
		ans = (ans % MD + MD) % MD;
		printf("%d\n", (int)ans);
	}
	return 0;
}
\end{verbatim}

	%\subsubsection{RMQ}
\begin{verbatim}
const int maxn = 100;
const int maxl = 7; //2^maxl >= maxn

int F[maxl][maxn];
int a[maxn], lg[maxn];
int n;

void init() {
    lg[0] = 0;
    for (int j, i = 1; i < maxn; lg[i++] = j - 1)
        for (j = 0; (1 << j) <= i; ++j);
}

void make_RMQ() {
    for (int i = 0; i < n; ++i) F[0][i] = a[i];

    for (int i = 1, k = 2; k <= n; ++i, k <<= 1)
        for (int j = 0; j <= n - k; ++j)
            F[i][j] = min(F[i - 1][j], F[i - 1][j + k / 2]);
}

inline int RMQ(int i, int j) {
    int k = lg[j - i + 1];
    return min(F[k][i], F[k][j - (1 << k) + 1]);
}
\end{verbatim}

	%\subsubsection{RMQ 2D}
\begin{verbatim}
const int maxn = 300 + 10;
const int maxl = 9; //2^maxl >= maxn

int F[maxl][maxl][maxn][maxn];
int a[maxn][maxn];
int lg[maxn];
int n, m;

void init() {
    lg[0] = 0;
    for (int j, i = 1; i < maxn; lg[i++] = j - 1)
        for (j = 0; (1 << j) <= i; ++j);
}

void make_RMQ() {
    for (int i = 0; i < n; ++i)
        for (int j = 0; j < m; ++j)
            F[0][0][i][j] = a[i][j];

    for (int i = 0, x = 1; x <= n; ++i, x <<= 1)
        for (int j = 0, y = 1; y <= m; ++j, y <<= 1) {
            if (!i && !j) continue;
            for (int p = 0; p <= n - x; ++p)
                for (int q = 0; q <= m - y; ++q)
                    if (i) F[i][j][p][q] = min(F[i - 1][j][p][q], F[i - 1][j][p + x / 2][q]);
                    else F[i][j][p][q] = min(F[i][j - 1][p][q], F[i][j - 1][p][q + y / 2]);
        }
}

inline int RMQ(int X1, int Y1, int X2, int Y2) {
    int x = lg[X2 - X1 + 1], y = lg[Y2 - Y1 + 1];
    return min(min(F[x][y][X1][Y1], F[x][y][X1][Y2 - (1 << y) + 1]),
        min(F[x][y][X2 - (1 << x) + 1][Y1], F[x][y][X2 - (1 << x) + 1][Y2 - (1 << y) + 1]));
}
\end{verbatim}


\section{\LARGE 字符串算法}
	\subsection{基础}
	\input{code/string/string}
	\subsubsection{ELFhash}
\begin{verbatim}
int ELFhash(char s[]) {
	unsigned long h = 0;
	for (int i = 0; i < strlen(s); ++i) {
		h = (h << 4) + s[i];
		unsigned long g = h & 0Xf0000000L;
		if (g) h ^= g >> 24;
		h &= ~g;
	}
	return h % maxhash;
}
\end{verbatim}

    \input{code/string/manacher}
    \input{code/string/min_representation}
    \subsubsection{kmp}
\begin{verbatim}
#include <cstdio>
#include <cstring>
#include <algorithm>
using namespace std;

const int N = 1000010;

int next[N];

int kmp(char *s, int n, char *t, int m) {
    int i, j;
    next[0] = -1;
    i = 0; j = -1;
    while (i < m) {
        if (j == -1 || t[i] == t[j]) {
            i++; j++;
            next[i] = (t[i] == t[j] ? next[j] : j);
        } else {
            j = next[j];
        }
    }
    i = j = 0;
    while (i < n && j < m) {
    if (j == -1 || s[i] == t[j]) {
        i++; j++;
    } else {
        j = next[j];
    }
}
return (j >= m ? i - m : -1);
}
\end{verbatim}

    \subsubsection{扩展kmp}
\begin{verbatim}
int ext[maxn]; // lcp(pat's suffix, pat)
int ex[maxn]; // lcp(pat's suffix, str)
//exp. str = "aaaba", pat = "aba", then ex[] = {1, 1, 3, 0, 1}, ext[] = {3, 0, 1}
//la = strlen(str), lb = strlen(pat);
void extkmp(char *str, char *pat, int ext[], int ex[]) {
    int p=0,k=1;
    while(pat[p] == pat[p+1]) p++;
    ext[0] = lb, ext[1] = p;
    for(int i=2;i<lb;i++){
        int x = k + ext[k] - i, y = ext[i - k];
        if (y < x) ext[i] = y;
        else{
            p = max(0, x);
            while (pat[p] == pat[p+i]) p++;
            ext[i] = p;
            k = i;
        }
    }
    p = k = 0;
    while(str[p] && str[p] == pat[p]) p++;
    ex[0] = p;
    for(int i=1;i<la;i++){
        int x = k + ex[k] - i, y = ext[i - k];
        if (y < x) ex[i] = y;
        else{
            p = max(0, x);
            while (pat[p] && pat[p] == str[p+i]) p++;
            ex[i] = p;
            k = i;
        }
    }
}
\end{verbatim}

    %\input{code/string/trie_tree}

	\subsection{进阶}
	\input{code/string/suffix_automata}
    \subsubsection{ac自动机}
\begin{verbatim}
int root, idx;
struct trie_node{
    int next[size];
    int fail;
    bool flag;
    void init(){
        fail = -1, flag = false;
        memset(next, 0, sizeof(next));
    }
}trie[maxn * leng];
int q[maxn * leng];
void trie_init(){
    root = idx = 0;
    trie[root].init();
}
void insert(char *s){
    int i, j, p = root;
    for(i=0;s[i];i++){
        j = s[i] - 'A';
        if(!trie[p].next[j]){
            trie[++idx].init();
            trie[p].next[j] = idx;
        }
        p = trie[p].next[j];
    }
    trie[p].flag = true;
}
void build(){
    int j, p;
    q[0] = root;
    for(int l=0,h=1;l<h;){
        p = q[l++];
        for(j=0;j<size;j++){
            if(trie[p].next[j]){
                q[h++] = trie[p].next[j];
                if(trie[p].fail == -1)
                    trie[trie[p].next[j]].fail = root;
                else{
                    trie[trie[p].next[j]].fail =
                        trie[trie[p].fail].next[j];

                    trie[trie[p].next[j]].flag |=
                        trie[trie[trie[p].fail].next[j]].flag;
                }
            }
            else{
                if(trie[p].fail != -1)
                    trie[p].next[j] = trie[trie[p].fail].next[j];
            }
        }
    }
}
\end{verbatim}

	\input{code/string/suffix_automata}
    \input{code/string/trie_graph}

\section{\LARGE 计算几何}
	\subsection{平面几何基础}
    \input{code/geometry/geometry}
    \subsubsection{圆}
\begin{verbatim}
struct Circle{
    Point o;
    double r;
    Circle(Point o = Point(), double r = 1) : o(o), r(r){}
    Circle(double x, double y, double r = 1) : o(x, y), r(r){}
};

int intersected_circle_line(Circle c, Line l){
    return sign(dist_line_point(l, c.o) - c.r) < 0;
}

int ip_circle_line(Circle c, Line l, Point &p1, Point &p2){
    Point a = l.p, b = l.q;
    double dx = b.x - a.x;
    double dy = b.y - a.y;
    double sdr = Sqr(dx) + Sqr(dy);
    double dr = sqrt(sdr);
    double d, disc, x, y;
    a.x -= c.o.x; a.y -= c.o.y;
    b.x -= c.o.x; b.y -= c.o.y;
    d = a.x * b.y - b.x * a.y;
    disc = Sqr(c.r) * sdr - Sqr(d);
    if (disc < -EPS) return 0;
    if (disc < +EPS){
        disc = 0;
    }
    else{
        disc = sqrt(disc);
    }
    x = disc * dx * (dy > 0 ? 1 : -1);
    y = disc * fabs(dy);
    p1.x = (+d * dy + x) / sdr + c.o.x;
    p2.x = (+d * dy - x) / sdr + c.o.x;
    p1.y = (-d * dx + y) / sdr + c.o.y;
    p2.y = (-d * dx - y) / sdr + c.o.y;
    return disc > EPS ? 2 : 1;
}

int ip_circle_circle(const Circle &c1, const Circle &c2, Point &p1, Point &p2){
    double mx = c2.o.x - c1.o.x, sx = c2.o.x + c1.o.x, mx2 = Sqr(mx);
    double my = c2.o.y - c1.o.y, sy = c2.o.y + c1.o.y, my2 = Sqr(my);
    double sq = mx2 + my2, d = -(sq - Sqr(c1.r - c2.r)) * (sq - Sqr(c1.r + c2.r));
    if (!sign(sq)) return 0;
    if (d + EPS < 0) return 0;
    if (d < EPS){
        d = 0;
    }
    else{
        d = sqrt(d);
    }
    double x = mx * ((c1.r + c2.r) * (c1.r - c2.r) + mx * sx) + sx * my2;
    double y = my * ((c1.r + c2.r) * (c1.r - c2.r) + my * sy) + sy * mx2;
    double dx = mx * d, dy = my * d;
    sq *= 2;
    p1.x = (x + dy) / sq; p1.y = (y - dx) / sq;
    p2.x = (x - dy) / sq; p2.y = (y + dy) / sq;
    return d > EPS ? 2 : 1; 
}

double circle_circle_intersection_area(Circle A, Circle B){
    double d, dA, dB, tx, ty;
    d = hypot(B.o.x - A.o.x, B.o.y - A.o.y);
    if ((d < EPS) || (d + A.r <= B.r) || (d + B.r <= A.r)){
        return Sqr((B.r < A.r) ? B.r : A.r) * PI;
    }
    if (d >= A.r + B.r){
        return 0;
    }
    dA = tx = (Sqr(d) + Sqr(A.r) - Sqr(B.r)) / d / 2;
    ty = sqrt(Sqr(A.r) - Sqr(tx));
    dB = d - dA;
    return Sqr(A.r) * acos(dA / A.r) - dA * sqrt(Sqr(A.r) - Sqr(dA)) + Sqr(B.r) * acos(dB / B.r) - dB * sqrt(Sqr(B.r) - Sqr(dB));
}

/*
 * return 2 points of tangency of c and p
 */
void circle_tangents(Circle c, Point p, Point &a, Point &b){
    double d = Sqr(c.o.x - p.x) + Sqr(c.o.y - p.y);
    double para = Sqr(c.r) / d;
    double perp = c.r * sqrt(d - Sqr(c.r)) / d;
    a.x = c.o.x + (p.x - c.o.x) * para - (p.y - c.o.y) * perp;
    a.y = c.o.y + (p.y - c.o.y) * para + (p.x - c.o.x) * perp;
    b.x = c.o.x + (p.x - c.o.x) * para + (p.y - c.o.y) * perp;
    b.y = c.o.y + (p.y - c.o.y) * para - (p.x - c.o.x) * perp;
}

/*
 * +0 : on circle;
 * +1 : inside circle;
 * -1 : outside circle;
 */
int on_circle(Circle c, Point a){
    return sign(c.r - dist(a, c.o));
}

/*
 * minimum circle that covers 2 points
 */
Circle cc2(Point a, Point b){
    return Circle(mp(a, b), dist(a, b) / 2);
}

Circle cc3(Point p, Point q, Point r){
    Circle c;
    if (on_circle(c = cc2(p, q), r) >= 0) return c;
    if (on_circle(c = cc2(p, r), q) >= 0) return c;
    if (on_circle(c = cc2(q, r), p) >= 0) return c;
    c.o = ccc(p, q, r);
    c.r = dist(c.o, p);
    return c;
}

Circle min_circle_cover(Point p[], int n){
    if (n == 1) return Circle(p[0], 0);
    if (n == 2) return cc2(p[0], p[1]);
    random_shuffle(p, p + n);
    Point *ps[4] = {&p[0], &p[1], &p[2], &p[3]};
    Circle c = cc3(*ps[0], *ps[1], *ps[2]);
    while(true){
        Point *b = p;
        for(int i = 1; i < n; i++){
            if (dist(p[i], c.o) > dist(*b, c.o)) b = &p[i];
        }
        if (on_circle(c, *b) >= 0) return c;
        ps[3] = b;
        for(int i = 0; i < 3; i++){
            swap(ps[i], ps[3]);
            if (on_circle(c = cc3(*ps[0], *ps[1], *ps[2]), *ps[3]) >= 0) break;
        }
    }
}
\end{verbatim}

    \subsubsection{垂心,内心,外心}
\begin{verbatim}
point ip(line u, line v) {
  double n = (u.p.y - v.p.y) * (v.q.x - v.p.x) - (u.p.x - v.p.x) * (v.q.y - v.p.y);
  double d = (u.q.x - u.p.x) * (v.q.y - v.p.y) - (u.q.y - u.p.y) * (v.q.x - v.p.x);
  double r = n / d;
  return point(u.p.x + r * (u.q.x - u.p.x), u.p.y + r * (u.q.y - u.p.y));
}

Line perpendicular(Line l, Point a){
    return Line(a, Point(a.x + l.p.y - l.q.y, a.y + l.q.x - l.p.x));
}

Point pedal(Line l, Point a){
    return ip(l, perpendicular(l, a));
}

Point mirror(Line l, Point a){
    Point p = pedal(l, a);
    return Point(p.x * 2 - a.x, p.y * 2 - a.y);
}

//垂心
Point perpencenter(Point a, Point b, Point c){
    Line u = perpendicular(Line(b, c), a);
    Line v = perpendicular(Line(a, c), b);
    return ip(u, v);
}

//内心
Point icc(Point A, Point B, Point C){
    double a = dist(B, C);
    double b = dist(C, A);
    double c = dist(A, B);
    double p = (a + b + c) / 2;
    double s = sqrt(p * (p - a) * (p - b) * (p - c));
    Point cp;
    cp.x = (a * A.x + b * B.x + c * C.x) / (a + b + c);
    cp.y = (a * A.y + b * B.y + c * C.y) / (a + b + c);
    return cp;
}

//外心
Point ccc(Point A, Point B, Point C){
    double a1 = B.x - A.x, b1 = B.y - A.y, c1 = (Sqr(a1) + Sqr(b1)) / 2;;
    double a2 = C.x - A.x, b2 = C.y - A.y, c2 = (Sqr(a2) + Sqr(b2)) / 2;;
    double d = a1 * b2 - a2 * b1;
    Point cp;
    cp.x = A.x + (c1 * b2 - c2 * b1) / d;
    cp.y = A.y + (a1 * c2 - a2 * c1) / d;
    return cp;
}
\end{verbatim}

    \subsubsection{一般多边形}
\begin{verbatim}
/*
 * if point a inside polygon p[n]
 */
int inside_polygon(Point p[], int n, Point a){
	double sum = 0;
	for(int i = 0; i < n; i++){
		int j = (i + 1) % n;
		if (on_lineseg(Line(p[i], p[j]), a)) return 0;
		double angle = acos(dot(a, p[i], p[j]) / dist(a, p[i]) / dist(a, p[j]));
		sum += sign(cross(a, p[i], p[j])) * angle;
	}
	return sign(sum);
}

/*
 * if lineseg l strickly inside polygon p[n]
 */
int lineseg_inside_polygon(Point p[], int n, Line l){
	for(int i = 0; i < n; i++){
		int j = (i + 1) % n;
		Line l1(p[i], p[j]);
		if (on_lineseg_exclusive(l, p[i])) return 0;
		if (intersected_exclusive(l, l1)) return 0;
	}
	return inside_polygon(p, n, mp(l.p, l.q));
}

/*
 * if lineseg l intersect convex polygon p[n]
 */
int intersect_convex_lineseg(Point p[], int n, Line l){
	if (n < 3) return 0;
	Point q[4];
	int k = 0;
	q[k++] = l.p;
	q[k++] = l.q;
	for(int i = 0; i < n; i++){
		if (on_lineseg(l, p[i])){
			q[k++] = p[i];
		}
		else{
			int j = (i + 1) % n;
			Line a(p[i], p[j]);
			Point tmp = ip(a, l);
			if (on_lineseg(l, tmp) && on_lineseg(a, tmp)) q[k++] = tmp;
		}
	}
	sort(q, q + k);
	for(int i = 0; i + 1 < k; i++){
		if (inside_polygon(p, n, mp(q[i], q[i + 1]))) return 1;
	}
	return 0;
}

#define crossOp(p1,p2,p3) sign(cross(p1,p2,p3))

Point isSS(Point p1, Point p2, Point q1, Point q2) {
	double a1 = cross(q1,q2,p1), a2 = -cross(q1,q2,p2);
	return (p1 * a2 + p2 * a1) / (a1 + a2);
}

vector<Point> convexCut(const vector<Point>&ps, Point q1, Point q2) {
	vector<Point> qs;
	int n = ps.size();
	for (int i = 0; i < n; ++i) {
		Point p1 = ps[i], p2 = ps[(i + 1) % n];
		int d1 = crossOp(q1,q2,p1), d2 = crossOp(q1,q2,p2);
		if (d1 >= 0)
			qs.push_back(p1);
		if (d1 * d2 < 0)
			qs.push_back(isSS(p1, p2, q1, q2));
	}
	return qs;
}

typedef double Tdata;
typedef Point Tpoint;

struct Tline {
	Tdata a, b, c;
	double ang;
	Tline() {}
	Tline(Tdata a, Tdata b, Tdata c) : a(a), b(b), c(c) { ang = atan2(b, -a); }
	void get() { scanf("%lf%lf%lf", &a, &b, &c); }
	bool operator <(Tline l) const { return sign(ang - l.ang) < 0 || sign(ang - l.ang) == 0 && sign(c - l.c) < 0; }
};

inline int side(Tline l, Tpoint p) { return sign(l.a * p.x + l.b * p.y + l.c); }

// change line from two point form to general form
// O(1)
// return line(general form)
inline Tline change_line(Tpoint a, Tpoint b) {
	Tdata tmp, A = a.y - b.y, B = b.x - a.x, C = cross(a, b);
	if (sign(A)) tmp = fabs(A);
	else tmp = fabs(B);
	return Tline(A / tmp, B / tmp, C / tmp);
}

// calculate the area of polygon
// O(N)
// be careful the sign of the area
Tdata cal_area(Tpoint *P, int N) {
	if (N < 3) return 0;
	Tdata ret = 0;
	P[N] = P[0];
	for (int i = 0; i < N; ++i) ret += cross(P[i], P[i + 1]);
	return ret / 2.0;
}

// intersection of half-planes
// O(N log N)
// ax + by + c >= 0
// P - points form the intersection, M - number of points
void inter_hplane(Tline *H, int N, Tpoint *P, int &M) {
	int *queue = new int[N + 1];
	sort(H, H + N);
	M = 0;
	for (int i = 0; i < N; ++i)	if (!i || sign(H[i].ang - H[queue[M - 1]].ang)) queue[M++] = i;
	int h = 0, t = 2;
	for (int i = 2; i < M; ++i) {
		while (h + 1 < t && side(H[queue[i]], inter_point(H[queue[t - 1]], H[queue[t - 2]])) < 0) --t;
		while (h + 1 < t && side(H[queue[i]], inter_point(H[queue[h]], H[queue[h + 1]])) < 0) ++h;
		queue[t++] = queue[i];
	}
	while (h + 1 < t && side(H[queue[h]], inter_point(H[queue[t - 1]], H[queue[t - 2]])) < 0) --t;
	while (h + 1 < t && side(H[queue[t - 1]], inter_point(H[queue[h]], H[queue[h + 1]])) < 0) ++h;
	M = 0;
	for (int i = h; i < t; ++i) queue[M++] = queue[i];
	queue[M] = queue[0];
	for (int i = 0; i < M; ++i) P[i] = inter_point(H[queue[i]], H[queue[i + 1]]);
	delete [] queue;
}

// get the core of polygon
// O(N log N)
Tpoint core_of_poly(Tpoint *P, int N) {
	Tline *H = new Tline[N];
	Tpoint *A = new Tpoint[N];
	int M;
	P[N] = P[0];
	for (int i = 0; i < N; ++i) H[i] = change_line(P[i], P[i + 1]);
	inter_hplane(H, N, A, M);
	Tpoint ret = A[0];
	delete [] H; delete [] A;
	return ret;
}

// get the length of segment in convex polygon
// O(N)
Tdata seg_in_convex_poly(Tpoint a, Tpoint b, Tpoint *P, int N) {
	int d1 = point_in_convex_poly(a, P, N), d2 = point_in_convex_poly(b, P, N);
	if (d2 == 1) swap(d1, d2), swap(a, b);
	if (d2 == 1) return dist(a, b);
	Tpoint p;
	P[N] = P[0];
	if (d1 == 1)
		for (int i = 0; i < N; ++i) {
			int d = inter_seg(a, b, P[i], P[i + 1], p);
			if (d == 1 || d == 2) return dist(a, p); // not including the boundaries, add "d == 3" for including the boundaries
		}
	else {
		int cnt = 0;
		Tpoint u, v;
		for (int i = 0; i < N; ++i) {
			int d = inter_seg(a, b, P[i], P[i + 1], p);
			if (d == 3) return 0; // on the boundaries
			if (cnt == 2) continue;
			if (d)
				if (!cnt) u = p, ++cnt;
				else if (u != p) v = p, ++cnt;
		}
		return cnt == 2 ? dist(u, v) : 0;
	}
}

// get the centroid of polygon
// O(N)
Tpoint cal_centroid(Tpoint *P, int N) {
	P[N] = P[0];
	Tpoint c(0, 0);
	Tdata s = 0;
	for (int i = 0; i < N; ++i) {
		Tdata tmp = cross(P[i], P[i + 1]);
		c += (P[i] + P[i + 1]) * tmp; s += tmp;
	}
	return c / (3 * s);
}
\end{verbatim}


	\subsection{空间几何基础}
    \input{code/geometry/geometry3d}
    \subsubsection{空间变换矩阵}
\begin{verbatim}
const int N = 4;
const int MD = 1000000007;
const int INF = 0x3f3f3f3f;
const double PI = acos(-1.0);
const double EPS = 1E-6;

struct Matrix{
	int n, m;
	double v[N][N];
	Matrix(){
		n = m = 4;
		rep(i, 4) rep(j, 4) v[i][j] = (i == j);
	}
	Matrix(int n, int m) : n(n), m(m){
		rep(i, n) rep(j, m) v[i][j] = 0;
	}
};

int n;
Matrix ret;
char s[11];

Matrix operator * (Matrix a, Matrix b){
	Matrix c(a.n, b.m);
	rep(i, c.n){
		rep(j, c.m){
			rep(k, a.m){
				c.v[i][j] += a.v[i][k] * b.v[k][j];
			}
		}
	}
	return c;
}

Matrix translate(){
	Matrix ret;
	double x;
	rep(i, 3){
		scanf("%lf", &x);
		ret.v[i][3] += x;
	}
	return ret;
}

Matrix scale(){
	Matrix ret;
	double x;
	rep(i, 3){
		scanf("%lf", &x);
		ret.v[i][i] *= x;
	}
	return ret;
}

Matrix rotate(){
	Matrix ret;
	double x, y, z, d;
	scanf("%lf%lf%lf%lf", &x, &y, &z, &d);
	double len = sqrt(Sqr(x) + Sqr(y) + Sqr(z));
	x /= len; y /= len; z /= len;
	d = d * PI / 180.0;
	ret.v[0][0] = (1 - cos(d)) * x * x + cos(d);
	ret.v[0][1] = (1 - cos(d)) * x * y - sin(d) * z;
	ret.v[0][2] = (1 - cos(d)) * x * z + sin(d) * y;

	ret.v[1][0] = (1 - cos(d)) * y * x + sin(d) * z;
	ret.v[1][1] = (1 - cos(d)) * y * y + cos(d);
	ret.v[1][2] = (1 - cos(d)) * y * z - sin(d) * x;

	ret.v[2][0] = (1 - cos(d)) * z * x - sin(d) * y;
	ret.v[2][1] = (1 - cos(d)) * z * y + sin(d) * x;
	ret.v[2][2] = (1 - cos(d)) * z * z + cos(d);

	return ret;
}

Matrix pow(Matrix now, int n){
	Matrix ret;
	while(n){
		if (n & 1) ret = now * ret;
		now = now * now;
		n >>= 1;
	}
	return ret;
}

Matrix dfs(int lev){
	Matrix now, tmp;
	while(true){
		scanf("%s", s);
		if (s[1] == 'r'){ // translate
			tmp = translate();
		}
		else if (s[1] == 'c'){ // scale
			tmp = scale();
		}
		else if (s[1] == 'o'){ // rotate
			tmp = rotate();
		}
		else if (s[1] == 'e'){ // repeat
			int k;
			scanf("%d", &k);
			tmp = dfs(lev + 1);
			tmp = pow(tmp, k);
		}
		else if (s[1] == 'n'){ // end
			break;
		}
		now = tmp * now;
	}
	return now;
}

void solve(){
	Matrix now;
	rep(i, n){
		now.n = 4, now.m = 1;
		rep(j, 3) scanf("%lf", &now.v[j][0]);
		now.v[3][0] = 1;
		now = ret * now;
		rep(i, 3) if (fabs(now.v[i][0]) < EPS) now.v[i][0] = 0;
		printf("%.2f %.2f %.2f\n", now.v[0][0], now.v[1][0], now.v[2][0]);
	}
}

int main(){
	while(~scanf("%d", &n)){
		if (!n) break;
		ret = dfs(0);
		solve();
		puts("");
	}
	return 0;
}
\end{verbatim}


	\subsection{凸包}
	\subsubsection{凸包}
\begin{verbatim}
// find the convex hull
Point __o;

bool cmp_p(Point a, Point b){
	int f = sign(a.X - b.X);
	if (f) return f < 0;
	return sign(a.Y - b.Y) < 0;
}

bool cmp(Point a, Point b){
	int f = sign(cross(o, a, b));
	if (f) return f > 0;
	return sign(abs(a - o) - abs(b - o)) < 0;
}

Point stack[1111]

int find_convex(Point p[], int n){
	__o = *min_element(p, p + n, cmp_p);
	sort(p, p + n, cmp);
	int top = 0;
	rep(i, n){
		while(top >= 2 && sign(cross(stack[top - 2], stack[top - 1], p[i])) <= 0) top--;
		stack[top++] = p[i];
	}
	rep(i, top) p[i] = stack[i];
	return top;
}

// -----intersection points convex hull--------
bool lcmp(Line u, Line v){
	int c = sign((u.p.x - u.q.x) * (v.p.y - v.q.y) - (v.p.x - v.q.x) * (u.p.y - u.q.y));
	return c < 0 || !c && sign(cross(u.p, u.q, v.p)) < 0;
}

/*
 * XXX sizeof(p) MUST be as large as n * 2
 * return # of points of resulting convex hull
 */
int ip_convex(Line l[], int n, Point p[]){
	for(int i = 0; i < n; i++){
		if (l[i].q < l[i].p) swap(l[i].p, l[i].q);
	}
	sort(l, l + n, lcmp);
	int n1 = 0;
	for(int i = 0, j = 0; i < n; i = j){
		while(j < n && parallel(l[i], l[j])) j++;
		if (j - i == 1){
			l[n1++] = l[i];
		}
		else{
			l[n1++] = l[i];
			l[n1++] = l[j - 1];
		}
	}
	n = n1;
	l[n + 0] = l[0];
	l[n + 1] = l[1];
	int m = 0;
	for(int i = 0, j = 0; i < n; i++){
		while(j < n + 2 && parallel(l[i], l[j])) j++;
		for(int k = j; k < n + 2 && parallel(l[j], l[k]);k++){
			p[m++] = ip(l[i], l[k]);
		}
	}
	return find_convex(p, m);
}

typedef double Tdata;
typedef Point Tpoint;
// get the diameter of convex polygon
// O(N)
// p1, p2 are the points forming diameter
Tdata diam_convex_poly(Tpoint *P, int N, Tpoint &p1, Tpoint &p2) {
	if (N == 1) {
		p1 = p2 = P[0];
		return 0;
	}
	double ret = -INF;
	for (int j = 1, i = 0; i < N; ++i) {
		while (sign(cross(P[i], P[i + 1], P[j + 1]) - cross(P[i], P[i + 1], P[j])) > 0) j = (j + 1) % N;
		ret = max(ret, max(dist2(P[i], P[j]), dist2(P[i + 1], P[j + 1])));
	}
	return ret;
}
\end{verbatim}

	\input{code/geometry/convex3d}
	\subsubsection{三维凸包n*logn}
\begin{verbatim}
typedef double Tdata;

const int MAXN = 1000 + 10;
const int MAXF = MAXN * 6;
const int MAXM = MAXN * 12;
const double EPS = 1E-6;

inline int sign(Tdata x) { return x < -EPS ? -1 : x > EPS ? 1 : 0; }

struct Tpoint {
    Tdata x, y, z;
    
    Tpoint() {}
    Tpoint(Tdata x, Tdata y, Tdata z) : x(x), y(y), z(z) {}
    void get() { scanf("%lf%lf%lf", &x, &y, &z); }
    bool operator <(Tpoint p) const {
        int s = sign(x - p.x); if (s) return s < 0;
        s = sign(y - p.y); if (s) return s < 0;
        return sign(z - p.z) < 0;
    }
    bool operator ==(Tpoint p) const { return !sign(x - p.x) && !sign(y - p.y) && !sign(z - p.z); }
    void operator -=(Tpoint p) { x -= p.x; y -= p.y; z -= p.z; }
    void operator +=(Tpoint p) { x += p.x; y += p.y; z += p.z; }
    void operator *=(Tdata c) { x *= c; y *= c; z *= c; }
    void operator /=(Tdata c) { x /= c; y /= c; z /= c; }
    Tpoint operator +(Tpoint p) const { return Tpoint(x + p.x, y + p.y, z + p.z); }
    Tpoint operator -(Tpoint p) const { return Tpoint(x - p.x, y - p.y, z - p.z); }
    Tpoint operator *(Tdata c) const { return Tpoint(x * c, y * c, z * c); }
    Tpoint operator /(Tdata c) const { return Tpoint(x / c, y / c, z / c); }
};

inline Tdata sqr(Tdata x) { return x * x; }

inline Tdata norm2(Tpoint p) { return sqr(p.x) + sqr(p.y) + sqr(p.z); }

inline Tdata norm(Tpoint p) { return sqrt(norm2(p)); }

inline Tpoint cross(Tpoint a, Tpoint b) { return Tpoint(a.y * b.z - b.y * a.z, a.z * b.x - b.z * a.x, a.x * b.y - b.x * a.y); }

inline Tpoint cross(Tpoint o, Tpoint a, Tpoint b) { return cross(a - o, b - o); }

inline Tdata det(Tpoint a, Tpoint b, Tpoint c) {
    #define D2(a, b, x, y) (a.x * b.y - a.y * b.x)
    return a.x * D2(b, c, y, z) - a.y * D2(b, c, x, z) + a.z * D2(b, c, x, y);
    #undef D2
}

inline Tdata dot(Tpoint a, Tpoint b) { return a.x * b.x + a.y * b.y + a.z * b.z; }

inline double volume(Tpoint p, Tpoint a, Tpoint b, Tpoint c) { return det(a - p, b - p, c - p); }

struct Tedge {
    int v;
    Tedge *prev, *next, *opp;
    Tedge() {}
    Tedge(int v, Tedge *prev, Tedge *next, Tedge *opp) : v(v), prev(prev), next(next), opp(opp) { }
};

struct Chull3D {
    Tpoint P[MAXN];
    int face[MAXF][3];
    Tedge mem[MAXM], *elist[MAXM];
    Tedge *Fcon[MAXN];
    Tedge *Pcon[MAXF];
    int del[MAXF];
    int mark[MAXN];
    map<int, int> lnk[MAXN];
    int N, F, nfree, col, face_num;
    Tdata vol, area;
    Tpoint cen;

    void alloc_memory() {
        nfree = 12 * N;
        Tedge *e = mem;
        for (int i = 0; i < nfree; ++i) elist[i] = e++;
    }

    inline int vol_sgn(int o, int a, int b, int c) {
        Tdata v = volume(P[o], P[a], P[b], P[c]);
        return sign(v);
    }

    inline void add_face(int a, int b, int c) {
        face[F][0] = a; face[F][1] = b; face[F][2] = c; del[F] = 0; Pcon[F] = NULL;
        lnk[a][b] = lnk[b][c] = lnk[c][a] = F++;
    }

    inline void add_edge(int i, int j) {
        Tedge *a = elist[--nfree], *b = elist[--nfree];
        *a = Tedge(j, NULL, Fcon[i], b); *b = Tedge(i, NULL, Pcon[j], a);
        if (Fcon[i] != NULL) Fcon[i]->prev = a;
        Fcon[i] = a;
        if (Pcon[j] != NULL) Pcon[j]->prev = b;
        Pcon[j] = b;
    }

    inline bool can_see(int p, int f) { return vol_sgn(p, face[f][0], face[f][1], face[f][2]) < 0; }

    //return 0 if all in one plane or line
    bool find_tet() {
        for (int i = 1; i < N; ++i) if (P[i].x < P[0].x) swap(P[i], P[0]);
        for (int i = 2; i < N; ++i) if (P[i].x > P[1].x) swap(P[i], P[1]);
        for (int i = 3; i < N; ++i)
            if (fabs(norm2(cross(P[0], P[1], P[i]))) > fabs(norm2(cross(P[0], P[1], P[2])))) swap(P[2], P[i]);
        if (cross(P[0], P[1], P[2]) == Tpoint(0, 0, 0)) return 0;
        for (int i = 4; i < N; ++i)
            if (fabs(volume(P[0], P[1], P[2], P[i])) > fabs(volume(P[0], P[1], P[2], P[3]))) swap(P[3], P[i]);
        if (!vol_sgn(0, 1, 2, 3)) return 0;

        for (int i = 0; i < 4; ++i) {
            int a = (i + 1) % 4, b = (i + 2) % 4, c = (i + 3) % 4;
            if (vol_sgn(i, a, b, c) < 0) swap(b, c);
            add_face(a, b, c);
        }
        return 1;
    }

    void update(int f1, int f2) {
        for (Tedge *l = Pcon[f1]; l != NULL; l = l->next) {
            int v = l->v;
            if (mark[v] != col && can_see(v, f2)) {
                mark[v] = col;
                add_edge(v, f2);
            }
        }
    }

    bool coplanar(int f1, int f2, int p1, int p2) {
        int vs[4], m = 0;
        for (int i = 0; i < 3; ++i) {
            int v = face[f1][i];
            if (v != p1 && v != p2) vs[m++] = v;
        }
        for (int i = 0; i < 3; ++i) vs[m++] = face[f2][i];
        return vol_sgn(vs[0], vs[1], vs[2], vs[3]) == 0;
    }

    int cal_face() {
        int E = 0, V = 0;
        memset(mark, 0, sizeof(mark));
        for (int i = 0; i < F; ++i)
            if (!del[i])
                for (int j = 0; j < 3; ++j) {
                    int k = lnk[face[i][(j + 1) % 3]][face[i][j]];
                    if (!del[k] && !coplanar(i, k, face[i][j], face[i][(j + 1) % 3])) ++E, mark[face[i][j]] = mark[face[i][(j + 1) % 3]] = 1;
                }
        for (int i = 0; i < N; ++i) if (mark[i]) ++V;
        return 2 + E / 2 - V;
    }

    double cal_volume() {
        double ret = 0;
        for (int i = 0; i < F; ++i)
            if (!del[i]) {
                Tpoint a = P[face[i][0]], b = P[face[i][1]], c = P[face[i][2]];
                ret += volume(Tpoint(0, 0, 0), a, b, c);
            }
        return fabs(ret) / 6.0;
    }

    double cal_area() {
        double ret = 0;
        for (int i = 0; i < F; ++i)
            if (!del[i]) {
                Tpoint a = P[face[i][0]], b = P[face[i][1]], c = P[face[i][2]];
                ret += fabs(norm(cross(a, b, c)) / 2.0);
            }
        return ret;
    }

    Tpoint cal_centroid() {
        Tpoint ret = Tpoint(0, 0, 0);
        for (int i = 0; i < F; ++i)
            if (!del[i]) {
                Tpoint a = P[face[i][0]], b = P[face[i][1]], c = P[face[i][2]];
                ret += (a + b + c) * volume(Tpoint(0, 0, 0), a, b, c);
            }
        return ret / cal_volume() / 24.0;
    }

    void get() {
        scanf("%d", &N);
        for (int i = 0; i < N; ++i) P[i].get();
        sort(P, P + N);
        N = unique(P, P + N) - P; F = 0;    

        alloc_memory();
        vol = 0; area = 0;
        memset(del, 0, sizeof(del));
        for (int i = 0; i < N; ++i) lnk[i].clear();
        if (!find_tet()) return;

        random_shuffle(P + 4, P + N);
        for (int i = 0; i < N; ++i) Fcon[i] = NULL;
        for (int i = 4; i < N; ++i)
            for (int j = 0; j < F; ++j)
                if (can_see(i, j)) add_edge(i, j);

        col = 0;
        int flag = 0;
        memset(mark, 0, sizeof(mark));
        for (int i = 4; i < N; ++i) {
            ++flag;
            for (Tedge *j = Fcon[i]; j != NULL; j = j->next) if (!del[j->v]) del[j->v] = flag;
            for (Tedge *next, *j = Fcon[i]; j != NULL; j = next) {
                int u = j->v;
                next = j->next;
                Tedge *p = j->opp->prev, *n = j->opp->next;
                if (p != NULL) p->next = n;
                else Pcon[u] = n;
                if (n != NULL) n->prev = p;
                elist[nfree++] = j; elist[nfree++] = j->opp;
                for (int k = 0; k < 3; ++k) {
                    int v = lnk[face[u][(k + 1) % 3]][face[u][k]];
                    if (!del[v]) {
                        add_face(face[u][k], face[u][(k + 1) % 3], i);
                        ++col;
                        update(u, F - 1); update(v, F - 1);
                    }
                }
                for (Tedge *next, *l = Pcon[u]; l != NULL; l = next) {
                    next = l->next;
                    Tedge *p = l->opp->prev, *n = l->opp->next;
                    if (p != NULL) p->next = n;
                    else Fcon[l->v] = n;
                    if (n != NULL) n->prev = p;
                    elist[nfree++] = l; elist[nfree++] = l->opp;
                }
            }
        }
        face_num = cal_face(); vol = cal_volume(); area = cal_area(); cen = cal_centroid();
    }
};
\end{verbatim}

	\input{code/geometry/dynamic_convex}
	\subsubsection{两凸包间最短距离}
\begin{verbatim}
const int maxn = 10000 + 10;
const double PI = acos(-1.0);
const double EPS = 1E-6;

struct Tpoint {
    double x, y;
};

Tpoint a[maxn], b[maxn];
int n, m;

inline double cross(double X1, double Y1, double X2, double Y2) {
    return X1 * Y2 - X2 * Y1;
}

double Area(Tpoint *a, int n) {
    double ret = 0;
    a[n] = a[0];
    for (int i = 0; i < n; ++i) ret += cross(a[i].x, a[i].y, a[i + 1].x, a[i + 1].y);
    return ret;
}

inline double Dist(Tpoint A, Tpoint B) {
    return sqrt((A.x - B.x) * (A.x - B.x) + (A.y - B.y) * (A.y - B.y));
}

inline double DistP2S(Tpoint P, Tpoint A, Tpoint B) {
    if ((B.x - A.x) * (P.x - A.x) + (B.y - A.y) * (P.y - A.y) < 0) return Dist(P, A);
    if ((A.x - B.x) * (P.x - B.x) + (A.y - B.y) * (P.y - B.y) < 0) return Dist(P, B);
    return fabs(cross(P.x - A.x, P.y - A.y, P.x - B.x, P.y - B.y)) / Dist(A, B);
}

double MinDist() {
    if (Area(a, n) < 0) reverse(a, a + n);
    if (Area(b, m) < 0) reverse(b, b + m);

    int p1 = 0, p2 = 0;
    for (int i = 0; i < n; ++i)
        if (a[i].x < a[p1].x) p1 = i;
    for (int i = 0; i < m; ++i)
        if (b[i].x > b[p2].x) p2 = i;

    int cnt = 0;
    double ret = dist(a[p1], b[p2]);
    while (cnt < n) {
        ret = min(ret, DistP2S(a[p1], b[p2], b[(p2 + 1) % m]));
        ret = min(ret, DistP2S(a[(p1 + 1) % n], b[p2], b[(p2 + 1) % m]));
        ret = min(ret, DistP2S(b[p2], a[p1], a[(p1 + 1) % n]));
        ret = min(ret, DistP2S(b[(p2 + 1) % m], a[p1], a[(p1 + 1) % n]));

        if (cross(a[(p1 + 1) % n].x - a[p1].x, a[(p1 + 1) % n].y - a[p1].y, b[p2].x - b[(p2 + 1) % m].x, b[p2].y - b[(p2 + 1) % m].y) > 0) p1 = (p1 + 1) % n, ++cnt;
        else p2 = (p2 + 1) % m;
    }

    return ret;
}
\end{verbatim}


	\subsection{平面}
	\input{code/geometry/half_plane_intersection}
	\subsubsection{动态半平面交}
\begin{verbatim}
#include <cstdio>
#include <set>
#include <map>

using namespace std;

const int N = 100000 + 10;
const int inf = 100001;
const double eps = 1e-10;

set<int> S;
int B[N];
double xval[N];
map<double, int> X;

void cal(int a1, int b1, int a2, int b2, double& x, double& y) {
	if (a2 == -inf) {
		x = 0;
		y = b1;
	}
	else {
		x = (b2 - b1) / double(a1 - a2);
		y = a1 * x + b1;
	}
}

int main() {
	int n, a, b, i;
	double x, y;
	char cmd[2];
	set<int>::iterator itr;
	S.insert(-inf);
	S.insert(0);
	B[inf] = 0;
	xval[inf] = inf;
	B[0] = 0;
	X[0] = 0;
	xval[0] = 0;
	scanf("%d", &n);
	for (i=1; i<=n; i++) {
		scanf("%s", cmd);
		if (cmd[0] == 'S') {
			scanf("%d %d", &a, &b);
			itr = S.lower_bound(a);
			if (*itr == a) {
				if (B[-*itr] >= b) continue; else B[-*itr] = b;
			}
			else {
				itr --;
				cal(a, b, *itr, B[-*itr], x, y);
				itr ++;
				if (y < *itr * x + B[-*itr] + eps) continue;
				itr = S.insert(a).first;
				B[-*itr] = b;
			}
			itr --;
			while (itr != S.begin() && *itr * xval[-*itr] + B[-*itr] < a * xval[-*itr] + b + eps) {
				x = xval[-*itr];
				X.erase(x);
				S.erase(itr --);
			}
			cal(a, b, *itr, B[-*itr], x, y);
			itr = S.find(a);
			itr ++;
			X.erase(xval[-*itr]);
			xval[-a] = x;
			X[x] = a;
			itr ++;
			while (itr != S.end() && *itr * xval[-*itr] + B[-*itr] < a * xval[-*itr] + b + eps) {
				itr --;
				S.erase(itr ++);
				x = xval[-*itr];
				X.erase(x);
				itr ++;
			}
			itr --;
			cal(a, b, *itr, B[-*itr], x, y);
			xval[-*itr] = x;
			X[x] = *itr;
		}
		else {
			scanf("%lf", &x);
			x = x * x;
			map<double, int>::iterator itr;
			if (x == 0) itr = X.lower_bound(0);
			else {
				itr = X.lower_bound(x);
				itr --;
			}
			printf("%.0lf\n", itr->second * x + B[-itr->second]);
		}
	}
	return 0;
}                                 
\end{verbatim}

	\input{code/geometry/rotate_calipers}
	\subsubsection{kd树,支持插入}
\begin{verbatim}

const int N = 500005,K = 2,D=6;
const LL inf = ((ULL)1<<63)-1;
//const int inf=~0U>>1;

struct kd{
    T x[K];
    kd(){rep(i,K)x[i]=0;}
} t[N];
int l[N],r[N];
int a[D],n,tot,root;

void insert(int &cur,kd p, int d) {
    if (!cur){
        cur=++tot;
        rep(i,K) t[cur].x[i] = p.x[i];
        l[cur]=r[cur]=0;
        return;
    }
    T dx = p.x[d] - t[cur].x[d];
    if (++d==K) d=0;
    insert(dx<0?l[cur]:r[cur],p,d);
}

T dis2(kd a,kd b) {
    T s=0;
    rep(i,K) s+=Sqr(a.x[i]-b.x[i]);
    return s;
}

void query(int cur, kd p, LL &ret, int d) {
    if (!cur) return;
    ret = min(ret, dis2(t[cur],p));
    T dx = p.x[d] - t[cur].x[d];
    if (++d == K) d = 0;
    if (dx < 0) {
        query(l[cur],p,ret,d);
        if (ret > Sqr(dx)) query(r[cur],p,ret,d);
    } else {
        query(r[cur],p,ret,d);
        if (ret > Sqr(dx)) query(l[cur],p,ret,d);
    }
}

void work() {
    root = tot = 0;
    T ans = inf, ret=0;
    kd p;
    rep(i,n){
        p.x[0] = (p.x[0] * a[0] + a[1]) % a[2];
        p.x[1] = (p.x[1] * a[3] + a[4]) % a[5];
        query(root, p, ans, 0);
        insert(root, p, 0);
        ret += ans * (i > 0);
    }
    printf("%I64d\n", ret);
}

int main() {
    int test;
    scanf("%d", &test);
    rep(cas,test){
        scanf("%d", &n);
        rep(i,D) scanf("%d", &a[i]);
        work();
    }
    return 0;
}
\end{verbatim}

	\subsubsection{knn询问距离最近K个点}
\begin{verbatim}
double cross(Point a,Point b,Point c){return (b.X-a.X)*(c.Y-a.Y)-(c.X-a.X)*(b.Y-a.Y);}

double dot(Point a,Point b,Point c){return (b.X-a.X)*(c.X-a.X)+(b.Y-a.Y)*(c.Y-a.Y);}

bool inpoly(Point a, Point *p, int n){
    int wn = 0;
    rep(i,n){
        Point p1 = p[i], p2 = p[(i + 1) % n];
        double s = cross(a, p1, p2);
        if (!s && dot(a, p1, p2) <= 0) return true;
        double d1 = p1.Y - a.Y, d2 = p2.Y - a.Y;
        if (s > 0 && d1 <= 0 && d2 > 0) ++wn;
        if (s < 0 && d2 <= 0 && d1 > 0) --wn;
    }
    return wn != 0;
}

const int N = 20000, M = 20;

int n, m, r;
Point p[N], poly[M];

const int K = 2;
struct kd {
    LL x[K];
    int id;
}t[N];

double dis2(kd a, kd b){
    double s = 0;
    rep(i,K) s += Sqr(a.x[i] - b.x[i]);
    return s;
}

struct cmpk {
    int k;
    cmpk(int _k): k(_k) {}
    bool operator()(kd a, kd b){ return a.x[k] < b.x[k]; }
};

void build(int l, int r, int d){
    if (r - l <= 1) return;
    int mid = (l + r) >> 1;
    nth_element(t + l, t + mid, t + r, cmpk(d));
    if (++d == K) d = 0;
    build(l, mid, d); build(mid + 1, r, d);
}

typedef priority_queue<pair<double, int> > heap;
void knn(int l, int r, int d, kd p, size_t k, heap &h){
    if (r - l < 1) return;
    int mid = (l + r) >> 1;
    h.push(make_pair(dis2(p, t[mid]), t[mid].id));
    if (h.size() > k) h.pop();
    double dx = p.x[d] - t[mid].x[d];
    if (++d == K) d = 0;
    if (dx < 0) {
        knn(l, mid, d, p, k, h);
        if (h.top().first > Sqr(dx)) knn(mid + 1, r, d, p, k, h);
    } else {
        knn(mid + 1, r, d, p, k, h);
        if (h.top().first > Sqr(dx)) knn(l, mid, d, p, k, h);
    }
}

void solve(){
    scanf("%d", &m);
    rep(i,m) {
        int x,y;
        scanf("%d%d",&x,&y);
        poly[i]=MP(x,y);
    }
    int cnt = 0;
    rep(i,n){
        if (inpoly(p[i], poly, m)) {
            t[cnt].x[0] = p[i].X; t[cnt].x[1] = p[i].Y;
            t[cnt++].id = i + 1;
        }
    }
    build(0, cnt, 0);
    int q;
    scanf("%d", &q);
    while (q--) {
        kd p;
        scanf("%lld%lld", &p.x[0], &p.x[1]);
        heap h;
        knn(0, cnt, 0, p, 2, h);
        int a, b;
        b = h.top().second; h.pop();
        a = h.top().second;
        printf("%d %d\n", a, b);
    }
}

int main(){
    int dat;
    scanf("%d", &dat);
    rep(cas,dat){
        printf("Case #%d:\n", cas+1);
        scanf("%d",&n);
        rep(i,n){
            int x,y;
            scanf("%d%d",&x,&y);
            p[i]=MP(x,y);
        }
        scanf("%d", &r);
        rep(id,r){
            printf("Region %d\n", id+1);
            solve();
        }
    }
}
\end{verbatim}

	\input{code/geometry/range_tree}

	\subsection{面积交}
	\subsubsection{圆与多边形面积交}
\begin{verbatim}
Point p[3];
double r;

double cross(Point a, Point b){
    return a.X * b.Y - a.Y * b.X;
}

double cross(Point a, Point b, Point c){
    return cross(b - a, c - a);
}

double dot(Point a, Point b){
    return a.X * b.X + a.Y * b.Y;
}

double dot(Point a, Point b, Point c){
    return dot(b - a, c - a);
}

double len(Line l){
    return abs(l.S - l.F);
}

double dis(Point p, Line l){
    return fabs(cross(p, l.F, l.S) / len(l));
}

bool inter(Line a, Line b, Point &p){
    double s1 = cross(a.F, a.S, b.F);
    double s2 = cross(a.F, a.S, b.S);
    if (!sign(s1 - s2)) return false;
    p  = (s1 * b.S - s2 * b.F) / (s1 - s2);
    return true;
}

Vec dir(Line l){
    return l.S - l.F;
}

Vec normal(Vec v){
    return Vec(-v.Y, v.X);
}

Vec unit(Vec v){
    return v / abs(v);
}

bool onseg(Point p, Line l){
    return sign(cross(p, l.F, l.S)) == 0 && sign(dot(p, l.F, l.S)) <= 0;
}

double arg(Vec a, Vec b){
    double d = arg(b) - arg(a);
    if (d > PI) d -= 2 * PI;
    if (d < -PI) d += 2 * PI;
    return d;
}

double area(Point a, Point b){
    double s1 = 0.5 * cross(a, b);
    double s2 = 0.5 * arg(a, b) * r * r;
    return fabs(s1) < fabs(s2) ? s1 : s2;
}

double area(){
    double s = 0;
    rep(i, n){
        Point O(0, 0), A = p[i], B = p[(i + 1) % 3];
        Line AB(A, B);
        double d = dis(O, AB);
        if (sign(d - r) >= 0){
            s += area(A, B);
        }
        else{
            Point P;
            inter(AB, Line(O, O + normal(dir(AB))), P);
            Vec v = sqrt(r * r - d * d) * unit(dir(AB));
            Point P1 = P - v, P2 = P + v;
            if (!onseg(P1, AB) && !onseg(P2, AB)){
                s += area(A, B);
            }
            else{
                s += area(A, P1);
                s += area(P1, P2);
                s += area(P2, B);
            }
        }
    }
    return fabs(s);
}

void init(){
    scanf("%d%d", &n, &r);
    rep(i, n){
        double x, y;
        scanf("%lf%lf", &x, &y);
        p[i] = Point(x, y);
    }
}

int main(){
    init();
    printf("%.12lf\n", area());
    return 0;
}
\end{verbatim}

	\subsubsection{k多边形面积交}
\begin{verbatim}
int n;
int v[MAXN]; // the number of vertexes
point p[MAXN][MAXV];
pair<double, int> c[MAXN * MAXV * 2];
double tot[MAXN + 1];

double pos(point p, line ln) {
    return dcmp(ln.second.X - ln.first.X) ?
        (p.X - ln.first.X) / (ln.second.X - ln.first.X) :
        (p.Y - ln.first.Y) / (ln.second.Y - ln.first.Y);
}

double area() {
    memset(tot, 0, sizeof(tot));
    for (int i = 0; i < n; ++i)
        for (int ii = 0; ii < v[i]; ++ii) {
            point A = p[i][ii], B = p[i][(ii + 1) % v[i]];
            line AB = line(A, B);
            int m = 0;
            for (int j = 0; j < n; ++j) if (i != j)
                for (int jj = 0; jj < v[j]; ++jj) {
                    point C = p[j][jj], D = p[j][(jj + 1) % v[j]];
                    line CD = line(C, D);
                    int f1 = dcmp(cross(A, B, C));
                    int f2 = dcmp(cross(A, B, D));
                    if (!f1 && !f2) {
                        if (i < j && dcmp(dot(dir(AB), dir(CD))) > 0) {
                            c[m++] = make_pair(pos(C, AB), 1);
                            c[m++] = make_pair(pos(D, AB), -1);
                        }
                    } else {
                        double s1 = cross(C, D, A);
                        double s2 = cross(C, D, B);
                        double t = s1 / (s1 - s2);
                        if (f1 >= 0 && f2 < 0) c[m++] = make_pair(t, 1);
                        if (f1 < 0 && f2 >= 0) c[m++] = make_pair(t, -1);
                    }
                }
            c[m++] = make_pair(0.0, 0);
            c[m++] = make_pair(1.0, 0);
            sort(c, c + m);
            double s = cross(A, B), z = min(max(c[0].first, 0.0), 1.0);
            for (int j = 1, k = c[0].second; j < m; ++j) {
                double w = min(max(c[j].first, 0.0), 1.0);
                tot[k] += s * (w - z);
                k += c[j].second;
                z = w;
            }
        }
    return tot[0];
}

/*
   tot[0] is the aera of union
   tot[n - 1] is the aera of intersection
   tot[k - 1] - tot[k] is the aera of region covered by k times
   */
\end{verbatim}

	\input{code/geometry/two_polygon_intersection}
	\input{code/geometry/circle_k_intersection}
	\input{code/geometry/rectangle_and_some_circle_intersection}

	\subsection{其他}
    \subsubsection{椭圆周长}
\begin{verbatim}
double const pi = atan2(0, -1.0);

double cal(double a, double b) {
    double e2 = 1.0 - b * b / a / a;
    double e = e2;
    double ret = 1.0;
    double xa = 1.0, ya = 2.0;
    double t = 0.25;
    for (int i = 1; i <= 10000; ++i) {
        ret -= t * e;
        t = t * xa * (xa + 2) / (ya + 2) / (ya + 2);
        xa += 2.0;
        ya += 2.0;
        e *= e2;
    }
    return 2.0 * pi * a * ret;
}

int main() {
    int _ca = 1;
    double a, b;
    int T;
    for (scanf("%d", &T); T--; ) {
        scanf("%lf %lf", &a, &b);
        if (a < b) swap(a, b);
        printf("Case %d: %.10lf\n", _ca++, cal(a, b));
    }
    return 0;
}
\end{verbatim}


\section{\LARGE 理论}
	\subsection{数学}
    \subsubsection{数学结论}
\begin{verbatim}
五边形定理
五边形数 n * (3 * n +- 1) / 2
(1-x)*(1-x^2)*(1-x^3)....=sigma{(-1)^k * x^(n * (3 * n (+-) 1) / 2)}
即f[n] = f[n - 1] + f[n - 2] - f[n - 5] - f[n - 7] + f[n - 12] + f[n - 15] - .....

fibonacci数性质:
f[n] = f[n - 1] + f[n - 2]
f[n + m + 1] = f[n] * f[m] + f[n + 1] * f[m + 1]
gcd(f[n], f[n + 1]) = 1
gcd(f[n], f[n + 2]) = 1
gcd(f[n], f[m]) = f[gcd(n, m)]
f[n+1]*f[n+1]-f[n]*f[n+2] = (-1)^n
sigma{f[i]^2, 1<=i<=n} = f[n]*f[n+1]
sigma{f[i], 0<=i<=n} = f[n+2] - 1
sigma{f[2*i-1],1<=i<=n} = f[2*n]
sigma{f[2*i],1<=i<=n} = f[2*n+1]-1
sigma{(-1)^i*f[i],0<=i<=n} = (-1)^n*(f[n+1]-f[n])+1
f[2*n-1]=f[n]^2-f[n-2]^2
f[2*n+1]=f[n]^2+f[n+1]^2
3*f[n]=f[n+2]+f[n-2]
f[n]=c(n-1,0)+c(n-2,1)+..c(n-1-m,m) (m<=n-1-m)
sigma{f[i]*i,1<=i<=n}=n*f[n+2]-f[n+3]+2

catalan数性质:
凸多边形三角剖分数
简单有序根树的计数
(0,0)走到(n,n)经过的点(a,b)满足a<=b的路径数
乘法结合问题
c[n+1] = (4 * n - 2) / (n + 1) * c[n]
c[n] = (2*n)!/(n!)/((n+1)!)

第一类stirling数性质
有正有负,其绝对值是n个元素的项目分作k个环排列的数量,s[n,k](n个人分成k组,每组再按特定顺序围圈)
s[n][0] = 0, s[1][1] = 1;
s[n+1][k]= = s[n][k - 1] + n * s[n][k]
|s[n][1]| = (n-1)!
s[n][k] = (-1)^(n+k)*|s[n][k]|
s[n][n-1] = -C(n,2)
x*(x-1)*(x-2)..(x-n+1) = sigma{s[n][k] * x ^k}

第二类stirling数性质
n个元素的集定义k个等价类的方法数目(n个人分成k组的方法数)
s[n][n] = s[n][1] = 1
s[n][k] = s[n - 1][k - 1] + k * s[n - 1][k]
s[n][n - 1] = C(n, 2)
s[n][2] = 2^(n-1)-1
s[n][k] = 1/(k!)sigma{(-1)^k-j * C(k, j) * j ^n, 1<=j<=k}

bell数性质
B[n] = sigma{s[n][k], 1<=k<=n}
B[n+1] = simga{C(n,k)*B[k], 0<=k<=n}
B[p+n] = B[n] + B[n + 1] (mod p)
B[p^m+n] = B[n] + B[n+1] (mod p)

多项式性质
f(x)不存在重根<=>gcd(f(x), f‘(x))的次数小于1次
多项式gcd可以用来判断两多项式是否有公共根

多项式取模
f[x] = 0 (mod m) 
m = m1 * m2 * m3 ... mk
Ti 表示 f[x] = 0 (mod mi)的解数,则T = T1 * T2 * T3...Tk

数论
a^n % b = a^(n % phi(b) + phi(b)) % b (n >= phi(b))
lucas定理 c(n, m) = c(n % p, m % p) * c(n / p, m / p) % p
lucas函数 满足 f(n, m) = f(n % p, m % p) * f(n / p, m / p) % p, 可以猜测满足

原根
2,4,p^k,2*p^k存在原根,存在原根则原根数量为phi(phi(n))
验证原根x = phi(n), x = p1^a1*p2^a2..pk^ak
原根满足t ^ (x / pi) != 1 (mod n)

x*x+y*y==n的整数解:
x*x+y*y==n的整数解个数num = 4 * sigma{H(d), d | n}
H(d) =
(1) 奇数 : (-1)^((d-1)/2)
(2) 偶数 : 0

平方和定理:
(1)费马平方和定理:
    奇质数能表示为两个平方数之和的充分必要条件是该素数被4除余1
(2)费马平方和定理的拓展定理:
    正整数能表示为两平方数之和的充要条件是在它的标准分解式中,形如素因子的指数是偶数
(3)Brahmagupta–Fibonacci identity
    如果两个整数都能表示为两个平方数之和,则它们的积也能表示为两个平方数之和。公式及拓展公式为
\end{verbatim}
$$(a^{2}+b^{2})(c^{2}+d^{2})=(ac-bd)^{2}+(ad+bc)^{2}=(ac+bd)^{2}+(ad-bc)^{2}$$
$$(a^{2}+n*b^{2})(c^{2}+n*d^{2})=(ac-n*bd)^{2}+n*(ad+bc)^{2}=(ac+n*bd)^{2}+n(ad-bc)^{2}$$
\begin{verbatim}
    从这个定理可以看出:如果不能表示为三个数的平方和,那么也就不能表示为两个数的平方和。
(4)四平方和定理:
    每个正整数都可以表示成四个整数的平方数之和
(5)表为3个数的平方和条件: 
    正整数能表示为三个数的平方和的充要条件是不能表示成的形式,其中和为非负 整数。

连分数
连分数(a+(n^0.5)) / b
开始时,i满足,(a+i)/b=floor((a+(n^0.5))/b),之后过程一样
如果不成功,则可以变换为(ab+((nb^2)^0.5))/(b^2),之后再来

杨氏矩阵又叫杨氏图表,它是这样一个矩阵,满足条件:

 
杨氏矩阵
(1)如果格子(i,j)没有元素,则它右边和上边的相邻格子也一定没有元素。
(2)如果格子(i,j)有元素a[i][j],则它右边和上边的相邻格子要么没有元素,要么有元素且比a[i][j]大。
1 ~ n所组成杨氏矩阵的个数可以通过下面的递推式得到:
f[1] = 1; f[2] = 2;
f[n] = f[n - 1] + (n - 1) * f[n - 2];

钩子公式:
对于给定形状,不同的杨氏矩阵的个数为:n!除以每个格子的钩子长度加1的积。其中钩子长度定义为该格子
右边的格子数和它上边的格子数之和。
\end{verbatim}

    \input{code/theory/cubic_solver}
    \input{code/theory/integeral}
    \subsubsection{高斯消元}
\begin{verbatim}
//在异或方程里,要求最小改变次数,那就从后往前面枚举,先枚举只有变量之后,前面的变量就确定了
void gauss(double p[M][M]){
    static double *b[M];
    rep(i, M) b[i] = tmp[i];
    rep(i, M){
        REP(j, i, M){
            if (sign(fabs(b[j][i]) - fabs(b[i][i])) > 0) swap(b[i], b[j]);
        }
        rep(j, M){
            if (i == j) continue;
            double rate = b[j][i] / b[i][i];
            rep(k, M + M) b[j][k] -= b[i][k] * rate;
        }
        double rate = b[i][i];
        rep(j, M + M) b[i][j] /= rate;
    }
    rep(i, M) rep(j, M) p[i][j] = b[i][j + M];
}
\end{verbatim}

    \subsubsection{FFT}
\begin{verbatim}
typedef complex<long double> Comp;
class FFT {
    public:
        FFT(int n);
        void forward(Comp a[]) {
            compute(a, r);
        }
        void reverse(Comp a[]){
            compute(a, ir);
            for (int i = 0; i < n; i++) a[i] /= n;
        }
    private:
        int n, p;
        vector<int> rb;
        Comp r[20];
        Comp ir[20];
        void compute(Comp a[], Comp* r);
};

FFT::FFT(int n) : n(n) , rb(n) , p(0) {
    while ((1 << p) < n) ++p;
    for(int i = 0; i < n; i++){
        int x = i, y = 0;
        for (int j = 0; j < p; ++j) {
            y = (y << 1) | (x & 1);
            x >>= 1;
        }
        rb[i] = y;
    }
    for(int i = 0; i < p; i++){
        long double angle = 2 * PI / (1 << (i + 1));
        ir[i] = Comp(cos(angle), sin(angle));
        r[i] = std::conj(ir[i]);
    }
}

void FFT::compute(Comp a[], Comp* r) {
    for (int i = 0; i < n; ++i) if (rb[i] > i) swap(a[i], a[rb[i]]);
    for (int len = 2; len <= n; len <<= 1) {
        Comp root = *r++;
        for (int i = 0; i < n; i += len) {
            Comp w(1, 0);
            for (int j = 0; j < len / 2; ++j) {
                Comp u = a[i + j];
                Comp v = a[i + j + len / 2] * w;
                a[i + j] = u + v;
                a[i + j + len / 2] = u - v;
                w *= root;
            }
        }
    }
}
\end{verbatim}

    \input{code/theory/linear_programming}
    \subsubsection{线性递推式n*n*logn}
\begin{verbatim}
/* f[i] = a[i], i < m; 
 * f[n] = b[0] * f[n - m] + ... + b[m - 1] * f[n - 1]; 
 * given a[], b[], m, n; find f[n]
 * O(M ^ 2 log N) 
 * !!!!m = 1 特判
 */

const int M = 222;
const int MD=1000000007;

LL n;
int u,d;
int p[M],q[M];
bool use[M];
LL a[M],b[M];

int calc(LL n,int m,LL a[],LL c[],int p=MD){
    LL v[M]={1%p},u[M<<1],msk=!!n;
    for(LL i=n;i>1;i>>=1) msk<<=1;
    for(LL x=0;msk;msk>>=1,x<<=1){
        fill_n(u,m<<1,0);
        int b=!!(n&msk);
        x|=b;
        if (x<m) u[x]=1%p;
        else{
            rep(i,m) for(int j=0,t=i+b;j<m;++j,++t) u[t]+=v[i]*v[j],u[t]%=p;
            fba(i,(m<<1)-1,m) for(int j=0,t=i-m;j<m;++j,++t) u[t]+=c[j]*u[i],u[t]%=p;
        }
        copy(u,u+m,v);
    }
    LL ret=0;
    rep(i,m) ret+=v[i]*a[i],ret%=p;
    return ret;
}

int main(){
    while(~scanf("%I64d",&n)){
        Cls(a);
        Cls(b);
        Cls(use);
        scanf("%d",&u);
        rep(i,u) scanf("%d",p+i);
        scanf("%d",&d);
        rep(i,d) scanf("%d",q+i);
        int top=0;
        rep(i,d) top=max(top,q[i]+1),use[q[i]]=true;
        b[0]=1;
        REP(i,1,top){
            rep(j,u) if (i>=p[j]) b[i]+=b[i-p[j]],b[i]%=MD;
        }
        rep(i,top) if (!use[i]) b[i]=0;
        a[0]=1;
        REP(i,1,top){
            fab(j,1,i) a[i]+=a[i-j]*b[j],a[i]%=MD;
        }
        reverse(b,b+top);
        printf("%d\n",calc(n,top-1,a,b));
    }
    return 0;
}
\end{verbatim}

    \subsubsection{线性递推式+fft}
\begin{verbatim}
/* f[i] = a[i], i < m; 
 * f[n] = b[0] * f[n - m] + ... + b[m - 1] * f[n - 1]; 
 * given a[], b[], m, n; find f[n]
 * O(M ^ 2 log N) 
 * !!!!m = 1 特判
 */
using namespace std;

typedef long long LL;
typedef long long ULL;
//typedef complex<double> Comp;

const int N = 10000;
const int MD = 5767169;
const long double EPS = 0.05;
const double PI = acos(-1.0);

typedef complex<double> Comp;
typedef Comp cp;

const Comp I(0, 1);
const int M = 1 << 15;

Comp fft_a[M], fft_b[M];
Comp tmp[M];

int n, a, b, p, q;

int fac[MD], inv[MD];

inline int ex_gcd(int a, int b, int &x, int &y){
    if (!a){
        x = 0;
        y = 1;
        return b;
    }
    int g = ex_gcd(b % a, a, x, y);
    int t = y;
    y = x;
    x = t - (b / a) * y;
    return g;
}

inline int qmod(int x, int n){
    int ret = 1, y = x % MD;
    while(n){
        if (n & 1) ret = (LL)ret * y % MD;
        y = (LL)y * y % MD;
        n >>= 1;
    }
    return ret;
}

inline int lucas(int n, int m){
    int ret = 1;
    while(n && m && ret){
        int nn = n % MD, mm = m % MD;
        if (nn < mm) return 0;
        else ret = (LL)ret * fac[nn] % MD * inv[mm] % MD * inv[nn - mm] % MD;
        n /= MD, m /= MD;
    }
    return ret;
}

void init(){
    fac[0] = 1;
    inv[0] = 1;
    inv[1] = 1;
    for(int i = 2; i < MD; i++){
        inv[i] = -(LL)(MD / i) * inv[MD % i] % MD;
    }
    for(int i = 1; i < MD; i++){
        fac[i] = ((LL)fac[i - 1] * i) % MD;
        inv[i] = ((LL)inv[i - 1] * inv[i]) % MD;
    }
}

void fft(Comp *a, int n, int f = 1){
    double arg = PI;
    for(int k = n >> 1; k; k >>= 1, arg *= 0.5) {
        cp wm(cos(arg), f * sin(arg)), w(1, 0);
        for (int i = 0; i < n; i += k, w *= wm) {
            int p = i << 1;
            if (p >= n) p -= n;
            for (int j = 0; j < k; ++j) tmp[i + j] = a[p + j] + w * a[p + k + j];
        }
        rep(i,n) a[i] = tmp[i];
    }
}

void calc(Comp *a, Comp *b, int n){
    fft(a, n, 1);
    fft(b, n, 1);
    rep(i, n) a[i] = a[i] * b[i];
    fft(a, n, -1);
    rep(i, n) a[i] /= n;
}

int calc(int n, int m){
    static int v[(M << 1) - 1], u[(M << 1) - 1], t[(M << 1) - 1];
    int size = 1;
    while (size <= 2 * m - 2) size *= 2;
    rep(i, 2 * m - 1) v[i] = 0;
    v[0] = 1;
    int now = 1;
    while(n){
        if (now < m){
            rep(i, 2 * m - 1) u[i] = 0;
            u[now] = 1;
        }
        else{
#ifdef FFT
            rep(i, size) fft_a[i] = fft_b[i] = Comp(0, 0);
            rep(i, m){
                fft_a[i] = Comp(u[i], 0);
                fft_b[i] = Comp(u[i], 0);
            }
            calc(fft_a, fft_b, size);
            rep(i, m * 2 - 1) u[i] = (LL)(fft_a[i].real() + EPS) % MD;
#else
            for(int i = 0; i < m; i++) t[i] = u[i], u[i] = 0;
            for(int i = 0; i < m; i++){
                for(int j = 0; j < m; j++){
                    u[i + j] = (u[i + j] + (LL)t[i] * t[j]) % MD;
                }
            }
#endif
            for(int i = 2 * m - 2; i >= m; i--){
                u[i - p] = (u[i - p] + (LL)a * u[i]) % MD;
                u[i - q] = (u[i - q] + (LL)b * u[i]) % MD;
                u[i] = 0;
            }
        }
        if (n & 1){
#ifdef FFT
            rep(i, size) fft_a[i] = fft_b[i] = Comp(0, 0);
            rep(i, m){
                fft_a[i] = Comp(u[i], 0);
                fft_b[i] = Comp(v[i], 0);
            }
            calc(fft_a, fft_b, size);
            rep(i, m * 2 - 1) v[i] = (LL)(fft_a[i].real() + EPS) % MD;
#else
            for(int i = 0; i < m; i++) t[i] = v[i], v[i] = 0;
            for(int i = 0; i < m; i++){
                for(int j = 0; j < m; j++){
                    v[i + j] = (v[i + j] + (LL)t[i] * u[j]) % MD;
                }
            }
#endif
            for(int i = 2 * m - 2; i >= m; i--){
                v[i - p] = (v[i - p] + (LL)a * v[i]) % MD;
                v[i - q] = (v[i - q] + (LL)b * v[i]) % MD;
                v[i] = 0;
            }
        }
        now <<= 1;
        n >>= 1;
    }
    return v[m - 1];
}

int main(){    
    init();
    while(~scanf("%d%d%d%d%d", &a, &b, &p, &q, &n)){
        if (p > q) swap(p, q), swap(a, b);
        int ret;
        if (p == q){
            if (n % p != 0){
                ret = 0;
            }
            else{
                ret = qmod(a + b, n);
            }
        }
        else if (q <= 1000){
            n += q - 1;
            ret = calc(n, q);
        }
        else{
            ret = 0;
            int x, y;
            int d = ex_gcd(p, q, x, y);
            if (n % d != 0){
                puts("0");
                continue;
            }
            x = ((LL)x * n / d % (q / d) + (q / d)) % (q / d);
            y = (n - p * x) / q;
            while(y >= 0){
                ret = (ret + (LL)qmod(a, x) * qmod(b, y) % MD * lucas(x + y, x) % MD) % MD;
                x += q / d;
                y -= p / d;
            }
        }
        printf("%d\n", (ret % MD + MD) % MD);
    }
    return 0;
}
\end{verbatim}

	\input{code/theory/newton}

	\subsection{数论}
	\subsubsection{O(n)求逆元}
\begin{verbatim}
inv[1]=1;
rep(i, MD){
	if (i < 2) continue;
	inv[i] = - MD / i * (LL) inv[MD % i] % MD;
	inv[i] = (inv[i] % MD + MD) % MD;
}
\end{verbatim}

    \subsubsection{中国剩余定理(非互质)}
\begin{verbatim}
LL exgcd(LL a,LL b,LL &x,LL &y) {
    if (!a){
        x = 0;
        y = 1;
        return b;
    }
    LL g = exgcd(b % a, a, x, y);
    LL t = y;
    y = x;
    x = t - (b / a) * y;
    return g;
}

LL CRT(const vector<LL>& m,const vector<LL>& b) {
    bool flag = false;
    LL x, y, i,d,result,a1,m1,a2,m2,Size=m.size();
    m1 = m[0]; a1 = b[0];
    for(i = 1; i < Size; ++i){
        m2 = m[i]; a2 = b[i];
        d = exgcd(m1, m2, x, y );
        if((a2-a1) % d != 0) flag = true;
        result = (x * ((a2-a1) / d ) % m2 + m2 ) % m2;
        a1 = a1 + m1 * result; //对于求多个方程
        m1 = (m1 * m2) / d;    //lcm(m1,m2)最小公倍数
        a1 = (a1 % m1 + m1) % m1;
    }
    if (flag) return -1;
    else return a1;
}
\end{verbatim}

	\subsubsection{取模}
\begin{verbatim}
#include <cstdio>
#include <cstring>
#include <algorithm>
#include <cmath>
using namespace std;

typedef long long LL;
typedef long long LL;

LL gcd(LL x, LL y) { 
    return !y ? x : gcd(y, x % y);
}

LL modular(LL a, LL b) { 
    return (a % b + b) % b; 
}

/** a * x + b * y == gcd(a, b) */
LL exgcd(LL a,LL b,LL &x,LL &y) {
    if (!a){
        x = 0;
        y = 1;
        return b;
    }
    LL g = exgcd(b % a, a, x, y);
    LL t = y;
    y = x;
    x = t - (b / a) * y;
    return g;
}

/** x * y % m == 1 */
LL invert(LL x, LL m)  { 
    LL a, b;
    exgcd(x, m, a, b);
    return modular(a, m);
}

/** x % m == a && x % n == b */
LL modular_system(LL m, LL a, LL n, LL b) { 
    LL g, k, l;
    g = exgcd(m, n, k, l);
    if ((a - b) % g) return -1;
    k *= (b - a) / g;
    k = modular(k, n / g);
    return modular(k * m + a, m / g * n);
}

/** x % m[i] == r[i] */
LL modular_system_array(LL m[], LL r[], int k) {
    LL M = m[0], R = r[0];
    for (int i = 1; R != -1 && i < k; i++) {
        R = modular_system(M, R, m[i], r[i]);
        M = M / gcd(M, m[i]) * m[i];
    }
    return R;
}

/** a * x % m == b */
LL modular_equation(LL a, LL m, LL b) { 
    return modular_system(m, b, a, 0) / a % m;
}

/** calculate r = x ^ y % m */
LL modular_pow(LL x, LL y, LL m) {
    LL r = 1 % m;
    for (; y; y >>= 1, x = x * x % m)
        if (y & 1) r = r * x % m;
    return r;
}

//a ^ x = b (mod n), n is prime
int mod_log(int a, int b, int n) {
    int m = (int)ceil(sqrt(n)), inv = Inv(mod_exp(a, m, n), n);
    id[0] = 0; mexp[0] = 1;
    for (int i = 1; i < m; ++i) id[i] = i, mexp[i] = (long long)mexp[i - 1] * a % n;
    sort(id, id + m, logcmp); sort(mexp, mexp + m);
    for (int i = 0; i < m; ++i) {
        int j = lower_bound(mexp, mexp + m, b) - mexp;
        if (j < m && mexp[j] == b) return i * m + id[j];
        b = (long long)b * inv % n;
    }
    return -1;
}

// A ^ x = B (mod C)
// return x, -1 means no solution
int Dlog(int A, int B, int C) {
    map<int, int> Hash;
    int D = 1 % C;
    for(int j = D, i = 0;i <= 100; j = (long long)j * A % C, ++i)
        if (j == B) return i;
    int d = 0;
    for (int g; (g = gcd(A, C)) != 1; ) {
        if (B % g)return -1;
        ++d; C /= g; B /= g; D = (long long)D * A / g % C;
    }
    int M = (int)ceil(sqrt((double)C));
    for (int j = 1 % C, i = 0; i <= M; j = (long long)j * A % C, ++i)
        if (Hash.find(j) == Hash.end()) Hash[j] = i;
    for (int j = mod_exp(A, M, C), i = 0; i <= M; D = (long long)D * j % C, ++i) {
        int tmp = Inval(D, B, C);
        if (tmp >= 0 && Hash.find(tmp) != Hash.end()) return i * M + Hash[tmp] + d;
    }
    return -1;
}
\end{verbatim}

	\input{code/theory/combination_number}
	\subsubsection{雅可比判别式(二次剩余)}
\begin{verbatim}
//n为合数,判别不一定正确
int Jacobi(int a, int n) {
	if (a == 0) return 0;
	if (a == 1) return 1;
	int s, n1, a1 = a, e = 0;
	while (!(a1 & 1)) a1 >>= 1, ++e;
	if (!(e & 1)) s = 1;
	else {
		int u = n % 8;
		if (u == 1 || u == 7) s = 1;
		else s = -1;
	}
	if (n % 4 == 3 && a1 % 4 == 3) s = -s;
	n1 = n % a1;
	if (a1 == 1) return s;
	return s * Jacobi(n1, a1);
}
\end{verbatim}

	\input{code/theory/prime}
	\input{code/theory/prime_random}
	\subsubsection{米勒测试}
\begin{verbatim}
LL mul_mod(LL x, LL y, LL n) {
    if (!x) return 0;
    return (((x & 1) * y) % n + (mul_mod(x >> 1, y, n) << 1) % n) % n;
}

LL pow_mod(LL a, LL x, LL n) {
    LL ret = 1;
    while (x) {
        if (x & 1) ret = mul_mod(ret, a, n);
        a = mul_mod(a, a, n); x >>= 1;
    }
    return ret;
}

bool millerRabin(LL a, LL n) {
    LL r = 0, s = n - 1;
    while (!(s & 1)) {
        s >>= 1;
        ++r;
    }
    LL x = pow_mod(a, s, n);
    if (x == 1 || x == n - 1) return 1;
    for (int j = 1; j < r; ++j) {
        x = mul_mod(x, x, n);
        if (x == 1) return 0;
        if (x == n - 1) return 1;
    }
    return 0;
}

/*
    if n < 1,373,653, it is enough to test a = 2 and 3;
    if n < 9,080,191, it is enough to test a = 31 and 73;
    if n < 4,759,123,141, it is enough to test a = 2, 7, and 61;
    if n < 2,152,302,898,747, it is enough to test a = 2, 3, 5, 7, and 11;
    if n < 3,474,749,660,383, it is enough to test a = 2, 3, 5, 7, 11, and 13;
    if n < 341,550,071,728,321, it is enough to test a = 2, 3, 5, 7, 11, 13, and 17.
    */
bool isPrime(LL n) {
    for (int i = 2; i < 1000 && i < n; ++i)
        if (n % i == 0) return 0;
    if (!millerRabin(2, n)) return 0;
    if (!millerRabin(3, n)) return 0;
    if (!millerRabin(5, n)) return 0;
    if (!millerRabin(7, n)) return 0;
    return 1;
}
\end{verbatim}

	\input{code/theory/pollard}
	\subsubsection{pell方程}
\begin{verbatim}
// x * x - D * y * y = 1, xn+yn*sqrt(d) = (x0+y0*sqrt(d))^n
// x * x - D * y * y = -1, xn+yn*sqrt(d) = (x0+y0*sqrt(d))^(2*n+1)
//
// x * x - D * y * y = -1, D为质数,有解即D!=3(mod 4)
// 当D==0(mod 4)时,无解
//
// a * x * x - b * y * y = c 
// get x0, y0 from x * x - a * b * y = 1
// get x1, y1 from a * x * x - b * y * y = c
// [xk] = [x0, by0] ^ k-1 * [x1]
// [yk] = [ay0, x0]            [y1]
//
bool pell( int D, int& x, int& y ) {  
    int sqrtD = sqrt(D + 0.0);  
    if( sqrtD * sqrtD == D ) return false;  
    int c = sqrtD, q = D - c * c, a = (c + sqrtD) / q;  
    int step = 0;  
    int X[] = { 1, sqrtD };  
    int Y[] = { 0, 1 };  
    while( true ) {  
        X[step] = a * X[step^1] + X[step];  
        Y[step] = a * Y[step^1] + Y[step];  
        c = a * q - c;  
        q = (D - c * c) / q;  
        a = (c + sqrtD) / q;  
        step ^= 1;  
        if( c == sqrtD && q == 1 && step ) {  
            x = X[0], y = Y[0];  
            return true;  
        }  
    }  
}

//{{{ pell x*x-d*y*y = -1

struct Matrix{
  int n, m;
  LL v[2][2];
}c, tmp, ans;

Matrix operator*(const Matrix &a, const Matrix &b){
  c.n = a.n, c.m = b.m;
  for(int i = 0; i < c.n; i++){
    for(int j = 0; j < c.m; j++){
      c.v[i][j] = 0;
      for(int k = 0; k < a.m; k++){
        c.v[i][j] = (c.v[i][j] + 1LL * a.v[i][k] * b.v[k][j]);
      }
    }
  }
  return c;
}

int n, l;
int base;
int a[N];

bool build(int n){
  base = 0;
  while(base * base <= n) base++;
  base--;
  if (base * base == n) return false;
  int k = base;
  int n_k = n - k * k;
  l = 0;
  a[l++] = k;
  while(true){
    int i1 = n_k - k % n_k;
    i1 += ((base - i1) / n_k) * n_k;
    a[l++] = (i1 + k) / n_k;
    if (a[l - 1] == 2 * base) break;
    k = i1;
    n_k = (n - k * k) / n_k;
  }
  return true;
}

void solve(){
  ans.n = 2, ans.m = 2;
  ans.v[0][0] = a[0], ans.v[0][1] = 1;
  ans.v[1][0] = 1, ans.v[1][1] = 0;
  for(int i = 1; i < l - 1; i++){
    tmp.n = 2, tmp.m = 2;
    tmp.v[0][0] = a[i], tmp.v[0][1] = 1;
    tmp.v[1][0] = 1, tmp.v[1][1] = 0;
    ans = ans * tmp;
  }
  if (ans.v[0][0] * ans.v[0][0] - n * ans.v[1][0] * ans.v[1][0] == -1){
    printf("%d %d\n", ans.v[0][0], ans.v[1][0]);
  }
  else{
    puts("No Solution");
  }
}

int main(){
  scanf("%d", &n);
  if (build(n)) solve();
}

//}}}
\end{verbatim}

    \subsubsection{二次剩余, 解数}
\begin{verbatim}
int Euler(int a, int p) {
    int ret = 1, s = a, k = (p - 1) / 2;
    while (k) {
        if (k & 1) ret = (long long)ret * s % p;
        s = (long long)s * s % p;
        k >>= 1;
    }
    if (ret != 1) ret = 0;
    else ret = 2;
    return ret;
}        

int cal(int p, int n, int d) {
    int pn = 1;
    for (int i = 0; i < n; ++i) pn *= p;
    d %= pn;
    if (d == 0) {
        int k = 1;
        for (int i = 0; i < n / 2; ++i) k *= p;
        return k;
    }
    int r, b = 0, pr, pb;
    while (d % p == 0) {
        d /= p;
        ++b;
    }
    if (b % 2 != 0) return 0;
    r = b / 2;
    pr = 1;
    for (int i = 0; i < r; ++i) pr *= p;
    if (p == 2) {
        n -= b;
        if (n < 2) return 1 * pr;
        if (n == 2 && d % 4 == 1) return 2 * pr;
        if(n > 2 && d % 8 == 1) return 4 * pr;
        return 0;
    }
    return pr * Euler(d, p);
}

// x^2 = d (% m)
int QuadraticResidue(int m, int d) {
    int ret = 1;
    for (int i = 2; i * i <= m; ++i)
        if (m % i == 0) {
            int j = 0, q = 1;
            while (m % i == 0) {
                m /= i;
                ++j;
                q *= i;
            }
            ret *= cal(i, j, d);
        }
    if (m > 1) ret *= cal(m, 1, d);
    return ret;
}
\end{verbatim}

    \input{code/theory/quadratic_residue_solution}

	\subsection{博弈论}
	\input{code/theory/nim_product}

\section{\LARGE 其他}
    \input{code/other/bignum}
    \subsubsection{计算器}
\begin{verbatim}
#include <cstdio>
#include <cstring>
#include <cctype>
#include <algorithm>
using namespace std;

const int INTMAX = 0x7fffffff;
typedef long long ll;

int x, ok;
struct Token { int t; ll val; } la;

inline bool fit(ll x) { return -INTMAX-1 <= x && x <= INTMAX; }

Token next_token() {
  while (isspace(x)) x = getchar();
  if (x == EOF) return (Token){EOF}; // XXX g++ only!!

  if (strchr("()+-*/%^", x)) {
    Token res = {x};
    x = getchar();
    return res;
  }
  
  ll z;
  for (z = 0; isdigit(x); x = getchar()) {
    z = z * 10 + x - '0';
    if (!fit(z)) ok = 0;
  }
  return (Token){'n', z};
}

int pow(int x, int y) {
  ll r = 1;
  if (x == 0 && y == 0 || y < 0) { ok = 0; return 1; }
  if (x == 0 || x == 1) return x;
  if (x == -1) y %= 2;
  
  while (y--) {
    r *= x;
    if (!fit(r)) { ok = 0; return 1; }
  }
  return r;
}

void shift() { la = next_token(); }

bool match(int t) {
  if (la.t == t) {
    shift();
    return 1;
  }
  return 0;
}

int exp();
int term();
int factor();
int unit();

int exp() {
  ll ans = term();
  while (1) {
    if (match('+')) {
      ans += term();
    } else if (match('-')) {
      ans -= term();
    } else break;
    
    if (!fit(ans)) ok = 0;
  }
  return ans;
}

int term() {
  ll t, ans = factor();
  while (1) {
    if (match('*')) {
      ans *= factor();
    } else if (match('/')) {
      if (t = factor())
        ans /= t;
      else
        ok = 0;
    } else if (match('%')) {
      if (t = factor())
        ans %= t;
      else
        ok = 0;
    } else break;
    if (!fit(ans)) ok = 0;
  }
  return ans;
}

int factor() {
  ll ans = unit();
  if (match('^'))
    ans = pow(ans, factor());
  return ans;
}

int unit() {
  ll ans, sign;

  sign = 1;
  while (match('-'))
    sign *= -1;

  if (la.t == 'n') {
    ans = sign * la.val;
    shift();
    if (!fit(ans)) ok = 0;
    return ans;
  }

  if (match('(')) {
    ans = sign * exp();
    if (!fit(ans)) ok = 0;
    if (match(')'))
      return ans;
  }

  ok = 0;
  return 1;
}

int main() {
  x = getchar(); shift();

  while (la.t != EOF) {
    ok = 1;
    int val = exp();
  
    if (ok)
      printf("%d\n", val);
    else
      puts("ERROR!");
  }
  
  return 0;
}
\end{verbatim}

	\subsubsection{最长公共上升子序列}
\begin{verbatim}
#include <cstdio>
#include <cstring>
#include <algorithm>

using namespace std;

#define rep(i, n) for(int i = 0; i < (n); i++)

int n, m;
int a[1010],b[1010];
int f[1010];

int LCIS() {
    memset(f, 0, sizeof(f));
    rep(i, n){
        int k = 0;
        rep(j, m){
            if(a[i] == b[j]){//如果a[i]==b[j] 
                if(f[j]<f[k]+1){//就在0到j-1之间,找一个b[k]小于a[i]的f[k]值最大的解
                    f[j]=f[k]+1;
                }
            }
            if(a[i]>b[j]){//0到j-1中,对于小于a[i]的,保存f值的最优解
                if(f[k]<f[j]){
                    k=j;
                }
            }
        }
    }
    int ans=0;
    rep(i, m){
        ans=max(ans,f[i]);
    }
    return ans;
}

int main() {
    int t;
    scanf("%d", &t);
    while(t--) {
        scanf("%d",&n);
        rep(i, n){
            scanf("%d",&a[i]);
        }
        scanf("%d",&m);
        rep(j, m){
            scanf("%d",&b[j]);
        }
        printf("%d\n",LCIS());
        if (t) printf("\n");
    }
    return 0;
}
\end{verbatim}

	\subsubsection{罗马数字}
\begin{verbatim}
map <string, int, less <string> > dict;
char nums[5000][20];

void gen_roman() {
	char *roman[13] = {"M", "CM", "D", "CD", "C", "XC", "L", "XL", "X", "IX", "V", "IV", "I"};
	int arab[13] = {1000, 900, 500, 400, 100, 90, 50, 40, 10, 9, 5, 4, 1};
	string key;

	for (int i = 0; i < 5000; ++i) {
		nums[i][0] = 0;
		for (int n = i,  j = 0; n; ++j)
			for ( ; n >= arab[j]; n -= arab[j]) 
				strcat(nums[i], roman[j]);
		key = nums[i];
		dict[key] = i;
	}
}

char *to_roman(int n) {
	if (n < 1 || n >= 5000) return 0;
	return nums[n];
}

int to_arabic(char *in) {
	string key = in;
	if (!dict.count(key)) return -1;
	return dict[key];
}
\end{verbatim}

	\input{code/other/max_sub_rect}
	\input{code/other/dlx_exact_cover}
	\input{code/other/dlx_multi_cover}
	\input{code/other/NQueens}
    \subsubsection{sgu313}
\begin{verbatim}
/*
There are L stations along a circular railway, numbered 1 through L. 
Trains travel in both directions, and take 1 minute to get from a station to the neighbouring one.
There are n employee's houses along the railway, and n offices, each house or office located near a railway station. 
You are to establish a one-to-one correspondence between houses and offices in such a way that total travel time (sum of travel times of each employee) is minimized.
*/

#include <cstdio>
#include <cstring>
#include <algorithm>

using namespace std;

const int maxn = 50000 + 10;

struct Tobj {
	int loc, sign;
};

Tobj obj[maxn * 2];
long long tab[maxn * 2];
int stack[maxn * 2];
int match[maxn];
int n, l;

bool cmp(Tobj i, Tobj j) {
	return i.loc < j.loc;
}

int main() {
	scanf("%d%d", &n, &l);
	for (int i = 1; i <= n; ++i) {
		scanf("%d", &obj[i].loc);
		obj[i].sign = i;
	}
	for (int i = 1; i <= n; ++i) {
		scanf("%d", &obj[n + i].loc);
		obj[n + i].sign = -i;
	}

	sort(obj + 1, obj + n * 2 + 1, cmp);

	int p = n + 1, delta = 0, now = 0;
	long long sum = 0, len = l - (obj[n * 2].loc - obj[1].loc);
	tab[delta + p] = len;
	for (int i = 2; i <= n * 2; ++i) {
		if (obj[i - 1].sign > 0) ++now;
		else --now;
		len = obj[i].loc - obj[i - 1].loc;
		sum += len * abs(now);
		tab[now + delta + p] += len;
	}
	for (int i = -n; i <= n; ++i) tab[i + p] += tab[i - 1 + p];

	long long ans = -1;
	int broken;
	for (int i = 1; i <= n * 2; ++i) {
		if (ans == -1 || sum < ans) {
			ans = sum;
			broken = i;
		}
		if (obj[i].sign > 0) {
			sum -= (long long)(l - tab[delta + p]) - tab[delta + p];
			++delta;
		}
		else {
			sum += (long long)(l - tab[delta - 1 + p]) - tab[delta - 1 + p];
			--delta;
		}
	}

	int top = 0, i = broken;
	do {
		stack[top++] = obj[i].sign;
		while (top >= 2 && (stack[top - 1] > 0) != (stack[top - 2] > 0)) {
			if (stack[top - 1] > 0) match[stack[top - 1]] = -stack[top - 2];
			else match[stack[top - 2]] = -stack[top - 1];
			top -= 2;
		}
		if ((++i) > n * 2) i = 1;
	} while (i != broken);

	printf("%I64d\n", ans);
	for (int i = 1; i <= n; ++i) {
		if (i > 1) printf(" ");
		printf("%d", match[i]);
	}
	printf("\n");

	return 0;
}
\end{verbatim}

	\input{code/other/MIPT015}
	\subsubsection{MIPT016最大不相交矩阵}
\begin{verbatim}
/*
Big rectangle A is defined via inequalities 0 <= x <= MAXX, 0 <= y <= MAXY, where x and y are cartesian coordinates. There are N marked points inside the rectangle.
Rectangles B and C have sides parrallel to the sides of A. Intersection of B and C has null area. B and C have no marked points inside.

Please, write program which can determine maximum sum area of the rectangles B and C.
*/

#include <cstdio>
#include <cstring>
#include <algorithm>

using namespace std;

const int maxn = 100 + 10;

struct Tpoint {
	double x, y;
};

Tpoint p[maxn];
double x[maxn], y[maxn], f[maxn], g[maxn];
double MAXX, MAXY;
int n;

inline bool cmpx(Tpoint i, Tpoint j) {
	return i.x < j.x;
}

inline bool cmpy(Tpoint i, Tpoint j) {
	return i.y < j.y;
}

int main() {
	scanf("%lf%lf%d", &MAXX, &MAXY, &n);
	for (int i = 0; i < n; ++i) {
		scanf("%lf%lf", &p[i].x, &p[i].y);
		x[i] = p[i].x; y[i] = p[i].y;
	}
	x[n] = 0; x[n + 1] = MAXX;
	y[n] = 0; y[n + 1] = MAXY;

	sort(x, x + n + 2);
	int m = unique(x, x + n + 2) - x;
	double area1 = 0, area2 = 0;
	for (int i = 1; i < m; ++i)
		if (x[i] - x[i - 1] > area1) area2 = area1, area1 = x[i] - x[i - 1];
		else if (x[i] - x[i - 1] > area2) area2 = x[i] - x[i - 1];
	double ans = (area1 + area2) * MAXY;

	sort(y, y + n + 2);
	m = unique(y, y + n + 2) - y;
	area1 = 0, area2 = 0;
	for (int i = 1; i < m; ++i)
		if (y[i] - y[i - 1] > area1) area2 = area1, area1 = y[i] - y[i - 1];
		else if (y[i] - y[i - 1] > area2) area2 = y[i] - y[i - 1];
	if ((area1 + area2) * MAXX > ans) ans = (area1 + area2) * MAXX;

	memset(f, 0, sizeof(f));
	memset(g, 0, sizeof(g));
	sort(p, p + n, cmpx);
	for (int i = 0; i < n; ++i)
		if (p[i].x > 0 && p[i].x < MAXX && p[i].y > 0 && p[i].y < MAXY) {
			double up = MAXY, down = 0, area = 0;
			for (int j = i + 1; j < n; ++j)
				if (p[j].x > p[i].x && p[j].y < up && p[j].y > down) {
					if ((p[j].x - p[i].x) * (up - down) > area) area = (p[j].x - p[i].x) * (up - down);
					if (p[j].y > p[i].y) up = p[j].y;
					else down = p[j].y;
				}
			if ((MAXX - p[i].x) * (up - down) > area) area = (MAXX - p[i].x) * (up - down);
			f[i] = area;

			up = MAXY; down = area = 0;
			for (int j = i - 1; j >= 0; --j)
				if (p[j].x < p[i].x && p[j].y < up && p[j].y > down) {
					if ((p[i].x - p[j].x) * (up - down) > area) area = (p[i].x - p[j].x) * (up - down);
					if (p[j].y > p[i].y) up = p[j].y;
					else down = p[j].y;
				}
			if (p[i].x * (up - down) > area) area = p[i].x * (up - down);
			g[i] = area;
		}

	for (int i = n - 2; i >= 0; --i)
		if (f[i + 1] > f[i]) f[i] = f[i + 1];
	for (int i = 1; i < n; ++i)
		if (g[i - 1] > g[i]) g[i] = g[i - 1];

	for (int i = 0; i < n; ++i)
		if (f[i] + g[i] > ans) ans = f[i] + g[i];

	memset(f, 0, sizeof(f));
	memset(g, 0, sizeof(g));
	sort(p, p + n, cmpy);
	for (int i = 0; i < n; ++i)
		if (p[i].x > 0 && p[i].x < MAXX && p[i].y > 0 && p[i].y < MAXY) {
			double right = MAXX, left = 0, area = 0;
			for (int j = i + 1; j < n; ++j)
				if (p[j].y > p[i].y && p[j].x < right && p[j].x > left) {
					if ((p[j].y - p[i].y) * (right - left) > area) area = (p[j].y - p[i].y) * (right - left);
					if (p[j].x > p[i].x) right = p[j].x;
					else left = p[j].x;
				}
			if ((MAXY - p[i].y) * (right - left) > area) area = (MAXY - p[i].y) * (right - left);
			f[i] = area;

			right = MAXX; left = area = 0;
			for (int j = i - 1; j >= 0; --j)
				if (p[j].y < p[i].y && p[j].x < right && p[j].x > left) {
					if ((p[i].y - p[j].y) * (right - left) > area) area = (p[i].y - p[j].y) * (right - left);
					if (p[j].x > p[i].x) right = p[j].x;
					else left = p[j].x;
				}
			if (p[i].y * (right - left) > area) area = p[i].y * (right - left);
			g[i] = area;
		}

	for (int i = n - 2; i >= 0; --i)
		if (f[i + 1] > f[i]) f[i] = f[i + 1];
	for (int i = 1; i < n; ++i)
		if (g[i - 1] > g[i]) g[i] = g[i - 1];

	for (int i = 0; i < n; ++i)
		if (f[i] + g[i] > ans) ans = f[i] + g[i];

	printf("%.2lf\n", ans);

	return 0;
}
\end{verbatim}

	\subsubsection{poj3016可并堆}
\begin{verbatim}
/*
    Given a sequence A of length n, seperate it into k parts and modify each element so that
    each part is increasing or decreasing strictly, and minimize the cost.
    modify a1, a2, ..., am into b1, b2, ..., bm, the cost is |a1 - b1| + |a2 - b2| + ... + |am - bm|
*/

#include <cstdio>
#include <cstring>
#include <algorithm>

using namespace std;

const int maxn = 2000 + 10;
const int maxk = 10 + 5;

struct node {
    int key, npl, left, right;
};

node heap[maxn];
int a[maxn], root[maxn], cnt[maxn];
int cost[maxn][maxn];
int f[maxk][maxn];
int n, k, m;

int merge(int a, int b, int typ) {
    if (!a) return b;
    if (!b) return a;
    if ((heap[a].key - heap[b].key) * typ < 0) swap(a, b);
    heap[a].right = merge(heap[a].right, b, typ);
    if (heap[heap[a].right].npl > heap[heap[a].left].npl) swap(heap[a].left, heap[a].right);
    heap[a].npl = heap[heap[a].right].npl + 1;
    return a;
}

int main() {
    while (scanf("%d%d", &n, &k), n || k) {
        for (int i = 1; i <= n; ++i) scanf("%d", &a[i]);
        heap[0].npl = -1;
        for (int i = 1; i <= n; ++i) {
            int x, y;
            m = x = y = 0;
            cnt[0] = i - 1;
            for (int j = i; j <= n; ++j) {
                heap[j].key = a[j] - j; heap[j].left = heap[j].right = heap[j].npl = 0;
                ++m; root[m] = cnt[m] = j; x += a[j] - j; y -= a[j] - j;
                while (m > 1 && heap[root[m - 1]].key > heap[root[m]].key) {
                    if ((cnt[m] - cnt[m - 1]) & 1) x -= a[root[m]] - root[m];
                    if ((cnt[m - 1] - cnt[m - 2]) & 1) x -= a[root[m - 1]] - root[m - 1];
                    root[m - 1] = merge(root[m], root[m - 1], 1); --m;
                    if ((cnt[m + 1] - cnt[m - 1]) & 1) x += a[root[m]] - root[m];
                    if ((cnt[m + 1] - cnt[m]) & 1 && (cnt[m] - cnt[m - 1]) & 1) {
                        y += (a[root[m]] - root[m]) * 2;
                        root[m] = merge(heap[root[m]].left, heap[root[m]].right, 1);
                    }
                    cnt[m] = cnt[m + 1];
                }
                cost[i][j] = x + y;
            }
            m = x = y = 0;
            cnt[0] = i - 1;
            for (int j = i; j <= n; ++j) {
                heap[j].key = a[j] + j; heap[j].left = heap[j].right = heap[j].npl = 0;
                ++m; root[m] = cnt[m] = j; x -= a[j] + j; y += a[j] + j;
                while (m > 1 && heap[root[m - 1]].key < heap[root[m]].key) {
                    if ((cnt[m] - cnt[m - 1]) & 1) x += a[root[m]] + root[m];
                    if ((cnt[m - 1] - cnt[m - 2]) & 1) x += a[root[m - 1]] + root[m - 1];
                    root[m - 1] = merge(root[m], root[m - 1], -1); --m;
                    if ((cnt[m + 1] - cnt[m - 1]) & 1) x -= a[root[m]] + root[m];
                    if ((cnt[m + 1] - cnt[m]) & 1 && (cnt[m] - cnt[m - 1]) & 1) {
                        y -= (a[root[m]] + root[m]) * 2;
                        root[m] = merge(heap[root[m]].left, heap[root[m]].right, -1);
                    }
                    cnt[m] = cnt[m + 1];
                }
                if (x + y < cost[i][j]) cost[i][j] = x + y;
            }
        }
        memset(f, -1, sizeof(f));
        f[0][0] = 0;
        for (int i = 1; i <= k; ++i)
            for (int j = i; j <= n; ++j)
                for (int p = i - 1; p < j; ++p)
                    if (f[i - 1][p] != -1)
                        if (f[i][j] == -1 || f[i - 1][p] + cost[p + 1][j] < f[i][j]) f[i][j] = f[i - 1][p] + cost[p + 1][j];
        printf("%d\n", f[k][n]);
    }
    return 0;
}
\end{verbatim}


%CJK的时候使用
%\end{CJK}
\end{document}
