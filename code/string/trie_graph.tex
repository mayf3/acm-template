\subsubsection{Trie图}
\begin{verbatim}
/*
trie图用于多模式串匹配,一般的题目都是要求生成的字符串中不包含一些字符串
建立trie图,然后从根节点开始走,只要不走到危险节点就可以了
注意:trie图的2个性质:1、一个节点的危险性和它后缀节点相同 2、根节点的直接儿子的后缀是根节点
*/
#include <string>
#include <queue>
#include <cstdio>
#include <cstring>
#include <iostream>
using namespace std;

class Trie{
    public:
        int tot;
        const static int kMaxN=100;

    public:
        char ch[kMaxN];
        bool danger[kMaxN];
        int g[kMaxN],next[kMaxN],suffix[kMaxN];

    public:
        Trie(){
            tot=1;
            memset(g,255,sizeof(g));
            memset(danger,0,sizeof(danger));
        }
        int getson(int node, char c, bool flag);
        void insert(string st);
        void makegraph();
};

int Trie::getson(int node, char c, bool flag){
    while (true){
        int now=g[node];
        while (now!=-1 && ch[now]!=c) now=next[now];
        if (now!=-1 || !flag) return now;
        if (node==1) return node;
        node=suffix[node];
    }
}

void Trie::insert(string st){
    int len=st.size();
    int now=1,son;
    for (int i=0; i<len; i++){
        son=getson(now,st[i],false);
        if (son==-1){
            ++tot;
            ch[tot]=st[i];
            next[tot]=g[now];
            g[now]=tot;
            now=tot;
        } else now=son;
    }
    danger[now]=true;
}

void Trie::makegraph(){
    queue<int> q;
    q.push(1);
    while (!q.empty()){
        int node=q.front(); q.pop();
        for (int p=g[node]; p!=-1; p=next[p]){
            q.push(p);
            if (node==1){
                suffix[p]=1;
            } else{
                suffix[p]=getson(suffix[node],ch[p],true);
            }
            danger[p]|=danger[suffix[p]];
        }
    }
}

int main(){
    string st;
    Trie graph;
    cin>>st;
    graph.insert(st);
    cin>>st;
    graph.insert(st);
    return 0;
}

\end{verbatim} 
