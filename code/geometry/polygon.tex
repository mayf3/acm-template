\subsubsection{一般多边形}
\begin{verbatim}
/*
 * if point a inside polygon p[n]
 */
int inside_polygon(Point p[], int n, Point a){
	double sum = 0;
	for(int i = 0; i < n; i++){
		int j = (i + 1) % n;
		if (on_lineseg(Line(p[i], p[j]), a)) return 0;
		double angle = acos(dot(a, p[i], p[j]) / dist(a, p[i]) / dist(a, p[j]));
		sum += sign(cross(a, p[i], p[j])) * angle;
	}
	return sign(sum);
}

/*
 * if lineseg l strickly inside polygon p[n]
 */
int lineseg_inside_polygon(Point p[], int n, Line l){
	for(int i = 0; i < n; i++){
		int j = (i + 1) % n;
		Line l1(p[i], p[j]);
		if (on_lineseg_exclusive(l, p[i])) return 0;
		if (intersected_exclusive(l, l1)) return 0;
	}
	return inside_polygon(p, n, mp(l.p, l.q));
}

/*
 * if lineseg l intersect convex polygon p[n]
 */
int intersect_convex_lineseg(Point p[], int n, Line l){
	if (n < 3) return 0;
	Point q[4];
	int k = 0;
	q[k++] = l.p;
	q[k++] = l.q;
	for(int i = 0; i < n; i++){
		if (on_lineseg(l, p[i])){
			q[k++] = p[i];
		}
		else{
			int j = (i + 1) % n;
			Line a(p[i], p[j]);
			Point tmp = ip(a, l);
			if (on_lineseg(l, tmp) && on_lineseg(a, tmp)) q[k++] = tmp;
		}
	}
	sort(q, q + k);
	for(int i = 0; i + 1 < k; i++){
		if (inside_polygon(p, n, mp(q[i], q[i + 1]))) return 1;
	}
	return 0;
}

#define crossOp(p1,p2,p3) sign(cross(p1,p2,p3))

Point isSS(Point p1, Point p2, Point q1, Point q2) {
	double a1 = cross(q1,q2,p1), a2 = -cross(q1,q2,p2);
	return (p1 * a2 + p2 * a1) / (a1 + a2);
}

vector<Point> convexCut(const vector<Point>&ps, Point q1, Point q2) {
	vector<Point> qs;
	int n = ps.size();
	for (int i = 0; i < n; ++i) {
		Point p1 = ps[i], p2 = ps[(i + 1) % n];
		int d1 = crossOp(q1,q2,p1), d2 = crossOp(q1,q2,p2);
		if (d1 >= 0)
			qs.push_back(p1);
		if (d1 * d2 < 0)
			qs.push_back(isSS(p1, p2, q1, q2));
	}
	return qs;
}

typedef double Tdata;
typedef Point Tpoint;

struct Tline {
	Tdata a, b, c;
	double ang;
	Tline() {}
	Tline(Tdata a, Tdata b, Tdata c) : a(a), b(b), c(c) { ang = atan2(b, -a); }
	void get() { scanf("%lf%lf%lf", &a, &b, &c); }
	bool operator <(Tline l) const { return sign(ang - l.ang) < 0 || sign(ang - l.ang) == 0 && sign(c - l.c) < 0; }
};

inline int side(Tline l, Tpoint p) { return sign(l.a * p.x + l.b * p.y + l.c); }

// change line from two point form to general form
// O(1)
// return line(general form)
inline Tline change_line(Tpoint a, Tpoint b) {
	Tdata tmp, A = a.y - b.y, B = b.x - a.x, C = cross(a, b);
	if (sign(A)) tmp = fabs(A);
	else tmp = fabs(B);
	return Tline(A / tmp, B / tmp, C / tmp);
}

// calculate the area of polygon
// O(N)
// be careful the sign of the area
Tdata cal_area(Tpoint *P, int N) {
	if (N < 3) return 0;
	Tdata ret = 0;
	P[N] = P[0];
	for (int i = 0; i < N; ++i) ret += cross(P[i], P[i + 1]);
	return ret / 2.0;
}

// intersection of half-planes
// O(N log N)
// ax + by + c >= 0
// P - points form the intersection, M - number of points
void inter_hplane(Tline *H, int N, Tpoint *P, int &M) {
	int *queue = new int[N + 1];
	sort(H, H + N);
	M = 0;
	for (int i = 0; i < N; ++i)	if (!i || sign(H[i].ang - H[queue[M - 1]].ang)) queue[M++] = i;
	int h = 0, t = 2;
	for (int i = 2; i < M; ++i) {
		while (h + 1 < t && side(H[queue[i]], inter_point(H[queue[t - 1]], H[queue[t - 2]])) < 0) --t;
		while (h + 1 < t && side(H[queue[i]], inter_point(H[queue[h]], H[queue[h + 1]])) < 0) ++h;
		queue[t++] = queue[i];
	}
	while (h + 1 < t && side(H[queue[h]], inter_point(H[queue[t - 1]], H[queue[t - 2]])) < 0) --t;
	while (h + 1 < t && side(H[queue[t - 1]], inter_point(H[queue[h]], H[queue[h + 1]])) < 0) ++h;
	M = 0;
	for (int i = h; i < t; ++i) queue[M++] = queue[i];
	queue[M] = queue[0];
	for (int i = 0; i < M; ++i) P[i] = inter_point(H[queue[i]], H[queue[i + 1]]);
	delete [] queue;
}

// get the core of polygon
// O(N log N)
Tpoint core_of_poly(Tpoint *P, int N) {
	Tline *H = new Tline[N];
	Tpoint *A = new Tpoint[N];
	int M;
	P[N] = P[0];
	for (int i = 0; i < N; ++i) H[i] = change_line(P[i], P[i + 1]);
	inter_hplane(H, N, A, M);
	Tpoint ret = A[0];
	delete [] H; delete [] A;
	return ret;
}

// get the length of segment in convex polygon
// O(N)
Tdata seg_in_convex_poly(Tpoint a, Tpoint b, Tpoint *P, int N) {
	int d1 = point_in_convex_poly(a, P, N), d2 = point_in_convex_poly(b, P, N);
	if (d2 == 1) swap(d1, d2), swap(a, b);
	if (d2 == 1) return dist(a, b);
	Tpoint p;
	P[N] = P[0];
	if (d1 == 1)
		for (int i = 0; i < N; ++i) {
			int d = inter_seg(a, b, P[i], P[i + 1], p);
			if (d == 1 || d == 2) return dist(a, p); // not including the boundaries, add "d == 3" for including the boundaries
		}
	else {
		int cnt = 0;
		Tpoint u, v;
		for (int i = 0; i < N; ++i) {
			int d = inter_seg(a, b, P[i], P[i + 1], p);
			if (d == 3) return 0; // on the boundaries
			if (cnt == 2) continue;
			if (d)
				if (!cnt) u = p, ++cnt;
				else if (u != p) v = p, ++cnt;
		}
		return cnt == 2 ? dist(u, v) : 0;
	}
}

// get the centroid of polygon
// O(N)
Tpoint cal_centroid(Tpoint *P, int N) {
	P[N] = P[0];
	Tpoint c(0, 0);
	Tdata s = 0;
	for (int i = 0; i < N; ++i) {
		Tdata tmp = cross(P[i], P[i + 1]);
		c += (P[i] + P[i + 1]) * tmp; s += tmp;
	}
	return c / (3 * s);
}
\end{verbatim}
