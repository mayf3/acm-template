\subsubsection{圆与多边形面积交}
\begin{verbatim}
Point p[3];
double r;

double cross(Point a, Point b){
	return a.X * b.Y - a.Y * b.X;
}

double cross(Point a, Point b, Point c){
	return cross(b - a, c - a);
}

double dot(Point a, Point b){
	return a.X * b.X + a.Y * b.Y;
}

double dot(Point a, Point b, Point c){
	return dot(b - a, c - a);
}

double len(Line l){
	return abs(l.S - l.F);
}

double dis(Point p, Line l){
	return fabs(cross(p, l.F, l.S) / len(l));
}

bool inter(Line a, Line b, Point &p){
	double s1 = cross(a.F, a.S, b.F);
	double s2 = cross(a.F, a.S, b.S);
	if (!sign(s1 - s2)) return false;
	p  = (s1 * b.S - s2 * b.F) / (s1 - s2);
	return true;
}

Vec dir(Line l){
	return l.S - l.F;
}

Vec normal(Vec v){
	return Vec(-v.Y, v.X);
}

Vec unit(Vec v){
	return v / abs(v);
}

bool onseg(Point p, Line l){
	return sign(cross(p, l.F, l.S)) == 0 && sign(dot(p, l.F, l.S)) <= 0;
}

double arg(Vec a, Vec b){
	double d = arg(b) - arg(a);
	if (d > PI) d -= 2 * PI;
	if (d < -PI) d += 2 * PI;
	return d;
}

double area(Point a, Point b){
	double s1 = 0.5 * cross(a, b);
	double s2 = 0.5 * arg(a, b) * r * r;
	return fabs(s1) < fabs(s2) ? s1 : s2;
}

double area(){
	double s = 0;
	rep(i, n){
		Point O(0, 0), A = p[i], B = p[(i + 1) % 3];
		Line AB(A, B);
		double d = dis(O, AB);
		if (sign(d - r) >= 0){
			s += area(A, B);
		}
		else{
			Point P;
			inter(AB, Line(O, O + normal(dir(AB))), P);
			Vec v = sqrt(r * r - d * d) * unit(dir(AB));
			Point P1 = P - v, P2 = P + v;
			if (!onseg(P1, AB) && !onseg(P2, AB)){
				s += area(A, B);
			}
			else{
				s += area(A, P1);
				s += area(P1, P2);
				s += area(P2, B);
			}
		}
	}
	return fabs(s);
}

void init(){
	scanf("%d%d", &n, &r);
	rep(i, n){
		double x, y;
		scanf("%lf%lf", &x, &y);
		p[i] = Point(x, y);
	}
}

int main(){
	init();
	printf("%.12lf\n", area());
	return 0;
}
\end{verbatim}
