\subsubsection{delaunay triangulation三角剖分}
\begin{verbatim}
#include <cstdio>
#include <cstring>
#include <algorithm>
#include <cmath>

using namespace std;

#define Oi(e) ((e)->oi)
#define Dt(e) ((e)->dt)
#define On(e) ((e)->on)
#define Op(e) ((e)->op)
#define Dn(e) ((e)->dn)
#define Dp(e) ((e)->dp)
#define Other(e, p) ((e)->oi == p ? (e)->dt : (e)->oi)
#define Next(e, p) ((e)->oi == p ? (e)->on : (e)->dn)
#define Prev(e, p) ((e)->oi == p ? (e)->op : (e)->dp)
#define V(p1, p2, u, v) (u = p2->x - p1->x, v = p2->y - p1->y)
#define C2(u1, v1, u2, v2) (u1 * v2 - v1 * u2)
#define C3(p1, p2, p3) ((p2->x - p1->x) * (p3->y - p1->y) - (p2->y - p1->y) * (p3->x - p1->x))
#define Dot(u1, v1, u2, v2) (u1 * u2 + v1 * v2)

#define MAXN 1000001

struct point {
  double x, y;
  struct edge *in;

  bool operator<(const point &p1) const {
    return x < p1.x || (x == p1.x && y < p1.y);
  }

  point(double x = 0, double y = 0) : x(x), y(y) {}
};

struct edge {
  point *oi, *dt;
  edge *on, *op, *dn, *dp;
};

struct gEdge {
  int u, v;
  double w;
} E[3 * MAXN];

point p[MAXN], *Q[MAXN];
edge mem[3 * MAXN], *elist[3 * MAXN];
int nfree, N, M;

void Alloc_memory() {
  nfree = 3 * N;
  edge *e = mem;
  for (int i = 0; i < nfree; ++i) 
    elist[i] = e++;
}

void Splice(edge *a, edge *b, point *v) {
  edge *next;
  if (Oi(a) == v) next = On(a), On(a) = b;
  else next = Dn(a), Dn(a) = b;
  if (Oi(next) == v) Op(next) = b;
  else Dp(next) = b;
  if (Oi(b) == v) On(b) = next, Op(b) = a;
  else Dn(b) = next, Dp(b) = a;
}

edge *Make_edge(point *u, point *v) {
  edge *e = elist[--nfree];
  e->on = e->op = e->dn = e->dp = e; 
  e->oi = u; e->dt = v;
  if (u->in == NULL) u->in = e;
  if (v->in == NULL) v->in = e;
  return e;
}

edge *Join(edge *a, point *u, edge *b, point *v, int side) {
  edge *e = Make_edge(u, v);
  if (side == 1) {
    if (Oi(a) == u) Splice(Op(a), e, u);
    else Splice(Dp(a), e, u);
    Splice(b, e, v);
  } else {
    Splice(a, e, u);
    if (Oi(b) == v) Splice(Op(b), e, v);
    else Splice(Dp(b), e, v);
  }
  return e;
}

void Remove(edge *e) {
  point *u = Oi(e), *v = Dt(e);
  if (u->in == e) u->in = e->on;
  if (v->in == e) v->in = e->dn;
  if (Oi(e->on) == u) e->on->op = e->op;
  else e->on->dp = e->op;
  if (Oi(e->op) == u) e->op->on = e->on;
  else e->op->dn = e->on;
  if (Oi(e->dn) == v) e->dn->op = e->dp;
  else e->dn->dp = e->dp;
  if (Oi(e->dp) == v) e->dp->on = e->dn;
  else e->dp->dn = e->dn;
  elist[nfree++] = e;
}

void Make_Graph() {
  M = 0;
  for (int i = 0; i < N; ++i) {
    point *u = &p[i];
    edge *e = u->in;
    do {
      point *v = Other(e, u);
      if (u < v) {
        E[M].u = u - p, E[M].v = v - p;
        E[M++].w = sqrt(double(u->x - v->x) * (u->x - v->x) + double(u->y - v->y) * (u->y - v->y));
      }
      e = Next(e, u);
    } while (e != u->in);
  }
}

void Low_tangent(edge *e_l, point *o_l, edge *e_r, point *o_r, edge **l_low, point **OL, edge **r_low, point **OR) {
  point *d_l = Other(e_l, o_l), *d_r = Other(e_r, o_r);
  while (1) {
    if (C3(o_l, d_l, o_r) > 0.0) e_l = Prev(e_l, d_l), o_l = d_l, d_l = Other(e_l, o_l);
    else if (C3(o_r, d_r, o_l) < 0.0) e_r = Next(e_r, d_r), o_r = d_r, d_r = Other(e_r, o_r);
    else break;
  }
  *OL = o_l, *OR = o_r; *l_low = e_l, *r_low = e_r;
}

void Merge(edge *lr, point *s, edge *rl, point *u, edge **tangent) {
  point *O, *D, *OR, *OL;
  edge *B, *L, *R;
  Low_tangent(lr, s, rl, u, &L, &OL, &R, &OR);
  *tangent = B = Join(L, OL, R, OR, 0);
  O = OL, D = OR;
  while (1) {
    edge *El = Next(B, O), *Er = Prev(B, D), *next, *prev;
    point *l = Other(El, O), *r = Other(Er, D);
    double l1, l2, l3, l4, r1, r2, r3, r4, cot_N, cot_P, cot_L, cot_R, u1, v1, u2, v2;
    V(l, O, l1, l2); V(l, D, l3, l4); V(r, O, r1, r2); V(r, D, r3, r4);
    double cl = C2(l1, l2, l3, l4), cr = C2(r1, r2, r3, r4);
    bool BL = cl > 0.0, BR = cr > 0.0;
    if (!BL && !BR) break;
    if (BL) {
      double dl = Dot(l1, l2, l3, l4);
      cot_L = dl / cl;
      while (1) {
        next = Next(El, O);
        V(Other(next, O), O, u1, v1); V(Other(next, O), D, u2, v2);
        double N1 = C2(u1, v1, u2, v2);
        if (!(N1 > 0.0)) break;
        cot_N = Dot(u1, v1, u2, v2) / N1;
        if (cot_N > cot_L) break;
        Remove(El);
        El = next; cot_L = cot_N;
      }
    }
    if (BR) {
      double dr = Dot(r1, r2, r3, r4);
      cot_R = dr / cr;
      while (1) {
        prev = Prev(Er, D);
        V(Other(prev, D), O, u1, v1); V(Other(prev, D), D, u2, v2);
        double P1 = C2(u1, v1, u2, v2);
        if (!(P1 > 0.0)) break;
        cot_P = Dot(u1, v1, u2, v2) / P1;
        if (cot_P > cot_R) break;
        Remove(Er);
        Er = prev; cot_R = cot_P;
      }
    }
    l = Other(El, O); r = Other(Er, D);
    if (!BL || (BL && BR && cot_R < cot_L)) B = Join(B, O, Er, r, 0), D = r;
    else B = Join(El, l, B, D, 0), O = l;
  }
}

void Divide(int s, int t, edge **L, edge **R) {
  int n = t - s + 1;
  if (n == 2) *L = *R = Make_edge(Q[s], Q[t]);
  else if (n == 3) {
    edge *a = Make_edge(Q[s], Q[s + 1]), *b = Make_edge(Q[s + 1], Q[t]), *c;
    Splice(a, b, Q[s + 1]);
    double v = C3(Q[s], Q[s + 1], Q[t]);
    if (v > 0.0) c = Join(a, Q[s], b, Q[t], 0), *L = a, *R = b;
    else if (v < 0.0) c = Join(a, Q[s], b, Q[t], 1), *L = c, *R = c;
    else *L = a, *R = b;
  } else if (n > 3) {
    int split = (s + t) / 2;
    edge *ll, *lr, *rl, *rr, *tangent;
    Divide(s, split, &ll, &lr); Divide(split + 1, t, &rl, &rr);
    Merge(lr, Q[split], rl, Q[split + 1], &tangent);
    if (Oi(tangent) == Q[s]) ll = tangent;
    if (Dt(tangent) == Q[t]) rr = tangent;
    *L = ll; *R = rr;
  }
}

const double INF = 1e+9;
int tn, cn;
struct triangle { 
  point *a, *b, *c; 
} tri[MAXN << 1];
point *cycle[MAXN];

void Make_Triangles() {
  tn = 0;
  /* first 2 and last 2 are boundary points */
  for (int i = 2; i < N - 2; ++i) {
    cn = 0;
    point *u = &p[i];
    edge *e = u->in;
    do {
      cycle[cn++] = Other(e, u);
      e = Next(e, u);
    } while (e != u->in);
    cycle[cn] = cycle[0];
    for (int j = 0; j < cn; j++) {
      point *x = cycle[j], *y = cycle[j + 1];
      if (u < x && u < y && x - p < N - 2 && y - p < N - 2) {
        tri[tn].a = u;
        tri[tn].b = x;
        tri[tn++].c = y;
      }
    }
  }
}

int main() {
  scanf("%d", &N);
  for (int i = 0; i < N; ++i)
    scanf("%lf%lf", &p[i].x, &p[i].y);
  /* add boundary points, required by Make_Triangles() */
  p[N++] = point(-INF, -INF);
  p[N++] = point(+INF, -INF);
  p[N++] = point(-INF, +INF);
  p[N++] = point(+INF, +INF);
  for (int i = 0; i < N; ++i)
    p[i].in = NULL;
  sort(p, p + N);
  for (int i = 0; i < N; i++) 
    Q[i] = p + i;
  Alloc_memory();
  edge *L, *R;
  Divide(0, N - 1, &L, &R);
//Make_Graph();
  Make_Triangles();
  return 0;
}
\end{verbatim}
