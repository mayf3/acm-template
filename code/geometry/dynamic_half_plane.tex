\subsubsection{动态半平面交}
\begin{verbatim}
#include <stdio.h>
#include <string.h>
#include <math.h>
#include <algorithm>
#include <map>
using namespace std;

const int N = 100010;
const double eps = 1e-8;

inline int sign(double x) { 
    return x < -eps ? -1 : x > eps; 
}

struct point {
    double x, y; 
    point(double x = 0, double y = 0) : x(x), y(y) {}
};

struct line {
    point p, q;
    double alpha;
    line() {}
    line(point a, point b) {
        p = a; q = b;
        alpha = atan2(q.y - p.y, q.x - p.x);
    }
    bool operator<(const line &l) const {
        return alpha < l.alpha;
    }
};

inline double cross(point a, point b, point c) {
    return (b.x - a.x) * (c.y - a.y) - (b.y - a.y) * (c.x - a.x);
}

point ip(line u, line v) {
    double n = (u.p.y - v.p.y) * (v.q.x - v.p.x) - (u.p.x - v.p.x) * (v.q.y - v.p.y);
    double d = (u.q.x - u.p.x) * (v.q.y - v.p.y) - (u.q.y - u.p.y) * (v.q.x - v.p.x);
    double r = n / d;
    return point(u.p.x + r * (u.q.x - u.p.x), u.p.y + r * (u.q.y - u.p.y));
}

inline int side(line l, point p) {
    return sign(cross(l.p, l.q, p));
}

typedef map<line, point> polygon;

void forward(polygon &s, polygon::iterator &it) {
    if (++it == s.end()) it = s.begin();
}

void backward(polygon &s, polygon::iterator &it) {
    if (it == s.begin()) it = s.end(); --it;
}

int main() {
    int n;
    double w, h;
    point o;
    while (~scanf("%d %lf %lf", &n, &w, &h)) {
        point A(0, 0);
        point B(w, 0);
        point C(w, h);
        point D(0, h);
        polygon p;
        p[line(A, B)] = B;
        p[line(B, C)] = C;
        p[line(C, D)] = D;
        p[line(D, A)] = A;
        double area = 2 * w * h;
        for (int i = 0; i < n; i++) {
            point A, B;
            scanf("%lf %lf", &A.x, &A.y);
            scanf("%lf %lf", &B.x, &B.y);
            line l(A, B);
            if (!p.empty()) {
                polygon::iterator b = p.lower_bound(l);
                polygon::iterator a = b; backward(p, a);
                point prev = a->second;
                if (side(l, prev) < 0) {
                    backward(p, b);
                    backward(p, a);
                    while (p.size() > 1 && side(l, a->second) <= 0) {
                        area -= cross(o, a->second, b->second);
                        p.erase(b);
                        b = a;
                        backward(p, a);
                    }
                    area -= cross(o, a->second, b->second);
                    forward(p, a); forward(p, b);
                    forward(p, a); forward(p, b);
                    while (p.size() > 1 && side(l, a->second) <= 0) {
                        area -= cross(o, prev, a->second);
                        prev = a->second;
                        p.erase(a);
                        a = b;
                        forward(p, b);
                    }
                    area -= cross(o, prev, a->second);
                    if (p.size() > 1) {
                        backward(p, a); backward(p, b);
                        p[l] = ip(l, b->first);
                        a->second = ip(a->first, l);
                        area += cross(o, p[l], b->second);
                        area += cross(o, a->second, p[l]);
                        b = a; backward(p, a);
                        area += cross(o, a->second, b->second);
                    } else {
                        p.clear();
                        area = 0;
                    }
                }
            }
            printf("%lf\n", area / 2);
        }
    }
    return 0;
}
\end{verbatim}
