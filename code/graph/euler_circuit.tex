\subsubsection{混合图欧拉回路}
\begin{verbatim}
#include<iostream>
#include<cstdio>
#include<algorithm>
#include<cstring>
#include<ctime>
using namespace std;

const int INF=0x7FFFFFFF;
const int maxn=5000;
int n,m,num,s,t,all,correct,flow;
int a[maxn][3],adj[maxn],next[maxn],vt[maxn],h[maxn];
int vh[maxn],st[maxn],f[maxn],sum[maxn],sign[maxn][2],fa[maxn];

int maxflow(int x, int ff) {
    if (x==t) return ff;
    int minh=t-s+2;
    for (int i=1; i<=sum[x]; i++){
        if (f[st[x]]>0){
            //因为欧拉回路中两点之间可能有多条连边,用邻接矩阵可能会出现问题
            if (h[x]==h[adj[st[x]]]+1){
                int k=maxflow(adj[st[x]],min(ff,f[st[x]]));
                if (k>0){
                    f[st[x]]-=k;
                    f[st[x]^1]+=k;
                    return k;
                }
            }
            minh=min(minh,h[adj[st[x]]]+1);
            if (h[s]>t-s+1) return 0;
        }
        st[x]=next[st[x]];
        if (st[x]==0) st[x]=vt[x];
    }
    --vh[h[x]];
    if (vh[h[x]]==0) h[s]=t-s+2;
    h[x]=minh;
    ++vh[h[x]];
    return 0;
}

void connect(int x,int y) {
    ++all;
    ++sum[x];
    adj[all]=y;
    next[all]=vt[x];
    vt[x]=all;
}

int ask(int x) {
    if (fa[x]==x) return x;
    fa[x]=ask(fa[x]);
    return fa[x];
}

void solve() {
    for (int i=2; i<=n; i++)
    if (ask(i)!=ask(i-1)){
        printf("impossible\n");
        return;
    }
    //并查集判断图的联通性
    s=0; t=n+1; correct=0; flow=0;
    for (int i=1; i<=n; i++)
    if ((abs(sign[i][0]-sign[i][1])%2)==1){
	    printf("impossible\n");
	    return;
	} else if (sign[i][0]>sign[i][1]){
	    connect(s,i);
	    f[all]=(sign[i][0]-sign[i][1])/2;
	    connect(i,s);
	    //将入度大于出度的点连到起点s
	    correct+=(sign[i][0]-sign[i][1])/2;
	} else if (sign[i][1]>sign[i][0]){
	    connect(i,t);
	    f[all]=(sign[i][1]-sign[i][0])/2;
	    connect(t,i);
	    //将出度大于入度的点连接到汇点t
	}
	for (int i=s; i<=t; i++) st[i]=vt[i];
	memset(h,0,sizeof(h));
	memset(vh,0,sizeof(vh));
	vh[0]=t-s+1;
	while (h[s]<=t-s+1) flow+=maxflow(s,INF);
	if (flow==correct) printf("possible\n"); else printf("impossible\n");
	//满流时成立
}

void prepare() {
    all=1;
    memset(sum,0,sizeof(sum));
    memset(vt,0,sizeof(vt));
    memset(sign,0,sizeof(sign));
    memset(f,0,sizeof(f));
    for (int i=1; i<=n; i++) fa[i]=i;
}

void init() {
    scanf("%d",&num);
    while (num--){
        scanf("%d%d",&n,&m);
        prepare();
        for (int i=1; i<=m; i++){
            scanf("%d%d%d",&a[i][0],&a[i][1],&a[i][2]);
            if (a[i][2]==0){
                if ((rand()%2)==0) swap(a[i][0],a[i][1]);
                //随机双向边的方向
                connect(a[i][0],a[i][1]);
                f[all]=1;
                //构图时只连接双向边,流的方向为随机后的方向
                connect(a[i][1],a[i][0]);
            }
            ++sign[a[i][0]][0]; ++sign[a[i][1]][1];
            int x=ask(a[i][0]), y=ask(a[i][1]);
            fa[x]=fa[y];
        }
        solve();
    }
}

int main() {
    srand(time(0));
    init();
    return 0;
}
\end{verbatim} 
