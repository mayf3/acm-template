\subsubsection{有上下界的最大流}

\begin{verbatim}
#include <algorithm>
#include <iostream>
#include <cstdio>
#include <cstring>
using namespace std;

const int INF=0x7FFFFFFF;

const int maxn=1001;
const int kMaxM=100001;

int n,m,s,t,ca,P,tot,S,T,NT,flow,maxtot;
int g[maxn],last[maxn],h[maxn],vh[maxn];
int a[kMaxM][4],f[kMaxM],adj[kMaxM],next[kMaxM];

void insert(int x, int y, int limit){
    f[tot]=limit;
    adj[tot]=y;
    next[tot]=g[x];
    g[x]=tot++;

    f[tot]=0;
    adj[tot]=x;
    next[tot]=g[y];
    g[y]=tot++;
}

int dfs(int now, int add){
    if (now==T) return add;
    int y, tmp, minh=NT+1, p=last[now];
    do{
        y=adj[p];
        if (f[p]>0 && p<maxtot){
            if (h[now]==h[y]+1){
                tmp=dfs(y,min(f[p],add));
                if (tmp!=0){
                    f[p]-=tmp;
                    f[p^1]+=tmp;
                    last[now]=p;
                    return tmp;
                }
            }
            minh=min(minh,h[y]+1);
            if (h[S]>NT) return 0;
        }
        p=next[p];
        if (p==-1) p=g[now];
    } while (p!=last[now]);

    if (--vh[h[now]]==0) h[S]=NT+1;
    h[now]=minh; ++vh[minh];
    return 0;
}

int getflow(){
    //a[i][0],a[i][1],a[i][2],a[i][3]表示(a[i][0],a[i][1])
    //这条边以及下界流量为a[i][2],上界流量为a[i][3]
    //tot是边的总数
    tot=0;
    memset(g,255,sizeof(g));
    for (int i=1; i<=m; i++)
    insert(a[i][0],a[i][1],a[i][3]-a[i][2]);
    int tmp=tot;
    //前面tot条边都是上界流量-下界流量
    //做第二次最大流的时候只能用这tot条边
    //所以在这里要用tmp纪录tot

    //S和T在第一次网络流中是超级源和超级汇。
    //对于一条下界边(u,v,lower),连(u,T,lower)和(S,v,lower)
    S=n+1; T=S+1;
    for (int i=1; i<=m; i++)
    {
        insert(a[i][0],T,a[i][2]);
        insert(S,a[i][1],a[i][2]);
    }
    //s和t是原图中的源和汇,第一次最大流要连一条(t,s,无穷大)的边
    insert(t,s,INF);

    memset(h,0,sizeof(h));
    memset(vh,0,sizeof(vh));
    for (int i=1; i<=T; i++) last[i]=g[i];
    //maxtot表示做网络流时所能经过的编号最大的边,NT表示有多少个顶点
    maxtot=tot; vh[0]=T; flow=0; NT=n+2;
    while (h[S]<=NT) flow+=dfs(S,INF);
    //第一次最大流,如果流量不等于所有边的下界总和,则方案不合法
    if (flow!=m*lower) return -1;

    //第二次最大流在残余图上做
    //且只能使用前tmp条边,源和汇是原图的源汇
    //f[tot-1]表示从原图的源流出的流量,加入到最后的流量中
    S=s; T=t; flow=f[tot-1];
    memset(h,0,sizeof(h));
    memset(vh,0,sizeof(vh));
    for (int i=1; i<=n; i++) last[i]=g[i];
    //若T比定点个数小,则设NT=n,所有的h[S]<=NT而不是<=T
    maxtot=tmp; vh[0]=n; NT=n;
    while (h[S]<=NT) flow+=dfs(S,INF);
    //flow就是这个有上下界流量的最大流
    return flow;
}
\end{verbatim}
