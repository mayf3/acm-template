\subsubsection{LCA oninle}
\begin{verbatim}
const int N = 10000, M = 20;

int n, m, q;
int tot;
vector<int> E[N];
int dep[N];
int dist[N];
int anc[M][N];
int tin[N], tout[N];
bool vis[N];

void dfs(int x){
    vis[x] = true;
    tin[x] = tot++;
    int y;
    rep(i, E[x].size()){
        y = E[x][i];
        if (vis[y]) continue;
        dep[y] = dep[x] + 1;
        anc[0][y] = x;
        dfs(y);
    }
    tout[x] = tot++;
}

bool isanc(int x, int y){
    return tin[x] <= tin[y] && tout[y] <= tout[x];
}

int lca(int x, int y){
    if (isanc(x, y)) return x;
    fba(j, M - 1, 0){
        if (isanc(anc[j][x], y)) continue;
        x = anc[j][x];
    }
    return anc[0][x];
}

int main(){
    while(~scanf("%d%d", &n, &m)){
        rep(i, n) E[i].clear();
        rep(i, n) vis[i] = false;
        int x, y;
        rep(i, m){
            scanf("%d%d", &x, &y);
            x--, y--;
            E[x].PB(y);
            E[y].PB(x);
        }
        tot = 0;
        dep[0] = 1;
        rep(i, n) anc[i][0] = 0;
        dfs(0);
        REP(j, 1, M) rep(i, n) anc[j][i] = anc[j - 1][anc[j - 1][i]];
        scanf("%d", &q);
        while(q--){
            scanf("%d%d", &x, &y);
            x--, y--;
            printf("%d\n", lca(x, y) + 1);
        }
    }
    return 0;
}
\end{verbatim}
