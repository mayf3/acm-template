\subsubsection{生成树计数}
	一个完全图$K_n$有$n^{n-2}$棵生成树,即$n$个节点的带标号无根树有$n^{n-2}$个。\par
	证明用到$\mathbf{prufer code}$:一棵$n$无根树的$\mathbf{prufer code}$是这样转化的:每次选一个编号最小的叶节点,删除它,并把它所连的父亲节点的编号写下,直到这棵树剩下2个节点为止。那么生成的这$n-2$个数组成的序列就是这棵树的$\mathbf{prufer code}$。$\mathbf{prufer code}$和树是一一对应的,而一个长度为$n-2$,每个数字的范围为$[1,n]$的序列一共有$n^{n-2}$种可能,所以$n$个节点的带标号无根树一共有$n^{n-2}$个。\par
	以上是Cayley公式,它的一个应用:$n$个节点,每个节点的度分别为$d_1,d_2,\ldots,d_n$,那么生成树的个数为$\frac{(n-2)!}{(d_1-1)!(d_2-1)!\cdots(d_n-1)!)}$。因为顶点$i$在序列中出现了$d_i-1$次。\par
	~\\
	~\\ \par
		对于完全二分图,两边的顶点分别为$n,m$,那么生成树的个数为$n^{m-1}*m^{n-1}$。\par
		~\\
		~\\ \par
\subsubsection{无向图的生成数计数-MatrixTree定理}
		给出一个无向图$G=(V,E)$,求生成树个数。做法是构造一个$n*n$的Kichhoff矩阵。矩阵的对角线$(i,i)$的位置填的是第$i$个顶点的度,对于$G$的边$(v_i,v_j)$在矩阵$(i,j)$和$(j,i)$的位置填-1(\textbf{若$(i,j)$有$k$条重边,那么矩阵$(i,j)$和$(j,i)$的位置填$-k$}),然后生成树的个数就是$n*n$的矩阵的$n-1$阶的行列式。具体做法就是删除任意的第$r$行$r$列,然后求矩阵的行列式。
	~\\
\subsubsection{无向带权图最小生成树计数}
		把所有的边按照边权从小到大排序,然后做Kruskal。\par
		假设已经处理了边权$w_i<w$的边,形成一个森林$T$,现在考虑所有边权为$w$的边。\par
		1、若一条边权为$w$的边$(u,v)$所连接的两个顶点在森林$T$中属于统一个块,那么$(u,v)$这条边是不可能存在于最小生成树的方案中,否则$(u,v)$可以存在于最小生成树的方案中。\par
		2、把所有边权为$w$,且根据森林$T$判断出可以存在于最小生成树方案中的边找出来,假设这些边集为$E$。我们可以把森林$T$中的每一个块缩成一个点,那么用$E$中的边去连接$T$,就形成了一些连通块。对于每一个连通块的方案数就是对这个连通块做一个\textbf{生成树计数}就可以了,然后把这些连通块各自的方案数相乘就是选择边权为$w$的边的方案数。\par
		3、算完边权为$w$的方案之后,就把这些边加入到$T$中,形成新的森林。\par
\subsubsection{带限制的最小生成树问题}
	\begin{itemize}
		\item Problem \par
			无向带权连通图,每条边是黑色或白色。让你求一棵最小权的恰好有K条白色边的生成树。
		\item Solution \par
			1、对于一个图,如果存在一棵生成树,它的白边数量为$x$,那么就称$x$是合法白边数。所有的合法白边数组成一个区间$[l,r]$。\par
			2、对于一个图,如果存在一棵最小生成树,它的白边数量为$x$,那么就称$x$是最小合法白边数。所有的最小合法白边数组成一个区间$[l,r]$。\par
			3、将所有白边追加权值$x$所得到的最小生成树,如果该树有$a$条白边。那么这可树就是$a$条白边最小生成树的一个最优解。

			~\\
			所以可以二分得到一个最大的$x$使得所求的最小生成树的白边的最小值和最大值所组成的一个区间$[l,r]$,若$K\in [l,r]$,则该最小生成树就是最优解。其实只要求出最大的一个$x$使得最小生成树中最大白边数量不小于$K$即可。记录答案的时候,必须把枚举的$x$加上,然后在最后减去$K*x$,如果直接在计算的时候加原来白边的长度的话,有可能超过$K$条边。\par
			对于黑、白不同的边,他们内部的顺序是一样的,所以一开始将黑白边分别排序,这样在二分判断的时候只需要$O(M)$的时间复杂度去合并排序的边了,其中$M$是边数。总的时间复杂度为$O(MlogW+NlogN)$其中$N$是顶点数,$W$是边权。

		\item Expansion \par
			1、限制某个节点$node$的度数恰好为$K$的最小生成树。解法就是把和$node$关联的边标记为白边,其余的边为黑边,然后就转化为上面的经典问题了。
	\end{itemize}

