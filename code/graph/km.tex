\subsubsection{二分图最大权匹配}
\begin{verbatim}
#include "template.cpp" 

/*
 * name     :     Kuhn_Munkres
 * usage     :    maximum weight bipartite matching
 * develop    :    mincost maxflow 
 * space complexity    :    O(N^2)
 * time complexity    :    O(N^3)
 * checked    :    no
 */

const int N = 105, inf = 0x3F3F3F3F;

int n;
int graph[N][N];
int match[N], slack[N], lx[N], ly[N];
bool vx[N], vy[N];

bool find(int x) {
    vx[x] = true;
    rep(y, n){
        if (vy[y]) continue;
        if (lx[x] + ly[y] == graph[x][y]){
            vy[y] = true;
            if (match[y] == -1 || find(match[y])){
                match[y] = x;
                return true;
            }
        }
        else slack[y] = min(slack[y], lx[x] + ly[y] - graph[x][y]);
    }
    return false;
}

int max_match(int n) {
    rep(i, n) {
        lx[i] = *max_element(graph[i], graph[i] + n);
        ly[i] = 0;
    }
    memset(match, -1, sizeof(match));
    rep(x, n){
        memset(slack, -1, sizeof slack);
        while(true){
            Cls(vx);
            Cls(vy);
            if (find(x)) break;
            int sub = inf;
            rep(i, n) if (!vy[i]) sub = min(sub, slack[i]);
            rep(i, n) if (vx[i]) lx[i] -= sub;
            rep(i, n) {
                if (vy[i]) ly[i] += sub;
                else slack[i] -= sub;
            }
        }
    }
    int res = 0;
    rep(i, n) res += graph[match[i]][i];
    return res;
}

int min_match(int n) {
    rep(i, n) rep(j, n) graph[i][j] *= -1;
    return -max_match(n);
}

int main() {
    while(~scanf("%d", &n)){
        rep(i, n) rep(j, n) scanf("%d", &graph[i][j]);
        printf("%d\n", max_match(n));
    }
    return 0;
}
\end{verbatim}
