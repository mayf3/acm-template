\subsubsection{Dijkstra最短路}
\begin{verbatim}
#include <cstdio>
#include <cstring>
#include <algorithm>
#include <map>

/*
 * name     :     dijkstra(STL)
 * usage     :    single-source shortest path(only non-negative weight)
 * develop    :    small label first optimization, negative circle
 * space complexity    :    O(M)
 * time complexity    :    O(NlogN)
 * checked    :    no
 */

const int N = 111111; //number of the vertices

int n, m;
int dist[N];
vector<PII> E[N];

int calc(int s, int e) {
    priority_queue<PII, vector<PII>, greater<PII> > Q;
    rep(i, n) dist[i] = -1;
    dist[s] = 0;
    Q.push(MP(0, s));
    int x, y, cost;
    while (!Q.empty()) {
        x = Q.top().Y, cost = Q.top().X;
        Q.pop();
        if (cost > dist[x]) continue;
        rep(i, E[x].size()){
            y = E[x][i].X, cost = E[x][i].Y;
            if (dist[y] == -1 || dist[y] != -1 && dist[x] + cost > dist[y]){
                dist[y] = dist[x] + cost;
                Q.push( make_pair(dist[y], y) );
            }
        }
    }
    return dist[e];
}

int main(){
    while(~scanf("%d%d", &n, &m)){
        rep(i, n) E[i].clear();
        int x, y, c;
        rep(i, m){
            scanf("%d%d%d", &x, &y, &c);
            x--, y--;
            E[x].PB(MP(y, c));
            E[y].PB(MP(x, c));
        }
        printf("%d\n", calc(0, n - 1));
    }
    return 0;
}
\end{verbatim}
