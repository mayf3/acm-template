\subsubsection{可持久化线段树}
持久化数据结构最重要的思想是无论询问还是插入操作,都不修改原来的数据,而是新建立节点。
~\\
~\\
\Large{题目描述} \par
\small
1. C l r d: Adding a constant d for every $A_i(l\leq i\leq r)$, and increase the time $t$ add by 1, this is the only operation that will cause the time increase. \par
2. Q l r: Querying the current sum of $A_i(l \leq i \leq r)$. \par
3. H l r t: Querying a history sum of $A_i(l \leq i\leq r)$ in time $t$. \par
4. B t: Back to time $t$. And once you decide return to a past, you can never be access to a forward edition anymore.\par
~\\
\Large{程序} \par
\small
\begin{verbatim}
#include <cmath>
#include <cstdio>
#include <cstring>
#include <vector>
#include <iostream>
#include <algorithm>
using namespace std;

const int kMaxN=100001;
const int MAXLIMIT=500000;

struct data{
    long long add;
    long long sum;
    data *lt,*rt;
    data(){}
    data(data *lt, data *rt, long long sum, int add=0):lt(lt),rt(rt),sum(sum),add(add){}
    inline long long getSum(int len){
        if (!this) return 0;
        return sum+add*len;
    }
};

data buffer[MAXLIMIT];
data* tree[kMaxN];
int n,m,total,number;
int record[kMaxN];

data *build(int l, int r){
    if (l==r){
        scanf("%d",&number);
        data *node=&buffer[total++];
        node->lt=node->rt=NULL;
        node->sum=number; node->add=0; 
        return node;
    }

    int mid=(l+r)/2;
    data *node=&buffer[total++];
    node->lt=build(l,mid);
    node->rt=build(mid+1,r);
    node->sum=node->lt->getSum(mid-l+1)+
        node->rt->getSum(r-mid);

    node->add=0;
    return node;
}

data *insert(int l, int r, data *node, int L, int R, long long Add, int add){
    if (L>r || l>R){
        if (Add!=0){
            buffer[total]=data(*node);
            buffer[total].add+=Add;
            return &buffer[total++];
        }
        return node;
    }
    if (L<=l && r<=R){
        buffer[total]=data(*node);
        buffer[total].add+=Add+add;
        return &buffer[total++];
    }

    int mid=(l+r)/2;
    data *lt=insert(l,mid,node->lt,L,R,
        Add+node->add,add);

    data *rt=insert(mid+1,r,node->rt,L,R,
        Add+node->add,add);

    long long sum=lt->getSum(mid-l+1)+
        rt->getSum(r-mid);

    buffer[total++]=data(lt,rt,sum,0);
    return &buffer[total-1];
}

long long ask(int l, int r, data *node, int L, int R, long long Add){
    if (L>r || l>R) return 0;
    if (L<=l && r<=R){
        return node->getSum(r-l+1)+Add*(r-l+1);
    }

    int mid=(l+r)/2;
    long long lt=ask(l,mid,node->lt,L,R,
        Add+node->add);

    long long rt=ask(mid+1,r,node->rt,L,R,
        Add+node->add);

    return lt+rt;
}


void init(){
    total=0;
    tree[0]=build(1,n);
}

void solve(){
    char opt[2];
    int L,R,add;
    int cur=0,Time;
    record[cur]=total;
    for (int i=0; i<m; i++){
        scanf("%s",opt);
        if (opt[0]=='C'){
            scanf("%d%d%d",&L,&R,&add);
            tree[cur+1]=insert(1,n,tree[cur],L,R,0,add);
            record[++cur]=total;
        } else
        if (opt[0]=='H'){
            scanf("%d%d%d",&L,&R,&Time);
            long long temp=ask(1,n,tree[Time],L,R,0);
            printf("%I64d\n",temp);
        } else
        if (opt[0]=='Q'){
            scanf("%d%d",&L,&R);
            long long temp=ask(1,n,tree[cur],L,R,0);
            printf("%I64d\n",temp);
        } else{
            scanf("%d",&cur);
            total=record[cur];
        }
    }
}

int main(){
    while (scanf("%d%d",&n,&m)!=EOF){
        init();
        solve();
    }
    return 0;
}
\end{verbatim}
