\subsubsection{生成N皇后一组解}
\begin{verbatim}
#include <cstdio>
#include <algorithm>
#include <cstring>

using namespace std;

const int maxn = 300 + 10;

int board[maxn], diagl[maxn * 2], diagr[maxn * 2];
int n, conflict;

void compute() {
    conflict = 0;
    memset(diagl, 0, sizeof(diagl));
    memset(diagr, 0, sizeof(diagr));
    for (int i = 1; i <= n; ++i) {
        int r = i, c = board[i];
        ++diagl[r - c + n]; ++diagr[r + c - 1];
    }
    for (int i = 1; i < (n << 1); ++i) {
        if (diagl[i]) conflict += diagl[i] - 1;
        if (diagr[i]) conflict += diagr[i] - 1;
    }
}

void del(int r, int c, int& sum) {
    if (--diagl[r - c + n]) --sum;
    if (--diagr[r + c - 1]) --sum;
}

void ins(int r, int c, int& sum) {
    if (++diagl[r - c + n] > 1) ++sum;
    if (++diagr[r + c - 1] > 1) ++sum;
}

void NQueens() {
    for (int i = 1; i <= n; ++i) board[i] = i;
    while (1) {
        random_shuffle(board + 1, board + n + 1);
        compute();
        while (conflict) {
            int tot = 0;
            for (int i = 1; i <= n; ++i)
                for (int j = i + 1; j <= n; ++j) {
                    int k = conflict;
                    del(i, board[i], k); del(j, board[j], k);
                    ins(i, board[j], k); ins(j, board[i], k);
                    if (k < conflict) {
                        swap(board[i], board[j]);
                        conflict = k; ++tot;
                    }
                    else {
                        ins(i, board[i], k); ins(j, board[j], k);
                        del(i, board[j], k); del(j, board[i], k);
                    }
                }
            if (!tot) break;
        }
        if (!conflict) break;
    }
    for (int i = 1; i <= n; ++i) printf("%d\n", board[i]);
}

int main(){
    scanf("%d", &n);
    NQueens();
    return 0;
}
\end{verbatim}
