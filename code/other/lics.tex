\subsubsection{最长公共上升子序列}
\begin{verbatim}
#include <cstdio>
#include <cstring>
#include <algorithm>

using namespace std;

#define rep(i, n) for(int i = 0; i < (n); i++)

int n, m;
int a[1010],b[1010];
int f[1010];

int LCIS() {
    memset(f, 0, sizeof(f));
    rep(i, n){
        int k = 0;
        rep(j, m){
            if(a[i] == b[j]){//如果a[i]==b[j] 
                if(f[j]<f[k]+1){//就在0到j-1之间,找一个b[k]小于a[i]的f[k]值最大的解
                    f[j]=f[k]+1;
                }
            }
            if(a[i]>b[j]){//0到j-1中,对于小于a[i]的,保存f值的最优解
                if(f[k]<f[j]){
                    k=j;
                }
            }
        }
    }
    int ans=0;
    rep(i, m){
        ans=max(ans,f[i]);
    }
    return ans;
}

int main() {
    int t;
    scanf("%d", &t);
    while(t--) {
        scanf("%d",&n);
        rep(i, n){
            scanf("%d",&a[i]);
        }
        scanf("%d",&m);
        rep(j, m){
            scanf("%d",&b[j]);
        }
        printf("%d\n",LCIS());
        if (t) printf("\n");
    }
    return 0;
}
\end{verbatim}
