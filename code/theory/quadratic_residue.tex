\subsubsection{二次剩余}
\begin{verbatim}
int Euler(int a, int p) {
    int ret = 1, s = a, k = (p - 1) / 2;
    while (k) {
        if (k & 1) ret = (long long)ret * s % p;
        s = (long long)s * s % p;
        k >>= 1;
    }
    if (ret != 1) ret = 0;
    else ret = 2;
    return ret;
}        

int cal(int p, int n, int d) {
    int pn = 1;
    for (int i = 0; i < n; ++i) pn *= p;
    d %= pn;
    if (d == 0) {
        int k = 1;
        for (int i = 0; i < n / 2; ++i) k *= p;
        return k;
    }
    int r, b = 0, pr, pb;
    while (d % p == 0) {
        d /= p;
        ++b;
    }
    if (b % 2 != 0) return 0;
    r = b / 2;
    pr = 1;
    for (int i = 0; i < r; ++i) pr *= p;
    if (p == 2) {
        n -= b;
        if (n < 2) return 1 * pr;
        if (n == 2 && d % 4 == 1) return 2 * pr;
        if(n > 2 && d % 8 == 1) return 4 * pr;
        return 0;
    }
    return pr * Euler(d, p);
}

// x^2 = d (% m)
int QuadraticResidue(int m, int d) {
    int ret = 1;
    for (int i = 2; i * i <= m; ++i)
        if (m % i == 0) {
            int j = 0, q = 1;
            while (m % i == 0) {
                m /= i;
                ++j;
                q *= i;
            }
            ret *= cal(i, j, d);
        }
    if (m > 1) ret *= cal(m, 1, d);
    return ret;
}
\end{verbatim}
