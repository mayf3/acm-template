\subsubsection{牛顿插值法}
\begin{verbatim}
#include <stdio.h>
#include <string.h>
#include <algorithm>
using namespace std;

typedef long long ll;

const ll R = 1000000007;
const int N = 1000 + 10;

int n, q;
ll fs[N][N], xs[N], ys[N];

inline ll modular(ll x, ll y) {
    return (x % y + y) % y;
}

ll exgcd(ll m, ll n, ll &a, ll &b) { 
    if (!n) return a = 1, b = 0, m;
    ll d = exgcd(n, m % n, b, a);
    b -= m / n * a;
    return d;
}

ll invert(ll x, ll m)  { 
    ll a, b;
    exgcd(x, m, a, b);
    return modular(a, m);
}

ll f(ll x) {
    ll y = ys[0];
    ll p = 1;
    for (int i = 1; i < n; i++) {
        p = modular(p * (x - xs[i - 1]), R);
        y = modular(y + p * fs[0][i], R);
    }
    return y;
}

int main() {
    scanf("%d %d", &n, &q);
    for (int i = 0; i < n; i++)
        scanf("%lld %lld", &xs[i], &ys[i]);
    for (int i = 0; i < n; i++)
        fs[i][i] = ys[i];
    for (int d = 1; d < n; d++)
        for (int i = 0, j = d; j < n; i++, j++)
            fs[i][j] = modular(fs[i + 1][j] - fs[i][j - 1], R) * invert(d, R) % R;
    while (q--) {
        ll x; scanf("%lld", &x);
        printf("%lld\n", f(x));
    }
    return 0;
}
\end{verbatim}
