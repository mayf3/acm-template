\subsubsection{FFT}
\begin{verbatim}
typedef complex<double> Comp;
typedef Comp cp;

const double PI = acos(-1);
const Comp I(0, 1);

const int N = 1<<18;
Comp tmp[N];
Comp a[N] = { }, b[N] = { };
int n,m,d;
LL ans;
LL c[N];
int v[N];

void fft(Comp *a,int n,int f=1){
    double arg = PI;
    for(int k = n >> 1; k; k >>= 1, arg *= 0.5) {
        cp wm(cos(arg), f * sin(arg)), w(1, 0);
        for (int i = 0; i < n; i += k, w *= wm) {
            int p = i << 1;
            if (p >= n) p -= n;
            for (int j = 0; j < k; ++j) tmp[i + j] = a[p + j] + w * a[p + k + j];
        }
        rep(i,n) a[i] = tmp[i];
    }
}

int calc(int n){
    fft(a,n,1);
    fft(b,n,1);
    rep(i,n) a[i] = a[i]*b[i];
    fft(a,n,-1);
    rep(i,n) a[i] /= n;
}

int main(){
    int T;
    scanf("%d",&T);
    rep(cas,T){
        scanf("%d",&n);
        rep(i,n) scanf("%d",v+i);
        int ma=0;
        rep(i,n) ma=max(ma,v[i]);
        rep(i,N) a[i]=b[i]=Comp(0,0);
        rep(i,n) b[v[i]]=a[v[i]]+=Comp(1,0);
        int top=1;
        while(top<=ma*2) top*=2;
        calc(top);
        LL ans=(LL)n*(n-1)*(n-2)/6;
        rep(i,top) c[i]=(LL)(a[i].real()+0.4);
        rep(i,n) c[v[i]*2]--;
        rep(i,top) c[i]/=2;
        //rep(i,top) cout<<i <<' '<<c[i]<<endl;
        REP(i,1,top) c[i]+=c[i-1];
        rep(i,n) ans-=c[v[i]];
        printf("%.7lf\n",ans/((double)n*(n-1)*(n-2)/6));
    }
    return 0;
}
\end{verbatim}
