\subsubsection{高斯消元}
\begin{verbatim}
//在异或方程里,要求最小改变次数,那就从后往前面枚举,先枚举只有变量之后,前面的变量就确定了
void gauss(int n, int m, double p[M][M]){
    static double tmp[M][M];
    static double *b[M];
    rep(i, n){
        rep(j, m){
            tmp[i][j] = p[i][j];
        }
    }
    rep(i, n){
        b[i] = tmp[i];
    }
    rep(i, n){
        REP(j, i, n){
            if (sign(fabs(b[j][i]) - fabs(b[i][i])) > 0) swap(b[i], b[j]);
        }
        rep(j, n){
            if (i == j) continue;
            double rate = b[j][i] / b[i][i];
            rep(k, m) b[j][k] -= b[i][k] * rate;
        }
        double rate = b[i][i];
        rep(j, m) b[i][j] /= rate;
    }
    rep(i, n){
        rep(j, m){
            p[i][j] = b[i][j];
        }
    }
}

//整数答案不超过LL,可以用辗转相除法做高斯消元
void gauss(int n, int m, T p[M][M], T& ret){
    static T tmp[M][M];
    static T *b[M];
    rep(i, n){
        rep(j, m){
            tmp[i][j] = p[i][j];
        }
    }
    rep(i, n){
        b[i] = tmp[i];
    }
    rep(i, n){
        REP(j, i, n){
            if (abs(b[j][i]) > abs(b[i][i])) swap(b[i], b[j]);
        }
        if (b[i][i] == 0){
            ret = 0;
            return;
        }
        REP(j, i + 1, n){
            if (b[j][i] == 0) continue;
            while(b[j][i]){
                if (abs(b[i][i]) > abs(b[j][i])) swap(b[i], b[j]);
                LL rate = b[j][i] / b[i][i];
                REP(k, i, m) b[j][k] = b[j][k] - b[i][k] * rate;
            }
        }
        ret *= b[i][i];
        ret = abs(ret) % MD;
    }
    rep(i, n){
        rep(j, m){
            p[i][j] = b[i][j];
        }
    }
}
\end{verbatim}
