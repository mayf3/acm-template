\subsubsection{多项式求逆}
\begin{verbatim}
#include <cstdio>
#include <complex>
#include <cmath>
#include <algorithm>
#include <iostream>
using std::copy;
using std::fill;

const long long mod_v = 17ll * (1 << 27) + 1;
const int MaxN = 10010;
long long a[MaxN], b[MaxN], c[MaxN];
long long eps[MaxN], inv_eps[MaxN];
int tot;

long long power(long long x, long long p) {
    long long v = 1;
    while(p) {
        if(p & 1) v = x * v % mod_v;
        x = x * x % mod_v;
        p >>= 1;
    }
    return v;
}

void init_eps(int n) {
    tot = n;
    long long base = power(3, (mod_v - 1) / n);
    long long inv_base = power(base, mod_v - 2);
    eps[0] = 1, inv_eps[0] = 1;
    for(int i = 1; i < n; ++i) {
        eps[i] = eps[i - 1] * base % mod_v;
        inv_eps[i] = inv_eps[i - 1] * inv_base % mod_v;
    }
}

long long inc(long long x, long long d) {
    x += d; 
    return x >= mod_v ? x - mod_v : x; 
}

long long dec(long long x, long long d) {
    x -= d; 
    return x < 0 ? x + mod_v : x; 
}

void transform(int n, long long *x, long long *w) {
    for(int i = 0, j = 0; i != n; ++i) {
        if(i > j) std::swap(x[i], x[j]);
        for(int l = n >> 1; (j ^= l) < l; l >>= 1);
    }
    for(int i = 2; i <= n; i <<= 1) {
        int m = i >> 1;
        for(int j = 0; j < n; j += i) {
            for(int k = 0; k != m; ++k) {
                long long z = x[j + m + k] * w[tot / i * k] % mod_v;
                x[j + m + k] = dec(x[j + k], z);
                x[j + k] = inc(x[j + k], z);
            }
        }
    }
}

void polynomial_inverse(int deg, long long* a, long long* b, long long* tmp) {
    if(deg == 1) {
        b[0] = power(a[0], mod_v - 2);
    } else {
        polynomial_inverse((deg + 1) >> 1, a, b, tmp);
        int p = 1;
        while(p < deg << 1) p <<= 1;
        copy(a, a + deg, tmp);
        fill(tmp + deg, tmp + p, 0);
        transform(p, tmp, eps);
        transform(p, b, eps);
        for(int i = 0; i != p; ++i) {
            b[i] = (2 - tmp[i] * b[i] % mod_v) * b[i] % mod_v;
            if(b[i] < 0) b[i] += mod_v;
        }
        transform(p, b, inv_eps);
        long long inv = power(p, mod_v - 2);
        for(int i = 0; i != p; ++i)
            b[i] = b[i] * inv % mod_v;
        fill(b + deg, b + p, 0);
    }
}

int main() {
    init_eps(2048);
    int n;
    std::cin >> n;
    for(int i = 0; i != n; ++i)
        std::cin >> a[i];
    polynomial_inverse(n, a, b, c);
    std::cout << "inverse: ";
    for(int i = 0; i != n; ++i)
        printf("%lld ", (b[i] + mod_v) % mod_v);
    std::cout << std::endl;
    return 0;
}
\end{verbatim}
