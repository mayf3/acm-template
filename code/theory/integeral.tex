\subsubsection{辛普森积分}
\begin{verbatim}
#include <stdio.h>
#include <string.h>
#include <algorithm>
#include <math.h>
using namespace std;

//{{{start
double f(double x){
    return 8 * sqrt(r * r - x * x) * sqrt(R * R - x * x);
}

double simpson(double width, double fa, double fb, double fc){
    return (fb + fa + 4 * fc) * width / 6;
}

double asr(double a, double b, double eps, double A){
    double c = (a + b) / 2;
    double fa = f(a), fb = f(b), fc = f(c);
    double L = simpson(c - a, fa, fc, f((c + a) / 2));
    double R = simpson(b - c, fc, fb, f((b + c) / 2));
    if (fabs(L + R - A) <= 15 * eps) return L + R + (L + R - A) / 15;
    return asr(a, c, eps / 2, L) + asr(c, b, eps / 2, R);
}

double asr(double a, double b, double eps){
    return asr(a, b, eps, simpson(b - a, f(a), f(b), f((b + a) / 2)));
}
//end}}}

/* simpson integral of f at [a, b] */
double simpson(double (*f)(double), double a, double b) {
    int n = (int)(10000 * (b - a)); n -= n % 2;
    double A = 0, B = 0, d = (b - a) / n;
    for (int i = 1; i < n; i += 2)
        A += f(a + i * d);
    for (int i = 2; i < n; i += 2)
        B += f(a + i * d);
    return (f(a) + f(b) + 4 * A + 2 * B) * d / 3;
}

/* romberg integral of f at [a, b] */
double romberg(double (*f)(double), double l, double r) {
    const int N = 18;
    double a[N][N], p[N];
    p[0] = 1;
    for (int i = 1; i < N; i++)
        p[i] = p[i - 1] * 4;
    a[0][0] = (f(l) + f(r)) / 2;
    for (int i = 1, n = 2; i < N; i++, n <<= 1) {
        a[i][0] = 0;
        for (int j = 1; j < n; j += 2)
            a[i][0] += f((r - l) * j / n + l);
        a[i][0] += a[i - 1][0] * (n / 2);
        a[i][0] /= n;
    }
    for (int j = 1; j < N; j++)
        for (int i = 0; i < N - j; i++)
            a[i][j] = (a[i + 1][j - 1] * p[j] - a[i][j - 1]) / (p[j] - 1);
    return a[0][N - 1] * (r - l);
}

/* helper function of adaptive_simpsons */
double adaptive_simpsons_aux(double (*f)(double), double a, double b, double eps,
        double s, double fa, double fb, double fc, int depth) {
    double c = (a + b) / 2, h = b - a;
    double d = (a + c) / 2, e = (c + b) / 2;
    double fd = f(d), fe = f(e);
    double sl = (fa + 4 * fd + fc) * h / 12;
    double sr = (fc + 4 * fe + fb) * h / 12;
    double s2 = sl + sr;
    if (depth <= 0 || fabs(s2 - s) <= 15 * eps)
        return s2 + (s2 - s) / 15;
    return adaptive_simpsons_aux(f, a, c, eps / 2, sl, fa, fc, fd, depth - 1) +
        adaptive_simpsons_aux(f, c, b, eps / 2, sr, fc, fb, fe, depth - 1);
}

/* Adaptive Simpson's Rule, integral of f at [a, b], max error of eps, max depth of depth */
double adaptive_simpsons(double (*f)(double), double a, double b, double eps, int depth) {
    double c = (a + b) / 2, h = b - a;
    double fa = f(a), fb = f(b), fc = f(c);
    double s = (fa + 4 * fc + fb) * h / 6;
    return adaptive_simpsons_aux(f, a, b, eps, s, fa, fb, fc, depth);
}
\end{verbatim}
