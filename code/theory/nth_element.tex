\subsubsection{根据前m项求第n项}
\begin{verbatim}
#include <cstdio>

typedef long long LL;

const int N = 222;
const int MOD = 1234567891;

int k, a, n, d;
int fac[N], inv[N];
int f[N], g[N], h[N];

int qmod(int a, int n, int p = MOD){
  int ret = 1;
  while(n){
    if (n & 1) ret = 1LL * ret * a % p;
    a = 1LL * a * a % p;
    n >>= 1;
  }
  return ret;
}

int C(int n, int m){
  if (n < m) return 0;
  return 1LL * fac[n] * inv[m] % MOD * inv[n - m] % MOD;
}

// calc the Nth element from f[0] ~ f[m] (f[x] = sigma(ai * x ^ i, 0 <= i <= m);
int calc(LL n, int m, int f[]){
  static int pre[N], suf[N];
  if (n <= m) return f[n];
  pre[0] = n % MOD;
  for(int i = 1; i <= m; i++) pre[i] = 1LL * pre[i - 1] * ((n - i) % MOD) % MOD;
  suf[m] = (n - m) % MOD;
  for(int i = m - 1; i >= 0; i--) suf[i] = 1LL * suf[i + 1] * ((n - i) % MOD) % MOD;
  int ret = 0;
  int now = (m & 1) ? -1 : 1;
  for(int i = 0; i <= m; i++){
    int tmp = 1LL * now * f[i] * inv[i] % MOD * inv[m - i] % MOD; 
    if (i) tmp = 1LL * tmp * pre[i - 1] % MOD;
    if (i < m) tmp = 1LL * tmp * suf[i + 1] % MOD;
    ret = (0LL + ret + tmp) % MOD;
    now = -now;
  }
  return ret;
}

// calc 0! ~ m! and 1/0! ~ 1/m!
void init(){
  fac[0] = 1;
  for(int i = 1; i < N; i++) fac[i] = 1LL * fac[i - 1] * i % MOD;
  inv[N - 1] = qmod(fac[N - 1], MOD - 2);
  for(int i = N - 2; i >= 0; i--){
    inv[i] = 1LL * inv[i + 1] * (i + 1) % MOD;
  }
}

int main(){
  init();
  int T;
  scanf("%d", &T);
  while(~scanf("%d%d%d%d", &k, &a, &n, &d)){
    f[0] = 0;
    for(int i = 1; i <= k + 3; i++){
      f[i] = (0LL + f[i - 1] + qmod(i, k)) % MOD;
    }
    g[0] = 0;
    for(int i = 1; i <= k + 3; i++){
      g[i] = (0LL + g[i - 1] + f[i]) % MOD;
    }
    h[0] = calc(a, k + 2, g);
    for(int i = 1; i <= k + 3; i++){
      h[i] = (0LL + h[i - 1] + calc(a + 1LL * i * d, k + 2, g)) % MOD;
      h[i] = (1LL * h[i] % MOD + MOD) % MOD;
    }
    int ret = calc(n, k + 3 , h);
    ret = (1LL * ret % MOD + MOD) % MOD;
    printf("%d\n", ret);
  }
}
\end{verbatim}
