\subsubsection{多项式运算(exp,ln,sqrt,inv,div)}
\begin{verbatim}
// BZOJ 3625
#include <algorithm>
#include <cstdio>

using std::swap;
using std::fill;
using std::copy;
using std::reverse_copy;
using std::reverse;

typedef int value_t;
typedef long long calc_t;
const int MaxN = 1 << 19;
const value_t mod_base = 119, mod_exp = 23;
const value_t mod_v = (mod_base << mod_exp) + 1;
const value_t primitive_root = 3;
int epsilon_num;
value_t eps[MaxN], inv_eps[MaxN], inv2;
value_t inv[MaxN];

value_t dec(value_t x, value_t v) { x -= v; return x < 0 ? x + mod_v : x; }
value_t inc(value_t x, value_t v) { x += v; return x >= mod_v ? x - mod_v : x; }
value_t pow(value_t x, value_t p) {
  value_t v = 1;
  for(; p; p >>= 1, x = (calc_t)x * x % mod_v)
    if(p & 1) v = (calc_t)x * v % mod_v;
  return v;
}

void init_eps(int num) {
  epsilon_num = num;
  value_t base = pow(primitive_root, (mod_v - 1) / num);
  value_t inv_base = pow(base, mod_v - 2);
  eps[0] = inv_eps[0] = 1;
  for(int i = 1; i != num; ++i) {
    eps[i] = (calc_t)eps[i - 1] * base % mod_v;
    inv_eps[i] = (calc_t)inv_eps[i - 1] * inv_base % mod_v;
  }
}

void transform(int n, value_t *x, value_t *w = eps) {
  for(int i = 0, j = 0; i != n; ++i) {
    if(i > j) swap(x[i], x[j]);
    for(int l = n >> 1; (j ^= l) < l; l >>= 1);
  }
  for(int i = 2; i <= n; i <<= 1) {
    int m = i >> 1, t = epsilon_num / i;
    for(int j = 0; j < n; j += i) {
      for(int p = 0, q = 0; p != m; ++p, q += t) {
        value_t z = (calc_t)x[j + m + p] * w[q] % mod_v;
        x[j + m + p] = dec(x[j + p], z);
        x[j + p] = inc(x[j + p], z);
      }
    }
  }
}

void inverse_transform(int n, value_t *x) {
  transform(n, x, inv_eps);
  value_t inv = pow(n, mod_v - 2);
  for(int i = 0; i != n; ++i) x[i] = (calc_t)x[i] * inv % mod_v;
}

void polynomial_inverse(int n, value_t *A, value_t *B) {
  static value_t T[MaxN];
  if(n == 1) {
    B[0] = pow(A[0], mod_v - 2);
    return;
  }

  int half = (n + 1) >> 1;
  polynomial_inverse(half, A, B);

  int p = 1;
  for(; p < n << 1; p <<= 1);

  fill(B + half, B + p, 0);
  transform(p, B);

  copy(A, A + n, T);
  fill(T + n, T + p, 0);
  transform(p, T);

  for(int i = 0; i != p; ++i) B[i] = (calc_t)B[i] * dec(2, (calc_t)T[i] * B[i] % mod_v) % mod_v;
  inverse_transform(p, B);
}

void polynomial_sqrt(int n, value_t *A, value_t *B) {
  static value_t T[MaxN];
  if(n == 1) {
    B[0] = 1; // sqrt A[0], here is 1
    return;
  }

  int p = 1;
  for(; p < n << 1; p <<= 1);

  int half = (n + 1) >> 1;
  polynomial_sqrt(half, A, B);
  fill(B + half, B + n, 0);
  polynomial_inverse(n, B, T);
  fill(T + n, T + p, 0);
  transform(p, T);

  fill(B + half, B + p, 0);
  transform(p >> 1, B);
  for(int i = 0; i != p >> 1; ++i) B[i] = (calc_t)B[i] * B[i] % mod_v;
  inverse_transform(p >> 1, B);
  for(int i = 0; i != n; ++i) B[i] = (calc_t)inc(A[i], B[i]) * inv2 % mod_v;
  transform(p, B);
  for(int i = 0; i != p; ++i) B[i] = (calc_t)B[i] * T[i] % mod_v;
  inverse_transform(p, B);
}

void polynomial_logarithm(int n, value_t *A, value_t *B) {
  static value_t T[MaxN];
  int p = 1;
  for(; p < n << 1; p <<= 1);
  polynomial_inverse(n, A, T);
  fill(T + n, T + p, 0);
  transform(p, T);

  // derivative
  copy(A, A + n, B);
  for(int i = 0; i < n - 1; ++i) B[i] = (calc_t)B[i + 1] * (i + 1) % mod_v;
  fill(B + n - 1, B + p, 0);
  transform(p, B);

  for(int i = 0; i != p; ++i) B[i] = (calc_t)B[i] * T[i] % mod_v;
  inverse_transform(p, B);

  // integral
  for(int i = n - 1; i; --i) B[i] = (calc_t)B[i - 1] * inv[i] % mod_v;
  B[0] = 0;
}

void polynomial_exponent(int n, value_t *A, value_t *B) {
  static value_t T[MaxN];
  if(n == 1) {
    B[0] = 1;
    return;
  }

  int p = 1; 
  for(; p < n << 1; p <<= 1);

  int half = (n + 1) >> 1;
  polynomial_exponent(half, A, B);
  fill(B + half, B + p, 0);

  polynomial_logarithm(n, B, T);
  for(int i = 0; i != n; ++i) T[i] = dec(A[i], T[i]);
  T[0] = inc(T[0], 1);
  transform(p, T);
  transform(p, B);
  for(int i = 0; i != p; ++i) B[i] = (calc_t)B[i] * T[i] % mod_v;
  inverse_transform(p, B);
}

void polynomial_division(int n, int m, value_t *A, value_t *B, value_t *D, value_t *R) {
  static value_t A0[MaxN], B0[MaxN];

  int p = 1, t = n - m + 1;
  while(p < t << 1) p <<= 1;

  fill(A0, A0 + p, 0);
  reverse_copy(B, B + m, A0);
  polynomial_inverse(t, A0, B0);
  fill(B0 + t, B0 + p, 0);
  transform(p, B0);

  reverse_copy(A, A + n, A0);
  fill(A0 + t, A0 + p, 0);
  transform(p, A0);

  for(int i = 0; i != p; ++i) A0[i] = (calc_t)A0[i] * B0[i] % mod_v;
  inverse_transform(p, A0);
  reverse(A0, A0 + t);
  copy(A0, A0 + t, D);

  for(p = 1; p < n; p <<= 1);
  fill(A0 + t, A0 + p, 0);
  transform(p, A0);
  copy(B, B + m, B0);
  fill(B0 + m, B0 + p, 0);
  transform(p, B0);
  for(int i = 0; i != p; ++i) A0[i] = (calc_t)A0[i] * B0[i] % mod_v;
  inverse_transform(p, A0);
  for(int i = 0; i != m; ++i) R[i] = (A[i] - A0[i]) % mod_v;
  fill(R + m, R + p, 0);
}

value_t tmp[MaxN];
value_t A[MaxN], B[MaxN], C[MaxN], T[MaxN];

int main() {
  int n, m;
  std::scanf("%d %d", &n, &m);
  int min_v = ~0u >> 1;
  for(int i = 0; i != n; ++i) {
    std::scanf("%d", tmp + i);
    if(min_v > tmp[i]) min_v = tmp[i];
  }

  inv2 = mod_v - mod_v / 2;

  int p = 1;
  for(; p < (m + min_v + 1) << 1; p <<= 1);
  init_eps(p);

  A[0] = 1;
  for(int i = 0; i != n; ++i) {
    int x = tmp[i];
    T[x - min_v] = 2;
    A[x] = mod_v - 4;
  }

  polynomial_inverse(m + min_v + 1, T, C);
  polynomial_sqrt(m + min_v + 1, A, B);
  B[0] = dec(1, B[0]);
  for(int i = 1; i <= m + min_v; ++i) B[i] = mod_v - B[i];
  for(int i = 0; i <= m; ++i) B[i] = B[i + min_v];
  fill(B + m + 1, B + p, 0);
  fill(C + m + 1, C + p, 0);
  transform(p, B);
  transform(p, C);
  for(int i = 0; i != p; ++i) B[i] = (calc_t)B[i] * C[i] % mod_v;
  inverse_transform(p, B);
  for(int i = 1; i <= m; ++i) std::printf("%d\n", B[i]);
  return 0;
}
\end{verbatim}
