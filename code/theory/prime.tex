\subsubsection{分解质因数}
\begin{verbatim}
#include <stdio.h>
#include <string.h>
#include <algorithm>
using namespace std;

const int M = 1000010;

int n, m, dn;
int fp[M], prime[M], pn;
int p[M], k[M], d[M];

/* O(M) */
void init() {
    pn = 0;
    for (int i = 2; i < M; i++) {
        if (!fp[i]) fp[i] = prime[pn++] = i;
        for (int j = 0; prime[j] * i < M; j++) {
            fp[prime[j] * i] = prime[j];
            if (i % prime[j] == 0) break;
        }
    }
}

/* assumes that n < M */
void factorize1(int n, int p[], int k[], int &m) {
for (m = 0; n > 1; m++) {
    p[m] = fp[n], k[m] = 0;
    while (n % p[m] == 0) 
    n /= p[m], k[m]++;
}
}

/* assumes that n < M * M */
void factorize(int n, int p[], int k[], int &m) {
m = 0;
for (int i = 0; n >= M && prime[i] * prime[i] <= n; i++)
if (n % prime[i] == 0) {
p[m] = prime[i], k[m] = 0;
while (n % p[m] == 0)
n /= p[m], k[m]++;
m++;
    }
    if (n < M)
    for (; n > 1; m++) {
        p[m] = fp[n], k[m] = 0;
        while (n % p[m] == 0) 
        n /= p[m], k[m]++;
    }
    if (n > 1)
    p[m] = n, k[m++] = 1;
}

void find_divisors(int p[], int k[], int m, int d[], int &dn) {
dn = 0; d[dn++] = 1;
for (int i = 0; i < m; i++) {
    for (int j = 0, z = dn; j < k[i] * z; j++, dn++)
    d[dn] = d[dn - z] * p[i];
}
sort(d, d + dn);
}

int main(){
    init();
    scanf("%d", &n);
    factorize(n, p, k, m);
    find_divisors(p, k, m, d, dn);
    for(int i = 0; i < dn; i++){
        printf("%d ", d[i]);
    }
    puts("");
    return 0;
}
\end{verbatim}
