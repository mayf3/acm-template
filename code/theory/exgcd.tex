\subsubsection{取模}
\begin{verbatim}
#include <cstdio>
#include <cstring>
#include <algorithm>
#include <cmath>
using namespace std;

typedef long long ll;
typedef long long LL;

ll gcd(ll x, ll y) { 
    return !y ? x : gcd(y, x % y);
}

ll modular(ll a, ll b) { 
    return (a % b + b) % b; 
}

/** a * x + b * y == gcd(a, b) */
LL exgcd(LL a,LL b,LL &x,LL &y) {
    if (!a){
        x = 0;
        y = 1;
        return b;
    }
    LL g = exgcd(b % a, a, x, y);
    LL t = y;
    y = x;
    x = t - (b / a) * y;
    return g;
}

/** x * y % m == 1 */
ll invert(ll x, ll m)  { 
    ll a, b;
    exgcd(x, m, a, b);
    return modular(a, m);
}

/** x % m == a && x % n == b */
ll modular_system(ll m, ll a, ll n, ll b) { 
    ll g, k, l;
    g = exgcd(m, n, k, l);
    if ((a - b) % g) return -1;
    k *= (b - a) / g;
    k = modular(k, n / g);
    return modular(k * m + a, m / g * n);
}

/** x % m[i] == r[i] */
ll modular_system_array(ll m[], ll r[], int k) {
    ll M = m[0], R = r[0];
    for (int i = 1; R != -1 && i < k; i++) {
        R = modular_system(M, R, m[i], r[i]);
        M = M / gcd(M, m[i]) * m[i];
    }
    return R;
}

/** a * x % m == b */
ll modular_equation(ll a, ll m, ll b) { 
    return modular_system(m, b, a, 0) / a % m;
}

/** calculate r = x ^ y % m */
ll modular_pow(ll x, ll y, ll m) {
    ll r = 1 % m;
    for (; y; y >>= 1, x = x * x % m)
        if (y & 1) r = r * x % m;
    return r;
}

//a ^ x = b (mod n), n is prime
int mod_log(int a, int b, int n) {
    int m = (int)ceil(sqrt(n)), inv = Inv(mod_exp(a, m, n), n);
    id[0] = 0; mexp[0] = 1;
    for (int i = 1; i < m; ++i) id[i] = i, mexp[i] = (long long)mexp[i - 1] * a % n;
    sort(id, id + m, logcmp); sort(mexp, mexp + m);
    for (int i = 0; i < m; ++i) {
        int j = lower_bound(mexp, mexp + m, b) - mexp;
        if (j < m && mexp[j] == b) return i * m + id[j];
        b = (long long)b * inv % n;
    }
    return -1;
}

// A ^ x = B (mod C)
// return x, -1 means no solution
int Dlog(int A, int B, int C) {
    map<int, int> Hash;
    int D = 1 % C;
    for(int j = D, i = 0;i <= 100; j = (long long)j * A % C, ++i)
        if (j == B) return i;
    int d = 0;
    for (int g; (g = gcd(A, C)) != 1; ) {
        if (B % g)return -1;
        ++d; C /= g; B /= g; D = (long long)D * A / g % C;
    }
    int M = (int)ceil(sqrt((double)C));
    for (int j = 1 % C, i = 0; i <= M; j = (long long)j * A % C, ++i)
        if (Hash.find(j) == Hash.end()) Hash[j] = i;
    for (int j = mod_exp(A, M, C), i = 0; i <= M; D = (long long)D * j % C, ++i) {
        int tmp = Inval(D, B, C);
        if (tmp >= 0 && Hash.find(tmp) != Hash.end()) return i * M + Hash[tmp] + d;
    }
    return -1;
}
\end{verbatim}
