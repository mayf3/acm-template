\subsubsection{组合数计算}
\begin{verbatim}
#include <cstdio>
#include <cmath>
#include <memory.h>

typedef long long T;

///Lib functions
T GCD(T a, T b) {
    return b ? GCD(b, a % b) : a;
}

T extendGCD(T a, T b, T & x, T & y) {
    if (!b)
        return x = 1, y = 0, a;
    T res = extendGCD(b, a % b, x, y), tmp = x;
    x = y, y = tmp - (a / b) * y;
    return res;
}

///for x^k
T power(T x, T k) {
    T res = 1;
    while (k) {
        if (k & 1)
            res *= x;
        x *= x, k >>= 1;
    }
    return res;
}

///for x^k mod m
T powerMod(T x, T k, T m) {
    T res = 1;
    while (x %= m, k) {
        if (k & 1)
            res *= x, res %= m;
        x *= x, k >>= 1;
    }
    return res;
}

/***************************************
  Inverse in mod p^t system
 ***************************************/
T inverse(T a, T p, T t = 1) {
    T pt = power(p, t);
    T x, y;
    y = extendGCD(a, pt, x, y);
    return x < 0 ? x += pt : x;
}

/***************************************
  Linear congruence theorem
  x = a (mod p)
  x = b (mod q)
  for gcd(p, q) = 1, 0 <= x < pq
 ***************************************/
T linearCongruence(T a, T b, T p, T q) {
    T x, y;
    y = extendGCD(p, q, x, y);
    while (b < a)
        b += q / y;
    x *= b - a, x = p * x + a, x %= p * q;
    if (x < 0)
        x += p * q;
    return x;
}

/***************************************
  prime table
  O(n)
 ***************************************/
const int PRIMERANGE = 1000000;
int prime[PRIMERANGE + 1];
int getPrime() {
    memset(prime, 0, sizeof(int) * (PRIMERANGE + 1));
    for (int i = 2; i <= PRIMERANGE; i++) {
        if (!prime[i])
            prime[++prime[0]] = i;
        for (int j = 1; j <= prime[0] && prime[j] <= PRIMERANGE / i; j++) {
            prime[prime[j] * i] = 1;
            if (i % prime[j] == 0)
                break;
        }
    }
    return prime[0];
}

/***************************************
  get factor of n
  O(sqrt(n))
  factor[][0] is prime factor
  factor[][1] is factor generated by this prime
  factor[][2] is factor counter

need: Prime Table
 ***************************************/
///you should init the prime table before
int factor[100][3], facCnt;
int getFactors(int x) {
    facCnt = 0;
    int tmp = x;
    for (int i = 1; prime[i] <= tmp / prime[i]; i++) {
        factor[facCnt][1] = 1, factor[facCnt][2] = 0;
        if (tmp % prime[i] == 0)
            factor[facCnt][0] = prime[i];
        while (tmp % prime[i] == 0)
            factor[facCnt][2]++, factor[facCnt][1] *= prime[i], tmp /= prime[i];
        if (factor[facCnt][1] > 1)
            facCnt++;
    }
    if (tmp != 1)
        factor[facCnt][0] = tmp, factor[facCnt][1] = tmp, factor[facCnt++][2] = 1;
    return facCnt;
}

/***************************************
  Easy Combination for
  C(n, k) that not exceeds limit of T
 ****************************************/
T combination(int n, int k) {
    T res = 1, g;
    if (k > n)
        return 0;
    if (n - k < k)
        k = n - k;
    for (int i = 0; i < k; i++) {
        if (res % (i + 1) == 0)
            res /= i + 1, res *= n - i;
        else if ((n - i) % (i + 1) == 0)
            res *= (n - i) / (i + 1);
        else {
            g = GCD(res, i + 1), res /= g;
            res *= (n - i) / ((i + 1) / g);
        }
    }
    return res;
}

/***************************************
  C(n, k) mod m for k < 50,
  O(k*k*lgk) m * m < T_MAX
 ***************************************/
T combinationModS(T n, T k, T m) {
    ///larger gate for more optimization
    ///too large gate may overflow somewhere
    const T gate = 1LL << 50;
    if (k > n || m == 1)
        return 0;
    if (n - k < k)
        k = n - k;
    T d[k], tmp = 1, i = 0, j, h, g;
    for (i = 0, j = n - k + 1; n - i >= j; i++) {
        d[i] = n - i;
        while (gate / d[i] >= j && n - i != j)
            d[i] *= j, j++;
    }
    for (j = 2, h = k; j <= h; h--) {
        tmp *= h;
        while (gate / tmp >= j && h != j)
            tmp *= j, j++;
        for (int s = 0; tmp != 1; s++) {
            g = GCD(tmp, d[s]);
            d[s] /= g, tmp /= g;
        }
    }
    int res = 1;
    for (j = 0; j < i; j++)
        d[j] %= m, res *= d[j], res %= m;
    return res;
}

/*****************************************
  C(n, k) mod m using prime table for sieving
  O(n)  m * m < TMAX
  limited by primetable, n could not be too large
need: prime table
 *****************************************/
T combinationModPri(T n, T k, T m) {
    if (k > n || m == 1)
        return 0;
    T result = 1, cnt = 0, temp;
    for (int i = 1; i < prime[0] && prime[i] <= n; i++) {
        temp = n, cnt = 0;
        while (temp)
            temp /= prime[i], cnt += temp;
        temp = n - k;
        while (temp)
            temp /= prime[i], cnt -= temp;
        temp = k;
        while (temp)
            temp /= prime[i], cnt -= temp;
        temp = prime[i];
        while (cnt) {
            if (cnt & 1)
                result *= temp, result %= m;
            temp *= temp, cnt >>= 1, temp %= m;
        }
        if (result == 0)
            return 0;
    }
    return result;
}

/*************************************************
  C(n, k) mod m
  O(k*lgm) m * m < T_MAX
 *************************************************/
T combinationModN(T n, T k, T m) {
    if (k > n || m == 1)
        return 0;
    k = (n - k < k) ? n - k : k;
    int pcnt = 0;
    T a = 1, b = 1, x, y, g;
    T pa = 1, pb = 1;  ///may over flow
    for (int i = 1; i <= k; i++) {
        a *= n - i + 1, b *= k - i + 1;
        while ((g = GCD(a, m)) > 1)
            pa *= g, a /= g;
        while ((g = GCD(b, m)) > 1)
            pb *= g, b /= g;
        g = GCD(pa, pb), pa /= g, pb /= g;
        while (pa % m == 0)
            pa /= m, pcnt++;
        while (pb % m == 0)
            pb /= m, pcnt--;
        b %= m, a %= m;
    }
    a *= pa / pb, a %= m;
    while (pcnt)
        return 0;
    extendGCD(b, m, x, y);
    if (x < m)
        x += m;
    x *= a, x %= m;
    return x;
}

/*******************************************
  C(n, k) mod p
  O(k) p*p <= TMAX
 *******************************************/
T combinationModP(T n, T k, T p) {
    if (k > n)
        return 0;
    if (n - k < k)
        k = n - k;
    T a = 1, b = 1, x, y;
    int pcnt = 0;
    for (int i = 1; i <= k; i++) {
        x = n - i + 1, y = i;
        while (x % p == 0)
            x /= p, pcnt++;
        while (y % p == 0)
            y /= p, pcnt--;
        x %= p, y %= p, a *= x, b *= y;
        b %= p, a %= p;
    }
    if (pcnt)
        return 0;
    extendGCD(b, p, x, y);
    if (x < 0)
        x += p;
    a *= x, a %= p;
    return a;
}

/*******************************************
  C(n, k) mod p^t
  O(k*lgn/lgp) p^2t < TMAX
 *******************************************/
T combinationModPt(T n, T k, T p, T t) {
    if (k > n)
        return 0;
    if (n - k < k)
        k = n - k;
    T pt = power(p, t);
    T a = 1, b = 1, x, y;
    int pcnt = 0;
    for (int i = 1; i <= k; i++) {
        x = n - i + 1, y = i;
        while (x % p == 0)
            pcnt++, x /= p;
        while (y % p == 0)
            pcnt--, y /= p;
        x %= pt, y %= pt, a *= x, b *= y;
        a %= pt, b %= pt;
    }
    if (pcnt >= t)
        return 0;
    extendGCD(b, pt, x, y);
    if (x < 0)
        x += pt;
    a *= x, a %= pt;
    return a * power(p, pcnt) % pt;
}

/*******************************************
  C(n, k) mod m
  O(k*lgn/lgp) m * m < TMAX
  p is fractor of m (depends on the smallest one)
need:
prime table
factor table
combinationModPt()
linearCongruence
 *******************************************/
///you need to init the prime table
T combinationModLi(T n, T k, T m) {
    if (k > n || m == 1)
        return 0;
    getFactors(m);
    T a, b, p, q;
    for (int i = 0; i < facCnt; i++) {
        if (!i)
            a = combinationModPt(n, k, factor[i][0], factor[i][2]), p = factor[i][1];
        else
            b = combinationModPt(n, k, factor[i][0], factor[i][2]), q = factor[i][1];
        if (!i)
            continue;
        a = linearCongruence(a, b, p, q), p *= q;
    }
    return a;
}

/*******************************************
  C(n, k) mod p
  Lucas's theorem for combination mod p
  O(p * lgn/lgp)
 *******************************************/
T lucas(T n, T k, T p) {
    T res = 1;
    while (n && k && res) {
        res *= combinationModP(n % p, k % p, p);
        res %= p, n /= p, k /= p;
    }
    return res;
}

/*******************************************
  a = n * (n - 1) * ... * (n - k + 1)
  b = m * (m - 1) * ... * (m - k + 1)
  c = ? from input
  if a * c / b is an integer
  this function will calculate this value module p^t
  and the c input is moduled by p^t, be sure that gcd(c, p^t) = 1
  O(len * lgn/lgp) , p^2t < TMAX
 *******************************************/
///the parameter &pcnt caches the factors consists of p
T productQuotient(T n, T m, T len, T p, T pt, T & c, T & pcnt) {
    if (!c || n < len)
        return c = 0;
    T & a = c, b = 1, x, y;
    for (int i = 1; i <= len; i++) {
        x = n - i + 1, y = m - i + 1;
        while (x % p == 0)
            x /= p, pcnt++;
        while (y % p == 0)
            y /= p, pcnt--;
        x %= pt, y %= pt, a *= x, b *= y;
        a %= pt, b %= pt;
    }
    extendGCD(b, pt, x, y);
    if (x < 0)
        x += pt;
    a *= x, a %= pt;
    return a;
}

/*******************************************
  C(n, k) mod p^t
  generalized Lucas's theorem for combination mod p
  O(p^t * lgn/lgp), p^2t < TMAX
 *******************************************/
T generalizedLucas(T n, T k, T p, T t) {
    if (k > n)
        return 0;
    if (n - k < k)
        k = n - k;
    if (t == 1)
        return lucas(n, k, p);
    T pt = power(p, t);
    T c = 1, pcnt = 0, ktable[100], ntable[100], ltable[100];
    int cnt = 0;
    for (; n || k; cnt++) {
        ktable[cnt] = k, ntable[cnt] = n, ltable[cnt] = k % pt;
        n -= k % pt, k -= k % pt, n /= p, k /= p;
    }
    for (--cnt; c && cnt >= 0; cnt--)
        productQuotient(ntable[cnt], ktable[cnt], ltable[cnt], p, pt, c, pcnt);
    if (!c || pcnt >= t)
        return 0;
    return c * power(p, pcnt) % pt;
}

/*******************************************
  C(n, k) mod m
  O(min(k, p^t * lgn/lgp)) m * m < TMAX
  p^t is fractor of m
need:
prime table
factor table
generalizedLucas
linearCongruence
 *******************************************/
///you need to init the prime table
T combinationModLucas(T n, T k, T m) {
    if (m == 1 || k > n)
        return 0;
    if (n - k < k)
        k = n - k;
    getFactors(m);
    T a, b, p, q;
    for (int i = 0; i < facCnt; i++) {
        if (!i)
            a = generalizedLucas(n, k, factor[i][0], factor[i][2]), p = factor[i][1];
        else
            b = generalizedLucas(n, k, factor[i][0], factor[i][2]), q = factor[i][1];
        if (!i)
            continue;
        a = linearCongruence(a, b, p, q), p *= q;
    }
    return a;
}

/********************************************
 ********************************************/
const T PTMAX = 10000;
T facmod[PTMAX];
void initFacMod(T p, T t = 1) {
    T pt = power(p, t);
    facmod[0] = 1 % pt;
    for (int i = 1; i < pt; i++) {
        if (i % p)
            facmod[i] = facmod[i - 1] * i % pt;
        else
            facmod[i] = facmod[i - 1];
    }
}

///you should init the facmod[] before
T factorialMod(T n, T & pcnt, T p, T t = 1) {
    T pt = power(p, t), res = 1;
    T stepCnt = 0;
    while (n) {
        res *= facmod[n % pt], res %= pt;
        stepCnt += n / pt, n /= p, pcnt += n;
    }
    res *= powerMod(facmod[pt - 1], stepCnt, pt);
    return res %= pt;
}
T combinationModPtFac(T n, T k, T p, T t = 1) {
    if (k > n || p == 1)
        return 0;
    if (n - k < k)
        k = n - k;
    T pt = power(p, t), pcnt = 0, pmcnt = 0;
    if (k < pt)
        return combinationModPt(n, k, p, t);
    initFacMod(p, t);
    T a = factorialMod(n, pcnt, p, t);
    T b = factorialMod(k, pmcnt, p, t);
    b *= factorialMod(n - k, pmcnt, p, t), b %= pt;
    pcnt -= pmcnt;
    if (pcnt >= t)
        return 0;
    a *= inverse(b, p, t), a %= pt;
    return a * power(p, pcnt) % pt;
}
T combinationModFac(T n, T k, T m) {
    getFactors(m);
    T a, b, p, q;
    for (int i = 0; i < facCnt; i++) {
        if (!i)
            a = combinationModPtFac(n, k, factor[i][0], factor[i][2]), p = factor[i][1];
        else
            b = combinationModPtFac(n, k, factor[i][0], factor[i][2]), q = factor[i][1];
        if (!i)
            continue;
        a = linearCongruence(a, b, p, q), p *= q;
    }
    return a;
}

/*********************************************************
  C(n, k) mod m with genelizedLucas and combinationModFac
  O(min(k, p^t))
 *********************************************************/
T combinationModPtAuto(T n, T k, T p, T t) {
    if (t > 6)
        return combinationModPtFac(n, k, p, t);
    return generalizedLucas(n, k, p, t);
}
T combinationModOdds(T n, T k, T m) {
    if (k > n || m == 1)
        return 0;
    getFactors(m);
    T a, b, p, q;
    for (int i = 0; i < facCnt; i++) {
        if (!i)
            a = combinationModPtAuto(n, k, factor[i][0], factor[i][2]), p = factor[i][1];
        else
            b = combinationModPtAuto(n, k, factor[i][0], factor[i][2]), q = factor[i][1];
        if (!i)
            continue;
        a = linearCongruence(a, b, p, q), p *= q;
    }
    return a;
}
\end{verbatim}
