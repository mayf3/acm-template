\subsubsection{线性递推式+fft}
\begin{verbatim}
/* f[i] = a[i], i < m; 
 * f[n] = b[0] * f[n - m] + ... + b[m - 1] * f[n - 1]; 
 * given a[], b[], m, n; find f[n]
 * O(M ^ 2 log N) 
 * !!!!m = 1 特判
 */
using namespace std;

typedef long long LL;
typedef long long ULL;
//typedef complex<double> Comp;

const int N = 10000;
const int MD = 5767169;
const long double EPS = 0.05;
const double PI = acos(-1.0);

typedef complex<double> Comp;
typedef Comp cp;

const Comp I(0, 1);
const int M = 1 << 15;

Comp fft_a[M], fft_b[M];
Comp tmp[M];

int n, a, b, p, q;

int fac[MD], inv[MD];

inline int ex_gcd(int a, int b, int &x, int &y){
    if (!a){
        x = 0;
        y = 1;
        return b;
    }
    int g = ex_gcd(b % a, a, x, y);
    int t = y;
    y = x;
    x = t - (b / a) * y;
    return g;
}

inline int qmod(int x, int n){
    int ret = 1, y = x % MD;
    while(n){
        if (n & 1) ret = (LL)ret * y % MD;
        y = (LL)y * y % MD;
        n >>= 1;
    }
    return ret;
}

inline int lucas(int n, int m){
    int ret = 1;
    while(n && m && ret){
        int nn = n % MD, mm = m % MD;
        if (nn < mm) return 0;
        else ret = (LL)ret * fac[nn] % MD * inv[mm] % MD * inv[nn - mm] % MD;
        n /= MD, m /= MD;
    }
    return ret;
}

void init(){
    fac[0] = 1;
    inv[0] = 1;
    inv[1] = 1;
    for(int i = 2; i < MD; i++){
        inv[i] = -(LL)(MD / i) * inv[MD % i] % MD;
    }
    for(int i = 1; i < MD; i++){
        fac[i] = ((LL)fac[i - 1] * i) % MD;
        inv[i] = ((LL)inv[i - 1] * inv[i]) % MD;
    }
}

void fft(Comp *a, int n, int f = 1){
    double arg = PI;
    for(int k = n >> 1; k; k >>= 1, arg *= 0.5) {
        cp wm(cos(arg), f * sin(arg)), w(1, 0);
        for (int i = 0; i < n; i += k, w *= wm) {
            int p = i << 1;
            if (p >= n) p -= n;
            for (int j = 0; j < k; ++j) tmp[i + j] = a[p + j] + w * a[p + k + j];
        }
        rep(i,n) a[i] = tmp[i];
    }
}

void calc(Comp *a, Comp *b, int n){
    fft(a, n, 1);
    fft(b, n, 1);
    rep(i, n) a[i] = a[i] * b[i];
    fft(a, n, -1);
    rep(i, n) a[i] /= n;
}

int calc(int n, int m){
    static int v[(M << 1) - 1], u[(M << 1) - 1], t[(M << 1) - 1];
    int size = 1;
    while (size <= 2 * m - 2) size *= 2;
    rep(i, 2 * m - 1) v[i] = 0;
    v[0] = 1;
    int now = 1;
    while(n){
        if (now < m){
            rep(i, 2 * m - 1) u[i] = 0;
            u[now] = 1;
        }
        else{
#ifdef FFT
            rep(i, size) fft_a[i] = fft_b[i] = Comp(0, 0);
            rep(i, m){
                fft_a[i] = Comp(u[i], 0);
                fft_b[i] = Comp(u[i], 0);
            }
            calc(fft_a, fft_b, size);
            rep(i, m * 2 - 1) u[i] = (LL)(fft_a[i].real() + EPS) % MD;
#else
            for(int i = 0; i < m; i++) t[i] = u[i], u[i] = 0;
            for(int i = 0; i < m; i++){
                for(int j = 0; j < m; j++){
                    u[i + j] = (u[i + j] + (LL)t[i] * t[j]) % MD;
                }
            }
#endif
            for(int i = 2 * m - 2; i >= m; i--){
                u[i - p] = (u[i - p] + (LL)a * u[i]) % MD;
                u[i - q] = (u[i - q] + (LL)b * u[i]) % MD;
                u[i] = 0;
            }
        }
        if (n & 1){
#ifdef FFT
            rep(i, size) fft_a[i] = fft_b[i] = Comp(0, 0);
            rep(i, m){
                fft_a[i] = Comp(u[i], 0);
                fft_b[i] = Comp(v[i], 0);
            }
            calc(fft_a, fft_b, size);
            rep(i, m * 2 - 1) v[i] = (LL)(fft_a[i].real() + EPS) % MD;
#else
            for(int i = 0; i < m; i++) t[i] = v[i], v[i] = 0;
            for(int i = 0; i < m; i++){
                for(int j = 0; j < m; j++){
                    v[i + j] = (v[i + j] + (LL)t[i] * u[j]) % MD;
                }
            }
#endif
            for(int i = 2 * m - 2; i >= m; i--){
                v[i - p] = (v[i - p] + (LL)a * v[i]) % MD;
                v[i - q] = (v[i - q] + (LL)b * v[i]) % MD;
                v[i] = 0;
            }
        }
        now <<= 1;
        n >>= 1;
    }
    return v[m - 1];
}

int main(){    
    init();
    while(~scanf("%d%d%d%d%d", &a, &b, &p, &q, &n)){
        if (p > q) swap(p, q), swap(a, b);
        int ret;
        if (p == q){
            if (n % p != 0){
                ret = 0;
            }
            else{
                ret = qmod(a + b, n);
            }
        }
        else if (q <= 1000){
            n += q - 1;
            ret = calc(n, q);
        }
        else{
            ret = 0;
            int x, y;
            int d = ex_gcd(p, q, x, y);
            if (n % d != 0){
                puts("0");
                continue;
            }
            x = ((LL)x * n / d % (q / d) + (q / d)) % (q / d);
            y = (n - p * x) / q;
            while(y >= 0){
                ret = (ret + (LL)qmod(a, x) * qmod(b, y) % MD * lucas(x + y, x) % MD) % MD;
                x += q / d;
                y -= p / d;
            }
        }
        printf("%d\n", (ret % MD + MD) % MD);
    }
    return 0;
}
\end{verbatim}
