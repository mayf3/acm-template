\subsubsection{组合数取模扩展1}
\begin{verbatim}
#include <cstdio>
#include <vector>

typedef long long LL;

using namespace std;

const int N = 100000 + 1;
const int MOD = 998244352;

int n, m, mod;
int fac[MOD];

int qmod(int a, int n, int p){
  int ret = 1; 
  while(n){
    if (n & 1) ret = 1LL * ret * a % p;
    a = 1LL * a * a % p;
    n >>= 1;
  }
  return ret;
}

int ex_gcd(int a,int b,int &x,int &y) {
    if (!a) return x = 0, y = 1, b;
    int g = ex_gcd(b % a, a, x, y);
    int t = y;
    y = x;
    x = t - (b / a) * y;
    return g;
}

int modular(int a, int b) { 
    return (a % b + b) % b; 
}

int invert(int x, int m)  { 
    int a, b;
    ex_gcd(x, m, a, b);
    return modular(a, m);
}

int china(const vector<int>& m,const vector<int>& b) {
    bool flag = false;
    int x, y, i,d,result,a1,m1,a2,m2;
    m1 = m[0]; a1 = b[0];
  for(i = 1; i < m.size(); ++i){
        m2 = m[i]; a2 = b[i];
        d = ex_gcd(m1, m2, x, y );
        if((a2-a1) % d != 0) flag = true;
        result = (x * ((a2-a1) / d ) % m2 + m2 ) % m2;
        a1 = a1 + m1 * result; 
        m1 = (m1 * m2) / d;   
        a1 = (a1 % m1 + m1) % m1;
    }
    if (flag) return -1;
    else return a1;
}

pair<int, int> work(int n, int p, int x){
  int a = n / x, b = n % x;
  int ret = 1LL * qmod(fac[x - 1], a, x) * fac[b] % x;
  int cnt = n / p, now = p;
  pair<int, int> ans = make_pair(1, 0);
  if (n > p) ans = work(n / p, p, x);
  return make_pair(1LL * ans.first * ret % x, cnt + ans.second);
}

int calc(int n, int m, int p, int c){
  int x = 1;
  for(int i = 0; i < c; i++) x = x * p;
  fac[0] = 1;
  for(int i = 1; i < x; i++){
    if (i % p == 0) fac[i] = fac[i - 1];
    else fac[i] = 1LL * fac[i - 1] * i % x;
  }
  pair<int, int> pa = work(n, p, x);
  pair<int, int> pb = work(m, p, x);
  pair<int, int> pc = work(n - m, p, x);
  int ret = pa.first, cnt = pa.second - pb.second - pc.second;
  ret = 1LL * ret * invert(pb.first, x) % x;
  ret = 1LL * ret * invert(pc.first, x) % x;
  return 1LL * ret * qmod(p, cnt, x) % x;
}

int check(int n, int p, int x){
  int tmp = 1;
  for(int i = 1; i <= n; i++){
    if (i % p == 0) continue;
    tmp = tmp * i % x;
  }
  return tmp;
}

int main(){
  while(~scanf("%d%d%d", &n, &m, &mod)){
    if (mod == 1){
      puts("0");
      continue;
    }
    vector<pair<int, int> > fac;
    vector<int> a, b;
    for(int i = 2; i * i <= mod; i++){
      if (mod % i) continue;
      fac.push_back(make_pair(i, 0));
      a.push_back(1);
      while(mod % i == 0){
        a[a.size() - 1] *= i;
        fac[fac.size() - 1].second ++;
        mod /= i;
      }
    }
    if (mod > 1){
      fac.push_back(make_pair(mod, 1));
      a.push_back(mod);
    }
    for(int i = 0; i < fac.size(); i++){
      b.push_back(calc(n, m, fac[i].first, fac[i].second));
    }
    int ret = china(a, b);
    printf("%d\n", ret);
  }
}
\end{verbatim}
