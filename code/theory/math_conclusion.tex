\subsubsection{数学结论}
\begin{verbatim}
五边形定理
五边形数 n * (3 * n - 1) / 2

fibonacci数性质:
f[n] = f[n - 1] + f[n - 2]
f[n + m + 1] = f[n] * f[m] + f[n + 1] * f[m + 1]
gcd(f[n], f[n + 1]) = 1
gcd(f[n], f[n + 2]) = 1
gcd(f[n], f[m]) = f[gcd(n, m)]
f[n+1]*f[n+1]-f[n]*f[n+2] = (-1)^n
sigma{f[i]^2, 1<=i<=n} = f[n]*f[n+1]
sigma{f[i], 0<=i<=n} = f[n+2] - 1
sigma{f[2*i-1],1<=i<=n} = f[2*n]
sigma{f[2*i],1<=i<=n} = f[2*n+1]-1
sigma{(-1)^i*f[i],0<=i<=n} = (-1)^n*(f[n+1]-f[n])+1
f[2*n-1]=f[n]^2-f[n-2]^2
f[2*n+1]=f[n]^2+f[n+1]^2
3*f[n]=f[n+2]+f[n-2]
f[n]=c(n-1,0)+c(n-2,1)+..c(n-1-m,m) (m<=n-1-m)
sigma{f[i]*i,1<=i<=n}=n*f[n+2]-f[n+3]+2

catalan数性质:
凸多边形三角剖分数
简单有序根树的计数
(0,0)走到(n,n)经过的点(a,b)满足a<=b的路径数
乘法结合问题
c[n+1] = (4 * n - 2) / (n + 1) * c[n]
c[n] = (2*n)!/(n!)/((n+1)!)

第一类stirling数性质
有正有负,其绝对值是n个元素的项目分作k个环排列的数量,s[n,k](n个人分成k组,每组再按特定顺序围圈)
s[n][0] = 0, s[1][1] = 1;
s[n+1][k]= = s[n][k - 1] + n * s[n][k]
|s[n][1]| = (n-1)!
s[n][k] = (-1)^(n+k)*|s[n][k]|
s[n][n-1] = -C(n,2)
x*(x-1)*(x-2)..(x-n+1) = sigma{s[n][k] * x ^k}

第二类stirling数性质
n个元素的集定义k个等价类的方法数目(n个人分成k组的方法数)
s[n][n] = s[n][1] = 1
s[n][k] = s[n - 1][k - 1] + k * s[n - 1][k]
s[n][n - 1] = C(n, 2)
s[n][2] = 2^(n-1)-1
s[n][k] = 1/(k!)sigma{(-1)^k-j * C(k, j) * j ^n, 1<=j<=k}

bell数性质
B[n] = sigma{s[n][k], 1<=k<=n}
B[n+1] = simga{C(n,k)*B[k], 0<=k<=n}
B[p+n] = B[n] + B[n + 1] (mod p)
B[p^m+n] = B[n] + B[n+1] (mod p)

多项式性质
f(x)不存在重根<=>gcd(f(x), f‘(x))的次数小于1次
多项式gcd可以用来判断两多项式是否有公共根

数论
a^n % b = a^(n % phi(b) + phi(b)) % b (n >= phi(b))
lucas定理 c(n, m) = c(n % p, m % p) * c(n / p, m / p) % p
lucas函数 满足 f(n, m) = f(n % p, m % p) * f(n / p, m / p) % p, 可以猜测满足

罗马数字性质
\end{verbatim}
