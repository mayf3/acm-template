\subsubsection{数学结论}
\begin{verbatim}
五边形定理
五边形数 n * (3 * n +- 1) / 2
(1-x)*(1-x^2)*(1-x^3)....=sigma{(-1)^k * x^(n * (3 * n (+-) 1) / 2)}
即f[n] = f[n - 1] + f[n - 2] - f[n - 5] - f[n - 7] + f[n - 12] + f[n - 15] - .....

fibonacci数性质:
f[n] = f[n - 1] + f[n - 2]
f[n + m + 1] = f[n] * f[m] + f[n + 1] * f[m + 1]
gcd(f[n], f[n + 1]) = 1
gcd(f[n], f[n + 2]) = 1
gcd(f[n], f[m]) = f[gcd(n, m)]
f[n+1]*f[n+1]-f[n]*f[n+2] = (-1)^n
sigma{f[i]^2, 1<=i<=n} = f[n]*f[n+1]
sigma{f[i], 0<=i<=n} = f[n+2] - 1
sigma{f[2*i-1],1<=i<=n} = f[2*n]
sigma{f[2*i],1<=i<=n} = f[2*n+1]-1
sigma{(-1)^i*f[i],0<=i<=n} = (-1)^n*(f[n+1]-f[n])+1
f[2*n-1]=f[n]^2-f[n-2]^2
f[2*n+1]=f[n]^2+f[n+1]^2
3*f[n]=f[n+2]+f[n-2]
f[n]=c(n-1,0)+c(n-2,1)+..c(n-1-m,m) (m<=n-1-m)
sigma{f[i]*i,1<=i<=n}=n*f[n+2]-f[n+3]+2

catalan数性质:
凸多边形三角剖分数
简单有序根树的计数
(0,0)走到(n,n)经过的点(a,b)满足a<=b的路径数
乘法结合问题
c[n+1] = (4 * n - 2) / (n + 1) * c[n]
c[n] = (2*n)!/(n!)/((n+1)!)

第一类stirling数性质
有正有负,其绝对值是n个元素的项目分作k个环排列的数量,s[n,k](n个人分成k组,每组再按特定顺序围圈)
s[n][0] = 0, s[1][1] = 1;
s[n+1][k]= = s[n][k - 1] + n * s[n][k]
|s[n][1]| = (n-1)!
s[n][k] = (-1)^(n+k)*|s[n][k]|
s[n][n-1] = -C(n,2)
x*(x-1)*(x-2)..(x-n+1) = sigma{s[n][k] * x ^k}

第二类stirling数性质
n个元素的集定义k个等价类的方法数目(n个人分成k组的方法数)
s[n][n] = s[n][1] = 1
s[n][k] = s[n - 1][k - 1] + k * s[n - 1][k]
s[n][n - 1] = C(n, 2)
s[n][2] = 2^(n-1)-1
s[n][k] = 1/(k!)sigma{(-1)^k-j * C(k, j) * j ^n, 1<=j<=k}

bell数性质
B[n] = sigma{s[n][k], 1<=k<=n}
B[n+1] = simga{C(n,k)*B[k], 0<=k<=n}
B[p+n] = B[n] + B[n + 1] (mod p)
B[p^m+n] = B[n] + B[n+1] (mod p)

多项式性质
f(x)不存在重根<=>gcd(f(x), f‘(x))的次数小于1次
多项式gcd可以用来判断两多项式是否有公共根

多项式取模
f[x] = 0 (mod m) 
m = m1 * m2 * m3 ... mk
Ti 表示 f[x] = 0 (mod mi)的解数,则T = T1 * T2 * T3...Tk

数论
a^n % b = a^(n % phi(b) + phi(b)) % b (n >= phi(b))
lucas定理 c(n, m) = c(n % p, m % p) * c(n / p, m / p) % p
lucas函数 满足 f(n, m) = f(n % p, m % p) * f(n / p, m / p) % p, 可以猜测满足

原根
2,4,p^k,2*p^k存在原根,存在原根则原根数量为phi(phi(n))
验证原根x = phi(n), x = p1^a1*p2^a2..pk^ak
原根满足t ^ (x / pi) != 1 (mod n)

x*x+y*y==n的整数解:
x*x+y*y==n的整数解个数num = 4 * sigma{H(d), d | n}
H(d) =
(1) 奇数 : (-1)^((d-1)/2)
(2) 偶数 : 0

平方和定理:
(1)费马平方和定理:
    奇质数能表示为两个平方数之和的充分必要条件是该素数被4除余1
(2)费马平方和定理的拓展定理:
    正整数能表示为两平方数之和的充要条件是在它的标准分解式中,形如素因子的指数是偶数
(3)Brahmagupta–Fibonacci identity
    如果两个整数都能表示为两个平方数之和,则它们的积也能表示为两个平方数之和。公式及拓展公式为
\end{verbatim}
$$(a^{2}+b^{2})(c^{2}+d^{2})=(ac-bd)^{2}+(ad+bc)^{2}=(ac+bd)^{2}+(ad-bc)^{2}$$
$$(a^{2}+n*b^{2})(c^{2}+n*d^{2})=(ac-n*bd)^{2}+n*(ad+bc)^{2}=(ac+n*bd)^{2}+n(ad-bc)^{2}$$
\begin{verbatim}
    从这个定理可以看出:如果不能表示为三个数的平方和,那么也就不能表示为两个数的平方和。
(4)四平方和定理:
    每个正整数都可以表示成四个整数的平方数之和
(5)表为3个数的平方和条件: 
    正整数能表示为三个数的平方和的充要条件是不能表示成的形式,其中和为非负 整数。

连分数
连分数(a+(n^0.5)) / b
开始时,i满足,(a+i)/b=floor((a+(n^0.5))/b),之后过程一样
如果不成功,则可以变换为(ab+((nb^2)^0.5))/(b^2),之后再来

杨氏矩阵又叫杨氏图表,它是这样一个矩阵,满足条件:

 
杨氏矩阵
(1)如果格子(i,j)没有元素,则它右边和上边的相邻格子也一定没有元素。
(2)如果格子(i,j)有元素a[i][j],则它右边和上边的相邻格子要么没有元素,要么有元素且比a[i][j]大。
1 ~ n所组成杨氏矩阵的个数可以通过下面的递推式得到:
f[1] = 1; f[2] = 2;
f[n] = f[n - 1] + (n - 1) * f[n - 2];

钩子公式:
对于给定形状,不同的杨氏矩阵的个数为:n!除以每个格子的钩子长度加1的积。其中钩子长度定义为该格子
右边的格子数和它上边的格子数之和。
\end{verbatim}
