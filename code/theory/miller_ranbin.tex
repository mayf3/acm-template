\subsubsection{米勒测试}
\begin{verbatim}
LL mul_mod(LL x, LL y, LL n) {
    if (!x) return 0;
    return (((x & 1) * y) % n + (mul_mod(x >> 1, y, n) << 1) % n) % n;
}

LL pow_mod(LL a, LL x, LL n) {
    LL ret = 1;
    while (x) {
        if (x & 1) ret = mul_mod(ret, a, n);
        a = mul_mod(a, a, n); x >>= 1;
    }
    return ret;
}

bool millerRabin(LL a, LL n) {
    LL r = 0, s = n - 1;
    while (!(s & 1)) {
        s >>= 1;
        ++r;
    }
    LL x = pow_mod(a, s, n);
    if (x == 1 || x == n - 1) return 1;
    for (int j = 1; j < r; ++j) {
        x = mul_mod(x, x, n);
        if (x == 1) return 0;
        if (x == n - 1) return 1;
    }
    return 0;
}

/*
    if n < 1,373,653, it is enough to test a = 2 and 3;
    if n < 9,080,191, it is enough to test a = 31 and 73;
    if n < 4,759,123,141, it is enough to test a = 2, 7, and 61;
    if n < 2,152,302,898,747, it is enough to test a = 2, 3, 5, 7, and 11;
    if n < 3,474,749,660,383, it is enough to test a = 2, 3, 5, 7, 11, and 13;
    if n < 341,550,071,728,321, it is enough to test a = 2, 3, 5, 7, 11, 13, and 17.
    */
bool isPrime(LL n) {
    for (int i = 2; i < 1000 && i < n; ++i)
        if (n % i == 0) return 0;
    if (!millerRabin(2, n)) return 0;
    if (!millerRabin(3, n)) return 0;
    if (!millerRabin(5, n)) return 0;
    if (!millerRabin(7, n)) return 0;
    return 1;
}
\end{verbatim}
