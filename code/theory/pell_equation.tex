\subsubsection{pell方程}
\begin{verbatim}
// x * x - D * y * y = 1, xn+yn*sqrt(d) = (x0+y0*sqrt(d))^n
// x * x - D * y * y = -1, xn+yn*sqrt(d) = (x0+y0*sqrt(d))^(2*n+1)
//
// x * x - D * y * y = -1, D为质数,有解即D!=3(mod 4)
// 当D==0(mod 4)时,无解
//
// a * x * x - b * y * y = c 
// get x0, y0 from x * x - a * b * y = 1
// get x1, y1 from a * x * x - b * y * y = c
// [xk] = [x0, by0] ^ k-1 * [x1]
// [yk] = [ay0, x0]            [y1]
//
bool pell( int D, int& x, int& y ) {  
    int sqrtD = sqrt(D + 0.0);  
    if( sqrtD * sqrtD == D ) return false;  
    int c = sqrtD, q = D - c * c, a = (c + sqrtD) / q;  
    int step = 0;  
    int X[] = { 1, sqrtD };  
    int Y[] = { 0, 1 };  
    while( true ) {  
        X[step] = a * X[step^1] + X[step];  
        Y[step] = a * Y[step^1] + Y[step];  
        c = a * q - c;  
        q = (D - c * c) / q;  
        a = (c + sqrtD) / q;  
        step ^= 1;  
        if( c == sqrtD && q == 1 && step ) {  
            x = X[0], y = Y[0];  
            return true;  
        }  
    }  
}

//{{{ pell x*x-d*y*y = -1

struct Matrix{
  int n, m;
  LL v[2][2];
}c, tmp, ans;

Matrix operator*(const Matrix &a, const Matrix &b){
  c.n = a.n, c.m = b.m;
  for(int i = 0; i < c.n; i++){
    for(int j = 0; j < c.m; j++){
      c.v[i][j] = 0;
      for(int k = 0; k < a.m; k++){
        c.v[i][j] = (c.v[i][j] + 1LL * a.v[i][k] * b.v[k][j]);
      }
    }
  }
  return c;
}

int n, l;
int base;
int a[N];

bool build(int n){
  base = 0;
  while(base * base <= n) base++;
  base--;
  if (base * base == n) return false;
  int k = base;
  int n_k = n - k * k;
  l = 0;
  a[l++] = k;
  while(true){
    int i1 = n_k - k % n_k;
    i1 += ((base - i1) / n_k) * n_k;
    a[l++] = (i1 + k) / n_k;
    if (a[l - 1] == 2 * base) break;
    k = i1;
    n_k = (n - k * k) / n_k;
  }
  return true;
}

void solve(){
  ans.n = 2, ans.m = 2;
  ans.v[0][0] = a[0], ans.v[0][1] = 1;
  ans.v[1][0] = 1, ans.v[1][1] = 0;
  for(int i = 1; i < l - 1; i++){
    tmp.n = 2, tmp.m = 2;
    tmp.v[0][0] = a[i], tmp.v[0][1] = 1;
    tmp.v[1][0] = 1, tmp.v[1][1] = 0;
    ans = ans * tmp;
  }
  if (ans.v[0][0] * ans.v[0][0] - n * ans.v[1][0] * ans.v[1][0] == -1){
    printf("%d %d\n", ans.v[0][0], ans.v[1][0]);
  }
  else{
    puts("No Solution");
  }
}

int main(){
  scanf("%d", &n);
  if (build(n)) solve();
}

//}}}
\end{verbatim}
